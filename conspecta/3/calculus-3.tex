\documentclass[a4paper]{report}

\usepackage{../mathstemplate}
\usepackage{wasysym}

\date{III семестр, осень 2023 г.}
\title{Матанализ. Неофициальный конспект}
\author{Лектор: Сергей Витальевич Кисляков\\ Конспектировал Леонид Данилевич}

\begin{document}
    \shorthandoff{"}
    \maketitle
    \tableofcontents
    \newpage
    \setcounter{lection}{0}


    \section{Литература}
    \numbers{
        \item Б. М. Макаров <<Теория меры и интеграла>>
        \item ? <<Интеграл Лебега>>
        \item Халмош <<Теория меры>>
    }


    \chapter{Теория меры}
    \newlection{6 сентября 2023 г.}

    Пусть $f: [a, b] \map \R$ --- ограниченная функция.

    Для того, чтобы интеграл Римана-Дарбу существовал, нам надо, чтобы она была какой-то хорошей --- с ограниченными колебаниями, часто просто требуется кусочная непрерывность.
    А как быть иначе?

    Запишем такое, не совсем верное рассуждение.
    \provehere[Рассуждение]{
        Пусть $|f(x)| \le M$ при $x \in [a, b]$.
        Разобьём отрезок $[-M, M]$ в объединение промежутков $[-M, M] = I_1 \sqcup \dots \sqcup I_k$, будем считать, что $\forall k: |I_k| < \eps$.

        Обозначим за $e_j \coloneqq f^{-1}(I_j)$.
        Видно, что $e_1 \sqcup \dots \sqcup e_k = [a, b]$ --- прообразы отрезков $I_j$ образуют разбиение $[a, b]$.

        Оценим суммы Дарбу следующим образом: \[S_{\Delta}f \le \sum\limits_{j = 1}^{k}\beta_j |e_j|\qquad s_{\Delta}f \ge \sum\limits_{j = 1}^{k}\alpha_j |e_j|\]
        где $|e|$ --- <<длина>> множества $e$.

        Заметим, что верхние и нижние суммы близки: $S_{\Delta}f - s_{\Delta}f = \sum\limits_{j = 1}^{b}(\beta_j - \alpha_j)|e_j| \le \eps \sum\limits_{j=1}^{k}|e_j| = \eps(b -a)$.

        Таким образом, проинтегрировали любую ограниченную функцию.
        В чём проблема?
    }
    Как естественным образом определить длину множества $|e|$?

    Надо, чтобы длина была аддитивной: $|e \sqcup f| = |e| + |f|$.
    \note{
        Можно определить длину на всех подмножествах $[a, b]$, но такое определение не конструктивно, и к тому же не единственно.
    }

    Пусть $I$ --- конечный промежуток, $\{I_j\}_{j=1}^{\infty}$ --- тоже конечные промежутки, такие, что $I = \bigsqcup\limits_{j = 1}^{\infty}I_j$.

    Также хочется, чтобы предельные переходы выполнялись: $|I| = \sum\limits_{j = 1}^{\infty}|I_j|$.
    Это называется \emph{счётной аддитивностью}.


    \section{Меры}
    Пусть $X$ --- множество, $\mathcal{A}$ --- система его подмножеств.
    Пока будем считать только, что $\o \in \mathcal{A}$.

    \definition[Функция множества]{
        \emph{Вещественная функция множества} $\phi: \mathcal{A} \map \R$, или \emph{комплексная функция множества} $\phi: \mathcal{A} \map \C$.
    }
    Вещественная функция множества $\phi: \mathcal{A} \map \R_{\ge 0}$ называется \emph{неотрицательной}.

    Иногда также разрешают функции приобретать значения на расширенной прямой $\phi: \mathcal{A} \map \overline{\R}_{\ge 0 } = [0, +\infty]$.

    \definition[Мера]{
        Аддитивная функция $\phi: \mathcal{A} \map \overline{\R}_{\ge 0}$.
    }
    Аддитивность означает, что в случае $e_1, \dots, e_k \in \mathcal{A}$, если $e \coloneqq \bigsqcup\limits_{j=1}^{k} e_j \in \mathcal{A}$, то $\phi(e) = \phi(e_1) + \dots + \phi(e_k)$.
    \note{
        Если $\phi$ --- аддитивная функция, то $\phi(\o) = \phi(\o) + \phi(\o)$, откуда $\phi(\o) = 0$ (формально, $\phi(\o) = \infty$ тоже подходит, но тогда из аддитивности $\phi \equiv +\infty$, это скучный случай).
    }
    \examples{
        \item Пусть $\mathcal{P}_0(\R)$ --- совокупность конечных промежутков, $\mathcal{P}(\R)$ --- совокупность всех промежутков.

        Тогда для обоих семейств можно ввести меру $\phi(I) = |I|$.

        Заметим, что аддитивность действительно выполняется: если отрезок $\angles{a, b}$ разбит на отрезки $\angles{a_0, a_1}, \dots, \angles{a_{n - 1}, a_n}$, где $a_0 = a, a_n = b$, то и правда
        \[b - a = \sum\limits_{j = 1}^{n}(a_j - a_{j - 1})\]
        \item То же самое можно сделать в $\R^n$: введём множества ограниченных и всех прямоугольных параллелепипедов. \gather{\mathcal{P}_0(\R^n) = \defset{I_1 \times \dots \times I_n}{\text{все }I_j\text{ --- конечные промежутки}} \\ \mathcal{P}(\R^n) = \defset{I_1 \times \dots \times I_n}{\text{все }I_j\text{ --- промежутки}}}

        Обозначим за $V_n$ объём на $\mathcal{P}$: $V_n(I_1 \times \dots \times I_n) = |I_1| \proddots |I_n|$, где бесконечность в произведении трактуется так: если есть хотя бы один нуль, то произведение равно нулю, иначе бесконечно.

        Почему эта мера аддитивна?

        Пусть $Q, Q_1, \dots, Q_k \in \mathcal{P}_0(\R^n)$, причём $Q = \bigsqcup\limits_{j = 1}^{k}Q_j$.

        \lemma{
            $V_n(Q) =\sum\limits_{j = 1}^{k}V_n(Q_j)$.
            \provehere{
                Пусть $f$ --- функция на $Q$, определим \[J(f) = \int\limits_{I_n} \left(\cdots \int\limits_{I_2}\left(\int\limits_{I_1}f(x_1, \dots, x_n)\d x_1\right)\d x_2 \cdots\right) \d x_n\]
                $J$, правда, определён не всегда --- иногда какая-то промежуточная функция может быть не интегрируема по Риману-Дарбу.

                $J$ корректно определена для некоторой совокупности функций, которые образуют линейное пространство.

                Рассмотрим $K = \delta_1 \times \dots \times \delta_n \subset Q$.
                Тогда для $\chi_K$ --- характеристической функции $K$ --- $J$ определён, причём $J(\chi_K) = |\delta_1| \proddots |\delta_n| = V_n(K)$.

                Отсюда видно, что так как $\chi_Q = \sum\limits_{j = 1}^{k}\chi_{Q_j}$, то $V_n(Q) = \sum\limits_{j=1}^{k}V_n(Q_j)$.
            }
        }
    }
    Пускай $\phi: \mathcal{A} \map [0, +\infty]$ --- аддитивная функция множеств. $\phi$ называется \emph{счётно аддитивной}, если для $a \in \mathcal{A}, \{a_j\}_{j=1}^{\infty} \subset \mathcal{A}$ верно $a = \bigsqcup\limits_{j=1}^{\infty}a_j \then \phi(a) = \sum\limits_{j = 1}^{\infty}\phi(a_j)$.
    \theorem{
        Объём в $\R^n$ --- счётно аддитивная функция на $\mathcal{P}_0(\R^n)$ (и на $\mathcal{P}(\R^n)$ тоже, но пока не надо).
        \provehere{
            \indentlemma{
                Пусть $Q_1, \dots, Q_k, Q \in \mathcal{P}_0(\R^n)$.
                \numbers{
                    \item Если $Q_1, \dots, Q_k$ попарно не пересекаются, и $\forall j: Q_j \subset Q$, то $\sum\limits_{j = 1}^{k}V_n(Q_j) \le V_n(Q)$.
                    \item Если $Q \subset \bigcup\limits_{j=1}^{k}Q_j$ (условий на дизъюнктность нет), то $V_n(Q) \le \sum\limits_{j = 1}^{k}V_n(Q_j)$.
                }
            }        {
                \numbers{
                    \item $\sum\limits_{j=1}^{k}\chi_{Q_j} \le \chi_Q$ (поточечно), применяем ранее определённый функционал $J$.
                    \item $\sum\limits_{j=1}^{k}\chi_{Q_j} \ge \chi_Q$ (поточечно), применяем ранее определённый функционал $J$.
                }
            }
            Пусть $Q, Q_j \in \mathcal{P}_0(\R^n)$, где $j \in \N$, $Q = \bigsqcup\limits_{j = 1}^{\infty}Q_j$.
            \bullets{
                \item Рассмотрим $k$ параллелепипедов $Q_1, \dots, Q_k \subset Q$. Применяя лемму, получаем $\sum\limits_{j = 1}^{k}V_n(Q_j) \le V_n(Q)$.
                Это верно для каждого $k$, переходя к пределу сразу получаем $\sum\limits_{j = 1}^{\infty}V_n(Q_j) \le V_n(Q)$.
                \note{Эта часть верна для любой аддитивной меры.}
                \item Докажем обратное: $\sum\limits_{j = 1}^{\infty}V_n(Q_j) \ge V_n(Q)$.

                Пусть $Q = I_1 \times \dots \times I_n$. Если $\exists s: I_s = 0$, то доказывать нечего.

                Выберем $\eps > 0$.
                Существуют замкнутые отрезки $\overline{I_1}, \dots, \overline{I_s}$, такие, что $\overline{I_j} \subset I_j$, причём для $\overline{Q} = \overline{I_1} \times \dots \times \overline{I_n}$ его объём уменьшился несильно: $V_n(Q) \le V_n\left(\overline{Q}\right)+ \eps$.

                Аналогично раздуем составляющие параллелепипеды: $\forall j \in \N$ построим $\tilde{Q_j} = \tilde{I_1} \times \dots \times \tilde{I_n}$, так что открытый интервал $\tilde{I_j} \supset I_j$, причём $V_n(\tilde{Q_j}) \le V_n(Q_j) + \frac{\eps}{2^j}$.

                Теперь замкнутый параллелепипед покрывается открытыми, значит, можно выбрать конечное подпокрытие, сразу получив оценку (для некоего $k \in \N$) \[V_n(Q) - \eps \le \sum\limits_{j = 1}^{k}\left(V_n(Q_j) + \frac{\eps}{2^j}\right) \le \sum\limits_{j = 1}^{\infty}\left(V_n(Q_j) + \frac{\eps}{2^j}\right)\] Устремляя $\eps \to 0$, получаем требуемое $V_n(Q) \le \sum\limits_{j = 1}^{\infty}V_n(Q_j)$.\qedhere
            }
        }
    }


    \section{Обобщения}

    \subsection{Область задания меры (системы множеств)}
    Пусть $X$ --- множество, $\mathcal{A}$ --- система его подмножеств ($\o \in \mathcal{A}$).

    \definition[Кольцо]{Система множеств $\mathcal{A}$, такая что $\forall a, b \in \mathcal{A}: (a \cap b), (a \cup b), (a \sm b) \in \mathcal{A}$.}
    \example[Кольцо]{
        Объединения конечного числа отрезков (или даже параллелепипедов~(\cref{multiplication_of_semirings})) ($\mathcal{P}(\R^n)$ или $\mathcal{P}_0(\R^n)$).
    }

    \definition[Алгебра]{Кольцо $\mathcal{A}$, такое что $X \in \mathcal{A}$.}
    \note{В алгебре $\forall a \in \mathcal{A}: a^\comp \in \mathcal{A}$.
    В частности, из-за законов де Моргана достаточно проверять только одно из $(a \cup b), (a \cap b) \in \mathcal{A}$}

    \definition[Полукольцо]{
        Система множеств $\mathcal{A} \subset 2^X$, такое что $\forall a, b \in \mathcal{A}: (a \cap b) \in \mathcal{A}$, а разность $(a \sm b)$ есть объединение конечного числа попарно непересекающихся подмножеств из $\mathcal{A}$.
    }
    \example[Полукольцо]{
        Отрезки и конечные отрезки (или даже параллелепипеды~(\cref{multiplication_of_semirings})) ($\mathcal{P}(\R^n)$ или $\mathcal{P}_0(\R^n)$).
    }

    Пусть $X, Y$ --- множества, $\mathcal{A} \subset 2^X, \mathcal{B} \subset 2^Y$ --- полукольца.
    \definition[Обобщённый прямоугольник]{
        Произведение $a \times b$, где $a \in \mathcal{A}, b \in \mathcal{B}$.
    }
    \theorem{\label{multiplication_of_semirings}
    Множество обобщённых прямоугольников $\mathcal{C} = \defset{a \times b}{a \in \mathcal{A}, b \in \mathcal{B}}$ есть полукольцо в $X \times Y$.
    \provebullets{
        \item $\o \in \mathcal{C}: \o \times \o = \o$.
        \item $(a_1 \times b_1) \cap (a_2 \times b_2) = (a_1 \cap a_2) \times (b_1 \cap b_2)$, поэтому $\mathcal{C}$ замкнуто относительно пересечения.
        \item Рассмотрим $u, v \in \mathcal{C}$. $u \sm v = u \sm (u \cap v)$, поэтому можно считать, что $v \subset u$.

        Пусть $u = a_1 \times b_1, v = a_2 \times b_2$. Так как $v \subset u$, то $b_2 \subset b_1, a_2 \subset a_1$.
        Пусть $a_1 \sm a_2 = \bigsqcup\limits_{s = 1}^{n}e_s$, $b_1 \sm b_2 = \bigsqcup\limits_{t = 1}^{m}f_t$.

        Несложно видеть, что $u \sm v = \left(a_2 \sqcup \bigsqcup\limits_{s = 1}^{n}e_s\right) \times \left(b_2 \sqcup \bigsqcup\limits_{t = 1}^{m}f_t\right) \sm (a_2 \times b_2)$, что есть объединение $(n + 1)(m + 1) - 1$ понятного обобщённого прямоугольника.
    }
    }
    \note{
        Даже если $\mathcal{A}$ и $\mathcal{B}$ --- кольца или алгебры, множества обобщённых прямоугольников могут всё равно образовывать лишь полукольцо.
    }
    \definition[Мера на полукольце]{Неотрицательная аддитивная функция множества (возможно, принимающая значения $+\infty$).}
    \ok
    Пусть $\mathcal{P}_0(\R)$ --- полукольцо конечных отрезков, $f: \R \map \R$ --- нестрого возрастающая функция.
    \definition[Квазидлина, порождённая $f$]{\label{quasilength}
    $\mu_f(\angles{a, b}) \bydef f(b) - f(a)$.
    }
    Эта квазидлина, понятное дело, аддитивна, но не для всех функций она счётно аддитивна.
    \counterexample{
        Пусть $f(x) = \all{0,x < 1 \\ 1,x \ge 1}$.
        Тогда $1 = f([0, 1)) \ne \sum\limits_{i = 1}^{\infty}f\left(\left[1 - \frac{1}{2^{i-1}}, 1 - \frac{1}{2^i}\right)\right) = 0$
    }
    \theorem{
        Пускай $\mathcal{A} \subset 2^X, \mathcal{B} \subset 2^Y$ --- полукольца, $\mu$ и $\nu$ на них --- (конечные) меры, определим меру на произведении
        \[\gamma(u \times v) \coloneqq \mu(u) \cdot \nu(v)\]
        Утверждается, что $\gamma$ аддитивна~(\cref{about-product})
    }
    \newlection{8 сентября 2023 г.}
    Пусть $\mathcal{A}$ --- полукольцо подмножеств $2^X$.
    \definition[Кольцо, порождённое $\mathcal{A}$]{
        $\mathcal{R}(\mathcal{A}) \bydef \defset{\bigsqcup\limits_{j = 1}^{k}d_j}{d_j \in \mathcal{A}}$ --- всевозможные конечные дизъюнктные объединения.
    }
    \lemma{
        $\mathcal{R}(\mathcal{A})$ есть кольцо подмножеств $2^X$.
        \provehere{
            Пусть $u = c_1 \sqcup \dots \sqcup c_s; v = d_1 \sqcup \dots \sqcup d_t$.
            \bullets{
                \item Проверим, что $\mathcal{R}(\mathcal{A})$ замкнуто относительно пересечения.
                В самом деле, \[u \cap v = \bigsqcup\limits_{i,j}(c_i \cap d_j)\]
                \item Проверим, что $\mathcal{R}(\mathcal{A})$ замкнуто относительно разности: индукция по $t$.

                \underline{База:} $t = 1$. \[(c_1 \sqcup \dots \sqcup c_s) \sm d_1 = (c_1 \sm d_1) \sqcup \dots \sqcup (c_s \sm d_1)\]

                \underline{Переход:} \[(c_1 \sqcup \dots \sqcup c_s) \sm (d_1 \sqcup \dots \sqcup d_t) = \Big((c_1 \sqcup \dots \sqcup c_s) \sm (d_1 \sqcup \dots \sqcup d_{t-1})\Big) \sm d_t\]
                \item Проверим, что $\mathcal{R}(\mathcal{A})$ замкнуто относительно объединения.
                \[u \cup v = (u \sm v) \sqcup (v \sm u) \sqcup (u \cap v)\qedhere\]
            }
        }
    }
    Пусть $\mathcal{B} \subset 2^X$ --- полукольцо.
    Среди всех колец, $\mathcal{C} \supset \mathcal{B}$ есть наименьшее --- это их пересечение.
    \fact{
        Это кольцо $\mathcal{C}$ получается, как $\mathcal{R}(\mathcal{B})$.
    }


    \section{Поговорим про интеграл}
    $\mathcal{A} \subset 2^X$ --- полукольцо, $\mu: \mathcal{A} \map [0, +\infty]$ --- мера.
    \definition[Простая функция (относительно $\mathcal{A}$)]{
        Функция вида $f = \sum\limits_{i=1}^{k}\alpha_i\chi_{e_i}$, где $e_i \in \mathcal{A}$, $\forall 1 \le i < j \le k: e_i \cap e_j = \o$.
    }
    Определим <<хиленький интеграл>>, который пока не будем обозначать $\int$:
    \definition[Интеграл от простой функции по мере $\mu$]{
        $I_\mu(f) = \sum\limits_{j = 1}^{k}\alpha_i\mu(e_i)$, если это имеет смысл (считается, что $0 \cdot \infty = 0$, но $(-\infty) + (+\infty)$ не определено).
    }
    \lemma{
        Интеграл от простой функции не зависит от её представления в виде суммы.
        \provehere{
            Пусть $f = \sum\limits_{i=1}^{k}\alpha_i\chi_{e_i} = \sum\limits_{j=1}^{m}\beta_j\chi_{e_j'}$, где $\alpha_i, \beta_j \ne 0$.

            Обозначим $A = \defset{x \in X}{f(x) \ne 0}$ (кстати, носитель $\supp f \bydef \Cl\defset{x \in X}{f(x) \ne 0}$).
            Очевидно, $(e_1, \dots, e_k)$, как и $(e_1', \dots, e_m')$ --- разбиения $A$.
            У них есть общее измельчение $e''$, причём на каждом элементе $e''_{i,j} \coloneqq e_i \cap e_j'$ выполняется $\alpha_i = \beta_j$, откуда оба интеграла от простой функции --- через $e$ и через $e'$ --- совпадают с определением через $e''$:
            \[\sum\limits_{i = 1}^{k}\alpha_i\mu(e_i) = \sum\limits_{i = 1}^{k}\alpha_i\mu\left(\bigsqcup\limits_{j = 1}^{m}e_{i,j}\right) = \sum\limits_{i = 1}^{k}\sum\limits_{j = 1}^{m}\alpha_i\mu(e_{i,j}) = \sum\limits_{j = 1}^{m}\beta_j\mu\left(\bigsqcup\limits_{i = 1}^{k}e_{i,j}\right) = \sum\limits_{j = 1}^{m}\beta_j\mu(e_j')\]
            Если какой-то $e_i$ бесконечен, то один из конечного числа кусочков, на которые мы его разобьём ($e_i \cap e_j'$) тоже будет бесконечным, поэтому в случае бесконечностей (если обе суммы определены) обе суммы будут бесконечностями одного знака.
        }
    }
    \properties[Интеграл от простой функции]{
        \item $I_\mu(c \cdot f) = c \cdot I_\mu(f)$
        \item Если $f, g$ --- простые функции, то $f + g$ --- тоже простая, причём $I_\mu(f + g) = I_\mu(f) + I_\mu(g)$ (если в сумме двух интегралов нет бесконечностей разных знаков).
        \provehere{
            Пусть $f = \sum\limits_{i = 1}^{k}\alpha_i\chi_{e_i}; \quad g = \sum\limits_{j = 1}^{m}\beta_j \chi_{e_j'}$, где $\alpha_i, \beta_j \ne 0$.

            Положим $A \coloneqq \bigsqcup\limits_i{e_i}; \quad B  \coloneqq \bigsqcup\limits_j{e_j'}$.
            Рассмотрим $(A \sm B), (B \sm A), (A \cap B)$ --- все они лежат в $\mathcal{R}(\mathcal{A})$.

            Будем считать, что $(e_1, \dots, e_k)$, как разбиение $A$, измельчено так, что оно уважает разбиение $(A \sm B) \sqcup (A \cap B) = A$.

            Аналогично считаем, что $e'$ уважает разбиение $(B \sm A) \sqcup (B \cap A) = B$.

            Теперь $\mathcal{E} \coloneqq \defset{e_i \in \{e_i\}_{i = 1}^{k}}{e_i \subset A \cap B}$ и $\mathcal{E}' \coloneqq \defset{e_j' \in \{e_j'\}_{j = 1}^{m}}{e_j \subset A \cap B}$ --- разбиения $A \cap B$.
            Измельчим те элементы, которые попали в $\mathcal{E}$ и $\mathcal{E}'$, теперь ещё считаем, что $e$ и $e'$ уважают друг друга.
            Можно считать, что и $f$, и $g$ определены на разбиениях $\{e_i\}_{i=1}^{k} \cup \{e_j'\}_{j=1}^{m}$, и теперь по определению $f + g$ является простой функцией, и $I(f + g) = I(f) + I(g)$.
        }
        \item Для двух простых интегрируемых функций $f \le g \then I_\mu(f) \le I_\mu(g)$.
        \provehere{
            Если интегралы --- бесконечности одного знака, то доказывать нечего.

            Иначе $I_\mu(g)$ и $I_\mu(-f)$ не являются бесконечностями разного знака, то есть определено \[I_\mu(g - f) = I_\mu(g) - I_\mu(f)\]
            Но $(g - f)$ --- функция неотрицательная, по определению её интеграл неотрицателен.
        }
    }
    \lemma{\label{subunion_is_less_than_the_whole}
    Пусть $\mathcal{A}$ --- полукольцо с мерой $\mu$; $a, a_1 \dots, a_k \in \mathcal{A}$.
    \bullets{
        \item Если $a_j$ попарно дизъюнктны, причём $a_j \subset a$, то $\sum\limits_{j = 1}^{k}\mu(a_j) \le \mu(a)$.
        \item Если $a \subset \bigcup\limits_{j = 1}^{k}a_j$ (условий на дизъюнктность нет), то $\mu(a) \le \sum\limits_{j = 1}^{k}\mu(a_j)$.
    }
    \provebullets{
        \item $I_\mu\left(\chi_{\bigcup a_j}\right) \le I_\mu\left(\chi_a\right)$ так как $\chi_{\bigcup a_j} \le \chi_a$.
        \item $I_\mu\left(\chi_{\bigcup a_j}\right) \ge I_\mu\left(\chi_a\right)$, так как $\chi_{\bigcup a_j} \ge \chi_a$.
    }
    }
    \theorem{\label{about-product}
    Пусть $\mathcal{A} \subset 2^X, \mathcal{B} \subset 2^Y$ --- полукольца обобщённых прямоугольников. Положим $\mathcal{C} \coloneqq \mathcal{A} \times \mathcal{B}$, это полукольцо подмножеств $X \times Y$.

    Пусть $\mu$ --- мера на $\mathcal{A}$, $\nu$ --- мера на $\mathcal{B}$.
    Определим произведение мер $(\mu \otimes \nu)(a \times b) \bydef \mu(a)\nu(b)$.

    Утверждается, что $\mu \otimes \nu$ --- мера на $\mathcal{C}$.
    \provehere{
        Докажем аддитивность. Пусть $P = a \times b$, причём $P = \bigsqcup\limits_{j = 1}^{k}P_j$, где $P_j = a_j \times b_j$.

        Проверим, что $(\mu \otimes \nu)(P) \overset{?}= \sum\limits_{j=1}^k(\mu \otimes \nu)(P_j)$.

        Разделим переменные: $\chi_{c \times d}(s, t) = \chi_c(s) \cdot \chi_d(t)$.

        Дано, что $\chi_P = \sum\limits_{j = 1}^{k}\chi_{P_j}$, то есть $\chi_a(s)\chi_b(t) = \sum\limits_{j = 1}^{k}\chi_{a_j}(s)\chi_{b_j}(t)$.

        Интегрируем ($I_{\nu,t}$ означает интеграл по мере $\nu$ функции от переменной $t$ при фиксированном $s$):
        \gather{
            I_{\nu, t}\left(\chi_a(s)\chi_b(t)\right) = \sum\limits_{j = 1}^{k}I_{\nu,t}\left(\chi_{a_j}(s)\cdot \chi_{b_j}(t)\right)\quad\then\quad
            \chi_a(s)\nu(b) = \sum\limits_{j = 1}^{k}\chi_{a_j}(s)\cdot \nu(b_j) \\
            I_{\mu}\left(\chi_a(s)\nu(b)\right) =\sum\limits_{j = 1}^{k} I_{\mu}\left(\chi_{a_j}(s)\cdot \nu(b_j)\right) \quad\then\quad
            \mu(a)\nu(b) = \sum\limits_{j = 1}^{k}\mu(a_j)\nu(b_j)
        }
        Данное доказательство также допускает бесконечные меры.
    }
    }
    \note{
        Пусть $\mu$ --- мера на полукольце $\mathcal{A}$.
        Для $e \in \mathcal{R}(\mathcal{A})$ положим $\overline{\mu}(e) = I_\mu(\chi_e)$.

        Введённая $\overline{\mu}$ --- мера на $\mathcal{R}(\mathcal{A})$, понятно, что это единственно возможное продолжение --- единственная (аддитивная) мера на $\mathcal{R}(\mathcal{A})$, такая, что её сужение на $\mathcal{A}$ совпадает с $\mu$.
    }
    \note{
        Если меру определять на кольце, а не на полукольце, то аддитивность достаточно проверять для двух множеств: $e_1, e_2 \in \mathcal{R}(\mathcal{A}) \then e_1 \cup e_2 \in \mathcal{R}(\mathcal{A})$.
    }

    \subsection{Про счётную аддитивность}
    \definition[Регулярная мера $\mu$]{Мера, удовлетворяющая условиям:
    \numbers{
        \item $\forall a \in \mathcal{A}: \mu(a) = \inf\defset{\mu(U)}{U \supset a;~U \text{ открыто};~U \in \mathcal{A}}$.
        \item $\forall a \in \mathcal{A}: \mu(a) = \sup\defset{\mu(U)}{K \subset a;~K \text{ компактно};~K \in \mathcal{A}}$.
    }
    }
    \example[Регулярная мера]{
        Мера Лебега на $\mathcal{P}_0(\R^n)$.
    }
    \precaution{
        Для полукольца возможно бесконечных параллелепипедов теорема Александрова не применима: $\R \times \{0\} \subset \R^2$ не регулярно сверху, всякий параллелепипед, содержащий $\R \times \{0\}$ уже имеет бесконечную меру.
    }
    \theorem[А. Д. Александров]{
        Пусть $X$ --- топологическое пространство, $\mathcal{A} \subset 2^X$ --- полукольцо, $\mu$ --- регулярная мера на $\mathcal{A}$.

        Утверждается, что $\mu$ счётно аддитивна.
        \provehere{
            Рассмотрим $a \in \mathcal{A}, \{a_j\}_{j = 1}^{\infty}\subset\mathcal{A}$. Пусть $a = \bigsqcup\limits_{j = 1}^{\infty}a_j$.
            Для доказательства $\mu(a) = \sum\limits_{j = 1}^{\infty}\mu(a_j)$ покажем неравенства в обе стороны.
            \bullets{
                \item $\forall k \in \N: \sum\limits_{j =1}^{k}\mu(a_j) \le \mu(a)$~(\cref{subunion_is_less_than_the_whole}), производим предельный переход.
                \item Если $\mu(a_j) = \infty$, или $\mu(a) = 0$, то доказывать нечего.

                Выберем $\eps > 0$.
                Найдём такие $U_j, K \in \mathcal{A}$, что $U_j$ открыты, $K$ компактно, $U_j \supset a_j, K \subset a$, причём $\mu(U_j) \le \mu(a_j) + \frac{\eps}{2^j}$ и $\mu(K) \ge \mu(a) - \eps$.

                Так как из открытого покрытия компакта можно выделить конечное подпокрытие (и пусть $N$ --- максимальный номер элемента подпокрытия), то \[\mu(a) - \eps \le \mu(K) \le \sum\limits_{j = 1}^{N}\mu(U_j) \le \sum\limits_{j = 1}^{N}\left(\mu(a_j) + \frac{\eps}{2^j}\right) = \sum\limits_{j = 1}^{\infty}\mu(a_j) + \eps\]
                Устремляя $\eps \to 0$, получаем необходимое.\qedhere
            }
        }
    }
    \examples[Счётно-аддитивные меры]{
        \item Пусть $X$ --- (возможно бесконечное) множество, $\mathcal{A}$ --- семейство всех его конечных подмножеств.
        Можно определить $\mu(a) = \#(a)$ --- мощность множества $a \in \mathcal{A}$.

        Она счётно-аддитивная, так как если $a = \bigsqcup\limits_{j = 1}^{\infty}a_j$, причём $a \in \mathcal{A}$, то почти все (кроме конечного числа) $a_j = \o$.
        \item Можно продолжить эту меру на $2^X$:
        \[\mu(b) = \all{\#(b),&b\text{ конечно}\\+\infty,&\text{ иначе}}\]
        \item Пусть $\{\xi_x\}_{x \in X} \subset \R_{\ge 0}$ --- числовое семейство.
        Можно определить $\nu: 2^{X} \map \overline{\R}_{\ge 0}, \nu(e) = \sum\limits_{x \in e}\xi_x$.

        Если семейство суммируемо, то мера конечна.
    }
    \newlection{20 сентября 2023 г.}
    Вспомним, что мы определяли квазидлину $\mu_f(\angles{a, b}) = f(b) - f(a)$ для возрастающей функции $f: \R \map \R$~(\cref{quasilength}).
    Это функция может быть не счётно аддитивной, что случается, если $f$ разрывна.

    Поправим это определение, чтобы мера стала счётно-аддитивной.
    Пусть $f: \angles{a, b} \map \R$ --- возрастающая функция.

    Рассмотрим $\mathcal{P}(\angles{a, b})$ --- полукольцо промежутков, содержащихся в $\angles{a, b}$, и произвольно доопределим $f$ на некотором открытом интервале, содержащем $\angles{a, b}$ (скажем, если $a \in \angles{a, b}$, то есть $\angles{a, b}$ замкнут слева, то положим $f(a - \eps) = f(a) - \eps$ для $\eps \in (0, 1)$).
    \definition[Стилтьесова длина]{
        Длина, определённая по формуле
        \[\mu_f(\angles{c, d}) = \all{f(d-) - f(c-),&\angles{c, d} = [c, d)\\f(d-) - f(c+),&\angles{c,d}=(c,d)\\f(d+) - f(c+),&\angles{c, d} = (c, d]\\f(d+) - f(c-),&\angles{c,d}=[c,d]}\]
        где $f(x_0+) \bydef \lim\limits_{x \to x_0+}f(x)$ и $f(x_0-) \bydef \lim\limits_{x \to x_0-}f(x)$.
    }
    \proposal{
        Стилтьесова длина счётно аддитивна.
        \provehere{
            Выполняется теорема Александрова.
            Проверим, например, что для полуинтервала $[c, d)$ мера регулярна.

            Рассмотрим $\eps > 0$, для открытого подмножества, содержащего $[c, d)$ выберем $(c - \delta, d)$.
            Для достаточно маленьких $\delta$: $f((c - \delta)+) > f(c-) - \eps$.
            В качестве компактного подмножества, содержащегося в $[c, d)$, выберем $[c, d - \delta]$.
            При достаточно маленьких $\delta$: $f((d - \delta)+) > f(d-) - \eps$.

            Также можно проверить регулярность для бесконечных промежутков.
        }
    }

    \subsection{Продолжение меры}
    Продолжать можно только счётно-аддитивные меры, иначе будет неоднозначно.
    \definition[$\sigma$-алгебра]{
        Такая алгебра множеств $\mathcal{A} \subset 2^X$, что она замкнута относительно счётных операций:
        если семейство $\{A_i\}_{i \in \N}$ лежит в $\mathcal{A}$, то $\bigcup\limits_{i \in \N}A_i \in \mathcal{A}$ и $\bigcap\limits_{i \in \N}A_i \in \mathcal{A}$.
    }
    \theorem{
        Пусть $\mathcal{C} \subset 2^X$ --- система подмножеств $X$.
        Тогда в $X$ есть наименьшая $\sigma$-алгебра, содержащая $\mathcal{C}$.
        \provehere{
            Пересечение любого множества $\sigma$-алгебр --- $\sigma$-алгебра.
            Хотя бы одна есть --- это $2^X$.
            Тогда в качестве наименьшей подойдёт пересечение всех $\sigma$-алгебр, содержащих $\mathcal{C}$.
        }
    }
    \theorem{
        Пусть $\mathcal{P}_0(\R^n)$ --- полукольцо всех конечных прямоугольных параллелепипедов, а $\mathcal{A}$ --- наименьшая $\sigma$-алгебра, содержащая $\mathcal{P}_0(\R^n)$.

        Тогда объём на $\mathcal{P}_0(\R^n)$ единственным образом продолжается до счётно аддитивной меры $\lambda_n$ --- \emph{$n$-мерной меры Лебега} на $\mathcal{A}$.
        \provehere{
            Мы это докажем здесь~(\cref{lebesgue-caratheodory}). Сейчас приведём схему доказательства.
            \bullets{
                \item Обозначим $n$-мерный объём на параллелепипедах из $\mathcal{P}_0(\R^n)$ за $v_n$.
                Построим $v_n \rightsquigarrow v_n^*$, заданную на $2^{\R^n}$, которая не будет даже аддитивной.

                Тем не менее, для всякого $P \in \mathcal{P}_0(\R^n)$: $v_n^*(a) = v_n(P)$
                \item Теперь сузим $v_n^*$ на некоторую $\sigma$-алгебру, содержащую $\mathcal{P}_0(\R^n)$, причём там эта функция будет уже и аддитивной, и счётно аддитивной.\qedhere
            }
        }
    }
    \fact{
        Все открытые, а значит, и все замкнутые множества, лежат в наименьшей $\sigma$-алгебре $\mathcal{A}$, содержащей $\mathcal{P}_0(\R^n)$.
        \provehere{
            Открытое множество представимо, как объединение кубов с рациональными координатами вершин, содержащихся в нём.
        }
    }
    Пусть $Y$ --- топологическое пространство.
    \definition[Борелевская $\sigma$-алгебра]{\label{borel-sets}
    Наименьшая $\sigma$-алгебра подмножеств множества $Y$, содержащая все открытые множества.
    Обозначают $\mathcal{B}(Y)$.
    }
    \note{
        Выше определённая $\mathcal{A}$ совпадает с $\mathcal{B}(\R^n)$.
    }
    \fact{
        Пусть $\mathcal{A}$ --- алгебра подмножеств множества $X$.
        Следующие утверждения эквивалентны.
        \numbers{
            \item $\mathcal{A}$ --- $\sigma$-алгебра.
            \item Для всех $A_i \in \mathcal{A}, i \in \N$ верно, что $\bigcup\limits_{i \in \N}A_i \in \mathcal{A}$.
            \item Для всех $A_i \in \mathcal{A}, i \in \N$ верно, что $\bigcap\limits_{i \in \N}A_i \in \mathcal{A}$.
            \item Для всех $A_i \in \mathcal{A}$, таких что $A_1 \subset A_2 \subset \dots$ верно, что $\bigcup\limits_{i \in \N}A_i \in \mathcal{A}$.
            \item Для всех $A_i \in \mathcal{A}$, таких, что $A_1 \supset A_2 \supset \dots$ верно, что $\bigcap\limits_{i \in \N}A_i \in \mathcal{A}$.
            \item Для всех $A_i \in \mathcal{A}$, таких, что $A_i \cap A_j = \o$ для $i \ne j$ верно, что $\bigsqcup\limits_{i \in \N}A_i \in \mathcal{A}$.
        }
        \provebullets{
            \item[$2 \iff 3$] Закон де Моргана.
            \item[$1 \iff (2 \land 3)$] По определению.
            \item[$2 \then 4$] Очевидно.
            \item[$4 \then 2$] Положим $\overline{A}_i \coloneqq A_1 \cup \dots \cup A_i$.
            Тогда $\overline{A}_i$ возрастают по включению, и $\bigcup\limits_{i \in \N}\overline{A}_i \in \mathcal{A}$.
            \item[$4 \iff 5$] Тоже закон де Моргана.
            \item[$4 \then 6$] Пусть $A_i \in \mathcal{A}$, причём $A_i \cap A_j = \o$ для $i \ne j$.
            Выберем $\tilde{A}_i \coloneqq A_1 \cup \dots \cup A_i$. Согласно (4) $\bigcup\limits_{i \in \N}\tilde{A}_i \in \mathcal{A}$.
            \item[$6 \then 4$] Пусть $A_i \in \mathcal{A}$, причём $A_i \subset A_{i + 1}$. Положим $e_1 = A_1$, $e_j = A_{j} \sm A_{j - 1}$ для $j \ge 2$.
            Тогда $e_i \cap e_j = \o$ для $i \ne j$, и $\bigsqcup\limits_{i \in \N}e_i = \bigcup\limits_{i \in \N}A_i \in \mathcal{A}$.
        }
    }
    \fact{
        Пусть $\mathcal{A} \subset 2^X$ --- $\sigma$-алгебра, $\mu$ --- мера на $\mathcal{A}$.
        Следующие условия эквивалентны.
        \numbers{
            \item $\mu$ счётно аддитивна.
            \item Если $A_i \in \mathcal{A}, A_i \cap A_j = \o$ для $i \ne j$, то $\mu\left(\bigsqcup\limits_{i \in \N}A_i\right) = \sum\limits_{i \in \N}\mu(A_i)$.
            \item Если $A_1 \subset A_2 \subset \dots$, то $\mu\left(\bigcup\limits_{i \in \N}A_i\right) = \lim\limits_{i \to \infty}\mu(A_i)$.
        }
        \provebullets{
            \item[$1 \iff 2$] Так как $\mathcal{A}$ --- $\sigma$-алгебра, то $\bigsqcup\limits_{i \in \N}A_i$ автоматически лежит в $\mathcal{A}$, и $1$ тавтологично $2$.
            \item[$2 \then 3$] Пускай $A_1 \subset A_2 \subset \dots$. Введём $e_1 = A_1$, $e_j = A_j \sm A_{j - 1}$ для $j \ge 2$. $e_i \cap e_j = \o$ для $i \ne j$.
            Тогда $\mu\left(\bigcup\limits_{i \in \N}A_i\right) = \mu\left(\bigsqcup\limits_{i \in \N}e_i\right) = \sum\limits_{i \in \N}\mu(e_i) = \lim\limits_{n \to \infty}\mu\left(\bigsqcup\limits_{i = 1}^{n}e_i\right) = \lim\limits_{n \to \infty}\mu(A_n)$.
            \item[$3 \then 2$] То же самое в обратном порядке.
        }
    }
    \precaution{
        Монотонность по убывающим последовательностям не выполняется:

        Рассмотрим на кольце $\mathcal{P}(\R)$ убывающие по включению множества $A_n \coloneqq (n, +\infty)$. Несмотря на то, что $A_1 \supset A_2 \supset \dots$, всё-таки $\mu\left(\bigcap\limits_{i \in \N}A_i\right) = \mu(\o) = 0 \ne \lim\limits_{n \to \infty}\mu(A_n) = \infty$.
    }
    \theorem{
        Если $B_i \in \mathcal{A}$, $B_1 \supset B_2 \supset \dots$, причём $\mu(B_1) < +\infty$, то $\mu\left(\bigcap\limits_{i \in \N}B_i\right) = \lim\limits_{j \to \infty}\mu(B_j)$.
        \provehere{
            Положим $A_i = B_1 \sm B_i$.

            Тогда $\bigcap\limits_{i \in \N}B_i = B_1 \sm \bigcup\limits_{i \in \N}A_i$, и $\mu\left(\bigcap\limits_{i \in \N}B_i\right) = \mu(B_1) - \mu\left(\bigcup\limits_{i \in \N}A_i\right) = \mu(B_1) - \lim\limits_{n \to \infty}\mu(A_n) = \mu(B_1) - \lim\limits_{n \to \infty}\left(\mu(B_1) - \mu\left(B_n\right)\right) = \lim\limits_{n \to \infty}\mu(B_n)$

            Так как $\mu(B_1)$ конечна, то все производимые вычитания справедливы --- не происходит вычитания бесконечности из бесконечности.
        }
    }
    \note{
        Если мера конечна, то справедливо и обратное.
    }
    \ok
    Пусть $X$ --- множество, $\mathcal{P}$ --- полукольцо его подмножеств, $\mu$ --- мера на $\mathcal{P}$ (аддитивная, но не факт, что счётно-аддитивная).

    \definition[Внешняя мера, построенная по $\mu$]{\label{outer-measure}
    Функция $\mu^*$, заданная на $2^X$, определяемая по формуле
    \[\mu^*(e) = \inf\defset{\sum\limits_{j \in \N}\mu(a_j)}{a_j \in \mathcal{P}, e \subset \bigcup\limits_{j \in \N}a_j}\]
    }
    \properties{
        \item $\mu^*(\o) = 0$. Так, покрытие счётным количеством пустых множеств имеет суммарную меру 0.
        \item $e_1 \subset e_2 \then \mu^*(e_1) \le \mu^*(e_2)$ --- монотонность.
        \item Внешняя мера совсем не факт, что является мерой (то есть аддитивна). Тем не менее, верна \emph{счётная полуаддитивность}: $e \subset \bigcup\limits_{i \in \N}e_i \then \mu^*(e) \le \sum\limits_{i \in \N}\mu^*(e_i)$.
        \provehere{
            Если хотя бы одно из $\mu^*(e_i)$ бесконечно, то доказывать нечего. Далее считаем, что $\forall i: \mu^*(e_i)$ конечно.

            Выберем $\eps > 0$.
            По определению внешней меры $\forall i, k \in \N: \exists a_{i,k} \in \mathcal{P}$, такие, что $\bigcup\limits_{k \in \N}a_{i,k} \supset e_i$, причём $\sum\limits_{k \in \N}\mu(a_{i,k}) \le \mu^*(e_i) + \frac{\eps}{2^i}$.

            Тогда $e \subset \bigcup\limits_{i,k \in \N}a_{i,k}$ и $\mu^*(e) \le \sum\limits_{i,k}\mu(a_{i,k}) = \sum\limits_{i \in \N}\sum\limits_{k \in \N}\mu(a_{i,k}) \le \sum\limits_{i}\mu^*(e_i) + \eps$.
        }
        \item Если $\mu$ счётно аддитивна, то $\mu^*\big|_{\mathcal{P}} = \mu$.
        \provehere{
            Для $b \in \mathcal{P}: \mu^*(b) \le \mu(b)$, так как можно выбрать покрытие из одного элемента.

            Докажем, что $\mu(b) \le \mu^*(b)$.
            Рассмотрим кольцо $\mathcal{R}(\mathcal{P})$ --- совокупность дизъюнктных объединений $e_1 \sqcup \dots \sqcup e_s$, где $e_i \in \mathcal{P}$.
            Мера $\mu$ единственным образом продолжается до меры $\overline{\mu}$ на $\mathcal{R}(\mathcal{P})$.

            \indentlemma{
                $\forall e \subset X: \mu^*(e) = \mu^{\triangle}(e) \coloneqq \inf\defset{\sum\limits_{j \in \N}\mu(c_j)}{\{c_j\}_{j \in \N} \subset \mathcal{P} \text{ и } e \subset \bigsqcup\limits_{i \in \N}c_i}$
            }{
                $\mu^*(e) \le \mu^{\triangle}(e)$, так как всякое дизъюнктное покрытие является покрытием.

                Если $e \subset \bigcup\limits_{i \in \N}a_i$, то можно рассмотреть дизъюнктное покрытие множествами $\overline{a}_i \coloneqq a_i \sm \left(a_1 \cup \dots \cup a_{i - 1}\right)$.

                Так как $\overline{a}_j \in \mathcal{R}(\mathcal{P})$ и $\overline{a}_j \subset a_j$, то $\overline{\mu}(\overline{a}_j) \le \mu(a_j)$.

                Согласно свойству $\mathcal{R}(\mathcal{P})$: $\overline{a}_j = \bigsqcup\limits_{s = 1}^{k_j}e_{j,s}$, где при данном $j$ все $e_{j,s}$ попарно не пересекаются.
                Но при разных $j$ они тем более не пересекаются, они лежат в разных $\overline{a}_j$.

                Таким образом, $\bigsqcup\limits_{j,s}e_{j,s} \supset e$, откуда $\mu^{\triangle}(e) \le \sum\limits_{j,s}\mu(e_{j,s}) = \sum\limits_{j}\overline{\mu}(\overline{a}_j) \le \sum\limits_{j}\mu(a_j)$.
                Переходя к инфимуму, получаем $\mu^{\triangle}(e) \le \mu^*(e)$.
            }
            Используя лемму, рассмотрим произвольное дизъюнктное покрытие $e_j \in \mathcal{P}$ такое, что $\bigsqcup\limits_{j \in \N}e_j \supset e$.
            Введём $\tilde{e}_j \coloneqq e_j \cap e$. Для них $\bigsqcup\limits_{j \in \N}\tilde{e}_j = e$.

            Согласно счётной аддитивности $\mu(e) = \sum\limits_{j \in \N}\mu(\tilde{e}_j) \le \sum\limits_{j \in \N}\mu(e_j)$.
            Переходя к инфимуму, получаем искомое.
        }
        \counterexample[Счётная аддитивность важна]{
            Пусть $l_f$ --- квазидлина, порождённая функцией $f(x) = \all{0,&x < 0\\1,&x \ge 0}$.

            Покажем, что внешняя мера $l^*_f$ везде равна нулю.
            Рассмотрим счётное покрытие прямой $\R = \bigsqcup\limits_{n \in \N_0}[n, n + 1) \sqcup \bigsqcup\limits_{n \in \Z}[-2^n, -2^{n - 1})$.
            Квазидлины всех составляющих полуинтервала равны $0$, значит, внешняя мера прямой равна $0$, но тогда по монотонности и внешние веры всех подмножеств тоже равны $0$.
        }
    }
    \newlection{27 сентября 2023 г.}

    \subsection{Предмера}
    Пускай $X$ --- множество.

    Вещи, обладающие свойствами внешней меры будут возникать у нас разными способами, поэтому удобно уже сейчас обобщить это понятие, аксиоматизировав его.
    \definition[Предмера]{
        Функция $\gamma: 2^X \map \R_+$, обладающая свойствами
        \numbers{
            \item $\gamma(\o) = 0$.
            \item Монотонность $a \subset b \then \gamma(a) \le \gamma(b)$.
            \item Счётная полуаддитивность $a \subset \bigcup\limits_{j \in \N}a_j \then \gamma(a) \le \sum\limits_{j \in \N}\gamma(a_j)$.
        }
    }
    \note{
        Из 3. следует 2., проверяется выбором $a_i = \all{b,&i=1\\\o,&i>1}$.
        Более того, можно не требовать положительности, она следует из монотонности по отношению к пустому множеству.
    }

    \begin{definition_env}[$\gamma$-измеримое множество $e \subset X$]
        \[\forall a \subset X: \gamma(a) = \gamma(a \cap e) + \gamma(a \sm e) = \gamma(a \cap e) + \gamma\left(a \cap e^\comp\right)\]
    \end{definition_env}
    \theorem[Лебег --- Каратеодори]{\label{lebesgue-caratheodory}
    Совокупность $\Sigma$ всех $\gamma$-измеримых множеств образует $\sigma$-алгебру, на которой функция $\gamma\big|_\Sigma$ счётно-аддитивна.
    \supplement{
        Если $\gamma = \mu^*$, где $\mu$ --- мера на полукольце $\mathcal{P}$, то все множества из $\mathcal{P}$ автоматически $\gamma$-измеримы.
    }
    \supplement{
        Если $\mu$ счётно аддитивна на исходном полукольце $\mathcal{P}$, то $\mu^*\big|_{\mathcal{P}} = \mu$.
    }
    \provebullets{
        \item Покажем, что $\Sigma$ --- алгебра множеств.
        \bullets{
            \item Определение симметрично относительно $e$ и $e^\comp$, поэтому $e \in \Sigma \iff e^\comp \in \Sigma$.
            \item $\o \in \Sigma$ прямо из определения.
            Используя предыдущий пункт, $X \in \Sigma$.
            \item Пусть $e_1, e_2 \in \Sigma$. Проверим, что $e_1 \cap e_2 \in \Sigma$.
            Рассмотрим произвольное $a \subset X$. Запишем измеримость для $e_1$ при пересечении с $a$ и измеримость для $e_2$ при пересечении с $a \cap e_1$.
            \gather{
                \gamma(a) = \gamma(a \cap e_1) + \gamma(a \cap e_1^\comp) \\
                \gamma(a \cap e_1) = \gamma(a \cap e_1 \cap e_2) + \gamma(a \cap e_1 \cap e_2^\comp) \\
            }
            Отсюда подстановкой получаем
            \[\gamma(a) = \gamma(a \cap e_1 \cap e_2) + \underbrace{\gamma\left(a \cap e_1 \cap e_2^\comp\right) + \gamma\left(a \cap e_1^\comp\right)}_{\text{хотим показать, что это }\gamma(a \cap (e_1 \cap e_2)^\comp)}\]
            Записав измеримость $e_1$ при пересечении с $a \cap \left(e_1^\comp \cup e_2^\comp\right)$, получаем
            \multline{\gamma\left(a \cap (e_1^\comp \cup e_2^\comp)\right) = \gamma\left(a \cap (e_1^\comp \cup e_2^\comp) \cap e_1\right) + \gamma\left(a \cap (e_1^\comp \cup e_2^\comp) \cap e_1^\comp\right) =\\= \gamma\left(a \cap e_1 \cap e_2^\comp\right) + \gamma\left(a \cap e_1^\comp\right)}
            \item Так как $(e_1 \cup e_2) = (e_1^\comp \cap e_2^\comp)^\comp$ и $(e_1 \sm e_2) = e_1 \cap e_2^\comp$, то $\Sigma$ --- действительно алгебра.
        }
        \item Проверим <<усиленную аддитивность>>: для произвольного $a \subset X$, $b_1, b_2 \in \Sigma, b_1 \cap b_2 = \o \then$
        \[\gamma(a \cap (b_1 \sqcup b_2)) = \gamma(a \cap b_1) + \gamma(a \cap b_2)\]
        Данный факт потребуется для доказательства того, что $\Sigma$ --- $\sigma$-алгебра.

        Доказательство напрямую следует из измеримости $b_1$ при пересечении с $a \cap (b_1 \cup b_2)$.

        Отсюда по индукции видно, что для попарно непересекающихся $b_1, \dots, b_n \in \Sigma$:
        \[\gamma\left( a \cap \bigsqcup\limits_{j = 1}^{n}b_j\right) = \sum\limits_{j = 1}^{n}\gamma(a \cap b_j)\]

        \item Проверим, что $\Sigma$ является $\sigma$-алгеброй.
        Для этого достаточно проверить, что для счётного семейства $b_i \in \Sigma: b \coloneqq \bigsqcup\limits_{j \in \N}b_j \in \Sigma$.

        \indent{Чтобы доказать измеримость множества $e$, достаточно проверить неравенство $\gamma(a) \ge \gamma(a \cap e) + \gamma(a \sm e)$, потому что неравенство в другую сторону следует из счётной полуаддитивности.
        Дополнительно можно считать, что $\gamma(a)$ конечно.}

        Выберем произвольное $a \in X$, для него
        \[\gamma(a) = \gamma(a \cap (b_1 \sqcup \dots \sqcup b_n)) + \gamma(a \sm (b_1 \sqcup \dots \sqcup b_n)) \ge \left(\sum\limits_{j = 1}^{n}\gamma\left(a \cap b_j\right)\right) + \gamma(a \sm b)\]
        Переходя к пределу $n \to \infty$, получаем
        \[\gamma(a) \ge \left(\sum\limits_{j \in\N}\gamma(a \cap b_j)\right) + \gamma(a \sm b)\]
        Так как $a \cap b = \bigsqcup\limits_{j \in \N}(a \cap b_j)$, то из счётной полуаддитивности $\gamma(a \cap b) \le \sum\limits_{j \in \N}\gamma(a \cap b_j)$.
        Отсюда
        \[\gamma(a) \ge \left(\sum\limits_{j = 1}^{\infty}\gamma(a \cap b_j)\right) + \gamma(a \sm b) \ge \gamma(a \cap b) + \gamma(a \sm b)\]
        \item Проверим, что $\gamma\big|_\Sigma$ --- <<усиленно счётно-аддитивная мера>>, то есть для счётного семейства дизъюнктных $b_j \in \Sigma$ $\left(b \coloneqq \bigsqcup\limits_{j \in \N}b_j\right)$ и произвольного $\forall a \in X$:
        \[\gamma\left( a\cap\bigsqcup\limits_{j \in \N}b_j\right) = \sum\limits_{j \in \N}\gamma(a \cap b_j)\]
        При $a = X$ свойство обращается в обычную счётную аддитивность, но усиленная даётся даром, так что докажем и её тоже.

        С одной стороны, из счётной аддитивности $\gamma$: $\gamma(a \cap b) \le \sum\limits_{j \in \N}\gamma(a \cap b_j)$.
        С другой стороны, \[\gamma( a\cap b) \ge \gamma(a \cap (b_1 \cup \dots \cup b_n)) = \sum\limits_{j = 1}^{n}\gamma(a \cap b_j)\]
        и можно перейти к пределу по $n$.
        \item Докажем первое дополнение.

        Достаточно показать, что для любого $e \in \mathcal{P}, a \in X$: $\mu^*(a) \ge \mu^*(a \cap e) + \mu^*(a \sm e)$, обратное следует из полуаддитивности внешней меры.

        Рассмотрим произвольное счётное покрытие $\{c_i\}_{i \in \N}$ множества $a$ элементами множества $\mathcal{P}$.
        \bullets{
            \item Во-первых, по определению внешней меры $\mu^*(a \cap e) \le \sum\limits_{i \in \N}\mu(c_i\cap e)$
            \item Во-вторых, оценим $\mu^*(a \sm e)$.

            Каждое $b_i \coloneqq c_{i} \sm e$ представимо в виде конечного объединения $b_i = \bigcup\limits_{j = 1}^{s_i}d_i^{(j)}$, где $d_i^{(j)} \in \mathcal{P}$ попарно дизъюнктны.

            $\{d_i^{(j)}\}_{i,j}$ --- счётная совокупность множеств из $\mathcal{A}$, покрывающая $a \sm e$.
            \item Таким образом
            \[\mu^*(a \cap e) + \mu^*(a \sm e) \le \sum\limits_{i \in \N}\underbrace{\left(\mu(c_i \cap e) + \sum\limits_{j = 1}^{s_i}\mu\left(d_i^{(j)}\right)\right)}_{\mu(c_i)}\]
            Переходя к инфимуму по всем покрытиям, получаем
            \[\mu^*(a \cap e) + \mu^*(a \sm e) \le \inf\sum\limits_{i = 1}^{\infty}\mu(c_i) = \mu^*(a)\]
        }
        \item Наконец, для доказательства второго дополнения сошлёмся на четвёртый пункт свойств внешней меры~(\cref{outer-measure}).
    }
    }
    \definition[Стандартное продолжение меры $\mu$ на полукольце $\mathcal{P} \subset 2^X$]{
        Построенные данным образом $\Sigma$, и сужение $\mu^*\big|_{\Sigma}$ --- счётно-аддитивная мера на $\sigma$-алгебре.
    }
    \examples{
        \item Пусть $v_n$ --- объём на системе конечных $n$-мерных прямоугольных параллелепипедов (со сторонами, параллельными координатным осям).

        Стандартное продолжение данной меры --- \emph{мера Лебега} $\lambda_n$, полученное множество $\Sigma \subset 2^X$ --- множество \emph{измеримых по Лебегу} множеств.
        Все Борелевские множества, разумеется, измеримы по Лебегу~(\cref{borel-sets}), но обратное неверно --- измеримых множеств сильно больше~(\cref{lebesgue-sets-are-frequent}).
        \item
        Пусть $\lambda_f$ --- Стилтьесова длина, порождённая нестрого возрастающей функцией $f$.
        Она счётно аддитивна на полукольце промежутков.
        Её стандартное продолжение --- \emph{мера Лебега-Стилтьеса}.
        Здесь полученные измеримые множества --- элементы $\Sigma$ --- вообще говоря, могут зависеть от $f$ (при одной функции, порождающей меру, множество $x \subset X$ измеримо, но не при другой)
    }


    \section{Структура измеримых множеств}

    \subsection{Множества меры нуль}
    \fact{
        Пусть $\gamma$ --- предмера на $X$, рассмотрим такое подмножество $e \subset X$, что $\gamma(e) = 0$.
        Тогда $e$ является $\gamma$-измеримым.
        В частности, все подмножества $e$ имеют меру $0$ (в частности измеримы).
        \provehere{
            Проверим, что $\forall a \subset X: \gamma(a) \ge \gamma(a \cap e) + \gamma(a \sm e)$.

            Это так: по монотонности $\gamma(a \cap e) \le \gamma(e) = 0$ и $\gamma(a \sm e) \le \gamma(a)$.
        }
    }
    Пусть $\gamma = \mu^*$, где $\mu$ --- счётно-аддитивная мера на полукольце $\mathcal{P}$.
    \fact{
        Множество $e \subset X$ --- множество меры нуль $\iff \forall \eps > 0: \exists$ счётное семейство $b_i \in \mathcal{P}$, таких, что $\bigcup\limits_{i}b_i \supset e$, и $\sum\limits_{i = 1}^{\infty}\mu(b_i) < \eps$.
    }
    \examples[Множества меры нуль]{
        \item Точка.
        \item Конечное или счётное число точек (например, $\Q$).
        \item Канторово множество --- на $n$-м шаге его мера равна $\left(\frac{2}{3}\right)^n$.
    }
    \proposal{\label{lebesgue-sets-are-frequent}
    Так как канторово множество континуально, то все его подмножества (коих $2^{|\R|}$) имеют меру нуль и измеримы по Лебегу.
    Отсюда получаем, что всего измеримых множеств на прямой $2^{|\R|}$, так как это уже мощность всех подмножеств прямой.

    С другой стороны, Борелевских множеств всего континуум.
    \provehere[Схема доказательства]{
        Пусть $\mathcal{A}_0$ --- все интервалы с рациональными границами. Их счётное число.
        Но это пока даже не алгебра.

        Обозначим за $\mathcal{A}_1$ все их счётные объединения, их континуально.
        Но это пока не $\sigma$-алгебра.

        За $\mathcal{A}_2$ обозначим все счётные пересечения множеств из $\mathcal{A}_1$.
        За $\mathcal{A}_3$ обозначим все счётные объединения множеств из $\mathcal{A}_2$.

        И так далее. Заведём трансфинитную индукцию, на первом несчётном ординале всё перестанет меняться.
        Объединение не более чем континуального числа континуальных множеств континуально.
    }
    }
    \definition[Свойство точек множества $X$ выполняется почти всюду]{
        Множество точек, где оно не выполняется, имеет меру нуль.
    }
    Пусть $\mathcal{P}$ --- полукольцо подмножеств $X$, $\mu$ --- счётно аддитивная мера на $\mathcal{P}$.
    Стандартное продолжение часто тоже будем обозначать через $\mu$, иногда через $\overline{\mu}$.

    \subsection{$\sigma$-множества и $\delta\sigma$-множества}
    \begin{definition_env}[$\sigma$-множество относительно $\mathcal{P}$]
        Объединение счётного семейства элементов $\mathcal{P}$.
    \end{definition_env}
    Все $\sigma$-множества измеримы.
    \proposal{
        Если $e \subset X$ $\mu$-измеримо, то $\mu(e) = \inf\defset{\mu(b)}{e \subset b, b \text{ --- $\sigma$-множество}}$.
        \provehere{
            Так как по определению $\mu(e) = \mu^*(e) = \inf\defset{\sum\limits_{i = 1}^{\infty}\mu(c_i)}{c_i \in \mathcal{P}, \bigcup\limits_{i}c_i \supset e}$
            то можно выбрать в качестве $b \coloneqq \bigcup\limits_{i}c_i$, $b$ --- $\sigma$-множество.
        }
    }
    \note{
        Любое $\sigma$-множество $b$ представимо в виде дизъюнктного объединения счётного числа элементов $d_j \in \mathcal{P}$.

        Так как $d_j$ дизъюнктны, то $\mu(b) = \sum\limits_{j = 1}^{\infty}\mu(d_j)$, вот такая простая формула меры $\sigma$-множества.
    }
    \theorem{\label{two-is-enough}
    Если $c$ --- $\mu$-измеримое множество, и $\mu(c) <\infty$, то $\exists$ убывающая по включению последовательность $\sigma$-множеств $b_k$, таких, что $\bigcap\limits_{k = 1}^{\infty}b_k \supset c$ и $\mu\left(\bigcap\limits_{k = 1}^{\infty}b_k\right) = \mu(c)$.
    Иначе говоря, если $\tilde{c} \coloneqq \bigcap\limits_{k = 1}^{\infty}b_k$, то $\mu(\tilde{c}\sm c) = 0$.
    \provehere{
        Положим $\tilde{b}_k$ --- $\sigma$-множество, такое, что $\tilde{b}_k \supset c$, причём $\mu\left(\tilde{b}_k\right) < \mu(c) + \frac{1}{k}$.
        Назначим $b_k = \tilde{b}_1 \cap \dots \cap \tilde{b}_k$.
        \indentlemma{
            Пересечение двух (а значит, и конечного числа) $\sigma$-множеств --- $\sigma$-множество.
        }{
            Если $u = \bigcup\limits_{i = 1}^{\infty}e_i, v = \bigcup\limits_{j = 1}^{\infty}g_j$, где $e_i, g_j \in \mathcal{P}$, то $u \cap v = \bigcup\limits_{i,j=1}^{\infty}(e_i \cap g_j)$
        }
        Согласно лемме $b_k$ --- $\sigma$-множество. Так как $\mu(b_k) \le \mu(c) + \frac{1}{k}$, то $\mu\left(\bigcap\limits_{k = 1}^{\infty}b_k\right)=\lim\limits_{k \to \infty}\mu(b_k) = \mu(c)$.
    }
    }
    \begin{definition_env}[$\delta\sigma$-множество относительно $\mathcal{P}$]
        Пересечение счётного семейства $\sigma$-множеств.
    \end{definition_env}

    \subsection{$\sigma$-конечность}
    \begin{definition_env}[$\sigma$-конечная мера $\mu$]
        Такая мера, что $\exists E_1 \subset E_2 \subset \dots$, все $E_i \in \Sigma$, все $\mu(E_i) < +\infty$, причём $X = \bigcup\limits_{i = 1}^{\infty}E_i$.
    \end{definition_env}
    \examples{
        \item Считающая мера на несчётном множестве не является $\sigma$-конечной.
        \item Мера Лебега в $\R^n$ $\sigma$-конечна.
    }
    \newlection{30 сентября 2023 г.}
    По-прежнему, $\mathcal{A}$ --- полукольцо, $\mu$ --- мера на $\mathcal{A}$, обозначим её стандартное продолжение тоже за $\mu$.
    \theorem{
        Пусть стандартное продолжение меры $\mu$ на полукольце $\mathcal{P}$ $\sigma$-конечно. Пусть $d \subset X$ --- $\mu$-измеримо.
        Тогда $\exists \delta\sigma$-множество $D \supset d$, такое, что $\mu(D \sm d) = 0$.
        \provehere{
            \indentlemma{
                Пространство $X$ $\sigma$-конечно, если и только если $\exists e_1, e_2, \dots \subset X$: ${\mu(e_i) < \infty}$ и $\bigsqcup\limits_{i = 1}^{\infty}e_i = X$.
            }{
                Как обычно, если $\sigma$-конечно, то $E_1 \subset E_2 \subset \dots$ в объединении дают $X$, рассмотрим $e_i \coloneqq E_{i + 1} \sm E_i$.
                Наоборот, если даны $e_i$, то $E_i \coloneqq \bigsqcup\limits_{j = 1}^{i}e_j$.
            }
            Выберем $\eps > 0$.

            Пусть $e_i \in \Sigma$ --- измеримы, причём $\mu(e_i) < \infty$ и $\bigsqcup\limits_{i = 1}^{\infty}e_i = X$.
            Обозначим за $d_i \coloneqq d \cap e_i$. Тогда $\forall i: \mu(d_i) < \infty$.
            Согласно~(\cref{two-is-enough}): $\exists \sigma$-множество $D_i: \mu(D_i \sm d_i) < \frac{\eps}{2^i}$.
            Тогда подойдёт $D \coloneqq \bigcup\limits_{i = 1}^{\infty}D_i$: $\mu(D \sm d) < \eps$.

            Но отсюда пересечение $D$ по всем $\eps = \frac{1}{N}$ даёт подходящее $\delta\sigma$-множество.
        }
    }

    \subsection{Полнота}
    Пусть $\mathcal{C}\subset2^X$ --- $\sigma$-алгебра, на которой задана счётно-аддитивная мера $\nu$.

    Пусть $\mathcal{A}$ --- полукольцо, лежащее в $\mathcal{C}$, $\mu$ --- счётно-аддитивная мера на $\mathcal{A}, \overline{\mu}$ --- стандартное продолжение меры (на $\mu$-измеримые множества, пусть они составляют $\Sigma$).

    Пусть $\nu$ --- мера на $\mathcal{C}$, такая, что $\nu\big|_\mathcal{A} = \mu$, причём $\mu$ --- $\sigma$-конечна.
    \definition[Полная мера]{\label{full-measure}
    Такая счётно-аддитивная мера $\nu$ на $\sigma$-алгебре $\mathcal{C}$, что $\forall b \in \mathcal{C}: \nu(b) = 0 \then \forall a \subset b: a \in \mathcal{C}$.
    }
    \theorem{\label{uniqueness_of_measure}
    Меры $\nu$ и $\overline{\mu}$ совпадают на $\Sigma \cap \mathcal{C}$, а если $\nu$ полна, то $\Sigma \subset \mathcal{C}$.
    \provebullets{
        \item Пусть $A \in \Sigma$ есть $\sigma$-множество относительно полукольца $\mathcal{A}$.
        Тогда $A = a_1 \sqcup a_2\sqcup \dots$, где $a_j \in \mathcal{A}$

        Отсюда $A\in \mathcal{C}$, причём $\nu(A) = \sum\limits_{j = 1}^{\infty}\nu(a_j) = \sum\limits_{j = 1}^{\infty}\mu(a_j) = \overline{\mu}(A)$.
        \item Пусть $B \in \Sigma$ --- $\delta\sigma$-множество относительно полукольца $\mathcal{A}$, то есть $B = \bigcap\limits_{k = 1}^{\infty}A_k$, где $A_k$ --- $\sigma$-множества (дополнительно считаем, что $A_1 \supset A_2 \supset \dots$).

        Тогда $B \in \mathcal{C}$.
        Если $\overline{\mu}(B) < \infty$, то множества $A_j$ тоже можно выбрать конечной меры.

        Тогда $\nu(B) = \lim\limits_{k\to\infty}\nu(A_k) = \lim\limits_{k\to\infty}\overline{\mu}(A_k) = \overline{\mu}(B)$.
        \item Пускай $E \in \mathcal{C} \cap \Sigma$. Найдётся такое $\delta\sigma$-множество $E_1 \supset E$: $\overline{\mu}(E_1 \sm E) = 0$
        (если $E$ бесконечно, то это следует из $\sigma$-конечности $\mu$), причём так как $E_1$ --- $\delta\sigma$-множество относительно $\mathcal{A}$, то про него уже известно, что $\nu(E_1) = \overline{\mu}(E_1)$.
        Тогда $E_1 \sm E$ тоже содержится в $\mathcal{C} \cap \Sigma$.
        \indentlemma{
            Если $b \in \mathcal{C} \cap \Sigma$, причём $\overline{\mu}(b) = 0$, то $\nu(b) = 0$.
        }{
            Найдётся $b_1$ --- $\delta\sigma$-множество, такое, что $b_1 \supset b$ и $\overline{\mu}(b_1) = 0$.
            Тогда $\nu(b_1) = \overline{\mu}(b_1) = 0$, откуда $\nu(b) \le \nu(b_1) = 0$.
        }
        Лемма влечёт $\nu(E_1 \sm E) = 0$. Отсюда на всех множествах из $\mathcal{C}\cap\Sigma$ меры $\overline{\mu}$ и $\nu$ совпадают.
%    Отсюда при условиях $\overline{\mu}(E) < +\infty$ и $E \in \mathcal{C} \cap \Sigma$, то $\overline{\mu}(E) = \nu(E)$.
%
%    Если же $\overline{\mu}(E) = \infty$
        \item Проверим, что если $\nu$ полна, то $\Sigma \subset \mathcal{C}$.

        Если $\overline{\mu}(e) = 0$, то $e \in \mathcal{C}$, так как найдётся $\delta\sigma$-множество $e_1 \supset e$: $\overline{\mu}(e_1) = 0$.
        Из полноты меры $\nu$ автоматически $e \in \mathcal{C}$.

        Теперь рассмотрим $D \in \Sigma$. Найдётся $\delta\sigma$-множество $\overline{D} \supset D$, такое, что $\overline{\mu}(\overline{D}\sm D) = 0$, то есть $\overline{D}\sm D \in \mathcal{C}$.
        Таким образом, $D \in \mathcal{C}$, причём $\nu(\overline{D} \sm D) = 0$.
    }
    }

    \subsection{Двоичные (диадические) кубы}
    \definition[Двоичный отрезок ранга $n$]{
        Отрезок вида $I_n^{(k)} \coloneqq \left[\frac{k}{2^n}, \frac{k + 1}{2^n}\right)$ (здесь $n,k \in \Z$).
    }

    Заметим, что $\forall n \in \Z: \bigsqcup\limits_{k \in \Z}I_n^{(k)} = \R$, причём любые двоичные отрезки либо не пересекаются, либо вложены.

    \definition[Двоичные кубы ранга $n$]{
        Произведения $I_1 \times \dots \times I_d$, где $I_j$ --- двоичные отрезки ранга $n$.
    }
    Любые двоичные кубы тоже либо не пересекаются, либо вложены.
    \theorem{
        Пусть $G$ --- открытое множество в $\R^n$.

        Тогда $G = \bigsqcup\limits_{j = 1}^{\infty}Q_j$, где $Q_j$ --- попарно не пересекающиеся двоичные кубы, (быть может, какие-нибудь $Q_j = \o$) (иными словами, $G$ --- дизъюнктное объединение не более чем счётного числа каких-то двоичных кубов).
        \provehere{
            Рассмотрим точки $x\in G$.
            Для каждой точки выберем двоичный куб $Q \ni x$, полностью содержащийся в $G$.

            Объединение всех таких кубов даст $G$.
            Чтобы кубы не пересекались, мы оставим только кубы положительного ранга, а среди них --- максимальные по включению.
            (Если множество неограниченное, то максимального включения среди \textbf{всех} кубов может не найтись, надо ограничить их размер, поэтому мы взяли только кубы положительного ранга)
        }
    }
    Вспомним про $\mathcal{P}_0(\R^n)$ --- полукольцо ограниченных прямоугольных параллелепипедов, на котором есть мера --- $n$-мерный объём $v_n$.
    $\Sigma$ --- $\sigma$-алгебра измеримых по Лебегу множеств (относительно $v_n$), $\lambda_n$ --- стандартное продолжение $v_n$.

    Теперь обозначим $\mathcal{D}(\R^n)$ --- полукольцо всех двоичных кубов в $\R^n$.
    Положим $\rho_n = v_n\big|_{\mathcal{D}(\R^n)}$.
    По теореме Лебега --- Каратеодори получаем стандартное продолжение $\mu$ на множество $\Sigma_1 \subset 2^{\R^n}$.

    Тогда $\mathcal{D}(\R^n) \subset \mathcal{C} = \Sigma$, откуда $\Sigma_1 \subset \Sigma$, $\mu = \lambda_n\big|_{\Sigma_1}$.

    Также понятно, что все открытые множества являются счётными объединениями кубов из $\mathcal{D}(\R^n)$, откуда $\Sigma \subset \Sigma_1$, то есть на самом деле $\Sigma = \Sigma_1$.

    Наконец, так как обе меры совпадают на $\mathcal{D}(\R^n)$, то они равны~(\cref{uniqueness_of_measure}).
    \newlection{4 октября 2023 г.}
    \theorem{\label{measure_is_invariant}
    Пусть $\lambda_n$ --- мера Лебега в $\R^n$, $\Sigma$ --- $\sigma$-алгебра измеримых по Лебегу множеств.
    \numbers{
        \item Мера Лебега инвариантна относительно сдвига: если $e \in \Sigma, t \in \R^n: e + t \in \Sigma, \lambda_n(e + t) = \lambda_n(e)$.
        \item Если $\nu$ --- мера, заданная на этой $\sigma$-алгебре $\Sigma$, и $\nu$ инвариантна относительно сдвига ($\forall {e \in \Sigma}$, ${t \in \R^n}: \nu(e + t) = \nu(e)$), то тогда $\exists c \ge 0: \forall e \in \Sigma: \nu(e) = c\lambda_n(e)$.
        \provebullets{
            \item[1.] Достаточно доказать, что внешняя мера $\rho = v_n^*$ инвариантна относительно сдвига.

            $\rho(a) = \inf\sum\limits_{j}v_n(e_j)$ по всем $e_j$, таким, что их объединения покрывают $a$.
            Но \[{\bigcup\limits_{j}e_j \supset a} \iff \bigcup\limits_{j}(e_j - t) \supset (a - t)\]

            Измеримость по Лебегу тоже легко проверить:
            \gather{\rho(a) = \rho(a \cap e) + \rho(a \sm e) \quad\iff\quad\rho(a - t) = \rho((a - t) \cap (e - t)) + \rho((a - t) \sm (e - t))}

            \item[2.] Обозначим за $c \coloneqq \frac{\nu(Q_0)}{\lambda_n(Q_0)}$, где $Q_0$ --- какой-то фиксированный двоичный куб ранга 0.
            Тем самым, $\nu(Q) = c v_n(Q)$ для любого двоичного куба ранга 1 (инвариантность при сдвиге).

            Может так случиться, что $c = 0$.
            Тогда в силу счётной аддитивности и $\sigma$-конечности мера всего пространства равна 0.

            Заметим, что $2^n$ кубов ранга $k$ дают в объединении куб ранга $k - 1$: \[\left[0, \frac1{2^{k-1}}\right)^n = \left(\left[0, \frac{1}{2^{k}}\right) \sqcup \left[\frac1{2^k}, \frac1{2^{k-1}}\right)\right)^n\]
            Тем самым, мы по индукции получаем, что на всех двоичных кубах меры $\nu$ и $\lambda_n$ отличаются в $c$ раз.

            Дальше применяя теорему о единственности для меры $\rho = \frac{\nu}{c}$, получаем, что $\rho \equiv \lambda_n$ --- объём можно задать на двоичных кубах.

            Полнота $\nu$ получается автоматически из того, что $\nu$ задана на всей $\Sigma$.
            В самом деле, всякое множество меры нуль является подмножеством $\delta\sigma$-множества меры нуль.
        }
    }
    }


    \section{Поведение меры Лебега при линейных отображениях}
    Пусть $T: \R^n \map \R^n$ --- линейное отображение, $e \in \Sigma$.
    Чему равна $\lambda_n(Te)$?

    Пусть $(X, \mathcal{A}_X)$ и $(Y, \mathcal{A}_Y)$ --- пары из множеств и $\sigma$-алгебр их подмножеств.
    \definition[Измеримое отображение $F: X \map Y$ (относительно данных $\sigma$-алгебр)]{\label{measured_map}
    Такое отображение $F$, что $\forall a \in \mathcal{A}_Y: F^{-1}(a) \in \mathcal{A}_X$.
    }
    В частном случае $\mathcal{A}_X = \mathcal{B}(X)$ и $\mathcal{A}_Y = \mathcal{B}(Y)$ измеримое отображение называется \emph{измеримым по Борелю}.
    \lemma{
        Всякое непрерывное отображение $F: X \map Y$ измеримо по Борелю.
        \provehere{
            Положим $\mathcal{C} \coloneqq \defset{e \in Y}{F^{-1}(e) \in \mathcal{B}(X)}$.
            $\mathcal{C}$ --- $\sigma$-алгебра, так как взятие прообраза коммутирует со всеми теоретико-множественными операциями (даже несчётными).

            Так как прообраз открытого открыт, то $\mathcal{C}$ содержит все открытые множества.
            Это сразу влечёт, что $\mathcal{C} \supset \mathcal{B}(Y)$, а тогда и подавно $\forall e \in \mathcal{B}(Y): F^{-1}(e) \in \mathcal{B}(X)$.\qedhere
        }
    }
    Для счётно-аддитивной меры $\nu$, заданной на $\mathcal{A}_X$ можно ввести её образ.
    \definition[Образ меры $\mu$ при (измеримом) отображении $F$]{
        Мера $\rho$, заданная на $\mathcal{A}_Y$ следующим образом: $\rho(e) =\mu(F^{-1}(e))$.
    }
    Пусть $F: \R^n \map \R^n$ непрерывно.
    Рассмотрим образ меры $\mu(e) \coloneqq \lambda_n(F^{-1}(e))$.
    Если $e \in \mathcal{B}(\R^n)$, то $F^{-1}(e) \in \mathcal{B}(\R^n)$, и формула имеет смысл: $\mu(e)$ определена.

    Иначе же, (если $e$ --- измеримое по Лебегу, но не борелевское (например, $e$ --- какое-то неприятное множество меры нуль)) может произойти что угодно.
    Его прообраз может быть вообще неизмеримым по Лебегу.

    Образ же даже Борелевского множества необязательно измерим по Лебегу.
    Так, $\eta(e) \coloneqq \lambda_n(\Phi(e))$ для непрерывного (даже инъективного) $\Phi$ может быть не определена на каком-то борелевском множестве.
    Чтобы таких проблем не было, надо требовать непрерывность обратного отображения.
    \fact{
        Пусть $G_1, G_2 \subset \R^n$ --- ограниченные открытые множества, $\Phi: G_1 \map G_2$ --- гомеоморфизм.
        Введём меру $\nu$ на $G_1$: $\nu(e) = \lambda_n(\Phi(e))$.
        Тогда $\nu$ корректно определена на $\mathcal{B}(G_1)$.}
    Пусть $a \subset G_1$ --- измеримое по Лебегу множество, $\lambda_n(a) = 0$.
    Тогда хочется, чтобы выполнялось $\nu(a) = 0$.
    В таком случае $\nu(e) = \lambda_n(\Phi(e))$ будет определена на всех измеримых по Лебегу множествах (всякое измеримое по Лебегу множество --- разность $\delta\sigma$-множества, и множества меры нуль).


    Пусть $G_1, G_2$ --- открытые множества в $\R^n$, $\Phi: G_1 \map G_2$ --- гомеоморфизм.
    В терминах измеримости сказанное выше можно переформулировать в виде:
    тогда $\Phi^{-1}$ измеримо по Борелю, и если $\Phi$ липшицево, то $\Phi^{-1}$ измеримо по Лебегу.
    \theorem{
        Пусть $\Phi: G_1 \map G_2$ --- $C$-липшицево отображение, пусть $A \subset G_1$ --- меры нуль.
        Тогда $\Phi(A)$ тоже имеет меру нуль.
        \provehere{
            \indentlemma{
                Пусть $e \subset \R^n$. Тогда $e$ есть множество меры нуль $\iff \forall \eps > 0: \exists \{a_i\}_{i \in \N}: e \subset \bigcup\limits_{i}a_i$, причём
                \[\sum\limits_{i = 1}^{\infty}(\diam a_i)^n < \eps\]
            }{
                \bullets{
                    \item[$\then$] $\lambda_n(e) = 0 \iff \lambda_n^*(e) = 0 \iff \forall \eps > 0: \exists \{Q_i\}_{i \in \N}$ --- такое семейство двоичных кубов, что $\sum\limits_{i = 1}^{\infty}v_n(Q_i) < \eps$.
                    Учитывая, что $v_n(Q_i) = \left(\frac{\diam Q_i}{\sqrt{n}}\right)^n$ сразу получаем $\sum\limits_{i = 1}^{\infty}(\diam Q_i)^n < n^{\nicefrac{n}2}\eps$.
                    \item[$\when$] Всякое множество $a_i$ содержится в кубе (необязательно двоичном) $Q_i$ со стороной $\diam(a_i)$ (проекция на любую координатную ось не больше $\diam(a_i)$).
                }

            }
            Пусть открытое $e \subset G_1$ имеет меру нуль, предположим, что $\dist(e, G_1^\comp) > 0$.
            Тем самым, $\forall \eps > 0: \exists a_i \subset \R^n: \bigcup\limits_{i \in \N}a_i \supset e, \sum\limits_{i = 1}^{\infty}(\diam(a_i))^n < \eps$.
            Можно считать, что все $a_i$ пересекают $e$, тогда при маленьких $\eps: a_i \subset G_1$.

            Тем самым, $\diam(\Phi(a_i)) \le C \cdot \diam(a_i)$, и $\sum\limits_{i = 1}^{\infty}\diam(\Phi(a_i))^n \le C^n \cdot  \eps$

            Если же $\dist(e, G_1^\comp) = 0$, то воспользуемся теоремой об исчерпывающей последовательности компактов~(\cref{ob-ischerpivauschey}). Найдутся компактные $K_i \subset G_1$, в объединении дающие $G_1$.
            Для множества меры нуль $a \subset G_1$ заметим, что оно является объединением счётного числа множеств $a_i = a \cap K_i$, отделённых от границы.
        }
    }
    \theorem[Об исчерпывающей последовательности компактов]{\label{ob-ischerpivauschey}
    Пусть $G \subset \R^n$ --- открыто, тогда существует $\exists \{K_i\}_{i \in \N}$: $K_i \subset \Int(K_{i + 1})$, причём $\bigcup\limits_{i}K_i = G$.
    \provehere{
        Если $G = \R^n$, то выберем $K_i = \overline{B_i}(0)$.

        Иначе положим $\tilde{K}_i = \defset{x \in G}{\dist(x, G^\comp) \ge \frac{1}{i}}$. Несложно видеть, что $\bigcup\limits_{i}\tilde{K}_i = G$ --- это следует из замкнутости $G^\comp$.
        Из непрерывности функции расстояния (она даже липшицева) $\tilde{K}_i$ тоже замкнуто.

        Наконец, $\tilde{K}_i \subset \Int\tilde{K}_{i + 1}$. Если $G$ неограничено, то $\tilde{K}_i$ может быть некомпактно хотя и замкнуто.
        Чтобы избежать этой проблемы, положим $K_i = \tilde{K}_i \cap \overline{B_i}(0)$.
    }
    }
    \note{
        В $\R^n$ любая координатная гиперплоскость имеет лебегову меру нуль: например, она представима в виде объединения счётного числа гиперквадратиков меры нуль.
    }
    Итак, с чего мы начали. Пусть $T: \R^n \map \R^n$ --- линейное отображение.
    \theorem{
        $\forall e \in \Sigma: \lambda_n(T e) = |\det T| \cdot \lambda_n(e)$, где определитель взят в каком-то ортонормированном базисе.
        \provebullets{
            \item Пусть $T$ --- невырожденное отображение, $\det T \ne 0$.
            Тогда это гомеоморфизм $\R^n$ на себя.
            В любом случае, $T$ липшицево, например, с константой $\|T\|$.

            Таким образом, если положить $\nu = \lambda_n\circ T$, то окажется, что $\nu$ --- корректно определённая счётно-аддитивная мера на $\sigma$-алгебре измеримых по Лебегу множеств.
            Заметим, что $\nu$ инвариантна относительно сдвига: $\forall t \in \R^n$. $\lambda_n(Te + Tt) = \lambda_n(Te)$.
            Таким образом~(\cref{measure_is_invariant}): $\exists c: \nu = c \lambda_n$. Осталось проверить, что $c = |\det T|$.

            \bullets{
                \item Если $T$ --- ортогональное преобразование, то оно сохраняет расстояния, и $\det T = \pm 1$.
                Выберем $B$ --- замкнутый шар положительного радиуса с центром в 0. Тогда $TB = B$, но мера шара не равна 0 (в него можно засунуть кубик положительного диаметра), откуда $c = 1$.
                \item \textbf{Следствия}. Если $E$ --- собственное линейное подпространство $\R^n$, то его мера $\lambda_n$ равна 0.
                Ортогональным преобразованием его можно перевести в координатную гиперплоскость.

                Другим следствие предыдущего пункта является то, что меру Лебега можно начинать строить с любого ортонормированного базиса в $\R^n$.
                Мера сохраняется при всяких поворотах и симметриях.
                \item Воспользуемся полярным разложением оператора. Это значит, что для невырожденного линейного $T: \R^n\map\R^n$: $\exists U, A: T = UA$, где $U$ --- ортогональный оператор, а $A$ --- эрмитов (диагональный в каком-то базисе).
                Тогда посчитаем для измеримого $a \in \R^n$ $\lambda_n(Ta) = \lambda_n(UAa) = \lambda_n(Aa)$
                Будем считать, что мера Лебега построена в том базисе, в котором $A$ диагонален.
                \[A = \vect{\alpha_1 & \cdots & 0 \\ \vdots&\ddots&\vdots\\0&\cdots&\alpha_n}\]
                Всякий куб $Q \subset \R^n$ после применения $A$ переходит в параллелепипед со сторонами $|\alpha_1|, \dots, |\alpha_n|$.
                Действительно, $\lambda_n(A Q) = |\alpha_1 \proddots \alpha_n| \cdot \lambda_n(Q) = |\det A| \cdot \lambda_n(Q)$.
            }
            \item Если $T$ вырождено, то $\Image(T)$ --- линейное подпространство в $\R^n$, так как $Te \subset T(\R^n)$, то мера $Te$ тоже нуль.
        }
    }


    \chapter{Интеграл Лебега}
    Пускай имеется тройка $(X, \Sigma, \mu)$, где $X$ --- множество, $\Sigma \subset 2^X$ --- $\sigma$-алгебра, $\mu$ --- счётно-аддитивная мера на $\Sigma$.

    Определим для некоторых функций $f: X \map \R$ интеграл $\int\limits_{X}f \d \mu$.

    Раньше мы уже определяли интеграл от простой функции  $f = \sum\limits_{i}c_i\chi_{e_i}$, равный $I(f) = \sum\limits_{i}c_i\mu(e_i)$.
    В качестве $e_i$ теперь можно брать произвольные измеримые множества, что уже сильно увеличивает разнообразие простых функций.
    \definition[Простая функция $g: X \map \R$ относительно $\sigma$-алгебры $\Sigma$]{
        Функция вида $g(x) = \sum\limits_{i = 1}^{n}c_i \chi_{e_i}$, $c_i \in \R, e_i \in \Sigma$.
        Можно считать, что $e_i \cap e_j = \o$ для $i \ne j$.
    }
    \newlection{18 октября 2023 г.}
    \theorem[Малая теорема Леви]{\label{levi-small}
    Пусть $g_1, g_2, \dots,$ --- счётное семейство неотрицательных простых функций;\ пусть $g$ --- ещё одна простая функция.
    Предположим, что $\forall x \in X: g_1(x) \le g_2(x) \le \dots$, причём $g_j(x) \underset{j \to \infty}\Map g(x)$ (можно записать $g_n(x) \nearrow g(x)$).
    Тогда $\lim\limits_{j \to \infty}I(g_j) = I(g)$.
    \provehere{
        Сложность заключается в том, что число ступенек у $g_j$ может неограниченно расти.

        Заметим, что так как $g_j$ неотрицательны, то $I(g_j)$ всегда определён (число из $\R$ или $+\infty$).

        Если $\exists j: I(g_j) = +\infty$, то $I(g) = +\infty$, и доказывать нечего.
        Далее считаем, что $\forall j: I(g_j) \in \R$.

        Так как $g$ простая, то $g = \sum\limits_{s = 1}^{n}c_s\chi_{e_s}$, где $e_s$ --- попарно дизъюнктные множества из $\Sigma$.
        Положим $g_j^s \coloneqq g_j \cdot \chi_{e_s}$.
        Эти функции тоже простые.

        Зафиксируем $s~(1 \le s \le n)$, зафиксируем $x \in e_s$, посмотрим на $\lim\limits_{j \to \infty}g_j^s(x) = c_s \chi_{e_s}$.
        Проверим предельное соотношение для интегралов: так как $s$ пробегает конечное множество значений, то достаточно доказать только для одного значения.
        Далее считаем, что $g = c \chi_e$.

        Тем самым, утверждение свелось к следующему: для $e \in \Sigma$, для последовательности простых функций $g_j$, таких, что поточечно $0 \le g_1 \le \dots \le g_j \le g_{j + 1} \le \dots \le c\chi_e = g$, причём $\forall x \in X: \lim\limits_{j \to \infty}g_j(x) = c\chi_e(x)$,
        необходимо и достаточно показать, что $I(g_j) \underset{j \to \infty}\Map I(c\chi_e) = c\mu(e)$.

        Рассмотрим $d \in (0, c)$. Положим $E_n \coloneqq \defset{x}{g_n(x) > d}$. Понятно, что $E_n \subset e$, причём $\bigcup\limits_{n = 1}^{\infty}E_n = e$.

        Обозначим $h_n = d \cdot \chi_{E_n}$. По определению $E_n$: $h_n \le g_n$. Таким образом, \[\underbrace{I(h_n)}_{d \cdot\mu(E_n)\underset{n \to \infty}\Map d\cdot \mu(e)} \le I(g_n) \le \underbrace{I(g)}_{c \cdot \mu(e)}\]
        Так как $I(g_j) \le I(g_{j + 1})$, то существует предел $V = \lim\limits_{j \to \infty}I(g_j)$.
        Отсюда $d \cdot \mu(e) \le V \le c\cdot\mu(e)$, причём это верно для любого $d < c$.
    }
    }


    \section{Измеримые отображения}
    Пускай $(X, \mathcal{A}_X)$, $(Y, \mathcal{A}_Y)$ --- множества и $\sigma$-алгебры соответствующих подмножеств.
    $F: X \map Y$.

    Вспомним определение измеримости~(\cref{measured_map}):
    \definition[Измеримое отображение $F: X \map Y$ (относительно данных $\sigma$-алгебр)]{
        Такое отображение $F$, что $\forall c \in \mathcal{A}_Y: F^{-1}(c) \in \mathcal{A}_X$.
    }
    Если в качестве $X, Y$ рассмотреть топологические пространства без определённых $\sigma$-алгебр, то в качестве этих $\sigma$-алгебр в $X, Y$ можно выбрать $\sigma$-алгебры борелевских множеств $\mathcal{B}(X), \mathcal{B}(Y)$.
    В таком случае $F$ называется \emph{измеримой по Борелю}.
    \theorem{
        Пусть в $Y$ содержится счётная база топологии $\mathcal{A}_Y$;\ пускай $\mathcal{D}$ --- какая-нибудь (даже необязательно счётная) база для топологии в $Y$.

        Если $\forall e \in \mathcal{D}: F^{-1}(e) \in \mathcal{A}_X$, то $F$ измеримо по Борелю.
        \provehere{
            Рассмотрим открытое $G \subset Y$, докажем, что $F^{-1}(G) \subset \mathcal{A}_X$.

            Представим $G = \bigcup\limits_{x \in G}a_x$, где $a_x \in \mathcal{D}$ содержит $x$.

            Пускай $\mathcal{A}_Y$ --- счётная база топологии в $Y$.
            Для любого $x \in G$: $\exists c_x \in \mathcal{A}_Y: x \in c_x \subset a_x$, где $a_x \in \mathcal{D}$.
            $\bigcup\limits_{x \in G}c_x = G$. Так как среди $c_x$ всего счётное число различных, то можно выбрать представителей --- счётное множество $X \subset G: \bigcup\limits_{x \in X}c_x = G$.
            Тогда и подавно $\bigcup\limits_{x \in X}a_x = G$.

            Отсюда $F^{-1}(G) \in \mathcal{A}_X$, так как $\sigma$-алгебра выдерживает счётные операции.

            Этого достаточно, так как $\defset{E \subset Y}{F^{-1}(E) \in \mathcal{A}_X}$ --- $\sigma$-алгебра, и если в неё содержатся все открытые множества, то и все борелевские содержатся в ней тоже.
        }
    }
    Пусть $(X, \Sigma_1), (Y, \Sigma_2), (Z, \Sigma_3)$ --- множества со своими $\sigma$-алгебрами.

    Рассмотрим композицию $X \overset{F}\Map Y \overset{\Phi}\Map Z$.
    \theorem{
        Композиция измеримых отображений измерима.\provehere{$\forall e_3 \in \Sigma_3: (\Phi \circ F)^{-1}(e) = F^{-1}(\Phi^{-1}(e))$.
        }
    }
    \fact{
        Пусть $X_1, X_2$ --- топологические пространства, $F: X_1 \map X_2$ --- непрерывно.
        Пусть $X_1, X_2$ наделены своими борелевскими $\sigma$-алгебрами. Тогда $F$ измеримо.
        \provehere{
            Определим $\mathcal{A} \coloneqq \defset{e \in X_2}{F^{-1}(e) \in \mathcal{B}(X_1)}$. $\mathcal{A}$ --- $\sigma$-алгебра, причём она содержит все открытые множества.
        }
    }
    \corollary{
        Рассмотрим композицию $X \overset{F}\Map Y \overset{\Phi}\Map Z$.
        $X$ --- пространство с $\sigma$-алгеброй $\mathcal{A}$, $Y, Z$ --- топологические пространства с борелевскими $\sigma$-алгебрами, $F$ измеримо, $\Phi$ непрерывно.
        Тогда $\Phi \circ F$ непрерывно.
    }
    \ok
    $(X, \mathcal{A})$ --- пространство с $\sigma$-алгеброй.
    Рассмотрим $f: X \map \R$.
    \proposal{
        $f$ измеримо, если выполнено любое из следующих условий.
        \numbers{
            \item $\forall a, b \in \R: f^{-1}((a, b)) \in \mathcal{A}$.
            \item $\forall a, b \in \R: f^{-1}([a, b]) \in \mathcal{A}$.
            \item $\forall a \in \R: f^{-1}((-\infty, a)) \in \mathcal{A}$.
            \item $\forall a \in \R: f^{-1}((a, +\infty)) \in \mathcal{A}$.
        }
        \provehere{
            Согласно предыдущей теореме $(1)$ сразу влечёт измеримость.

            Проверим $(3) \then (1)$. Так как $(-\infty, d] = \bigcap\limits_{n = 1}^{\infty}(-\infty, d + \nicefrac1n)$, то \[f^{-1}((a, b)) = f^{-1}\left((-\infty, b) \sm (-\infty, a]\right) = f^{-1}\left((-\infty, b) \sm \bigcap\limits_{n = 1}^{\infty}(-\infty, a + \nicefrac{1}{n})\right)\]
            Всё остальное делается аналогично.
        }
    }
    \definition[Лебеговы множества функции $f$]{
        Для $a \in \R$ это множества вида $\defset{x}{f(x) < a}$, $\defset{x}{f(x) \le a}$, $\defset{x}{f(x)>a}$, $\defset{x}{f(x)\ge a}$.
    }
    Таким образом, для проверки того, что $f$ измерима, достаточно проверять измеримость только её Лебеговых множеств (достаточно какого-то одного типа).
    \ok
    Теперь рассмотрим отображение $F: X\map \R^n$, где $F(x) = \vect{f_1(x) \\ \vdots \\ f_n(x)}$ --- столбец координатных функций.
    \proposal{
        $F$ измеримо $\iff$ все $f_j$ измеримы.
        \provewthen{
            Пускай $I_1, \dots, I_n$ --- интервалы. Параллелепипеды $P = I_1 \times \dots \times I_n$ образуют базу топологии в $\R^n$.
            Достаточно доказать на базе, что $F^{-1}(P) \in \mathcal{A}$.
            $x \in F^{-1}(P) \iff F(x) \in P \iff \forall j: 1 \le j \le n\then f_j(x) \in I_j$. $F^{-1}(P) = \bigcap\limits_{j = 1}^{n}f_j^{-1}(I_j)$.
        }{
            Рассмотрим координатную проекцию $\pi_j: \R^n \map \R$. $\pi_j \circ F$ измеримо.
        }
    }
    \proposal{
        Пусть $f_1, f_2: X \map \R$ --- измеримы.
        Тогда измеримыми являются функции
        \bullets{
            \item $\alpha f_1 + \beta f_2$, где $\alpha, \beta \in \R$.
            \item $f_1 \cdot f_2$.
            \item $\frac{f_1}{f_2}$, если $\forall x \in X: f_2(x) \ne 0$.
        }
        \provehere{
            Пускай $F: X \map \R^2; \qquad F = \vect{f_1 \\ f_2}$.
            Согласно предыдущей теореме, оно измеримо.
            Скомпонуем $\psi \circ F$, где $\psi: \arr{ccc}{\R^2 &\map& \R\\(x, y) &\mapsto& \alpha x + \beta y \text{ или }xy}$
            Для частного: $\psi: \arr{ccc}{\R \times \R \sm \{0\}&\map& \R\\(x,y)&\mapsto&\frac{x}{y}}$
        }
    }
    Ниже нам будет удобно определять функцию $f$, принимающую бесконечные значения.
    \[f: X \map \overline{\R} \bydef \R \cup \{-\infty, +\infty\}\]
    Про такую функцию говорят, что она \emph{измерима}, если $f^{-1}(+\infty), f^{-1}(-\infty) \in \mathcal{A}$ и $f\big|_{f^{-1}(\R)}$ измерима в обычном понимании.

    К таким функциям можно применять примерно всё то, что уже доказано, только не надо складывать бесконечности разных знаков.
    \fact{
        Если $f$ (возможно) принимает значение $+\infty$, и все множества $\defset{x}{f(x) < a}$ лежат в $\mathcal{A}$, то $f$ измерима.
        \provehere{
            $f^{-1}(\R) = \bigcup\limits_{n = 1}^{\infty}f^{-1}(-\infty, n)$; $\defset{x}{f(x)=+\infty}=f^{-1}\left(\overline{\R}\right) \sm f^{-1}(\R)$.
        }
    }


    \section{Грани и предельные переходы}
    \theorem{
        Пусть $f_n: X \map \R$ --- измеримые функции. Пусть $f(x) = \inf\limits_{n}f_n(x)$.
        Для простоты считаем, что $\forall x: f_n(x)$ ограничены снизу.

        Тогда $f$ измерима.
        \provehere{
            Рассмотрим $\defset{x}{f(x) < a} = \bigcup\limits_{n}\defset{x}{f_n(x) < a}$.
        }
    }
    \corollary{
        Если функция $g(x) = \sup\limits_{n}f_n(x)$ всюду конечна, то функция $g$ измерима.
    }
    \corollary{
        Пусть $f_n$ измеримы, и $\forall x$: числовая последовательность $\{f_n(x)\}$ ограничена.
        Тогда $\varlimsup\limits_{n}f_n(x) = \varliminf\limits_{n}f_n(x)$ тоже измеримы.
        \provehere{
            Например, $\varlimsup\limits_{n}f_n(x) = \inf\limits_{k}\sup\limits_{n \ge k}f_n(x)$.
        }
    }
    \theorem{
        Пусть $f_n$ всюду конечны и измеримы, пусть $f_n(x) \underset{n \to \infty}\Map f(x)$, где $f(x)$ --- тоже конечна.
        Тогда $f$ измерима.
        \provehere{
            Это следствие из предыдущего.
        }
    }
    \note{
        Пусть $f_n(x) \underset{n \to \infty}\Map f(x)$, но допустимо, чтобы $f(x)$ принимало значения $\pm \infty$. (При этом $\forall n: f_n$ конечна)

        Тогда всё равно $f$ измерима.
        \provehere{
            Пусть $f(x_0) = +\infty$. Тогда $\forall N \in \N: \exists k \in \N: \forall n \ge k: f_n(x_0) \ge N$.

            Тем самым, $\defset{x_0}{f(x_0) = +\infty} = \bigcap\limits_{N \in \N}\bigcup\limits_{k\in\N}\bigcap\limits_{n \ge k}\defset{x_0}{f_n(x_0) \ge N}$.
        }
    }
    \definition[Ступенчатая функция]{
        $f: X \map \R$, такая, что $\exists E_1, E_2, \dots \in \mathcal{A}: E_i \cap E_j = \o$ при $i \ne j$ и $f\big|_{E_i}$ постоянна (скажем, равна $c_i$).
    }
    Иными словами, ступенчатая функция --- функция вида $\sum\limits_{i = 1}^{\infty}c_i \chi_{E_i}$.
    \note{
        Всякая ступенчатая функция измерима.
    }
    \theorem{
        Если $g: X \map \R$ --- измеримая функция, то $\exists$ последовательность ступенчатых функций $f_n$, такая, что $f_n \rightrightarrows g$.

        Если же $g \ge 0$, то $\exist$ простые функции $f_n: f_n(x) \nearrow g(x)$ поточечно.
        \provehere{\down
        \bullets{
            \item[1] Выберем $n \in \N$, рассмотрим двоичные интервалы $I_{j,n} = \left[\frac{j}{2^n}, \frac{j + 1}{2^n}\right)$.
            При фиксированном $n: \bigcup\limits_{j \in \Z}I_{j,n} = \R$.

            Пусть $E_{j,n} \coloneqq g^{-1}(I_{j,n}) \in \mathcal{A}$.
            Определим $f_n(x) = \frac{j}{2^n}$ при $x \in E_{j,n}$.
            Иными словами, бьётся ось ординат, и если функция $g$ принимает значение в неком двоичном отрезке, то $f_n(x)$ равно нижней границе этого отрезка.

            Тогда $\forall x: 0 \le g(x) - f_n(x) \le \frac{1}{2^n}$.

            Заметим, что $\forall x: f_n(x) \le f_{n + 1}(x)$.
            \item[2] Аналогично предыдущему пункту, берём полуинтервалы $\left[\frac{j}{2^n}, \frac{j + 1}{2^n}\right)$, и строим $f_n$ точно так же.
            Они сходятся к $g(x)$, но, увы, не простые.
            Тогда положим $\tilde{f}_n(x) = \min(f_n(x), n)$.
            Здесь $\tilde{f}_n(x)$ уже простые, по-прежнему возрастают монотонно, и всё ещё сходятся к $g$.
        }
        }
    }
    \note{
        Пусть $g$ принимает ещё и значения $\pm \infty$.
        Тогда можно построить последовательность ступенчатых $f_n$, как в теореме, определённых на $g\big|_{g(x)\text{ конечно}}$.

        Доопределим $\hat{f}_n(x) = \all{f_n(x),&g(x) \in \R \\ +\infty,& g(x) = +\infty \\ -\infty,&g(x) = -\infty}$.

        Тогда это всё ещё ступенчатые функции, и естественно считать, что они сходятся к $g$ равномерно.
        На том множестве, где $g(x)$ конечно, $|f_n(x) - g(x)|$ равномерно сходится к нулю, а если $g(x)$ бесконечно, то разность, конечно, не определена, но $f_n(x) = g(x)$.

        Похожую вещь можно применить и ко второму пункту теоремы.
    }


    \section{Интеграл}
    Сначала научимся интегрировать неотрицательные измеримые функции.

    Пусть $(X, \Sigma, \mu)$ --- пространство с мерой, то есть $\mu$ --- счётно-аддитивная мера, заданная на $\Sigma$.

    Предположим, что $\mu$ --- полная мера~(\cref{full-measure}).
    Если это не так, то можно продолжить $\mu$ по Лебегу --- Каратеодори.
    Тогда в целом ничего особо не поменяется. Так, в предположении $\sigma$-конечности для продолжения меры $\tilde{\mu}$ на $\tilde{\Sigma}$: $\forall a \in \tilde{\Sigma}: \mu(a) < +\infty \then \exists b \in \Sigma: b \supset a, \tilde{\mu}(b \sm a) = 0$.

    Пусть $f$ --- неотрицательная измеримая функция на $X$ (возможно, принимающая значения $+\infty$).

    Определим \emph{интеграл} $J(f) = \sup\defset{I(g)}{g\text{ --- простая, }0 \le g \le f}$.
    \note{
        Хотя $f$ разрешается принимать бесконечные значения, по определению простые функции --- суммы $\sum\limits_{j = 1}^{N}c_j \chi_{e_j}$, где $c_j \in \R$ (множества $e_j$ можно считать дизъюнктными).
    }
    \definition[Суммируемая (интегрируемая) функция $f$]{
        $J(f) < +\infty$.
    }
    \properties[Совсем немного простых свойств]{
        \item Если $f$ неотрицательная простая функция, то $J(f) = I(f)$.
        \item Если $f_1 \le f_2$ --- неотрицательные измеримые, то $J(f_1) \le J(f_2)$.
    }
    \newlection{25 октября 2023 г.}
    Пусть $a, b$ --- два числа.
    Для их минимума и максимума иногда используются обозначения
    \[a \fmax b \bydef \max(a, b) \qquad a \fmin b \bydef \min(a, b)\]
    В частности, это используется для поточечного максимума или минимума функций: \[(f \fmax g)(x) \bydef f(x) \fmax g(x) \bydef \max(f(x), g(x))\]
    \theorem[Леви, для неотрицательных функций (теорема о монотонной сходимости)]{
        Пусть $f_n$ --- измеримые функции, $0 \le f_1 \le f_2 \le \dots$
        Пускай $f(x) = \lim\limits_{n \to \infty}f_n(x)$.
        Тогда $J(f) = \lim\limits_{n \to \infty}J(f_n)$.
        \provehere{
            Если $\exists n \in \N: J(f_n) = +\infty$, то доказывать нечего: тогда начиная с этого места $J(f_{\ge n}) = J(f) = +\infty$.
            Отметим, что $f$ измерима, как предел измеримых.

            Теперь будем считать, что $\forall n: J(f_n) < +\infty$.
            Понятно, что $J(f) \ge \lim\limits_{n \to \infty}J(f_n)$, так как
            \[\defset{g}{0 \le g \le f, g\text{ --- простая}} \supset \bigcup\limits_{n \in \N}\defset{g_n}{0 \le g_n \le f_n, g_n\text{ --- простая}}\]
            Далее мы доказываем, что $J(f) \le \lim\limits_{n \to \infty}J(f_n)$.

            $\forall n: \exists$ простая функция $\psi_n: 0 \le \psi_n \le f_n, I(\psi_n) \ge J(f_n) - \frac{1}{2^{2n}}$.
            Сделаем так, чтобы $\{\psi_n\}$ возрастали: $\phi_n \coloneqq \psi_1 \fmax \dots \fmax \psi_n$.
            Отметим, что $\phi_n$ --- тоже простые функции.
            \indentlemma{
                Почти всюду (для всех $x$, кроме множества меры нуль) ${\lim\limits_{n \to \infty}\phi_n(x) = f(x)}$.
            }{
                Обозначим $e_n = \defset{x}{\phi_n(x) < f_n(x) - \frac{1}{2^n}}$.
                Заметим, что тогда всё ещё $\phi_n + \frac{1}{2^n}\chi_{e_n} \le f_n$.
                Слева стоит простая функция, откуда $\underbrace{I\left(\phi_n + \frac{1}{2^n}\chi_{e_n}\right)}_{I(\phi_n) + \frac{1}{2^n}\mu(e_n)} \le J(f_n)$.
                Так как $I(\phi_n) \ge J(f_n) - \frac{1}{2^{2n}}$, то $\mu(e_n) \le \frac{1}{2^n}$.

                Обозначим $E_n = \bigcup\limits_{k \ge n}e_k$.
                Его мера тоже не очень большая: $\mu(E_n) \le \sum\limits_{k \ge n}\mu(e_k) \le \sum\limits_{k \ge n}\frac{1}{2^k} = \frac{1}{2^{n - 1}}$.
                Так как имеется вложенность $E_1 \supset E_2 \supset \dots$, то $E \coloneqq \bigcap\limits_{n \ge 0}E_n$ имеет меру нуль.

                Осталось заметить, что $\phi_n(x) \underset{n \to \infty}\Map f(x)$ везде кроме $E$.
            }
            Так как по определению $J(f) \bydef \sup\defset{I(g)}{0 \le g \le f, g \text{ --- простая}}$, то достаточно доказать, что для всякой простой функции $g \le f$: $I(g) \le \lim\limits_{n \to \infty}I(\phi_n)$.

            Пусть $E \subset X$ --- множество меры нуль, на котором $\phi_n(x) \underset{n \to \infty}{\centernot\Map} f(x)$.
            Положим $\tilde{g} = g \fmin \chi_{E\inv}$ (занулим $g(x)$ при $x \in E$).
            Так как $\mu(E) = 0$, то $I(g) = I(\tilde{g})$.

            Пусть $\tilde{\phi}_n \coloneqq \phi_n \fmin \tilde{g}$.
            Согласно лемме, $\lim\limits_{n \to \infty}\tilde{\phi}_n(x) = \tilde{g}(x)$ всюду.
            Значит, согласно малой теореме Леви~(\cref{levi-small}) $\lim\limits_{n \to \infty}I(\tilde{\phi}_n) = I(\tilde{g})$.
            Но так как $\tilde{\phi}_n \le \phi_n$, а $I(\tilde{g}) = I(g)$, то действительно $\lim\limits_{n \to \infty}I(\phi_n) \ge I(g)$.
        }
    }


    \section{Применения теоремы Леви. Свойства интеграла}
    \fact{\label{supp-size}Пусть $f \ge 0$ --- измеримая функция, положим $A \coloneqq \defset{x}{f(x) \ne 0}$.

    Тогда $J(f) = 0 \iff \mu(A) = 0$.
    \provewthen{
        $J(f) = \sup I(g)$, где $0 \le g \le f$. Из монотонности меры всякая такая $g$ сосредоточена на множестве меры нуль.
        Считая интеграл $g$ по определению, получаем нуль.
    }{\up\up
        \indentlemma[Неравенство Чебышёва]{\label{tschebyschev-inequality}
            Пускай $h \ge 0$ --- неотрицательная измеримая функция, $\lambda > 0$.
            Тогда $\mu\defset{x}{h(x) > \lambda} \le \frac{1}{\lambda}J(h)$.
        }{
            Пусть $e = \defset{x}{h(x)>\lambda}$.
            Заметим, что $h \ge \lambda \chi_e$, из монотонности интеграла $J(h) \ge \lambda \mu(e)$.
        }
        Пусть $A_n = \defset{x}{f(x) > \frac{1}{n}}$.
        $A = \bigcup\limits_{n \ge 1}A_n$.
        Согласно неравенству Чебышёва $\mu(A_n)\le n J(f) = 0$. Таким образом, $\mu(A) = 0$.
    }}
    \note{
        Теорема Леви сохраняет силу, если неравенство $f_n(x) \le f_{n + 1}(x)$ выполнено почти всюду (нарушаются на множестве меры нуль), и стремление $f_n(x) \underset{n \to \infty}\Map f(x)$ тоже имеется почти всюду.
    }
    \properties[Свойства интеграла]{
        \item Линейность интеграла.

        Пусть $f, g \ge 0$ --- измеримые функции, $\alpha, \beta \ge 0$. Тогда $J(\alpha f + \beta g) = \alpha J(f) + \beta J(g)$.
        \provehere{
            Выбираем последовательность простых функций $0 \le u_n \nearrow f$ и $0 \le v_n \nearrow g$ почти всюду, воспользуемся линейностью предела и теоремой Леви:
            \[J(\alpha f + \beta g) = \lim\limits_{n \to \infty}I(\alpha u_n + \beta v_n) = \lim\limits_{n \to \infty}(\alpha I(u_n) + \beta I(v_n)) = \alpha J(f) + \beta J(g)\qedhere\]
        }
        \item Счётная аддитивность по множеству.
        Пусть $f \ge 0$ --- измеримая функция, положим $\nu(e) = J(f \cdot \chi_e)$ для $e \in \Sigma$.
        Тогда $\nu$ --- счётно аддитивная мера на $\sigma$-алгебре $\Sigma$.
        \provehere{
            Аддитивность следует из линейности интеграла.

            Для проверки счётной аддитивности удостоверимся в монотонной непрерывности: пусть $E_1 \subset E_2 \subset \cdots$, где $E_j \in \Sigma$.

            Определим $E \coloneqq \bigcup\limits_{j = 1}^{\infty}E_j$. Надо проверить, что $J(f \cdot \chi_{E}) = \lim\limits_{n \to \infty}J(f \cdot \chi_{E_n})$.

            $f \cdot \chi_E = \lim\limits_{n \to \infty}f \cdot \chi_{E_n}$, значит, можно воспользоваться теоремой Леви.
        }
    }

    \section{Интегралы от знакопеременных функций}
    Пускай $f$ --- измеримая функция на $X$, возможно, принимающая значения $\pm \infty$.
    Представим $f = f_+ - f_-$, где $f_+ = f \fmax 0, f_- = -(f \fmin 0)$.
    Тогда $|f| = f_+ + f_-$, причём $f_+$ и $f_-$ измеримы, и обе неотрицательны.

    \definition[$f$ обладает интегралом]{
        $J(f_+) < +\infty$, или $J(f_-) < +\infty$.
        В таком случае $J(f) \bydef J(f_+) - J(f_-)$.
    }
    \definition[$f$ суммируема (интегируема)]{
        Она обладает конечным интегралом, то есть $J(f_+), J(f_-) < +\infty$.
    }
    \proposal{
        $f$ суммируема $\iff |f|$ суммируема.
        \provetwhen {$J(|f|) = J(f_+) + J(f_-) < +\infty$.}{$f_+, f_- \le |f|$.}
    }

    \subsection{Про линейность интеграла}
    Пусть $f = g - h$, где $g, h \ge 0$.
    Тогда во всяком случае $g \ge f_+$ и $h \ge f_-$: \[f = g - h \then f \le g\text{, а так как }g \ge 0\text{, то }f_+ \le g\text{ тоже;}\qquad f_- = (-f)_+\]
    \proposal{
        Если $f = g - h$, где $g, h$ измеримы и неотрицательны, причём хотя бы одно из $J(g), J(h)$ конечно, то $f$ обладает интегралом $J(f) = J(g) - J(h)$.
        \provehere{
            Рассмотрим случай, когда $J(g) < +\infty$, в случае $J(h) < +\infty$ всё аналогично.

            Тогда $J(f_+) < +\infty$, и $f$ по определению обладает интегралом.
            \[f = f_+ - f_- = g - h \quad\then\quad f_+ + h = g + f_-\]
            Для неотрицательных функций известна аддитивность, откуда $J(f_+) + J(h) = J(g) + J(f_-)$.
            Перенося в противоположные части конечные слагаемые $J(f_+)$ и $J(g)$, получаем
            \[J(h) - J(g) = J(f_-) - J(f_+)\]
            Умножая обе части на $-1$, получаем искомое.
        }
    }
    \corollary{
        Если $f, g$ суммируемы (и, вообще говоря, знакопеременны), то $f + g$ тоже суммируема, и $J(f + g) = J(f) + J(g)$.
        \provehere{
            \[(f_+ - f_-) + (g_+ - g_-) = (f_+ + g_+) - (f_- + g_-)\qedhere\]
        }
    }
    \fact{Если $f$ суммируема, $\alpha \in \R$, то $J(\alpha f) = \alpha J(f)$.}
    \properties[Ещё свойства интеграла]{
        \item Основная оценка интеграла: если $f$ обладает интегралом, то $|J(f)| \le J(|f|)$.
        \item Если $f, g$ --- измеримы, и обладают интегралами, причём $f \le g$, то $J(f) \le J(g)$.
    }
    Для $e \in \Sigma$ и измеримой функции $f: X \map \overline{\R}$, имеющей интеграл, имеется обозначение
    \encircle{\int\limits_{e}f \d \mu = J(f \cdot \chi_e)}
    \theorem[Абсолютная непрерывность интеграла]{
        Пускай $f$ --- суммируемая функция.
        Тогда $\forall \eps > 0: \exists \delta > 0$: если $e \in \Sigma$, $\mu(e) < \delta$, то $\int\limits_{e}|f| \d \mu < \eps$.
        \provehere{
            От противного: пусть $\exists \eps > 0: \forall \delta > 0: \exists e \in \Sigma: \mu(e) < \delta$, но $\int\limits_{e}|f| \d \mu \ge \eps$.

            Рассмотрим последовательность $\delta_n = \frac{1}{2^n}$.
            Для каждого $\delta_n$ найдётся $e_n \in \Sigma$: $\mu(e_n) \le \frac{1}{2^n}$, но $\int\limits_{e_n}|f|\d\mu \ge \eps$.

            Пусть $E_n = \bigcup\limits_{k \ge n}e_k$, тогда из монотонности $\int\limits_{E_n}|f|\d\mu \ge \eps$.
            С другой стороны. $\mu(E_n) \le \sum\limits_{k \ge n}\mu(e_k) \le \sum\limits_{k \ge n}\frac{1}{2^k} = \frac{1}{2^{k - 1}}$.

            Таким образом, $\mu(E_n) \underset{n \to \infty}\Map 0$, но с другой стороны $E_1 \supset E_2 \supset \cdots$
            Положим $E \coloneqq \bigcap\limits_{n \in \N}E_n$. Из счётной аддитивности $\int\limits_{E}|f|\d\mu = \lim\limits_{n \to \infty}\int\limits_{E_n}|f|\d\mu$, но левый интеграл равен нулю, как интеграл по множеству меры нуль, а правый предел --- хотя бы $\eps$.
        }
    }
    \fact{
        Если $f$ --- суммируемая функция, то $\defset{x}{f(x) \ne 0}$ $\sigma$-конечно.
        \provehere{
            Применить неравенство Чебышёва~(\cref{tschebyschev-inequality}).\[\defset{x}{f(x) \ne 0} = \bigcup\limits_{n > 0}\bigdefset{x}{|f|(x) \ge \frac{1}{n}}\qedhere\]
        }
    }
    \theorem[Общая теорема Леви]{
        Пускай $f_1, f_2, \dots$ --- измеримые функции, монотонно возрастающие: $f_n \le f_{n + 1}$.

        Предположим, что $f_1$ суммируема. Тогда $J(f_n) \underset{n \to \infty}\Map J(f)$, где $f(x) = \lim\limits_{n \to \infty}f_n(x)$.
        \provehere{
            Положим $h_j(x) = f_j(x) - f_1(x)$, и применим теорему Леви для неотрицательных функций.
        }
    }
%    \corollary{
%        Пусть $f = \lim\limits_{n \to \infty}f_n$, где $f_1 \le f_2 \le \dots$.
%        $f$ суммируема $\iff \sup\limits_{n}J(f_n) < +\infty$, при этом $J(f) = \lim\limits_{n \to \infty}J(f_n)$.
%    }
    \theorem[Вариант теоремы Леви для рядов]{
        Пусть $u_n$ --- неотрицательные суммируемые функции, $u(x) \coloneqq \sum\limits_{n = 1}^{\infty}u_n(x)$.
        Тогда $u$ суммируема $\iff \sum\limits_{n = 1}^{\infty}\int\limits_{X}u_n \d \mu < +\infty$.
    }
    В случае монотонной сходимости почти всегда почти всё можно делать, а если сходимость не монотонна, то есть следующая теорема.
    \newlection{1 ноября 2023 г.}
    \theorem[Лебег, о мажорируемой сходимости]{
        Пусть $f, g$ --- измеримые функции, ${f_n \almost{n \to \infty} f}$.
        Предположим, что у $f_n$ есть общая суммируемая мажоранта: $|f_n(x)| \le g(x)$ и $\int\limits_{X}g\d\mu < +\infty$. Тогда $\int\limits_{X}f_n \d \mu \underset{n \to \infty}\Map \int\limits_{X}f\d\mu$.
        \provehere{
        % Покажем, что почти всюду $\varlimsup\limits_{n \to \infty}|f_n(x) - f(x)| = 0$. Что там в скобочках написано?
            Так как $g$ --- мажоранта, то везде на $X: |f_n(x) - f(x)| \le 2g(x)$.

            Положим $h_k(x) \coloneqq \sup\limits_{n \ge k}|f_n(x) - f(x)|$, заметим, что $h_k \searrow 0$.
            Так как $0 \le h_0(x) \le 2g(x)$, то $h_0$ суммируема, откуда по теореме Леви: $\int\limits_{X}h_k(x)\d\mu \underset{k \to \infty}\Map 0$.

            Осталось применить принцип двух полицейских для проинтегрированного неравенства: \[0 \le |f_k(x) - f(x)| \le h_k(x) \quad \then \quad \int\limits_{X}0\d\mu \le \int\limits_{X}|f_k(x) - f(x)|\d\mu \le \int\limits_{X}h_k(x)\d\mu\qedhere\]
        }
    }
    \counterexample{
        Совсем без мажоранты ничего не получится. Если $X = \R$, и $f_n = n\chi_{[0, \frac1n]}$, то $f$ сходятся к нулю почти всюду, но интегралы у всех $f_n$ единичные.
    }
    \lemma[Фату]{
        Пусть $f_n \ge 0$ --- измеримые функции, тогда $\int\limits_{X}\left(\varliminf\limits_{n \to \infty}f_n\right) \d\mu \le \varliminf\limits_{n \to \infty}\int\limits_{X}f_n\d\mu$.
        \provehere{
            Положим $h_k(x) \coloneqq \inf\limits_{n \ge k}f_n(x)$.
            Заметим, что $h_k(x) \nearrow h(x) \coloneqq \varliminf\limits_{n \to \infty}f_n(x)$.
            \[\int\limits_{X}h\d\mu = \lim\limits_{k \to \infty}\int\limits_{X}h_k\d\mu = \varliminf\limits_{k \to \infty}\int\limits_{X}h_k\d\mu \le \varliminf\limits_{n \to \infty}\int\limits_{X}f_n \d\mu\qedhere\]
        }
    }
    \corollary{
        Если измеримые $g_n \almost{n \to \infty} g$, и $\int\limits_{X}|g_n|\d\mu \le C$, то $g$ суммируема, причём $\int\limits_{X}|g|d\mu \le C$.
    }


    \section{Виды сходимости}
    \bullets{
        \item Сходимость почти всюду: мера множества, где сходимости нет, равна нулю.
        \item Сходимость по мере: $f_n \underset{n \to \infty}{\overset{\mu}\Map} f \overset{def}\iff \forall \eps > 0: \mu\bigdefset{x \in X}{|f_n(x) - f(x)| > \eps} \underset{n \to \infty}\Map 0$.
    }
    \definition[Последовательность Коши по мере]{
        Последовательность измеримых функций $f_n$, такая, что $\forall \eps > 0: \lim\limits_{(n, m) \to \infty}\mu\bigdefset{x \in X}{|f_n(x) - f_m(x)| > \eps} = 0$.
    }
    \fact{
        \down\numbers{
            \item Если $\mu(X) < \infty$, то из сходимости $f_n \almost{n \to \infty} f$ следует сходимость по мере $f_n \underset{n \to \infty}{\overset{\mu}\Map} f$.
            \item Из сходимости по мере $f_n \underset{n \to \infty}{\overset{\mu}\Map} f$ следует, что найдётся сходящаяся подпоследовательность $n_1 < n_2 < \dots$: $f_{n_k} \almost{k \to \infty} f$.

            Докажем даже более сильное утверждение: для последовательности Коши по мере $f_n$ найдутся измеримая $f$, и сходящаяся к ней подпоследовательность $n_1 < n_2 < \dots$: $f_{n_k} \almost{k \to \infty} f$.
        }
        \provenumbers{
            \item Пусть $\eps > 0$, обозначим $A_n \coloneqq \defset{x \in X}{\exists k \ge n: |f_k(x) - f(x)| \ge \eps}$. Они вложены: ${A_1 \supset A_2 \supset \dots}$

            Отметим, что $A_n$ измеримы, и пусть $A \coloneqq \bigcap\limits_{n = 1}^{\infty}A_n$. На множестве $A$ нет сходимости: ${f_n \centernot\Map f}$.
            Но раз есть сходимость почти всюду, то $\mu(A) = 0$, то есть (так как мера конечна) ${\mu(A_n) \underset{n \to \infty}\Map 0}$.
            \item
            Найдутся такие $N_1 \le N_2 \le \dots$, что $\mu\defset{x \in X}{\forall n, m \ge N_k: |f_n(x) - f_m(x)| \ge \frac{1}{2^k}} \le \frac{1}{2^k}$.

            Положим $E_k \coloneqq \bigdefset{x \in X}{\abs{f_{N_k}(x) - f_{N_{k+1}}(x)} \le \frac{1}{2^k}}$. $\mu(X \sm E_k) \le \frac{1}{2^k}$.
            Пусть $\tilde{E}_k = \bigcap\limits_{n \ge k}E_n$, тогда $\mu\left(X \sm \tilde{E}_k\right) = \mu\left(\bigcup\limits_{n \ge k}(X \sm E_n)\right) \le \sum\limits_{n \ge k}\mu(X \sm E_n) \le \frac{1}{2^{k-1}}$.

            Если $x \in \tilde{E}_k \then \forall n \ge k: \abs{f_{N_n}(x) - f_{N_{n+1}}(x)} \le \frac{1}{2^n}$, то есть $\forall x \in \tilde{E}_k: \sum\limits_{j = 1}^{\infty}\abs{f_{N_j}(x) - f_{N_{j+1}}(x)}$ сходится.

            А тогда эта сумма сходится и на $\bigcup\limits_{k = 1}^{\infty}\tilde{E}_k$. Это влечёт $\forall x \in \bigcup\limits_{k = 1}^{\infty}\tilde{E}_k: \exists \lim\limits_{k \to \infty}f_{N_k}(x)$.
            К этому пределу $f_{N_k}$ сходятся почти всюду: мера $X \sm \bigcup\limits_{k = 1}^{\infty}\tilde{E}_k = \bigcap\limits_{k = 1}^{\infty}(X \sm \tilde{E}_k)$ равна нулю.
        }
    }
    \fact{
        Также для последовательности Коши по мере $f_n$ найдётся измеримая $f$, такая, что $\underset{n \to \infty}{\overset{\mu}\Map} f$.
        \provehere{
            Выберем, как выше, подпоследовательность $f_{n_k} \almost{k \to \infty} f$.

            Зафиксируем $\eps, \delta > 0$.

            Так как подпоследовательность --- тоже последовательность Коши, то $\exists N_1 \in \N: \forall m, l > N_1: \mu\defset{x \in X}{\abs{f_{n_m} - f_{n_l}} > \eps} < \delta$.
            Устремляя $l \to \infty$, получаем $\forall m > N_1: \mu\defset{x \in X}{\abs{f_{n_m} - f} > \eps} \le \delta$.

            Далее из того, что исходная последовательность --- тоже последовательность Коши, получаем, что $\exists N_2: \forall m, l > N_2: \defset{x \in X}{\abs{f_m - f_l} > \eps} < \delta$.

        Из неравенства треугольника $\forall n > \max(n_{N_1}, N_2): \mu\defset{x \in X}{\abs{f_n - f} > 2\eps} \le \delta$.
        }
    }
    \counterexamples{
    \item Последовательность функций $f_n \coloneqq \chi_{[n, n + 1]}$ сходится почти всюду к $0$, но сходимости по мере нет, так как мера бесконечна.
    \item Пусть $H_n = \sum\limits_{k = 1}^{n}\frac{1}{k}$ --- гармоническое число.
    Обозначим за $\{x\}$ дробную часть числа $x\in\R$.

        $f_n \coloneqq \all{\chi_{[\{H_n\}, \{H_{n+1}\}]},&\{H_n\} < \{H_{n+1}\} \\ \chi_{[\{H_n\}, 1]} + \chi_{[0, \{H_{n+1}\}]},&\{H_n\} > \{H_{n+1}\}}$ --- последовательность Коши по мере, которая сходится по мере к $0$, но сходимости нет нигде на $[0, 1]$.
    }

    \section{Классы $L^p$}
    Определим $L^p(\mu) \bydef \defset{f: X \map \R}{f \text{ --- измерима, и }\int\limits_{X}|f|^p \d\mu < \infty}$.
    Как видно из первой буквы, класс назван в честь Лебега.

    В дальнейшем мы будем считать, что $p \ge 1$.

    Функции $f \in L^p(\mu)$ отвечает норма $\|f\|_{L^p} = \left(\int\limits_{X}|f|^p\d\mu\right)^{\nicefrac1p}$.
    По этой норме, как и по всякой другой, можно построить метрику $d(\_,\_)$.
    \theorem{
        В случае $p \ge 1: d$ --- реально метрика на $L^p(\mu)$.
        Чтобы выполнялась положительная определённость ($\|f\| = 0 \iff f = 0$), будем рассматривать функции определённые с точностью до меры нуль на $X$.
        Иными словами $\|f\| = 0 \iff f = 0$ почти всюду~(\cref{supp-size}).
        \provehere{
            \indentlemma[Неравенство Гёльдера]{
                Пусть $p, q > 1$ --- сопряжённые показатели (${\nicefrac1p + \nicefrac1q = 1}$), тогда для $f \in L^p, g \in L^q: fg \in L^1$, и $\|fg\|_{L^1} \le \|f\|_{L^p}\|g\|_{L^q}$.
            }{
                Неравенство однородное, можно считать $\|f\|_{L^p} = \|g\|_{L^q} = 1$ --- для этого надо заменить $f \rightsquigarrow \frac{f}{\left(\int\limits_{X}|f|^p\right)^{\nicefrac{1}{p}}}$ и $g \rightsquigarrow \frac{g}{\left(\int\limits_{X}|g|^q\right)^{\nicefrac{1}{q}}}$.

                Для $a, b > 0$ имеется неравенство Юнга (доказывали через выпуклость $\exp$): $\frac{a^p}{p} + \frac{b^q}{q} \ge ab$.
                Применяя его, получаем $|f(x)|\cdot|g(x)| \le \frac{|f(x)|^p}{p} + \frac{|g(x)|^q}{q}$.
                Интегрируя, получаем искомое $\int\limits_{X}|f(x)|\cdot|g(x)|\d\mu \le \frac{1}{p} + \frac{1}{q} = 1$.
            }

            Теперь проверим неравенство треугольника $\|f + g\|_{L^p} \le \|f\|_{L^p}+ \|g\|_{L^p}$.
            Для данной нормы оно носит название \emph{неравенства Минковского}.
            \[\|f + g\|_{L^p}^p = \int\limits_{X}|f + g|^p \d\mu = \int\limits_{X}|f + g|\cdot|f + g|^{p - 1}\d\mu \le \int\limits_{X}|f|\cdot|f + g|^{p - 1}\d\mu+\int\limits_{X}|g|\cdot|f + g|^{p - 1}\d\mu \circlesign{\le}\]
            Применив к каждому слагаемому неравенство Гёльдера ($|f + g|^{p - 1} \in L^q$, так как $\int\limits_{X}|f + g|^{(p - 1)q}\d\mu = \int\limits_{X}|f + g|^{p}\d\mu \le 2^p\int\limits_{X}\max(f, g)^p\d\mu\le 2^p\int\limits_{X}\left(|f|^p + |g|^p\right)\d\mu$), получаем
            \[\circlesign{\le} \left(\|f\|_{L^p} + \|g\|_{L^p}\right)\left(\int\limits_{X}|f + g|^{(p-1)q}\d\mu\right)^{\frac{1}{q}} = \left(\|f\|_{L^p} + \|g\|_{L^p}\right)\left(\int\limits_{X}|f + g|^{p}\d\mu\right)^{\frac{1}{q}}\]
            Далее делим обе части неравенства на $\left(\int\limits_{X}|f + g|^{p}\d\mu\right)^{\frac{1}{q}}$, и остаётся \[\underbrace{\left(\int\limits_{X}|f + g|^{p}\d\mu\right)^{1 - \frac{1}{q}}}_{\|f + g\|_{L^p}} \le \|f\|_{L^p} + \|g\|_{L^p}\qedhere\]
        }
    }
    \theorem{
        $L^p(\mu)$ --- полно.
        \provehere{
            Рассмотрим последовательность Коши $f_n \in L^p(\mu)$, и пусть $E_{k,l} \coloneqq \defset{x \in X}{\abs{f_k(x) - f_l(x)} > \delta}$.

            По определению последовательности Коши $\forall \eps > 0: \exists N \in \N: k, l \ge N \then \int\limits_{X}|f_k - f_l|^p < \eps^p$.

            Тогда $\forall k, l \ge N: \mu(E_{k,l}) = \mu\defset{x \in X}{\abs{f_k(x) - f_l(x)}^p > \delta^p} \le \frac{\eps^p}{\delta^p}$. Значит, $f_n$ --- последовательность Коши по мере.

            Пусть $f_{k_j} \almost{j \to \infty} f$, тогда $\forall \eps > 0: \exists N \in \N: \forall j, s > N: \int\limits_{X}|f_{k_j} - f_{k_s}|^p < \eps$. Устремляя $s \to \infty$, по лемме Фату получаем $\int\limits_{X}|f_{k_j} - f|^p \le \eps$.
            Значит, $f$ --- предел подпоследовательности $f_{k_j}$, и из неравенства треугольника и фундаментальности можно показать, что $f$ --- предел.
        }
    }
    \newlection{8 ноября 2023 г.}

    \subsection{Приближение функций из класса $L^p$}
    В дальнейшем часто будем обозначать меру множества $X$ за $|X|$.

    \theorem{
        Пусть $(X, \Sigma, \mu)$ --- пространство с полной мерой.
        Тогда простые функции образуют плотное множество в $L^p(\mu)$ при $1 \le p < +\infty$.
        \provehere{
            Всякая простая функция имеет вид $\phi = \sum\limits_{j = 1}^{N}\alpha_j\chi_{e_j}$, дизъюнктные $e_j \in \Sigma$.
            Если $\phi \in L^p(\mu)$, то меры всех $e_j$, таких, что $\alpha_j \ne 0$, конечны.

            Пусть $f \in L^p(\mu)$, разложим $f = f_+ - f_-$.
            Приблизим $f_+$ и $f_-$ по отдельности.
            Тем самым, без потери общности $f \ge 0$.

            Раз $f$ измерима, то существует последовательность простых функций $\phi_n \in L^p(\mu): 0 \le \phi_n \le f$, $\phi_n \nearrow f$ почти всюду.

            Так как $f - \phi_n \searrow 0$ почти всюду, то $|f - \phi_n|^p \searrow 0$ почти всюду.
            Применяем теорему Леви, и действительно получаем, что $\int\limits_{X}|f - \phi_n|^p\d\mu \to 0$.
        }
    }
    Пусть мера $\mu$ получена продолжением по Лебегу --- Каратеодори из меры $\nu$ на полукольце $\mathcal{A} \subset \Sigma$.
    Простые функции, полученные из полукольца $\mathcal{A}$ (то есть вида $u = \sum\limits_{j =1}^{N}\alpha_j \chi_{a_j},a_j \in \mathcal{A}$) будем называть \emph{элементарными}.
    \theorem{
        При сделанных предположениях элементарные функции образуют плотное множество в $L^p(\mu)$.
        \provehere{
            Выберем $\eps > 0$.
            Пускай $f \in L^p(\mu)$. $\exists \phi = \sum\limits_{j = 1}^{k}\alpha_j \chi_{e_j}, e_j \in \Sigma$ --- простая функция, хорошо приближающая $f: \|f - \phi\|_{L^p} < \eps$.
            Теперь достаточно приблизить $\phi$, или даже каждое слагаемое $\phi$ элементарными функциями.

            Для всякого $\delta > 0,e_j \in \Sigma$ найдём множество $a_j \in \mathcal{R}(\mathcal{A}): \|\chi_{a_j} - \chi_{e_j}\|_{L^p} < \delta$.

            $\phi \in L^p(\mu) \then \forall j: \mu(e_j) < \infty \then \forall j: \exists A_j$ --- $\sigma$-множество, такое, что $\mu(A_j \sm e_j) < \frac{\delta}{2}$.

            Как $\sigma$-множество, $A_j = \bigcup\limits_{k = 1}^{\infty}b_k, b_k \in \mathcal{A}$.
            Положим $a_j^{(s)} = \bigcup\limits_{k = 1}^{s}b_k \in \mathcal{R}(\mathcal{A})$.

            Но тогда \[\int\limits_{X}|\chi_{a_j^{(s)}} - \chi_{e_j}|^p \d\mu = \int\limits_{X}|\chi_{a_j^{(s)}} - \chi_{e_j}|\d\mu \le \int\limits_{X}|\chi_{a_j^{(s)}}-  \chi_{{A_j}}|\d\mu+\int\limits_{X}|\chi_{A_j}-  \chi_{e_j}|\d\mu \underset{\text{при больших $s$}}\le \frac{\delta}{2} + \frac{\delta}{2}\qedhere\]
        }
    }
    \corollary{
        Линейные комбинации характеристических функций конечных прямоугольных параллелепипедов (или диадических кубов) образуют плотное множество в $L^p(\R^n)$ ($1 \le p < +\infty$).
    }
    \corollary{
        Непрерывные функции с компактным носителем плотны в $L^p(\R^n)$.
        \provehere{
            Достаточно доказать, что для любого двоичного куба $K$: $\exists$ непрерывная функция $v$ с компактным носителем $\|\chi_K - v\|_{L^p} < \eps$.
            Приблизим $\chi_{\left[\frac{l-1}{2^k},\frac{l}{2^k}\right]}$ ломаной, которая равна $1$ на $\left[\frac{l-1}{2^k},\frac{l}{2^k}\right]$, и равна нулю вне $\nicefrac\eps2$-окрестности $\left[\frac{l-1}{2^k},\frac{l}{2^k}\right]$.

            Теперь если $n$ --- любое, то $K = I_1 \times \dots \times I_n$, перемножим функции, приближающие $I_j$.
        }
    }
    Пусть $t \in \R^n, f$ --- функция на $\R^n$.
    Тогда \emph{сдвиг} $f$ на $t$ --- это $f_t(x) = f(x + t)$ (иногда пишут минус).
    \theorem[Непрерывность сдвига в среднем]{
        Если $f \in L^p(\R^n), 1 \le p < +\infty$, то ${\|f - f_t\|_{L^p}\underset{t \to 0}\Map 0}$.
        \provehere{
            Пусть $\eps > 0$. Найдём $v$ --- непрерывную функцию с компактным носителем, такую, что $\|f - v\|_{L^p(\R^n)} < \eps$.
            \[\|f - f_t\|_{L^p(\R^n)} < \|f - v\|_{L^p(\R^n)} + \|v - v_t\|_{L^p(\R^n)} + \|v_t - f_t\|_{L^p(\R^n)} \le 2\eps + \|v - v_t\|_{L^p(\R^n)}\]
            Осталось доказать ту же теорему для непрерывной функции с компактным носителем, а она очевидна из теоремы Кантора --- $v$ равномерно непрерывна.

            Чуть подробнее: выберем $\delta > 0$, найдётся шар $\overline{B}$, такой, что он содержит $\delta$-окрестность $\supp(v)$. На нём $\forall \eps' > 0: \exists \delta' \in (0, \delta): |x - y| < \delta' \then \abs{v(x) - v(y)} < \eps'$.
            Интегрируя по шару $\overline{B}$ с конечной мерой, получаем $\|v - v_t\| \le |\overline{B}|\eps'^{\nicefrac1p}$ и $\eps'$ можно сделать сколь угодно малым.
        }
    }
    \note[Следствие неравенства Гёльдера]{
        $\mu(X) < +\infty \then L^p(\mu) \subset L^s(\mu)$ для $p \ge s$.
    }
    \provehere{
        При $p = s$ доказывать нечего, считаем $p > s$. Положим $r = \frac{p}{s} > 1$, к нему есть сопряжённый показатель $r'$.

        Пускай $f \in L^p(\mu)$.
        \[\int\limits_{X}|f|^s\d\mu = \int\limits_{X}|f|^s \cdot 1 \d\mu \le \left(\int\limits_{X}\left(|f|^s\right)^r\d\mu\right)^{\nicefrac{1}{r}} \cdot \left(\int\limits_{X}(1)^{r'}\d\mu\right)^{\nicefrac{1}{r'}} = \left(\int\limits_{X}|f|^p\d\mu\right)^{\nicefrac{1}{r}} \cdot \mu(X)^{\nicefrac{1}{r'}}\]
    }
    Отсюда видно, что $\|f\|_{L^s(\mu)} \le \|f\|_{L^p(\mu)} \cdot \mu(X)^{\frac{1}{sr'}}$, это особенно красиво при \emph{вероятностной мере} --- $\mu(X) = 1$.

    В случае бесконечной меры ($\mu(X) = \infty$) следствие можно применять к функциям, сосредоточенных на множествах конечной меры.
    \ok
    Введём ещё пространство $L^{\infty}(\mu)$ --- множество функций, таких, что $\exists A \in \R_{\ge 0}: |f| \le A$ почти всюду.

    $L^{\infty}(\mu)$ --- класс всех \emph{существенно ограниченных} функций.

    Если $f$ --- существенно ограниченная функция, то среди всех существенных верхних границ $\defset{K}{|f(x)| \le K\text{ почти всюду}}$ найдётся инфимум:
    Назовём её \[\esssup f = \inf\defset{K}{K\text{ есть существенная верхняя грань для $f$}}\]
    \theorem{
        Пусть $A = \esssup f$, тогда $A$ --- существенная граница $f$.
        \provehere{
            Пусть $n \in \N$, тогда $A + \frac{1}{n}$ --- существенная верхняя граница $f$.
            Тем самым, $\exists E_n: |E_n| = 0, |f(x)| \le A + \frac{1}{n}$ при $x \notin E_n$.
            Выберем $E = \bigcup\limits_{n = 1}^{\infty}E_n$. Тогда $|f(x)| \le A$ при $x \notin E$, но $\mu(E) = 0$.
        }
    }
    \note{
        Пусть $f$ существенно ограниченна, $A = \esssup f$. Тогда $\exists E: \mu(E) = 0$ и ${\sup\limits_{x \in X \sm E}f(x) = A}$.
    }
    \definition[Норма $f \in L^{\infty}(\mu)$]{
        $\|f\|_{L^{\infty}(\mu)} = \esssup\limits_{X}|f|$.
    }
    Если в пространстве $L^{\infty}$ отождествить функции, отличающиеся на множестве меры нуль, то $\|\_\|$ станет нормой.

    Расстояние между функциями в данном пространстве $d(f, g) = \|f - g\|$, неравенство треугольника здесь очевидно:
    \[\|u + v\|_{L^{\infty}} \le \|u\|_{L^{\infty}} + \|v\|_{L^{\infty}}\]
    \theorem{
        $L^{\infty}(\mu)$ полно.
        \provehere{
            Пусть $\{f_n\}$ --- последовательность Коши в $L^{\infty}(\mu)$, то есть $\esssup\limits_{x \in X}|f_n(x) - f_m(x)| \underset{(n,m)\to \infty}\Map 0$.

            Тогда найдутся множества $E_{n,m}: \essssup|f_n - f_m| = \sup\limits_{x \notin E_{n,m}}|f_n(x) - f_m(x)|$.

            Положим $E = \bigcup\limits_{n,m}E_{n,m}$, $\mu E = 0$. Тогда $\left\{f_n\big|_{X \sm E}\right\}_{n}$ --- последовательность Коши на пространстве ограниченных функций на $X\sm E$.
            Тем самым, $f_n\rightrightarrows f$ равномерно на $X \sm E$.
            Доопределим $f$ на $E$ как угодно, её класс эквивалентности в $L^{\infty}$ не поменяется.
        }
    }
    В неравенстве Гёльдера до сих пор рассматривались $p, p' : \frac{1}{p} + \frac{1}{p'} = 1$ при $1 < p,p' < \infty$.
    Если же подставить одно из $p, p'$ равным $1$, то второе станет равным $\infty$.
    Естественно считать $1$ и $\infty$ сопряжёнными показателями.

    Неравенство Гёльдера говорило, что $\int\limits_{X}|fg|\d\mu \le \|f\|_{L^p(\mu)} \cdot \|g\|_{L^{p'}(\mu)}$.
    \fact{
        Неравенство Гёльдера сохраняется при $p = 1$ или $p = \infty$.
        \provehere{
            Пусть $p = 1$. $|f(x)|\cdot|g(x)| \le |f(x)| \cdot \|g\|_{L^{\infty}(\mu)}$ почти всюду.
            Интегрируя это неравенство, получаем
            \[\int\limits_{X}|f|\cdot|g|\d\mu \le \int\limits_{X}|f(x)|\d\mu\cdot\|g\|_{L^{\infty}(\mu)} = \|f\|_{L^{1}(\mu)}\cdot\|g\|_{L^{\infty}(\mu)}\qedhere\]
        }
    }
    \fact{
        Пусть $\mu(X) = 1$.
        Тогда $\|f\|_{L^p(\mu)} \le \|f\|_{L^{\infty}(\mu)}$ при любом $p < \infty$.
        \provehere{
            \[\|f\|_{L^p} = \left(\int\limits_{X}|f|^p\d\mu\right)^{\nicefrac1p} \le \left(\int\limits_{X}\|f\|_{L^\infty(\mu)}^p\d\mu\right)^{\nicefrac1p} = \|f\|_{L^\infty(\mu)}\cdot\mu(X)\]
            В частности, из данного доказательства следует, что при $\mu(X) < +\infty$: $L^p(\mu) \supset L^{\infty}(\mu)$.
        }
    }
    Пусть $\mu(X) = 1$.
    Зафиксируем измеримую $f$, рассмотрим возрастающую функцию \[p \mapsto \|f\|_{L^p(\mu)}\]
    Если $f \notin L^p(\mu)$, то будем считать $\|f\|_{L^p} = \infty$.
    \exercise{
        $\lim\limits_{p \to \infty}\|f\|_{L^p} = \|f\|_{L^{\infty}}$.
    }

    \subsection{Связь интегралов Лебега и Римана}
    \theorem{
        Пусть $f$ --- функция на отрезке $\angles{a, b}$, интегрируемая по Риману --- Дарбу.
        Тогда $f$ суммируема, и интеграл Лебега такой же.
        \provehere{
            В данной постановке простые функции --- линейные комбинации характеристических функций отрезков, $\phi = \sum\limits_{j = 1}^{N}c_j \chi_{I_j}$.
            В этой лекции они назывались элементарными.

            Простые функции интегрируемы и по Риману, и по Лебегу, и интеграл у них один и тот же.

            Пусть $\angles{a, b} = I_1 \sqcup \dots \sqcup I_k$ --- разбиение $\Delta = \{I_1, \dots, I_k\}$.

            Зададим $\phi_{\Delta} = \sum\limits_{j = 1}^{k}\left(\sup\limits_{I_j} f\right)\chi_{I_j}, \psi_{\Delta} = \sum\limits_{j = 1}^{k}\left(\inf\limits_{I_j}f\right)\chi_{I_j}$.
            Тогда $\int\limits_{\angles{a,b}}\phi_{\Delta}$ --- верхняя сумма Дарбу для $f$ по отрезку $\angles{a, b}$, $\int\limits_{\angles{a, b}}\psi_{\Delta}$ --- нижняя сумма Дарбу.

            Понятно, что $\psi_\Delta \le f \le \phi_\Delta$ всюду на $\angles{a, b}$, причём для измельчения $\Delta'$ верно, что
            \[\psi_\Delta \le \psi_{\Delta'} \le f \le \phi_{\Delta'} \le \phi_\Delta\]
            Критерием интегрируемости по Риману является то, что $\osc\limits_{I_j}f $ могут быть сколь угодно малыми, то есть $\forall \eps > 0: \exists \Delta: \int\limits_{\angles{a,b}}(\phi_{\Delta} - \psi_{\Delta}) \le \eps$.

            Выберем последовательность $\eps_n = \frac{1}{n}$, построим разбиения $\Delta_n$ так, что каждое следующее является измельчением предыдущего.

            Тогда $\int\limits_{\R}(\phi_n - \psi_n)\d\lambda = \int\limits_{\R}|\phi_n - \psi_n|\d\lambda < \frac{1}{n}$.

            Отсюда следует, что существует последовательность индексов $n_j$, таких, что $\phi_{n_j} - \psi_{n_j} \underset{n \to \infty}\Map 0$ почти всюду.
            Таким образом, $\psi_{n_j}$ и $\phi_{n_j}$ стремятся к $f$ почти всюду, тем самым $f$ измерима!

            Теперь $\int\limits_{\R^n}\psi_{n_j} \le \text{интеграл Лебега или Римана }f \le \int\limits_{\R^n}\phi_{n_j}$.
        }
    }
    \intfact[Теорема Лебега]{
        Функция $f$ на конечном отрезке интегрируема по Риману $\iff$ множество точек разрыва $f$ имеет меру нуль.
    }
    \note{
        Пусть $f \ge 0$, $f$ интегрируема в смысле Римана несобственным образом на конечном или бесконечном интервале $\angles{\alpha, \beta}$.
        Тогда $f$ суммируема на $\angles{\alpha, \beta}$.
        \provehere{
            Например, пусть особенность на конце $\beta$: $f$ интегрируема по Риману на любом интервале $\angles{\alpha, \beta - \delta}$, причём $\exists \lim\limits_{\delta \to 0}\int\limits_{\alpha}^{\beta - \delta}f(x)\d x$.
            Пускай $f_n = f \cdot \chi_{\angles{\alpha, \beta - \frac{1}{n}}}$. Тогда $f_n \nearrow f$, по теореме Леви предельная функция тоже суммируема, причём её интеграл --- предел интегралов $f_n$.
        }
    }
    \note{
        Если функция знакопеременна, то интегрировать всё ещё бывает полезно в несобственном смысле: $\frac{\sin x}{x}$ не суммируема на $[0, \infty)$, но можно писать
        \[\int\limits_{0}^{\infty}\frac{\sin x}{x}\d x = \lim\limits_{R \to \infty}\int\limits_{0}^{R}\frac{\sin x}{x}\d x = \frac{\pi}2\]
    }
    \newlection{15 ноября 2023 г.}


    \section{Теоремы Тонелли и Фубини}
    Рассмотрим два пространства с мерой $(X, \mathcal{A}, \mu), (Y, \mathcal{B}, \nu)$ ($\mathcal{A}, \mathcal{B}$ --- $\sigma$-алгебры, $\mu, \nu$ --- счётно-аддитивные меры на $\mathcal{A}$ и $\mathcal{B}$ соответственно).

    Рассмотрим полукольцо $\mathcal{P} = X \times Y$ обобщённых прямоугольников: $c \in \mathcal{P} \iff c = a \times b$ для $a \in \mathcal{A}, b \in \mathcal{B}$.

    \proposal{
        Мера $\lambda \coloneqq \mu \otimes \nu$ на $\mathcal{P}$ ($\lambda(a \times b) \coloneqq \mu(a)\nu(b)$) счётно-аддитивна.
        \provehere{
            Выберем $\{a_j\}_{j = 1}^{\infty}\subset\mathcal{A}, \{b_j\}_{j = 1}^{\infty}\subset\mathcal{B}, \{c_j\}_{j = 1}^{\infty}\subset\mathcal{P}$ так, что $c_j = a_j \times b_j$.
            Пусть $c_j$ дизъюнктны; положим $c \coloneqq \bigsqcup\limits_{j = 1}^{\infty}c_j$, пусть $c \in \mathcal{P}$, то есть $\exists a \in \mathcal{A}, b \in \mathcal{B}: c = a \times b$.

            Надо проверить, что $\lambda(c) = \sum\limits_{j = 1}^{\infty}\lambda(c_j)$.

            Рассмотрим равенство $\chi_a(x)\chi_b(y) = \sum\limits_{j = 1}^{\infty}\chi_{a_j}(x)\chi_{b_j}(y)$. При каждом фиксированном $x$ обе части --- измеримые функции от $y$.

            Интегрируя, получаем по теореме Леви \[\chi_a(x)\underbrace{\int\limits_{Y}\chi_b(y)\d\nu(y)}_{\nu(b)} = \sum\limits_{j = 1}^{\infty}\chi_{a_j}(x)\underbrace{\int\limits_{Y}\chi_{b_j}\d\nu(y)}_{\nu(b_j)}\]
            Это равенство опять интегриурется, уже по $x$. В результате действительно получаем $\mu(a)\nu(b) = \sum\limits_{j = 1}^{\infty}\mu(a_j)\nu(b_j)$.
        }
    }
    Применяя теорему Лебега --- Каратеодори, можно продолжить меру $\lambda$, результат тоже обозначают $\mu \otimes \nu$, и называют \emph{произведением мер} $\mu$ и $\nu$.

    Пусть имеется несколько пространств с мерой $(X_1, \mu_1), \dots, (X_n, \mu_n)$.
    Можно определить меру произведения $\mu_1 \otimes \dots \otimes \mu_n$.
    В произведении, вообще говоря, надо указать порядок, но оказывается, что произведение мер ассоциативно.
    \example{
        Рассмотрим $\R^{n+k}=\R^n\times\R^k$. Пусть $\lambda_n, \lambda_k$ --- стандартные меры Лебега на $\R^n$ и $\R^k$.
        Тогда оказывается, что $\lambda_n \otimes \lambda_k = \lambda_{n + k}$.

        Можно заметить, что на обобщённых прямоугольниках мера произведения одна и та же, и применяя теорему об единственности, получаем $\lambda_n \otimes \lambda_k = \lambda_{n + k}$.
        (причём на самом деле неважно, что обобщённые прямоугольники берутся из евклидова пространства, это проверяет ассоциативность в общем виде)
    }
    Пускай $(X, \mathcal{A}, \mu)$, $(Y, \mathcal{B}, \nu)$ --- пространства со счётно-аддитивными мерами, обе меры полны и обе $\sigma$-конечны.
    В теоремах Тонелли и Фубини теоретически можно обойтись и без этих двух условий, но требуются дополнительные слова. Пусть $\lambda = \mu \otimes \nu$.
    \theorem[Тонелли]{
        Пусть $f$ --- $\lambda$-измеримая функция на $X \times Y$, $f \ge 0$. Тогда
        \numbers{
            \item Для $\mu$-почти всех $x \in X$: $f(x, \_)$ измерима на $Y$.
            \item Функция $\phi(x) \coloneqq \int\limits_{Y}f(x, \_)\d\nu$ измерима на $X$.
            \item $\int\limits_{X}\phi(x)\d\mu(x) = \int\limits_{X \times Y}f \d\lambda$.
        }
        \provehere{
            Назовём измеримую функцию $f \ge 0$ допустимой, если она определена на $X \times Y$, и удовлетворяет всем трём условиям.
            \numbers{
                \item Если $a \in \mathcal{A}, b \in \mathcal{B}$, то $\chi_{a \times b}$ допустима: $\chi_{a \times b}(x, y) = \chi_a(x)\chi_b(y)$.
                \item Неотрицательные элементарные функции, построенные по полукольцу $\mathcal{P}$, допустимы:
                \bullets{
                    \item Если $f, g$ --- допустимы, $\alpha, \beta \ge 0$, то $\alpha f + \beta g$ тоже допустима.
                    \item Если $f, g$ --- допустимы и $f$ суммируема, причём $0 \le g \le f$, то $f - g$ тоже допустима.
                    \provehere{
                        Пусть $\phi(x) = \int\limits_{X}f(x, \_)\d\nu$. В силу 3. она суммируема, откуда $\phi$ конечна почти всюду. Пусть $\psi(x) = \int\limits_{X}g(x, \_)\d\nu$.
                        Так как $\psi \le \phi$, то $\psi$ тоже конечна почти всюду, тогда дальше всё хорошо.
                    }
                }
                \item Пусть $f_n$ --- допустимые функции на $X \times Y$, пусть $0 \le f_1 \le f_2 \le \dots$, пусть $f(x, y) = \lim\limits_{n \to \infty}f_n(x, y)$. Автоматически $f$ измерима.
                Тогда $f$ тоже допустима.
                \provehere{
                    Пускай $E_n = \defset{x \in X}{f_n(x, \_)\text{ не измерима}}$. $\forall n: \mu E_n = 0$, так как $f_n$ допустимы. Положим $E \coloneqq \bigcup\limits_{n \in \N}E_n$, $\mu E = 0$.

                    $x \notin E \then$ все функции $f_n(x, \_)$ измеримы на $Y$. Имеется монотонная сходимость $f_n \nearrow f$, значит $f(x, \_)$ тоже измерима на $Y$ при $x \notin E$.

                    Построим $\phi(x) = \int\limits_{Y}{f(x, \_)\d\nu}, \phi_n(x) = \int\limits_{Y}f_n(x, \_)\d\nu$.
                    По теореме Леви (относительно меры $\nu$) для $x \notin E: \phi(x) = \lim\limits_{n \to \infty}\phi_n(x)$.
                    Тем самым $\phi$ измерима, как предел измеримых функций.

                    Более того, $\phi_n \nearrow \phi$, опять по теореме Леви (относительно меры $\mu$): \[\int\limits_{X}\phi\d\mu = \lim\limits_{n \to \infty}\int\limits_{X}\phi_n\d\mu = \lim\limits_{n \to \infty}\int\limits_{X \times Y}f_n \d\lambda \underset{\text{теорема Леви относительно $\lambda$}}= \int\limits_{X \times Y}f\d\lambda\qedhere\]
                }
                \item Пусть $f_n$ --- допустимые функции на $X \times Y$, пусть $f_1 \ge f_2 \ge \dots$, пусть $f(x, y) = \lim\limits_{n \to \infty}f_n(x, y)$. Автоматически $f$ измерима.
                Если $f_1$ суммируема, то $f$ тоже допустима.
                \provehere{Аналогично предыдущему пункту.}
                \item Если $A \subset X \times Y$ --- $\sigma$-множество. то $\chi_A$ допустима.
                \provehere{
                    Представим $A$ в виде $A = \bigsqcup\limits_{j = 1}^{\infty}A_j$.
                    $\sum\limits_{j = 1}^{N}\chi_{A_j} \underset{n \to \infty}\nearrow \chi_A$.
                }
                \item Если $A \subset X \times Y$ --- $\delta\sigma$-множество конечной меры $\lambda$, то $\chi_A$ допустима.
                \provehere{
                    Представим $A$ в виде $\bigcap\limits_{j = 1}^{\infty}A_j$, где $A_1 \supset A_2 \supset \dots$, $A_j$ --- $\sigma$-множества конечной меры.
                    $\chi_{A_j} \underset{n \to \infty}\searrow \chi_A$.
                }
                \item Если $e \subset X \times Y$ измеримо, и $\lambda(e) = 0$, то $\chi_e$ допустимо.
                \provehere{
                    Пусть $\overline{e}$ --- $\delta\sigma$-множество, такое, что $\overline{e} \supset e$, и $\lambda\left(\overline{e}\right) = 0$.

                    Тогда $\chi_e \le \chi_{\overline{e}}$. $\chi_{\overline{e}}$ допустима, в частности, $\chi_{\overline{e}}(x, \_)$ измерима на $Y$ для почти всех $x \in X$.
                    Обозначив $\overline{\phi}(x) = \int\limits_{Y}\chi_{\overline{e}}(x, \_)\d\nu$ видим, что $\overline{\phi}$ измерима на $X$, а так как
                    \[\int\limits_{X}\overline{\phi}\d\mu = \int\limits_{X \times Y}\chi_{\overline{e}}\d\lambda = 0\]
                    то $\overline{\phi}(x) = 0$ для почти всех $x \in X$.

                    Пусть $E = \defset{x \in X}{\overline{\phi}(x) \ne 0}$.
                    Для $x \notin E: \int\limits_{Y}\chi_{\overline{e}}(x, \_)\d\nu = 0$. Иными словами, ${\nu\defset{y \in Y}{(x, y) \in \overline{e}} = 0}$.

                    Но тогда из полноты меры (здесь мы ей пользуемся в первый раз) $\nu\defset{y \in Y}{(x, y) \in e} = 0$.
                    Тогда любая функция, сосредоточенная на $e$, измерима, в частности, $\chi_e(x, \_)$ измерима на $Y$.

                    Зная измеримость $\chi_e$ уже несложно доказать, что в пунктах 2 и 3 все интегралы равны нулю: в частности, $\phi(x) = \int\limits_{Y}\chi_e(x, \_)\d\nu$ равна нулю всюду кроме $E$.
                }
                \item Если $A \subset X \times Y$ --- измеримое множество относительно меры $\lambda$, причём $\lambda(A) < +\infty$, то $\chi_A$ допустима.
                \provehere{
                    $\exists \delta\sigma$-множество $\overline{A} \supset A$, такое, что $\lambda(\overline{A}\sm A) = 0$.
                    Применим второй $\bullet$ из 2. $\chi_A = \chi_{\overline{A}} - \chi_{A \sm \overline{A}}$.
                }
                \item Пусть $f$ --- простая функция относительно $\sigma$-алгебры $\lambda$-измеримых множеств, $f \ge 0$. Иными словами,
                \[f = \sum\limits_{j = 1}^{N}\alpha_j\chi_{e_j}, \alpha_j \ge 0, e_i \cap e_j = \o \text{ (при $i \ne j$)}\]
                Если $\forall j: \lambda e_j < +\infty$, то $f$ допустима.
                \item Пусть $f$ --- неотрицательная измеримая функция на $X \times Y$, $\lambda\defset{(x, y)}{f(x, y) \ne 0} < +\infty$. Тогда $f$ допустима.
                \provehere{
                    $\exists f_n$ --- простые функции, $0 \le f_n \le f$, $f_n \nearrow f$. Все $f_n$ допустимы, значит и $f$ допустима.
                }
                \item Все неотрицательные измеримые функции допустимы.
                \provehere{
                    $\exists X_1 \subset X_2 \subset \dots$, такие, что $X = \bigcup\limits_{i}X_i$, и все $\mu(X_i) < +\infty$. Аналогично $\exists Y_1 \subset Y_2 \subset \dots$, такие, что $Y = \bigcup\limits_{i}Y_i$, и все $\mu(Y_i) < +\infty$.
                    (Здесь мы пользуемся $\sigma$-конечностью в первый раз).

                    Положим $f_n(x, y) = f(x, y)\chi_{X_n}(x)\chi_{Y_n}(y)$. $f_n$ из пункта 10, значит, $f$ допустима, так как $f_n \nearrow f$.
                }
            }
        }
    }
    \theorem[Фубини]{
        Пусть $(X, \mu), (Y, \nu)$ --- два пространства с мерой, $\lambda = \mu \otimes \nu$.

        Если $f \in L^1(\lambda)$, то \bullets{
            \item Для почти всех $x\in X: \phi(x) \coloneqq \int\limits_{Y}f(x, \_)\d\nu$ суммируема на $X$.
            \item $\int\limits_{X}\phi\d\mu = \int\limits_{X \times Y}f\d\lambda$.
        }
        \provehere{
            $f$ суммируема $\then f_+, f_-$ суммируемы по $\lambda$.
            К каждой из них применима теорема Тонелли.
            Вычитаем заключения теоремы Тонелли для $f_+$ и $f_-$.
        }
    }
    \problem{
        Придумать функцию $f$, такую, что $\phi(x) \coloneqq \int\limits_{Y}f(x, \_)\d\nu$ суммируема, но $f \notin L^1(\lambda)$.
    }

    \subsection{Как применять}
    Пусть $f$ --- $\lambda$-измеримая функция (про знак ничего не известно).

    Чтобы доказать, что $f$ суммируема, надо доказать, что $|f|$ суммируема.

    По теореме Тонелли $|f|$ суммируема $\iff \int\limits_{X}\int\limits_{Y}|f|(x, y)\d \nu(y)\d \mu(x)$ конечен.
    Если интеграл сошёлся, то $f$ тоже суммируема, и для исходной функции тоже можно сводить интеграл к повторному.


    \section{Свёртки. Приближение функций с помощью свёрток}
    Пускай $f, g$ --- измеримые функции на $\R^n$.
    \definition[Свёртка $f * g$]{
        $(f * g)(x) = \int\limits_{\R^n}f(y)g(x - y)\d y$.
        Свёртка определена в тех точках, где интеграл определён.
    }
    Рассмотрим $L: \R^n \times \R^n \map \R^n \times \R^n, (x, y)\mapsto (y, x - y)$.
    $L$ линейно, значит, $L, L^{-1}$ измеримы по Лебегу.
    Определив $T: \R^n\times\R^n \map \R, (u, v) \mapsto f(u)g(v)$, видим, что $T$ измерима, откуда $(T \circ L)(x, y) = f(y)g(x - y)$ тоже измерима.
    \theorem{\label{p-equals-1}
    Если $f, g \in L^1(\R^n)$, то $(f * g)$ определена почти всюду, и $\|f * g\|_{L^1} \le \|f\|_{L^1} \|g\|_{L^1}$.
    \provehere{
        Рассмотрим $\phi(x, y) = |f(x)|\cdot |g(x - y)|$. Она неотрицательна, применяем теорему Тонелли:
        \multline{\int\limits_{\R^n \times \R^n}\phi\d\lambda_{2n} = \int\limits_{\R^n}\left(~\int\limits_{\R^n}|f(y)|\cdot|g(x-y)|\d x\right)\d y = \\
            = \int\limits_{\R^n}|f(y)|\left(~\int\limits_{\R^n}|g(x - y)|\d x\right)\d y = \int\limits_{\R^n}|f(y)|\left(~\int\limits_{\R^n}|g(x)|\d x\right)\d y = \|f\|_{L^1}\|g\|_{L^1}}

        По теореме Тонелли $\phi$ суммируема, тем самым, $(x, y) \mapsto f(y)g(x - y)$ тоже суммируема.
        По теореме Фубини $(f * g)(x)$ определена для почти всех $x$, причём она суммируема.
        \[\int\limits_{R^n}|(f * g)(x)| \d x \le \int\limits_{\R^n}\int\limits_{\R^n}|f(y)g(x-y)|\d y\d x \le \|f\|_{L^1}\|g\|_{L^1}\qedhere\]
    }
    }

    \note{
        Неформально говоря, если сворачивать $f$ с какими-то хорошими свойствами, и $g$ с какими-то другими хорошими свойствами, то свёртка обладает всеми хорошими свойствами каждой из них.
        В этом мы убедимся на некоторых примерах.
    }

    \statement{
        $f * g = g * f$ всегда, когда существует:
        \[(f * g)(x) = \int\limits_{\R^n}f(x - y)g(y)\d y =\left\|\arr{c}{z = x - y \\ y = x - z}\right\| = \int\limits_{\R^n}f(z)g(x - z)\d z\]
    }
    \exercise{
        Свёртка ассоциативна: $f * (g * h) = (f * g) * h$ всегда, когда существует.
    }
    \newlection{22 ноября 2023 г.}

    \subsection{Меры с плотностью}
    Пусть $(X, \Sigma, \mu)$ --- пространство с мерой. Пусть $\phi \ge 0$ --- измеримая функция на $X$.

    Можно определить меру, индуцированную функцией $\phi$: $\nu(e) = \int\limits_{e}\phi\d\mu$ для $e \in \Sigma$.
    Тогда $\phi$ называется плотностью меры $\nu$ относительно $\mu$.

    \subsubsection{Куда должна бить $\phi$?}
    \numbers{
        \item Можно считать, что $\phi: X \map \R$, но, возможно, меняет знак.
        Надо предположить, что либо $\phi_+$, либо $\phi_-$ суммируемы.

        Тогда сохраняется счётная аддитивность: $e = \bigsqcup\limits_{i = 1}^{\infty}e_i \then \nu(e) = \sum\limits_{i = 1}^{\infty}\nu(e_i)$.
        Тем не менее, всякие монотонности могут перестать выполняться, так как функция перестала быть неотрицательной.
        \item Можно считать, что $\phi: X \map \C$.
        Комплексный интеграл берётся отдельно по вещественной и мнимой частям:
        \[\int\limits_{e}(\alpha + \beta i)\d\mu = \int\limits_{e}\alpha\d\mu + i \int\limits_{e}\beta\d\mu\]
    }
    В обоих случаях $\nu$ перестаёт быть мерой в заявленном определении, это просто какая-то счётно-аддитивная функция множества, и $\phi$ во всех случаях называется её плотностью.

    \subsubsection{Интегрирование по мере $\nu$}
    Пусть $\nu$ определена, как выше.
    \fact{
        Если $g \ge 0$ --- измеримая функция на $X$, то $\int\limits_{X}g\d\nu = \int\limits_{X}g\phi\d\mu$.
        \provehere{
            Формула верна для $g = \chi_e$: \[\int\limits_{X}\chi_e\d\nu = \nu(e) = \int\limits_{X}\chi_e\phi\d\mu\]
            Значит, формула верна для неотрицательных простых функций.

            Теперь пусть $g$ --- произвольная измеримая. Существуют неотрицательные простые $g_n \nearrow g$, применяем теорему Леви (два раза, в левой и правой частях равенства).
        }
    }
    \corollary{
        Неотрицательная функция $g$ $\nu$-суммируема $\iff g\phi$ $\mu$-суммируема.
    }
    \corollary{
        $h$ (возможно, меняющая знак) $\nu$-суммируема $\iff h\phi$ $\mu$-суммируема, причём $\int\limits_{X}h\d\nu = \int\limits_{X}h\phi\d\mu$.
    }

    \subsection{Образ меры}
    Пусть $(X, \mathcal{A}), (Y, \mathcal{B})$ --- два пространства, $\mathcal{A}, \mathcal{B}$ --- $\sigma$-алгебры подмножеств в $X$ и $Y$ соответственно.

    Пусть $F: X \map Y$ измеримо относительно $(\mathcal{A}, \mathcal{B})$.

    Пускай  $\mu \ge 0$ --- счётно-аддитивная мера на $(X, \mathcal{A})$.
    Её \emph{образ} $F^0(\mu) \eqqcolon \nu$ --- счётно-аддитивная мера на $(Y, \mathcal{B})$, такая, что $\nu(b) = \mu(F^{-1}(b))$.
    Счётная аддитивность следует из того, что прообраз уважает все теоретико-множественные операции.

    \subsubsection{Интегрирование по мере $\nu$}
    Пусть $\nu$ определена, как выше.
    \[\int\limits_{Y}\chi_e\d\nu = \nu(b) = \int\limits_{X}\chi_{F^{-1}(b)}\d\mu\]
    Заметим, что $\chi_{F^{-1}(b)} = \chi_b \circ F$.
    \fact{
        Если $g \ge 0$ --- измеримая функция на $X$, то $\int\limits_{X}g\d\nu = \int\limits_{X}g\circ F\d\mu$.
        \provehere{
            Формула верна для $g = \chi_e$.
            Значит, формула верна для неотрицательных простых функций.

            Теперь пусть $g$ --- произвольная измеримая. Существуют неотрицательные простые $g_n \nearrow g$, применяем теорему Леви (два раза, в левой и правой частях равенства).
        }
    }
    Все замечания из предыдущего раздела повторяются:
    \corollary{
        Неотрицательная функция $g$ $\nu$-суммируема $\iff g\circ F$ $\mu$-суммируема.
    }
    \corollary{
        $h$ (возможно, меняющая знак) $\nu$-суммируема $\iff h\circ F$ $\mu$-суммируема, причём $\int\limits_{X}h\d\nu = \int\limits_{X}h\circ F\d\mu$.
    }
    Данная формула очень полезна при замене переменной в интеграле.

    Например, ранее записанное равенство $\int\limits_{\R^n}f(x - y)\d y = \int\limits_{\R^n}f(\xi)\d\xi$ видно из данной формулы при $F(y) = x - y$ --- здесь образ меры будет ей самой.

    \subsection{Свойства свёртки}
    Пусть $1 \le p \le \infty$.
    \theorem{
        Если $g \in L^1(\R^n)$, $f \in L^p(\R^n)$, то $\|f * g\|_{L^p(\R^n)} \le \|f\|_{L^p(\R^n)}\|g\|_{L^1(\R^n)}$.
        \provehere{
            Пусть $p = \infty$. Тогда это следует из оценки
            \[|(f * g)(x)| \le \int\limits_{\R^n}|f(x - y)| \cdot |g(y)| \d y \le \underbrace{\|f\|_{L^{\infty}}}_{\esssup(f)~}\int\limits_{\R^n}|g(y)|\d y\]
            При $p = 1$ доказано выше~(\cref{p-equals-1}).

            Теперь пусть $1 < p < \infty$, $q$ --- сопряжённый к $p$ показатель.
            \[\int\limits_{\R^n}|(f * g)(x)|^p\d x = \int\limits_{\R^n}\left(~\int\limits_{\R^n}|f(x - y)||g(y)|\d y\right)^p \d x \circlesign{=}\]
            Определим из меры Лебега новую меру с плотностью $|g(y)|$: $\nu(e) \coloneqq \int\limits_{e}|g(y)|\d y$.
            Это конечная мера на $\R^n$, так как $g \in L^1$.
            \[\circlesign{=} \int\limits_{\R^n}\left(~\int\limits_{\R^n}|f(x - y)| \cdot 1\d \nu(y)\right)^p \d x \circlesign{\le}\]
            Теперь применим неравенство Гёльдера относительно данной меры $\nu$.
            \multline{
                \circlesign{\le} \int\limits_{\R^n}\left[\left(~\int\limits_{\R^n}|f(x - y)|^p\d\nu(y)\right)^{\frac{1}{p}}\left(~\int\limits_{\R^n}1^q\d\nu\right)^{\frac{1}{q}}\right]^p \d x=\\= \int\limits_{\R^n}\left(~\int\limits_{\R^n}|f(x - y)|^p|g(y)|\d y\right)\d x \cdot \|g\|_{L^1}^{\frac{p}{q}} \underset{\text{\cref{p-equals-1}}}{\le} \|f\|^p_{L^p}\cdot \|g\|_{L^1} \cdot \|g\|_{L^1}^{\frac{p}{q}}
            }
            Тем самым, $\|f * g\|^p_{L^p} \le \|g\|_{L^1}^{\frac{p}{q} + 1} \|f\|^p_{L^p}$ и $\|f * g\|_{L^p} \le \|f\|_{L^p}\|g\|_{L^1}$.
        }
    }
    \exercise[Неравенство Юнга]{
        Пусть $f \in L^s(\R^n), g \in L^t(\R^n)$, где $s, t > 1$.
        Предположим, что $\frac{1}{r} = \frac{1}{s} + \frac{1}{t} - 1$, и пусть $r \ge 1$.
        Тогда $\|f * g\|_{L^r} \le \|f\|_{L^s}\|g\|_{L^t}$.
    }
    \exercise{
        Если $\frac{1}{p} + \frac{1}{q} = 1$, и $1 < p, q < \infty$, то $\|f*g\|_{L^{\infty}} \le \|f\|_{L^p} \|g\|_{L^q}$, и при этом $f * g$ непрерывна и стремится к нулю на $\infty$ \comment{Что? Не верится, что свёртка двух произвольных функций непрерывна}.
    }
    \fact{\label{strange-fact}
        Пусть $g \in L^1(\R^n)$, $g \ge 0$, $\int\limits_{\R^n}g(x)\d x = 1$. Тогда для $f \in L^p(\R^n)$ при $1 \le p < \infty$:
        \[\|f * g - f\|^p_{L^p} \le \int\limits_{\R^n}\int\limits_{\R^n}|f(x - y) - f(x)|^p g(y)\d y \d x\]
        \provehere{
            $(f * g)(x) - f(x) = \int\limits_{\R^n}f(x - y)g(y)\d y - \int\limits_{\R^n}f(x)g(y)\d y = \int\limits_{\R^n}(f(x - y) - f(x))g(y)\d y$
            Возьмём модуль, возведём в степень $p$, и проинтегрируем:
            \[\int\limits_{\R^n}|(f * g)(x) - f(x)|^p \d x = \int\limits_{\R^n}\left(~\int\limits_{\R^n}{|f(x - y) - f(x)|g(y)}\d y\right)^p \d x\]
            Далее вводим меру $\nu(e) = \int\limits_{e}g(y)\d y$, и опять применяем неравенство Гёльдера к ${|f(x - y) - f(x)| \cdot 1}$ (применяем к выражению внутри скобок):
            \gather{\int\limits_{\R^n}\left(|f(x - y) - f(x)| \cdot 1\right)\d\nu(y) \le \left(~\int\limits_{\R^n}|f(x - y) - f(x)|^p\d\nu(y)\right)^{\frac{1}{p}} \cdot \underbrace{\left(~\int\limits_{\R^n}1^q\d\nu(y)\right)^{\frac{1}{q}}}_{1} = \\
            = \left(~\int\limits_{\R^n}|f(x - y) - f(x)|^p g(y)\d y\right)^{\frac{1}{p}}
            }
            Далее, подставляя внутрь скобок полученную оценку, получаем
            \[\int\limits_{\R^n}\left(\left(~\int\limits_{\R^n}{|f(x - y) - f(x)|^p g(y)}\d y\right)^{\frac{1}{p}}\right)^p \d x\]
        }
    }
    Будем обозначать пространство \emph{бесконечно дифференцируемых функций на $\R^n$ с компактным носителем} значком $\mathcal{D}(\R^n)$.
    \theorem{
        Пусть $u$ --- непрерывна, с компактным носителем, $f \in L^p(\R^n)$. Тогда $f * u$ непрерывна.
        Если $u \in \mathcal{D}(\R^n)$, то $f * u \in C^{\infty}(\R^n)$.
        \provehere{
            \[(f * u)(x) = \int\limits_{\R^n}f(y)u(x - y)\d y \circlesign{=} \]
            Проверим непрерывность в $x_0 \in \R^n$, рассмотрим $B_r(x_0)$.
            Пусть $S \coloneqq \supp u$.
            Можно считать, что интегрирование берётся по компакту $K \coloneqq \overline{B_r(x_0)} - S$.
            \[\underset{\text{при $x \in B_r(x_0)$}}{\circlesign{=}} \int\limits_{K}f(y)u(x - y)\d y\]
            Заметим, что для всякой последовательности $x_j \to x_0: u(x_j - y) \to u(x_0 - y)$, откуда при $x_j$, близких к $x_0$: $\exists C$: $|f(y)u(x_j - y)| \le C|f(y)|$, и можно применить теорему Лебега о мажоранте:
            \indentlemma{
                Пусть $K \subset \R^n$ --- компактное множество, $I = (a, b)\subset \R$ --- интервал, $v: K \times I \map \R$.
                Пусть $\exists \der{}{t}v(x, t)$, и она непрерывна на $(a, b)$ при всяком фиксированном $x \in K$.
                Также предположим наличие суммируемой мажоранты $w$ на $K$: для всех $t \in (a, b): \abs{\der{}{t}v(x, t)} \le w(x)$.

                Определим $\phi(t) \coloneqq \int\limits_{K}v(x, t)\d x$ (предполагаем, что $v(x, t)$ суммируема при всяком $t$). Тогда $\phi$ дифференцируема на $(a, b)$, и
                \[\phi'(t_0) = \int\limits_{K}\der{}{t}v(x, t)\d x\]
            }{
                Выберем последовательность $t_n \to t_0$, запишем разностное отношение
                $\frac{\phi(t_n) - \phi(t_0)}{t_n - t_0} = \int\limits_{K}\frac{v(x, t_n) - v(x, t_0)}{t_n - t_0}\d x \underset{n \to \infty}\Map\phi'(t_0)$.
                По формуле Лагранжа $\frac{v(x, t_n) - v(x, t_0)}{t_n - t_0} = \der{}{t}v(x, \xi)$ для некой $\xi \in (t_n, t_0)$.
                $\abs{\der{}{t}v(x, \xi)} \le w(x)$, значит, у подынтегральной функции есть мажоранта, и можно перейти к пределу.
            }
            Заметим, что $(f * u)(x) = \int\limits_{K}f(y)u(x - y)\d y$, и её действительно можно дифференцировать бесконечно много раз:
        \[(f * u)^{(k)}(x) = \int\limits_{K}f(y)u^{(k)}(x - y)\d y\qedhere\]
        %$u$ ограничена, значит, есть мажоранта
        }
    }
    \note{
        Пусть $\supp f = A, \supp g = B$. Тогда $\supp(f * g) \subset A + B$.
    }
    \definition[Аппроксимативная единица для $\R^n$]{
        Последовательность функций $g_j \in L^1(\R^n)$, $g_j \ge 0, \int\limits_{\R^n}g_j = 1, \forall \delta > 0: \int\limits_{|y| \ge \delta}g_j(y)\d y \underset{j \to \infty}\Map 0$
    }
    \theorem{
        Пусть $g_j$ --- аппроксимативная единица, $f \in L^p(\R^n)$.

        Тогда $f * g_j \underset{j \to \infty}{\overset{L^p}\Map} f$, то есть $\|f * g_j - f\|_{L^p} \underset{j \to \infty}\Map 0$.

        Если $f$ непрерывна с компактным носителем (достаточно потребовать равномерной непрерывности и ограниченности), то $f * g_j \rightrightarrows f$
        \provehere{
            При $1 \le p < \infty$: \multline{\|(f * g_j) - f\|^p_{L^p} \underset{\text{\cref{strange-fact}}}{\le} \int\limits_{\R^n}\int\limits_{\R^n}|f(x - y) - f(x)|^p g_j(y)\d x \d y = \int\limits_{\R^n}\int\limits_{\R^n}|f(x - y) - f(x)|^p\d x \cdot g_j(y)\d y =\\= \int\limits_{|y| < \delta}g_j(y)\int\limits_{\R^n}|f(x - y) - f(x)|^p\d x \d y + \int\limits_{|y| \ge \delta}g_j(y)\int\limits_{\R^n}\underbrace{|f(x - y) - f(x)|^p}_{\le 2^p\left(|f(x - y)|^p + |f(y)|^p\right)}\d x\d y}
            $2^p\int\limits_{\R^n}|f(x - y)|^p + |f(y)|^p\d x \le 2^{p + 1}\int\limits_{\R^n}|f(x)|^p\d x$, и это конечно.
            Значит, по определению аппроксимативной единицы второе слагаемое мало при больших $j$.

            Для первого слагаемого применим непрерывность сдвигов для $f \in L^p(\R^n), 1 \le p < \infty$: $\forall \eps > 0: \exists \delta$: $|y| < \delta \then \int\limits_{\R^n}|f(x - y) - f(x)|^p \d x < \eps$. Значит, оно тоже маленькое.

            Теперь проверим равномерную сходимость.
            Так как $f$ имеет компактный носитель, то она равномерно непрерывна: $\forall \eps > 0: \exists \delta > 0: |x_1 - x_2| < \delta \then |f(x_1) - f(x_2)| < \eps$. Пусть $K = \max\limits_{x \in \R^n}|f(x)|$.
            \gather{(f * g_j)(x) - g(x) = \int\limits_{\R^n}f(x - y)g_j(y)\d y - f(x) = \int\limits_{\R^n}\left(f(x - y) - f(x) \right)g_j(y)\d y\\
            \abs{(f * g_j)(x) - g(x) } \le \int\limits_{|y| < \delta}|f(x - y) - f(x)|g_j(y)\d y + \int\limits_{|y| \ge \delta}\left(|f(x - y)| + |f(x)|\right)g_j(y)\d y \le \eps + 2K\int\limits_{|y| \ge \delta}g_y(y)\d y \underset{j \to \infty}\Map 0}
        }
    }
    \fact{
        Пространство $\mathcal{D}(\R^n)$ плотно в $L^p(\R^n)$ для $1 \le p < +\infty$.
        \provehere{
            Построим специальную аппроксимативную единицу. \numbers{
                \item Пускай $\phi: \R \map \R$ --- бесконечно дифференцируемая неотрицательная функция с компактным носителем, $\phi = 0$ вне $(-1, 1)$.

                Например, $\phi(x) = \all{\exp\left(\frac{1}{x^2 - 1}\right),& |x| <1 \\ 0,&|x| \ge 1}$.
                \item Положим $a \coloneqq\int\limits_{\R}\phi \d x$. Положим $\psi \coloneqq \frac{\phi}{a}$. Это функция с единичным интегралом.
                \item $\Psi(x_1, \dots, x_n) \coloneqq \psi(x_1)\proddots \psi(x_n)$.
                Это функция с единичным интегралом, сосредоточенная в кубе со стороной 2 и центром в нуле.
                \item В качестве аппроксимативной единицы выберем $\Psi_j(x) = j^n\Psi\left(jx\right)$.
                Интеграл по-прежнему равен единице, так как якобиан скалярного умножения на $j$ в $\R^n$ равен $j^n$.
            }
            Теперь выберем $\eps > 0$, и функции $f \in L^p(\R^n)$ сопоставим $g$ с компактным носителем, такую, что $\|f - g\|_{L^p} < \eps$.
            При больших $j$:
            \[|g * \Psi_j - g\|_{L^p} < \eps\qedhere\]
        }
    }
    \newlection{6 декабря 2023 г.}

    \subsection{Слегка другой способ построения аппроксимативной единицы}
    Этот способ практически повторяет способ с предыдущей лекции, но тут не требуется уметь строить бесконечно дифференцируемую функцию с компактным носителем.

    Пусть $\Phi: \R^n \map \R^n$ --- суммируемая функция, отнормируем её так, что $\int\limits_{\R^n}\Phi(x) \d x = 1$.
    Теперь положим в качестве аппроксимативной единицы $\Phi_j(t) = j^n \Phi(j t)$.
    Интеграл по всему пространству $\Phi_j$ равен $1$, так как якобиан домножения на $j$ в $\R^n$ --- это $j^n$.
    Наконец,
    \[\int\limits_{|x| > \delta}\Phi_j(x)\d x = \int\limits_{|x| > \delta}j^n \Phi(j x)\d x = \int\limits_{|x| > j\delta}\Phi(x)\d x \underset{j \to \infty}\Map 0\]

    Также можно вместо дискретного параметра $j$ выбрать непрерывный параметр $t$.
    Пусть $t$ мало, и пусть $\phi_t = \frac{1}{t^n}\Phi\left(\frac{x}{t}\right)$ --- аппроксимативные единицы.

    Тогда для $g \in L^p(\R^n)$: $g * \phi_t \underset{t \to 0}{\overset{L^p}\Map} g$ --- разумеется правда, так как вместо обычной сходимости можно рассматривать сходимости по последовательностям, стремящимся к нулю.

    Пусть $K \subset \R^n$ --- компактное множество положительной меры ($\abs{K} > 0$).
    Можно также положить $\phi = \frac{\chi_K}{|K|}$, и $\phi_t(x) = \frac{1}{t}\phi\left(\frac{x}{t}\right)$.
    Это ещё один пример аппроксимативной единицы.


    \section{Преобразования меры при дифференцируемом отображении}
    Пусть $G \subset \R^n$ --- открыто, пусть $F: G \map \R^n$ --- непрерывно дифференцируемая инъекция, и пусть дифференциал невырожден везде в $G$.

    Пусть $e \subset G$ --- измеримое множество.
    Научимся вычислять $|F(e)|$.
    \theorem{
        $|F(e)| = \int\limits_{e}\abs{J_F(x)}\d x$, где $J_F(x)$ --- якобиан отображения $F$ в точке $x$.
    }
    Прежде чем приступить к доказательству данной теоремы, вспомним несколько вещей.

    Так, в условиях теоремы (можно рассмотреть исчерпывающую последовательность компактов для $G$, на них производная ограничена и $F$ липшицева) для всякого измеримого $e \subset G$: $F(e)$ измеримо по Лебегу, и $|e| = 0 \then |F(e)| = 0$.

    Рассмотрим некоторое расширение понятия меры. Пусть $(S, \Sigma)$ --- пространство с $\sigma$-алгеброй.
    \definition[Знакопеременная мера]{
        Счётно-аддитивная (необязательно положительная) функция $\nu: \Sigma \map \R$.
        Счётная аддитивность понимается в обычном смысле: для непересекающихся $e_j \in \Sigma: \nu\left(\bigsqcup\limits_{j = 1}^{\infty}e_j\right) = \sum\limits_{j = 1}^{\infty}\nu(e_j)$.

        Из определения сразу следует, что сходимость должна быть абсолютной: формула верна с любой перестановкой.
    }
    Иногда такую меру называет \emph{зарядом} --- если обычная мера является аналогом массы, распределённой по всему пространству, то знакопеременная --- аналог заряда, который сокращается при разных знаках.
    \examples[Знакопеременная мера]{
        \item Пусть $g \in L^1(\R^n)$. Можно определить $\nu(e) = \int\limits_{e}g(x)\d x$. Интеграл счётно-аддитивен, значит, $\nu$ --- знакопеременная мера.
        \item Есть и другие меры (например, $\delta_0$ --- мера всякого множества равна $1$, если оно содержит $0$, и $0$ иначе.)
    }
    Пусть $(S, \Sigma, \lambda)$ --- пространство с мерой $\lambda \ge 0$; предположим, что $\lambda$ --- $\sigma$-конечна.
    Пусть $\nu$ --- знакопеременная мера на $\Sigma$.
    \definition[$\nu$ абсолютно непрерывна относительно $\lambda$]{
        $\forall e \in \Sigma: \lambda(e) = 0 \then \nu(e) = 0$.
    }
    \intfact[Теорема Радона --- Никодима]{
        Следующие два условия эквивалентны.
        \bullets{
            \item $\nu$ абсолютно непрерывна относительно $\lambda$.
            \item $\exists g \in L^1(\lambda): \forall e \in \Sigma: \nu(e) = \int\limits_{e}g \d \lambda$.
        }
    }
    Это весьма фундаментальная теорема, и у неё довольно длинное непростое доказательство.
    Нам эту теорему докажут в курсе функционального анализа, так как там есть некий трюк с гильбертовыми пространствами, позволяющий существенно упростить доказательство.

    Если такая $g$ существует, то она называется \emph{плотностью} меры $\mu$ относительно меры $\lambda$.
    Также $g$ зовут \emph{производная Радона --- Никодима}, причины к чему мы увидим ниже.
    \note{
        Плотность абсолютно непрерывной меры $\nu$ единственна.
        \provehere{
            Если $g_1, g_2$ --- две различные плотности, то $\forall e: \nu(e) = \int\limits_{e}g_1 \d\lambda = \int\limits_{e}g_2\d\lambda$, и значит $\int\limits_{e}(g_1 - g_2)\d\lambda = 0$.
            Рассматривая положительные и отрицательные части этой разности, получим, что она равна нулю почти всюду, что и требовалось.
        }
    }
    \note{
        $\nu \ge 0 \iff $ плотность $g \ge 0$.
        \provehere{
            $g$ не может быть отрицательной на множестве положительной меры $e$, иначе бы было $\nu(e) < 0$
        }
    }
    Вернёмся к ситуации дифференцируемого отображения $F: G \map \R^n$.
    Определим $\nu(e) = |F(e)|$ --- образ меры Лебега $\lambda_n$ на $F(G)$ при отображении $F^{-1}$.
    $\nu$ --- мера на $G$, определённая на той же алгебре $\lambda_n$-измеримых подмножеств $\Sigma \subset 2^G$ (да?).

    Факт $|e| = 0 \then \nu(e) = 0$ как раз и говорит, что $\nu$ абсолютно непрерывна относительно меры Лебега $\lambda_n$ на $G$.

    Рассмотрим $U = F(G)$ --- открытое множество в $\R^n$.
    Пусть $K_j$ --- исчерпывающая последовательность компактов для $U$, и положим $L_j \coloneqq F^{-1}(K_j)$.
    Тогда $L_j$ --- исчерпывающая последовательность компактов для $G$, но нам важно даже не то, что они компактны, а то, что $\forall j: \nu(L_j) < \infty$.

    Теперь рассмотрим \emph{сужение меры} $\nu$ на $L_j$ --- это не совсем сужение в теоретико-множественном смысле, а просто мера, определённая на подмножествах $L_j$.
    Это сужение конечно, и тогда по теореме Радона --- Никодима $\exists g \in L^1(L_j)$, такая, что $\forall e \in L_j: \nu(e) = \int\limits_{e}g_j\d\lambda$.

    Теперь рассмотрим $g_k$ для $k > j$. $\nu(e) = \int\limits_{e}g_k\d\lambda$.
    Понятно, что $g_k\big|_{L_j}$ --- плотность меры $\nu$ на $L_j$, и получается, что эти плотности согласованы: из единственности плотности $g_k\big|_{L_j} = g_j$ почти всюду.
    Тогда они <<склеиваются>> в одну измеримую функцию $g: G \map \R$, для которой всё верно: $\forall e \in G: |F(e \cap L_j)| = \int\limits_{e \cap L_j}g \d\lambda$, и можно перейти к пределу.
    Не исключено, что полученная $g$ не суммируема, что естественно --- образ маленького множества даже при дифференцируемом отображении может очень сильно растянуться.
    \corollary[Вывод из теоремы Радона --- Никодима]{
        $\exists g: G \map \R_{\ge 0}$, измеримая по Лебегу, такая, что $|F(e)| = \int\limits_{e}g(x)\d x$.
    }
    Таким образом, такая функция $g$ найдётся, осталось её как-то найти, а точнее, доказать, что это $|J_F(x)|$.

    \question{Как искать плотность $h$ у (вообще говоря знакопеременной) меры $\nu$ на $G$, если известно, что такая плотность есть?}
    Известно, что $\nu(e) = \int\limits_{e}h(x)\d x$.
    Предположим, что данная функция $h$ локально суммируема: $\forall x \in G: \exists U \ni x: \int\limits_{U}|h(x)|\d x < \infty$.
    Функция, полученная из теоремы Радона --- Никодима, именно такая.

    Значит, плотность можно искать по кусочкам: $\forall e \in U \cap G: \nu(e) = \int\limits_{e}h(x)\chi_U(x)\d x$.
    \intfact[Теорема о дифференцировании]{
        Пусть $B_r$ --- шар (или куб) радиуса $r$ с центром в $0 \in \R^n$.
        Тогда в этих условиях ($h \in L^1(\R^n)$ --- локально суммируема, $\nu(e) = \int\limits_{e}h(x)\d x$) почти всюду $h(x) = \lim\limits_{r \to 0}\frac{1}{|B_r|}\nu(x + B_r)$.
    }
    В случае одномерного пространства $n = 1$ это оказывается теоремой Ньютона --- Лейбница.
    Если $h$ непрерывна в $x$, то доказательство работает то же самое, но оказывается, что это правда не только в случае непрерывной $h$.

    \theorem[Слабая теорема о дифференцировании]{
        $\forall h \in L^1(\R^n): \exists$ последовательность $r_n$: $r_n \underset{n \to \infty}\Map 0$, и почти всюду \[\frac{1}{\abs{B_{r_n}}}\int\limits_{x + B_{r_n}}h(y)\d y \underset{n \to \infty}\Map h(x)\]
        \provehere{
            Это, конечно, частный случай теоремы о дифференцировании, но зато доказывается проще.

            Построим аппроксимативную единицу по функции $\phi \coloneqq \frac{1}{|B_1|}\chi_{B_1}$.

            Она будет иметь вид $\phi_r(x) = \frac{1}{r^n}\phi\left(\frac{x}{r}\right) = \frac{1}{r^n|B_1|}\phi\left(\frac{x}{r}\right) = \frac{1}{|B_r|}\chi_{B_r}(x)$.
            Известно, что $h * \phi_r \overset{L^1(\R^n)}{\underset{r \to \infty}\Map} h$.

            Запишем \[h * \phi_r = \int\limits_{\R^n}h(y)\phi_r(x - y)\d y = \int\limits_{\R^n}h(y)\frac{1}{|B_r|}\chi_{B_r}(x - y)\d y = \frac{1}{|B_r|}\int\limits_{|x - y| < r}h(y)\d y\]
            Справа как раз стоит выражение, про которое мы хотим показать, что стремится к $h(x)$. Сходимость в $L^1$ означает сходимость по мере, а раз имеется сходимость по мере, то имеется последовательность $r_n$, стремящаяся к нулю, такая что $h * \phi_{r_n} \underset{n \to \infty}\Map h$ почти всюду.
        }
    }
    Осталось доказать, что всюду \[\frac{|F(x + B_r)|}{|B_r|} \underset{r \to 0}\Map |J_F(x)|\label{jacobian}\tag{$*$}\]
    Доказав это, мы получим, что так как левая часть почти всюду сходится к плотности $h$, и тогда окажется, что плотность как раз равна модулю якобиана.
    \lemma[Об искажении]{
        Пусть $H: U \map \R^n$, выберем $x_0 \in U$, пусть $H$ непрерывно дифференцируемо и $\d H(x_0, \_) = E$, тождественный оператор.
        Положим $y_0 \coloneqq H(x_0)$.
        Утверждается, что $\forall \eps \in (0, 1): \exists \delta > 0: r \in (0, \delta) \then B_{r(1 - \eps)}(y_0) \subset H(B_r(x_0)) \subset B_{r(1 + \eps)}(y_0)$.
        \provehere{
            Выберем маленький $u \in \R^n$. По определению дифференцируемости $H(x_0 + u) = y_0 + \underbrace{\d H(x_0, u)}_{u} + v(u)$, где $v(u) = o(|u|)$.

            Выберем $\eps \in (0, 1)$, по определению $o$-маленького: $\exists \delta > 0: |u| < \delta \then |v(u)| < \eps|u|$. Тогда, действительно, $r < \delta \then |H(x_0 + u) - y_0| \le |u| + |v(u)| \le (1 + \eps)|u| \le (1 + \eps)r$.
            Это доказывает правое включение.

            Для доказательства левого включения возьмём $H^{-1}$, определённое в некоторой окрестности $y_0$, причём $\d H^{-1}(y_0, \_) = E$.
            Здесь уже доказано $\forall \eps > 0: \exists \delta > 0: \rho < \delta \then H^{-1}(B_{\rho}(y_0)) \subset B_{\rho(1 + \eps)}(x_0)$.

            Применяя $H$ ко включению, получаем $B_{\rho}(y_0) \subset H\left(B_{\rho(1 + \eps)}\right)\left(x_0\right)$.
            Обозначив $r = \rho(1 + \eps)$, получаем искомое включение, так как $B_{r(1 - \eps)} \subset B_{\frac{r}{1 + \eps}}$.
        }
    }
    Приступим к доказательству \eqref{jacobian}.
    Пусть $A = \d F(x_0, \_)$.
    \[\frac{\nu(B_r(x_0))}{|B_r(x_0)|} = \frac{|F(B_r(x_0))|}{|B_r(x_0|)} = \frac{\abs{AA^{-1}F(B_r(x_0))}}{\abs{B_r(x_0)}} = |\det A|\frac{\abs{A^{-1}F(B_r(x_0))}}{|B_r(x_0)|} =\Big\|\phi \coloneqq A^{-1}F\Big\|= |J_F(x_0)|\frac{|\phi(B_r(x_0))|}{|B_r(x_0)|}\]
    Но раз $\det \d_\phi(x_0, \_) =E$, то по лемме об искажении и принципу двух полицейских ${\frac{|\phi(B_r(x_0))|}{|B_r(x_0)|} \underset{r \to 0}\Map 1}$:
    \[\frac{r^n (1 - \eps)^n v}{r^n v} = \frac{|B_{r(1 - \eps)}(y_0)|}{|B_r(x_0)|} \le \frac{|\phi(B_r(x_0))}{|B_r(x_0)|}\le \frac{|B_{r(1 + \eps)}(y_0)|}{|B_r(x_0)|} = \frac{r^n(1 + \eps)^n v}{r^n v}\]
    \newlection{9 декабря 2023 г.}


    \section{Мера Лебега на поверхностях}
    Пусть теперь $m \le n$, и для $U \subset \R^m$ имеется гладкое инъективное $f: U \map \R^n$ со всюду невырожденным дифференциалом.

    Тогда $\forall e \subset U: f(e)$ --- какая-то $m$-мерная поверхность в $\R^n$, и её $n$-мерная мера равна нулю, но есть же у поверхности какая-то площадь, и хочется уметь её вычислять.

    \subsection{Частный случай линейного $f$}
    Пусть $A: \R^m \map \R^n$ --- линейный оператор.
    Тогда $A(\R^m)$ --- линейное подпространство размерности $m$ в $\R^n$, и в нём есть какая-то своя $m$-мерная мера Лебега.

    Пусть $e \subset \R^m$ --- измеримо.
    Как вычислить $A(e)$?

    Выберем ортонормированные базисы $e_1, \dots, e_m$ --- в $\R^m$ и $g_1, \dots, g_m$ --- в $A(\R^m)$.
    Пусть $T$ --- матрица оператора в этих базисах: $T_{i,j} = \angles{A e_i, g_j}$.
    Тогда $\lambda(Ae) = |\det T|\cdot|e|$.

    В этой формуле есть тот недостаток, что при определении $T$ используется произвольно выбранный базис $g_1, \dots, g_m$ в $A(\R^m)$.
    От этой проблемы можно избавиться так: $(\det T)^2 = \det(T^t T)$.
    Оказывается, матрица $T^t T$ выглядит приятно:
    \[\left(T^t T\right)_{i,k} = \sum\limits_{j = 1}^{m}\angles{A e_i, g_j}\angles{A e_k, g_j}\]
    Так как $g_{j}$ --- ортонормированный базис, то выше написано скалярное произведение $\angles{Ae_i, A e_k}$.

    Обозначим эту матрицу за $\Gamma(A)$: $\Gamma(A)_{i,k} = \angles{A e_i, A e_k}$.
    Эта $\Gamma(A)$ называется \emph{матрицей Грама} для оператора $A$.
    В частности, $\det\Gamma(A)$ --- \emph{определитель Грама} для оператора $A$.

    Тем самым, для линейного $f = A: \lambda(Ae) = (\det \Gamma(A))^{\nicefrac12}|e|$.
    Несложно догадаться, что для нелинейного $f$ формула будет такая (хотя мы ещё не определили меру $f(e)$, но догадаться-то можно): \[\lambda(f(e)) = \int\limits_{e}(\det \Gamma(\d_f(x, \_)))^{\nicefrac12}\d x\]
    Самым главным вопросом является --- как определить меру Лебега на такой поверхности.

    \subsection{$p$-мера Хаусдорфа}
    Здесь произвольное $p > 0$.
    Нам понадобятся только случаи $p \in \N$, но теорию удобнее развивать для всех положительных $p$.
    Также всё это можно провернуть в произвольном метрическом пространстве.

    Пусть $e \subset \R^n$ --- любое (необязательно измеримое) подмножество.
    Пусть $\eps > 0$.
    Рассмотрим всевозможные не более, чем счётные, покрытия этого множества системой множеств $\{a_j\}$ ($e \subset \bigcup\limits_{j}a_j$), таких, что $\forall j: \diam(a_j) \le \eps$.
    Назовём любое такое покрытие $\eps$-покрытием.

    Положим $\mu_p(e, \eps) = \inf\sum\limits_{j}\left(\frac{\diam a_j}2\right)^p$, где инфимум берётся по всем таким покрытиям $\{a_j\}$.
    Двойка в знаменателе стоит <<по традиции>>, чтобы $\mu_p$ было больше похоже на меру Лебега.

    \fact{
        $\eps_1 < \eps_2 \then \mu_p(e, \eps_1) \ge \mu_p(e, \eps_2)$.
        \provehere{
            Все покрытия диаметра не более $\eps_2$ являются и покрытиями диаметра не более $\eps_1$.
        }
    }
    Раз так, то $\exists \lim\limits_{\eps \to 0}\mu_p(e, \eps) = \sup\limits_{\eps > 0}\mu_p(e, \eps) \bydef \mu_p^*(e)$ (где-то супремум конечен, где-то бесконечен).
    \fact{
        $\mu_p^*$ --- предмера на $\R^n$.
        \provebullets{
            \item $\mu_p^*(\o) = 0$.
            \item $e_1 \subset e_2 \then \mu_p^*(e_1) \le \mu_p^*(e_2)$, так как при уменьшении множества $\eps$-покрытий становится больше, то есть $\forall \eps > 0: \mu_p(e_1, \eps) \le \mu_p(e_2, \eps)$.
            \item Осталась счётная полуаддитивность: пусть $e \subset \bigcup\limits_{k = 1}^{\infty}e_k$, тогда надо проверить, что ${\mu^*_p(e) \le \sum\limits_{k}\mu^*_p(e_k)}$.
            \provehere{
                Можно считать, что $\forall k: \mu^*_p(e_k) < \infty$, иначе доказывать нечего.

                Возьмём $\eps > 0, \delta > 0$, для каждого $k$ выберем $\eps$-покрытие $\{a_{k,j}\}_{j}$ множества $e_k$, такое, что $\sum\limits_{j}\left(\frac{\diam a_{k,j}}{2}\right)^p < \mu_p(e_k, \eps) + \frac{\delta}{2^k}$.
                Так как мера Хаусдорфа больше каждого из этих чисел, то они все конечны.

                Таким образом, $\{a_{k,j}\}_{k,j}$ --- $\eps$-покрытие $e$, откуда $\mu_p(e, \eps) \le \sum\limits_{j,k}\left(\frac{\diam a_{k,j}}{2}\right)^p \le \sum\limits_{k}\mu_p(e_k, \eps) + \delta$.
                Устремляя $\delta \to 0$, получаем $\mu_p(e, \eps) \le \sum\limits_{k}\mu_p(e_k, \eps)$.
            }
        }
    }
    Теперь можно применить теорему Лебега --- Каратеодори, и получить настоящую меру.
    Хочется узнать, какие множества будут измеримыми после данной процедуры.
    \fact{
        Если $\dist(e_1, e_2) > 0$, то $\mu^*_p(e_1\sqcup e_2) = \mu_p^*(e_1) + \mu_p^*(e_2)$.
        \provehere{
            В определении $\mu_p(e, \eps)$ можно ограничиться $\eps$-покрытиями $\{a_j\}$, такими, что $\forall j:a_j \cap e \ne \o$.

            В силу счётной полуаддитивности $\mu_p^*(e_1 \sqcup e_2) \le \mu_p^*(e_1) + \mu_p^*(e_2)$.

            Пусть $\delta = \dist(e_1, e_2)$, рассмотрим $\eps < \delta$.
            Пусть $\{a_j\}$ --- $\eps$-покрытие объединения, такое, что $\forall j: a_j \cap (e_1 \cup e_2) \ne \o$. Тогда для каждого $j: a_j \cap e_1 = \o$ либо $a_j \cap e_2 = \o$. Стало быть \multline{\mu_p(e_1 \sqcup e_2, \eps) = \sum\limits_{j}\left(\frac{\diam (a_j)}{2}\right)^p =\\= \sum\limits_{j: a_j \cap e_1 \ne \o}\left(\frac{\diam (a_j)}{2}\right)^p + \sum\limits_{j: a_j \cap e_2 \ne \o}\left(\frac{\diam (a_j)}{2}\right)^p \ge \mu_p(e_1, \eps) + \mu_p(e_2, \eps)\qedhere}
        }
    }
    \theorem{
        Пусть $(X, d)$ --- метрическое пространство, пусть $\eta$ --- предмера на $X$, причём $\forall e_1, e_2 \subset X: \dist(e_1, e_2) > 0 \then \eta(e_1 \sqcup e_2) = \eta(e_1) + \eta(e_2)$. Тогда все замкнутые множества в $X$ $\eta$-измеримы.
        \provehere{
            Пусть $F \subset X$ замкнуто, проверим, что $\forall a \in X: \eta(a) = \eta(a \cap F) + \eta(a \sm F)$.
            Из полуаддитивности достаточно проверять $\eta(a) \ge \eta(a \cap F) + \eta(a \sm F)$.

            Пусть $F_n \coloneqq \defset{x \in X}{\dist(x, F) < \frac{1}{n}}$. Из замкнутости $F$: $\bigcap\limits_{n \ge 1}F_n = F$.
            Пусть $C_n \coloneqq a \sm F_n$.

            Если $x \in a \cap F$, а $y \in C_n$, то разумеется $\dist(x, y) \ge \nicefrac1n$.
            Таким образом, заведомо $\eta(a) \ge \eta(a \cap F) + \eta(C_n)$.
            Так как $C_n$, возрастая, в объединении дают $a \sm F$, то хочется проверить, что $\lim\limits_{n \to \infty}\eta(C_n) = \eta(a \sm F)$.
            Запишем
            \[a \sm F = \bigcup\limits_{n = 1}^{\infty}C_n = C_n \sqcup\bigsqcup\limits_{j \ge n}(C_{j+1} \sm C_{j})\]
            Из счётной полуаддитивности $\eta(a \sm F) \le \eta(C_n) + \sum\limits_{j \ge n}\eta(C_{j+1}\sm C_j)$, но с другой стороны ${\eta(a \sm F)\ge \eta(C_n)}$.
            Значит, необходимо и достаточно доказать, что $\sum\limits_{j \ge n}\eta(C_{j+1}\sm C_j)\underset{j \to \infty}\Map 0$. Но это просто значит, что ряд $\sum\limits_{j = 1}^{\infty}\eta(C_{j+1}\sm C_j)$ сходится.

            Разобьём ряд на два: $\sum\limits_{j = 2k}\eta(C_{j+1}\sm C_j) + \sum\limits_{j = 2k+1}\eta(C_{j+1}\sm C_j)$.
            Теперь все подряд идущие слагаемые отстоят друг от друга на положительное расстояние: $\dist(C_{2k+1}\sm C_{2k}, C_{2k-1}\sm C_{2k-2}) > 0$.

            Стало быть, $\forall n: \sum\limits_{j = 2k}^{j \le n}\eta(C_{j+1}\sm C_j) \le \eta\left(\bigcup\limits_{j = 1}^{n}С_{j}\right) \le \eta(a)$.
            Аналогично сходится ряд ${\sum\limits_{j = 2k+1}\eta(С_{j+1}\sm С_j)}$.
        }
    }
    \corollary{
        Все борелевские множества $\mu_p^*$-измеримы.
    }
    \proposal{
        Пусть $E_1, E_2$ --- два евклидовых пространства (возможно, разных размерностей).
        Пусть $a \subset E_1, \Phi: a \map E_2$ --- $C$-липшицево отображение.

        Тогда $\mu_p^*(\Phi(a)) \le C^p(\mu_p^*(a))$.
        \provehere{
            Пусть $\{b_j\}$ --- $\eps$-покрытие $a$.
            Тогда $\diam(\Phi(b_j)) \le C\eps$. Иными словами, $\{\Phi(b_j)\}_j$ --- $C\eps$-покрытие множества $\Phi(a)$.
            Таким образом, $\mu_p(\Phi(a), C\eps) \le C^p\cdot \mu_p(a, \eps)$.
        }
    }
    \corollary{
        Мера Хаусдорфа не меняется при изометриях.
    }
    \theorem{
        Пусть $a \subset \R^n$ --- какое-то.
        Пусть $\exists p > 0: \mu_p^*(a) < \infty$. Тогда $\forall s > p: \mu_s^*(a) = 0$.
        \provehere{
            Выберем $\eps > 0$, так как $\mu_p^*(a) < \infty$, то $\exists \eps$-покрытие $\{b_j\}$ множества $a$, такое, что $\sum\limits_{j}\left(\frac{\diam b_j}2\right)^p \le \mu_p^*(a) + 1$.
            Тогда $\sum\limits_{j}\left(\frac{\diam b_j}{2}\right)^s = \sum\limits_{j}\left(\frac{\diam b_j}{2}\right)^p\underbrace{\left(\tfrac{\diam b_j}{2}\right)}_{\le \nicefrac\eps2}{}^{s - p}$.
            Тем самым, \[\sum\limits_{j}\left(\frac{\diam b_j}{2}\right)^s \le \left(\frac{\eps}2\right)^{s-p}(\mu^*_p(a) + 1) \underset{\eps \to 0}\Map 0\]
        }
    }
    \corollary{
        Пусть $\alpha \coloneqq \inf\defset{p > 0}{\mu^*_p(a) < \infty}$. Тогда $\mu_s^*(a) = \all{0, & s > \alpha \\ +\infty, & s < \alpha\\\text{что-то},&s = \alpha}$.
    }
    \definition[Хаусдорфова размерность $a\subset \R^n$]{
        Вот это число $\alpha$, отвечающее $a$.
        Обозначается $\dim_H(a)$.
    }
    \intfact{Хаусдорфова размерность стандартного Канторова множества равна $\frac{\log 2}{\log 3}$.}
    \theorem{
        $\dim_H(\R^n) = n$.
        \provehere{
            Пусть $K = [0, 1]^n$ --- куб в $\R^n$.
            Докажем, что $\mu_n^*(K) \in (0, \infty)$ (или что $\mu_n(K) \in (0, \infty)$, куб --- борелевское множество, поэтому неважно).
            Так как счётное число кубов покрывает всё $\R^n$ (и при сдвиге мера не меняется), то легко показать, что размерность $\R^n$ такая же, как и размерность куба.

            Пусть $\{e_j\}$ --- произвольное $\eps$-покрытие куба $K$, и пусть $b_j$ --- шар радиуса $\diam(e_j)$, содержащий $e_j$.
            Тогда $\{b_j\}$ --- тоже покрытие, откуда $\sum\limits_{j}\abs{b_j} \ge |K|$.
            $n$-мерная мера Лебега шара $B_r \subset \R^n$ пропорциональна $r^n$.
            Для конкретики положим, что она равна $c r^n$.
            \[1 = |K|\le c\sum\limits_{j}\left(\diam(e_j)\right)^n = 2^n c\cdot \sum\limits_{j}\left(\frac{\diam(e_j)}2\right)^n\]
            В другую сторону пойдём так: разобьём куб $K$ на диадические кубы ранга $s$, пусть $K_i$ --- диадические кубы ранга $s$, содержащиеся в $K$.
            Тогда $\left\{\overline{K}_i\right\}_i$ образуют $\eps$-покрытие $K$ при $\eps = 2^{-s}\sqrt{n}$.
            Отсюда получаем оценку
            \[\mu_n^*\left(K, 2^{-s}\sqrt{n}\right) \le \sum\limits_{i}\left(\frac{\diam K_i}2\right)^n = 2^{sn} \cdot \left(\frac{2^{-s}\sqrt{n}}2\right)^n = \left(\frac{\sqrt{n}}{2}\right)^n\qedhere\]
        }
    }
    Размерность $m$-мерного подпространства в $\R^n$ равна $m$, например, потому что оно изометрично $\R^m$.

    Обозначим за $\mu_p$ результат продолжения предмеры $\mu_p^*$ по теореме Лебега --- Каратеодори.
    \fact{
        $\exists C_{(m)}: C_{(m)}\mu_n = \lambda_n$.
        \provehere{
            $\mu_n$ не меняется при изометриях, в частности, при сдвиге.
            Значит, по теореме единственности, они пропорциональны.
        }
    }
    Пусть $m \le n$ (нас интересует на самом деле случай $m < n$, если равно, то всё и так известно), посмотрим на $\mu_m$.

    Понятно, что $C_{(m)}\mu_m$ совпадает с $m$-мерной мерой Лебега на всех $m$-мерных подпространствах в $\R^n$.
    Обозначим её за $\lambda_m$, что имеет смысл, так как оно везде совпадает с $\lambda_m$, где $\lambda_m$ определено.

    Теперь научимся вычислять $\lambda_m(f(e))$, где $e \subset \R^m, f: \underbrace{U}_{\supset e} \map \R^n$ --- гладкая инъекция.
    \newlection{13 декабря 2023 г.}
    \theorem{
        При сделанных предположениях $\lambda_m(f(e)) = \int\limits_{e}|\det \Gamma(\d_f(x, \_))|^{\nicefrac12}\d x$.
        \provebullets{
            \item Сначала обоснуем вообще, что $\dim_H(f(e)) = m$.
%
%        Пусть ситуация такова: $U \supset \overline{U}_1 \supset e$, где $U_1$ открыто.
%        Тогда $f$ липшицева на $\overline{U}_1$, откуда $\mu_m^*(f(e)) < \infty$.
%        Кроме этого, $f(e)$ --- борелевское множество, откуда $\mu_m(f(e)) < \infty$.
%
%    Если же ситуация не такова, то можно взять исчерпывающую последовательность компактов.
            \indent{
                \theorem[Регулярность меры Лебега]{
                    $\forall$ измеримого по Лебегу $e \subset \R^k$, $\forall \eps > 0: \exists \text{ открытое }U \supset e: \abs{U \sm e} < \eps$.
                    Кроме того, $|e| = \sup\limits_{K \subset e}|K|$, где $K$ --- компактны (запись другая, так как мера любого компактного множества конечна).
                    \provebullets{
                        \item Пусть $|e| < \infty$.
                        Так как оно измеримо, то $\exists$ параллелепипеды $\{P_j\}_j$, такие, что $\bigcup\limits_{j}P_j \supset e, \biggl|\bigcup\limits_{j}P_j\biggr| < |e| + \eps$.
                        Можно считать, что они открыты: параллелепипед с номером $j$ можно раздуть так, что $\tilde{P}_j \supset P_j$ открыт, и $\bigl|\tilde{P}_j\bigr| < \frac{\eps}{2^j}$.
                        Положим $U \coloneqq \bigcup\limits_{j = 1}^{\infty}\tilde{P}_j$, тогда $U$ открыто, и $|U| < |e| + 2\eps$.

                        Теперь если $|e| = \infty$, то воспользуемся $\sigma$-конечностью: пусть $\R^k = \bigcup\limits_{s = 1}^{\infty}B_s$, тогда подберём открытые $U_s \supset (e \cap B_s): |U_s \sm (e \cap B_s)| < \frac{\eps}{2^s}$. Объединение $U \coloneqq \bigcup\limits_{s}U_s$ подходит: $U \supset e$, и $|U \sm a| < \eps$.
                        \item Из предыдущего пункта можно найти подходящее замкнутое множество: пусть $a \coloneqq \R^k \sm e$, найдётся открытое $U \supset a, |U \sm a| < \eps$, тогда для замкнутого $F \coloneqq \R^k \sm U: |e \sm F| = |U \sm a| < \eps$.

                        Чтобы сделать $F$ компактным, возьмём пересечения $F_s \coloneqq F \cap \overline{B_s}$, где $\R^k = \bigcup\limits_{s = 1}^{\infty}B_s$.
                        Тогда легко видеть, что $|e| = \sup\limits_{s}|F_s|$.
                    }
                }
            }
            Возьмём исчерпывающую последовательность компактов $K_1 \subset K_2 \subset \cdots \subset \R^m$ ($K_i \subset \Int K_{i+1}$).

            Рассмотрим измеримое $E \subset K_j$. $|E| = C_{(m)}\mu_m^*(E) < +\infty$.
            Тем самым, согласно регулярности, найдутся вложенные компакты $\Phi_1 \subset \Phi_2 \subset \cdots \subset E$, такие, что $|\Phi_j| \underset{j \to \infty}\Map |E|$.
            Положим $\Phi \coloneqq \bigcup\limits_{j = 1}^{\infty}\Phi_j$.

            А раз так, то $f(E) = f(E \sm \Phi) \cup \bigcup\limits_{j = 1}^{\infty}f(\Phi_j)$.
            Значит, $f(E)$ измеримо --- это объединение множества меры нуль $f(E \sm \Phi)$, и счётного числа компактов $f(\Phi_j)$.

            Получается, на измеримых подмножествах $K_j$ корректно определена мера $\rho \coloneqq \lambda_m(f)$ ($\rho(e) \coloneqq \lambda_m(f(e))$).

            \item Пусть $\rho(e) = \lambda_m(f(e))$ --- мера на $U$.
            \indentlemma{
                $\rho$ абсолютно непрерывна относительно меры Лебега $\lambda_m$ на $U$.
            }{
                Пусть $\lambda_m(e) = 0$.
                Тогда $\mu_m(e) = 0$, и так как $f$ --- локально липшицева, то $\lambda_m(f(e)) = 0$.
            }
%            $\rho(e) = C_{(m)}\mu_m^*(F(e))$.
%        Пусть $|e| = C_{(m)}\mu_m^*(e) = 0$, где $e \in K_j$.
%        Тогда из локальной липшицевости $\mu_m^*(F(e)) = 0$.

            По теореме Радона --- Никодима: $\exists g_j: K_j \map \R$ --- неотрицательные измеримые функции, такие, что
            \[\forall e \subset K_j: \rho(e) = \lambda_m(f(e)) = \int\limits_{e}g_j(x)\d x\label{density}\tag{$*$}\]
            Далее, как и раньше, $g_{j + 1}$ почти всюду совпадает с $g_j$, поэтому $\exists g: U \map \R$ --- искомая плотность меры.

            \item Осталось показать, что $g(x) = |\det \Gamma(\d_f(x, \_))|^{\nicefrac12}$, а это делается с помощью слабой теоремы о дифференцируемости.

            А именно, из слабой теоремы о дифференцировании $\exists r_n \underset{n \to \infty}\Map 0$: $g(x) = \lim\limits_{n \to \infty}\frac{1}{|B_{r_n}(x)|}\int\limits_{B_{r_n}(x)}g(y) \d y$ почти всюду.

            Теперь достаточно показать $\forall x \in U: \lim\limits_{r \to 0}\frac{\lambda_m(f(B_r(x)))}{|B_r(x)|} = |\det \Gamma(\d_f(x, \_))|^{\nicefrac12}$.

            Пусть $x_0 \in U, y_0 \coloneqq f(x_0)$, тогда $\Pi \coloneqq \d_f(x_0, \R^m)$ --- $m$-мерная касательная плоскость к $f$ в точке $x_0$.
            Для удобства будем считать $y_0 = 0$ (заменим $f \rightsquigarrow f - y_0$).

            В предыдущем семестре была доказана теорема о том, что $\exists V \ni 0$, такая, что $A \coloneqq f(U) \cap V$ задаётся в виде $A = \defset{h(u) \coloneqq (u,\phi(u))}{u \in W}$, где $W \subset \Pi$ --- окрестность нуля, $\phi: \Pi \map \Pi^\perp$ --- некоторое непрерывно дифференцируемое отображение, причём $\d_\phi(x_0, \_) = \0$.

            Пусть $P: \R^n \map \Pi$ --- ортогональный проектор.
            Тогда $P$ и $h$ взаимно обратны ($P \circ h = \id$).

            Из непрерывной дифференцируемости $\phi$: $\forall \eps > 0: \exists \rho > 0: |u| \le \rho \then h$ --- липшицево с константой не выше $1 + \eps$:
            \[h(u_1) - h(u_2) \le |u_1 - u_2| + \underbrace{|\phi(u_1) - \phi(u_2)|}_{\le |\d_{\phi}(v, u_1 - u_2)|}\text{, где некая точка $v \in [u_1, u_2]$.}\]
            Устроим $\tilde{f}: U \map \Pi, \tilde{f}(x) = P f(x)$.
            Заметим, что меры $f(B_r(x_0))$ и $\tilde{f}(B_r(x_0))$ близки.
            В самом деле, из $1$-липшицевости $P$: всегда $\mu_m(P f(B_r(x_0))) \le \mu_m(f(B_r(x_0)))$, и при $r \le \rho: \mu_m(f(B_r(x_0))) = \mu_m(h(\tilde{f}(B_r(x_0)))) \le (1 + \eps)^m\mu_m(\tilde{f}(B_r(x_0)))$.
            Отсюда видно, что \[\lim\limits_{r \to 0}\frac{\mu_m(\tilde{f}(B_r(x_0)))}{|B_r(x_0)|} = \lim\limits_{r \to 0}\frac{\mu_m(f(B_r(x_0)))}{|B_r(x_0)|}\]
            Но так как $\tilde{f}$ --- отображение между двумя $m$-мерными евклидовыми пространствами, то можно записать $\lim\limits_{r \to 0}\frac{\mu_m(f(B_r(x_0)))}{|B_r(x_0)|} = \lim\limits_{r \to 0}\frac{\mu_m(\tilde{f}(B_r(x_0)))}{|B_r(x_0)|} = |J_{\tilde{f}}(x)|$.

            Осталось заметить, что образом $\d_f(x_0, \_)$ является $\Pi$.
            Значит, $P$ не меняет дифференциал, $\d_{\tilde{f}}(x_0, \_) = \d_f(x_0, \_)$.
        }
    }
    Рассмотрим оператор $A: \R^m \map \R^n$, где $m < n$.
    Пусть $e_j$ --- стандартный базис в $\R^m$, тогда $Ae_j$ --- столбцы матрицы $A$.

    Можно посчитать определитель Грама матрицы $A$: $\det(\angles{A e_i, A e_j})_{i,j}$ по формуле Бине --- Коши: $\det(A^t A) = \sum\limits_{I \subset [n], |I| = m}\det((A^t)_I)\det(A^I) = \sum\limits_{I \subset [n], |I| = m}\det(A^I)^2$.
    \examples[Меры поверхностей]{
        \item Пусть $m = 1$.
        Рассмотрим путь $\gamma: (a, b) \map \R^n$. $\gamma = \vect{\gamma_1 \\ \vdots \\ \gamma_n}$.
        Тогда $\d_{\gamma}(x_0, \_) = \vect{\gamma_1'(x_0) \\ \vdots \\ \gamma_n'(x_0)}$, и $\abs{\Gamma(\d_{\gamma}(x_0, \_))} = \sqrt{|\gamma_1^2(x_0) + \cdots + \gamma_n^2(x_0)|}$.
        Если $\gamma$ инъективна, то получается формула длины пути $\int\limits_{a}^{b}\sqrt{|\gamma_1^2(x) + \cdots + \gamma_n^2(x)|}\d x$.
        \item Пусть $G \subset \R^2$, и отображение представляет собой график: $F(x, y) = (x, y, \phi(x, y))$, где $\phi: \R \times \R \map \R$.
        Здесь матрица Якоби $J_F(x, y) = \vect{1  & 0 \\ 0 & 1 \\ \partial_1\phi(x, y) & \partial_2\phi(x, y)}$, где $\partial_i$ --- производная по $i$-му аргументу.
        Тем самым, $\mu_2(F(e)) = \int\limits_{e}\sqrt{1 + (\partial_1 \phi)^2 + (\partial_2 \phi)^2}\d\lambda_2$.
    }
    \section{Некоторые конкретные интегралы}
    \numbers{
        \item Заинтересуемся сходимостью $\int\limits_{\R^n \sm B_1}|x|^{\alpha}\d x$.
        Сделаем полярную замену $r, \phi_1, \dots, \phi_{n - 1}$. Интеграл обращается в $\int\limits_{1}^{\infty}r^{n - 1}r^{\alpha}\d r \cdot \Psi(\phi_1, \dots, \phi_n)$.
        Отсюда видно, что так как $\Psi(\phi_1, \dots, \phi_n)$ интегрируемо, то интеграл сходится $\iff n - 1 + \alpha < -1 \iff \alpha < -n$.
        \item Вычислим $\int\limits_{0}^{\infty}e^{-x^2}\d x$.
        Разумеется, интеграл суммируем. Пусть $I \coloneqq \int\limits_{-\infty}^{\infty}e^{-x^2}\d x$.
        \[I^2 = \int\limits_{-\infty}^{\infty}\int\limits_{-\infty}^{\infty}e^{-x^2}e^{-y^2}\d x\d y \circlesign{=}\]
        Так как функция $e^{-x^2}e^{-y^2}$ суммируема на плоскости, то по теореме Фубини он равен двойному интегралу:
        \[\circlesign{=}\iint\limits_{\R^2}e^{-x^2 - y^2}\d x\d y = \int\limits_{0}^{2\pi}\int\limits_{0}^{\infty}e^{-r^2}r\d r\d \phi = \left\|t = r^2\right\| = 2\pi \cdot \frac{1}{2}\int\limits_{0}^{\infty}e^{-t}\d t = \pi\]
        Тем самым, $I = \sqrt{\pi}$, и из чётности функции $e^{-x^2}$:
        \encircle{\int\limits_{0}^{\infty}e^{-x^2}\d x = \frac{\sqrt{\pi}}{2}}
    }


    \chapter{Элементы общей теории меры}
    Пусть $(X, \Sigma)$ --- пространство с $\sigma$-алгеброй, на которой мы будем рассматривать разные меры.

    Так как у этих мер могут быть разные множества меры нуль, то при пополнении одной меры могут появиться новые элементы $\Sigma$, на которые непонятно, как продолжать другую.
    Поэтому здесь никаких предположений о полноте мер скорее не будет.

    Здесь \emph{мера} --- счётно-аддитивная и, вообще говоря, комплексная функция $\nu: \Sigma \map \C$.
    Вспомним, что счётная аддитивность означает $\nu\left(\bigsqcup\limits_{j}e_j\right) = \sum\limits_{j}\nu(e_j)$.

    В силу технических соображений мы запретим мере принимать бесконечные значения, хотя на самом деле их можно и разрешить.

    Пусть $\nu = \alpha + i\beta$, где $\alpha, \beta$ --- вещественные меры ($\alpha = \Re(\nu), \beta = \Im(\nu)$).
    Понятно, что $\nu$ счётно-аддитивна $\iff \alpha, \beta$ каждая счётно-аддитивна.

    \proposal{
        Для всяких непересекающихся $e_j$: ряд $\sum\limits_{j}\nu(e_j)$ сходится абсолютно.
        \provehere{
            По определению сумма не зависит от перестановок слагаемых.
        }
    }
    \theorem{\label{bounded}
    Множество значений любой комплексной (\textbf{конечной}) меры ограничено.
    \provehere{
        В пределах данного доказательства назовём множество $a \in \Sigma$ <<плохим>>, если $\sup\defset{|\nu(b)|}{b \subset a, b \in \Sigma} = +\infty$.
        Пусть наше пространство --- $(X, \Sigma)$.
        \indentlemma{
            Если $e$ --- плохое множество, и $e = e_1 \sqcup e_2$, то хотя бы одно из $e_1$ и $e_2$ --- плохое.
        }{
            От противного: если оба хорошие, то и $e$ хорошее, так как $\forall b \subset e: b = (b \cap e_1) \cup (b \cap e_2)$, и меры $(b \cap e_1)$ и $(b \cap e_2)$ ограничены.
        }
        \indentlemma{
            Если $e$ --- плохое множество, то $\exists F, G: F \sqcup G = e$, такие, что $|\nu(F)|, |\nu(G)| \ge 1$.
        }{
            Так как $e$ --- плохое, то $\exists F: |\nu(F)| \ge |\nu(e)| + 1$. Но тогда для $G \coloneqq e \sm F: |\nu(G)| = |\nu(e) - \nu(F)| \ge |\nu(F)| - |\nu(e)| \ge 1$.
        }
        От противного: пусть $X$ --- плохое.
        Тогда $\exists a, b: a \sqcup b = X$, где $|\nu(a)|, |\nu(b)| \ge 1$.
        Одно из них плохое, обозначим его через $F_1$, а второе обозначим $G_1$.
        Применяя ту же самую конструкцию к $F_1$, получим $F_1 = F_2 \sqcup G_2$, где $|\nu(F_2)|, |\nu(G_2)| \ge 1$, причём $F_2$ --- плохое.

        И так далее, в итоге $X = \bigsqcup\limits_{j = 1}^{\infty}G_j \sqcup \bigcap\limits_{j = 1}^{\infty}F_j$.

        Тогда $\nu\left(\bigsqcup\limits_{j = 1}^{\infty}G_j\right) = \sum\limits_{j = 1}^{\infty}\nu(G_j)$, ряд не сходится (члены не стремятся к нулю, например), противоречие.
    }
    }
    \newlection{20 декабря 2023 г.}
    \section{Разложение Хана}
    Сужение меры можно определить так $\mu\big|_E(a) \bydef \mu(a \cap E)$.

    Иногда сужение определяют немного по-другому: $\mu\big|_E(a)$ --- мера, заданная как $\mu$, но теоретико-множественно суженная на $\defset{b \in \Sigma}{b \subset E}$.

    Разницы особой нет (конечно, предполагается, что $E \in \Sigma$), можно считать и так, и так: для множеств $a \subset E$ оба определения идентичны.
    \theorem[Хан]{
        Пусть $\mu$ --- \textbf{конечная} вещественная мера на $(X, \Sigma)$.
        Тогда $\exists E, F \in \Sigma: E \sqcup F = X$, такие, что $\mu\big|_E \ge 0, \mu\big|_F \le 0$. Такое $E$, что $\mu\big|_E \ge 0$ называется \emph{множеством положительности} (аналогично, $F$ --- \emph{множество отрицательности}).
        \provehere{
            $\defset{\mu(b)}{b \in \Sigma}$ ограничено~(\cref{bounded}), пусть $M = \sup\limits_{b \in \Sigma}\mu(b)$. Так как $\mu(\o) = 0$, то $M \ge 0$.
            Рассмотрим два случая.
            \bullets{
                \item $M = 0$. Тогда $\mu \le 0$, и $F \coloneqq X, E \coloneqq \o$ подходят.
                \item $M > 0$. По определению супремума $\exists b_k \in \Sigma: M - \frac{1}{2^k} \le \mu(b_k) \le M$.
                \indentlemma{
                    Пусть $b \subset X$, $\mu(b) = M - \eps$. Если измеримое $e \subset b$, то $\mu(e) \ge -\eps$.
                }{
                    От противного: пусть $\exists e \subset b_k: \mu(e) < -\eps$. Тогда мера $b_k \sm e$ больше супремума $M$.
                }
                Положим $\overline{b_k} = \bigcup\limits_{n \ge k}b_k$.
                Оценим $\mu\left(\overline{b_k}\right)$ снизу:
                \gather{\mu\left(b_k \cup b_{k+1} \cup \cdots \cup b_n\right) = \mu(b_k) + \mu(b_{k+1} \sm b_k) + \mu\left(b_{k+2}\sm(b_k \cup b_{k+1})\right) + \cdots + \mu(b_{n} \sm (\cdots))\ge\\\ge \left(M - \frac{1}{2^k}\right) - \frac{1}{2^{k+1}} - \frac{1}{2^{k+2}} - \cdots - \frac{1}{2^n} \ge M - \frac{1}{2^{k-1}}}
                Переходя к пределу в силу счётной аддитивности, получаем $\mu\left(b_k \cup b_{k+1} \cup \cdots\right) \ge M - \frac{1}{2^{k-1}}$.

                Теперь заметим, что $\overline{b_1} \supset \overline{b_2} \supset \cdots$ Положим $E \coloneqq \bigcap\limits_{k = 1}^{\infty}\overline{b_k}$.
                В силу монотонной непрерывности $\mu(E) = \lim\limits_{k \to \infty}\mu(\overline{b_k}) = M$ (формально, хотя бы $M$, но так как $M$ --- супремум всех мер, то больше быть не может).

                Теперь в силу леммы все подмножества $E$ имеют неотрицательную меру. Положим $F \coloneqq X \sm E$, все подмножества $F$ имеют неположительную меру (от противного: если $f \subset F$ имеет положительную меру, то $\mu(E \cup f) > M$).\qedhere
            }
        }
    }
    Такие $E, F \subset X$ из теоремы --- \emph{разложение Хана}.
    Оно единственно с точностью до множества меры нуль --- если $A \subset E$ имеет меру нуль, то все его измеримые подмножества тоже имеют меру нуль, и $(E \sm A, F \cup A)$ --- тоже разложение Хана.

    Теперь можно определить \emph{положительную} и \emph{отрицательную части меры} $\mu^+(a) \bydef \mu(a \cap E)$ и $\mu^-(a) \bydef -\mu(a \cap F)$.
    Тогда $\mu(a) = \mu^{+}(a) - \mu^-(a)$, тем самым любая конечная вещественная мера есть разность двух неотрицательных.

    \definition[Модуль меры]{
        $|\mu|(a) \bydef \mu^+(a) + \mu^-(a)$.
    }
    Введём естественный частичный порядок на мерах: $\mu \le \nu \iff \forall e \in \Sigma: \mu(e) \le \nu(e)$.
    \proposal{
        $\mu^+$ есть наименьшая из неотрицательных мер $\nu: \nu \ge \mu$.
        \provehere{
            Пусть $E, F$ --- разложение Хана для $\mu$.
            Пусть неотрицательная $\nu \ge \mu$. Тогда $\nu(a) = \nu(a \cap E) + \nu(a \cap F) \ge \mu(a \cap E) = \mu^+(a)$.
        }
    }
    \note{
        $\mu^- = (-\mu)^+$.
    }
    \note{
        Пусть $g: X \map \R, g(x) = \all{1,&x \in E \\ -1,&x \in F}$. Тогда $\mu(e) = \int\limits_{e}g\d\abs{\mu}$.
        \provehere{
            \[\int\limits_{e}g\d|\mu| = \int\limits_{e \cap E}\d|\mu| - \int\limits_{e \cap F}\d|\mu| = \int\limits_{e \cap E}\d\mu^+ - \int\limits_{e \cap F}\d\mu^- = \mu^+(e) - \mu^-(e) = \mu(e)\qedhere\]
        }
    }
    \section{Интеграл комплексных функций}
    Пусть $(X, \Sigma)$ --- пространство с $\sigma$-алгеброй, и $\rho$ --- неотрицательная счётно-аддитивная мера.

    Пусть $\phi: X \map \C$ является $\Sigma$-измеримым (прообраз любого борелевского множества измерим).
    Разложим $\phi = \alpha + i\beta$, где $\alpha, \beta$ --- вещественные измеримые функции.

    Определим $\int\limits_{a}\phi \d\rho \bydef \int\limits_{a}\alpha\d\rho + i\int\limits_{a}\beta\d\rho$.
    Данный интеграл обладает всеми свойствами, которые от него ожидаются --- аддитивность, $\C$-линейность, счётная аддитивность по множеству, счётная аддитивность по функции.

    Основную оценку интеграла удобно доказывать так:
    \gather{\exists \theta \in \R: e^{i\theta}\int\limits_{a}\phi\d\rho \in \R_{\ge 0} \\ \abs{\int\limits_{a}\phi\d\rho} = \Re\int\limits_{a}e^{i\theta}\phi\d\rho = \abs{\int\limits_{a}\Re(e^{i\theta}\phi)\d\rho} \le \int\limits_{a}|e^{i\theta}\phi|\d\rho = \int\limits_{a}|\phi|\d\rho}

    Многие комплексные меры (как счётно-аддитивные функции множеств) как раз происходят из таких интегралов.

    Пускай $\mu$ --- комплексная мера на $\Sigma$.
    \definition[Полная вариация $\mu$ на $a \in \Sigma$]{$|\mu|(a) = \sup\defset{\sum\limits_{j = 1}^{N}|\mu(e_j)|}{e_1\sqcup \dots\sqcup e_N = a}$}
    \theorem{
        $|\mu|$ есть неотрицательная конечная счётно-аддитивная мера на $\Sigma$.
        \provehere{См. (\cref{abs-mes-prop})}
    }
    \note{
        Если $\mu$ вещественна, то $|\mu|(a) = \mu^+(a) + \mu^-(a)$, где $|\mu|$ понимается, как полная вариация.

        Иными словами, не зря модуль меры и её полную вариацию обозначают одинаково.
        \provehere{
            Если $a \in \Sigma$, и $e_1 \sqcup \cdots \sqcup e_n = a$, то $\sum\limits_{j =1}^{n}|\mu(e_j)| \le \sum\limits_{j = 1}^{n}(\mu^+(e_j) + \mu^-(e_j)) = \mu^+(a) + \mu^-(a)$.

            Обратно, разобьём $a =  (a \cap E) \cup (a \cap F)$ (где $E,F$ --- разложение Хана). Тогда $|\mu|(a) \ge |\mu(a \cap E)| + |\mu(a \cap F)| = \mu^+(a) + \mu^-(a)$.
        }
    }
    \lemma{
        Пусть $\rho_1, \rho_2$ --- комплексные меры на $\Sigma$, $\alpha,\beta \in \C$. Тогда $\forall a \in \Sigma: |(\alpha \rho_1 + \beta \rho_2)|(a) \le |\alpha||\rho_1|(a) + |\beta||\rho_2|(a)$.
        \provehere{ Пусть $a = e_1 \sqcup \cdots \sqcup e_n$. Оценим
            \[\sum\limits_{j = 1}^{n}|(\alpha \rho_1 + \beta \rho_2)(e_j)| \le |\alpha|\sum\limits_{j = 1}^{n}|\rho_1(e_j)| + |\beta|\sum\limits_{j = 1}^{n}|\rho_2(e_j)| \le |\alpha||\rho_1|(a) + |\beta||\rho_2|(a)\]
            и перейдём к супремуму по всем разбиениям $a = e_1 \sqcup \cdots \sqcup e_n$.
        }
    }
    \corollary{
        Если $\rho$ --- комплексная мера, то $\forall a: |\rho|(a) < +\infty$.
        \provehere{
            Разложим $\rho = \rho_1 + i\rho_2$. $\rho_1, \rho_2$ --- вещественные меры, их модули принимают конечные значения.
        }
    }
    \proposal{
        Пусть $\nu$ --- неотрицательная (необязательно конечная) мера на $\Sigma$, пусть $g$ --- комплексная суммируемая (относительно $\nu$) функция.
        Определим $\mu(e) \coloneqq \int\limits_{e}g\d\nu$.

        Тогда $|\mu|(a) = \int\limits_{a}|g|\d\nu$.
        \provenumbers{
            \item[0.] Пусть $g \in L^1(\nu)$. Тогда $|\mu|(a) \le \int\limits_{a}|g|\d\nu$.
            Действительно, пусть $e_1 \sqcup \cdots \sqcup e_n = a$. Тогда $\sum\limits_{j = 1}^{n}|\mu(e_j)| = \sum\limits_{j = 1}^{n}\abs{\int\limits_{e_j}g\d\nu} \le \sum\limits_{j = 1}^{n}\int\limits_{e_j}|g|\d\nu = \int\limits_{a}|g|\d\nu$.
            \item Пусть $g = \sum\limits_{j = 1}^{k}c_j \chi_{e_j}$ --- простая функция, где мы считаем, что $e_j \in \Sigma$ попарно не пересекаются.

            Рассмотрим разбиение $a = (a \cap e_1) \sqcup \cdots \sqcup (a \cap e_k) \sqcup \left(a \sm \bigcup\limits_{j = 1}^{k}e_j\right)$.
            \[|\mu|(a) \ge \sum\limits_{j = 1}^{k}|\mu(a \cap e_j)| +\underbrace{ \abs{\mu\left(a \sm \bigcup\limits_{j = 1}^{k}e_j\right)}}_{= 0} = \sum\limits_{j = 1}^{k}\abs{\,\int\limits_{a \cap e_j}g\d\nu} = \sum\limits_{j = 1}^{k}|c_j|\nu(a \cap e_j) = \int\limits_{a}|g|\d\nu\]
            \item Пусть $g \in L^1(\nu)$ ($g: X \map \C$).
            Выберем $\eps > 0$. Тогда так как простые функции плотны в $L^1$, то $\exists h: X \map \C$ --- простая функция, такая, что $\int\limits_{X}|g - h|\d\nu < \eps$ (отдельно приблизим вещественную и мнимую части).

            Пусть $\mu_1(a) = \int\limits_{a}h\d\nu$, и положим $\mu_2 \coloneqq \mu - \mu_1$.
            \[\mu(a) = \int\limits_{a}h\d\nu + \int\limits_{a}(g - h)\d\nu = \mu_1(a) + \mu_2(a)\]
            \[\all{|\mu|(a) \le |\mu_1|(a) + |\mu_2|(a) \\ |\mu_1|(a) \le |\mu|(a) + |\mu_2|(a) \qquad \then \quad |\mu_1|(a) - |\mu_2|(a) \le |\mu|(a) \le |\mu_1|(a) + |\mu_2|(a)}\]
            Оценим $\mu_2(a)$:
            \[|\mu_2|(a) \le \int\limits_{a}|g - h|\d\nu \le \int\limits_{X}|g - h|\d\nu \le \eps\]
            Отсюда получается
            \[\int\limits_{a}|h|\d\nu - \eps \le |\mu|(a) \le \int\limits_{a}|h|\d\nu + \eps\]
            И наконец можно заменить $h$ на $g$, при этом мы ошибёмся не больше, чем на $\eps$.
            \[\int\limits_{a}|g|\d\nu - 2\eps \le |\mu|(a) \le \int\limits_{a}|g|\d\nu + 2\eps\]
            Устремим $\eps \to 0$.
        }
    }
    Теперь докажем теорему, сформулированную ранее.
    \theorem{
        $|\mu|$ есть неотрицательная конечная счётно-аддитивная мера на $\Sigma$.
        \provehere{\label{abs-mes-prop}
        Пусть $\mu$ --- комплексная мера на $\Sigma$, разложим $\mu = \mu_1 + i\mu_2$, где $\mu_1, \mu_2$ --- вещественные меры.
        Пусть $\nu = \mu_1^+ + \mu_2^+ + \mu_1^- + \mu_2^-$. Все четыре слагаемых --- абсолютно непрерывные меры относительно $\nu$, то есть по теореме Радона --- Никодима они представляются через плотность: $\exists g_{1,2}^\pm \in L^1(\nu): \mu_{1,2}^\pm(e) = \int\limits_{e}g_{1,2}^\pm\d\nu$.

        Тогда $|\mu|(e) = \int\limits_{e}|g|\d\nu$, и действительно получаем, что $|\mu|$ --- неотрицательная конечная счётно-аддитивная мера на $\Sigma$.
        }}

    \subsection{Интеграл по комплексной мере}
    Пусть $g: X \map \C$ измерима относительно комплексной меры $\mu = \mu_1 + i\mu_2$.

    Определим $\int\limits_{a}g\d\mu \bydef \int\limits_{a}g\d\mu_1^+-\int\limits_{a}g\d\mu_1^-+i\left(\int\limits_{a}g\d\mu_2^+-\int\limits_{a}g\d\mu_2^-\right)$

    В данном определении предполагается, что $g$ суммируема относительно всех четырёх мер.
    Оказывается, $g \in L^1(\mu_{1,2}^\pm) \iff g \in L^1(|\mu|)$, так как $\mu_{1,2}^\pm \le |\mu|$.
    \lemma[Линейность по мере]{
        Пусть $\rho, \sigma$ --- две комплексные меры на $\Sigma$, пусть $g$ суммируема относительно $|\rho|$ и $|\sigma|$.
        Утверждается, что \[\int\limits_{a}g\d\,(\rho+\sigma) = \int\limits_{a}g \d\rho + \int\limits_{a}g \d\sigma\]
        \provehere{
            Утверждается, что $\exists \lambda$ --- положительная мера на $\Sigma$, относительно которой $\rho,\sigma$ абсолютно непрерывны: $\exists u, v: X \map \C$, такие, что $\rho(e) = \int\limits_{e}u\d\lambda$ и $\sigma(e) = \int\limits_{e}v\d\lambda$.
            Например, в качестве $\lambda$ можно взять $\rho_1^+ + \rho_1^- + \rho_2^+ + \rho_2^- + \sigma_1^+ + \sigma_1^- + \sigma_2^+ + \sigma_2^-$.

            Разложим на вещественную и мнимую части $u = u_1 + i u_2, v =  v_1 + i v_2$.
            \[\int\limits_{a}g\d\rho = \int\limits_{a}g u_1^+\d\lambda - \int\limits_{a}g u_1^-\d\lambda + i\left(\int\limits_{a}g u_2^+\d\lambda - \int\limits_{a}g u_2^-\d\lambda\right) = \int\limits_{a}gu\d\lambda\]
            Аналогично $\int\limits_{a}g\d\sigma = \int\limits_{a}gv\d\lambda$.
            \[\int\limits_{a}g\d\rho + \int\limits_{a}g\d\sigma = \int\limits_{a}g(u + v)\d\lambda = \int\limits_{a}g\d\,(\rho+\sigma)\]
        }
    }

    \section{Разложение Лебега}
    Пусть $(X,\Sigma,\lambda)$ --- пространство с неотрицательной $\sigma$-конечной мерой $\lambda \ge 0$.

    Пусть $\rho$ --- комплексная мера на $\Sigma$.
    \definition[$\rho$ абсолютно непрерывна относительно $\lambda$]{
        $\lambda(e) = 0 \then \rho(e) = 0$.
    }
    \definition[$\rho$ сингулярна относительно $\lambda$]{
        $\exists a \in \Sigma: \lambda(a) = 0$ и $\forall e \subset X \sm a$: $|\rho|(e) = 0$.
    }
    \example{
        Стандартная мера Лебега $\lambda_1$ на $\R$ взаимно сингулярна точечной $\delta$-мере $\delta_0$.
    }
    \theorem[Лебег]{
        Для произвольной комплексной меры $\rho$ на $\Sigma$ существует и единственно представление $\rho = \rho_1 + \rho_2$, где $\rho_1$ абсолютно непрерывна относительно $\lambda$, а $\rho_2$ сингулярна относительно $\lambda$.
        \provehere{
            Положим $A \coloneqq \sup\defset{|\rho|(e)}{e \in \Sigma, \lambda(e) = 0}$. Тогда $\exists e_n \in \Sigma: |\rho|(e_n) > A - \frac{1}{n}$.

            Положим $\overline{e_n} = e_n \cup e_{n + 1} \cup \cdots$. Тогда $\overline{e_1} \supset \overline{e_2} \supset \cdots$, и положим $E = \bigcap\limits_{n = 1}^{\infty}\overline{e_n}$.

            В силу монотонной непрерывности $|\rho|(E) = \lim\limits_{n \to \infty}|\rho|(\overline{e_n}) = A$. Так как $\forall n: \lambda(\overline{e_n}) = 0$, то $\lambda(E) = 0$.

            Разложим $\rho_1(e) \coloneqq \rho(e \sm E), \rho_2(e) \coloneqq \rho(e \cap E)$. В таком представлении $\rho_2$ очевидно сингулярна относительно $\lambda$. Абсолютную непрерывность $\rho_1$ относительно $\lambda$ докажем от противного.

            Пусть $\exists b \in \Sigma, b \subset X \sm E: |\rho_1|(b) > 0, \lambda(b) = 0$.
            Тогда определим $E_1 = E \cup b$, и окажется, что $|\rho|(E_1) = |\rho|(E) + |\rho|(b) \ge A + |\rho_1|(b) > A$, противоречие.
        }
    }
    \example{
        Определим рекурсивно канторову лестницу $C: [0, 1] \map [0, 1]$.
        \[\begin{tikzpicture}
              \draw[->] (-0.5,0) -- (3.5,0) node[above]{$x$};
              \draw[->] (0,-0.5) -- (0,3.5) node[right]{$y$};
              \fill (0, 0) circle (1.5pt) node[below left] {$0$};
              \fill (1, 0) circle (1.5pt) node[below] {$\nicefrac13$};
              \fill (2, 0) circle (1.5pt) node[below] {$\nicefrac23$};
              \fill (3, 0) circle (1.5pt) node[below] {$1$};
              \fill (0,1.5) circle (1.5pt) node[left] {$\nicefrac12$};
              \fill (0,3) circle (1.5pt) node[left] {$1$};
              \newcommand{\cantor}[5]{ %l, r, d, u, depth
                  \pgfmathsetmacro{\depth}{\numexpr#5}

                  \draw ({(#1+2*#2)/3},{(#3+#4)/2}) -- ({(2*#1+#2)/3},{(#3+#4)/2});

                  \ifnum\depth<5
                  \cantor{#1}{(2*#1+#2)/3}{#3}{(#3+#4)/2}{#5+1}
                  \cantor{(#1+2*#2)/3}{#2}{(#3+#4)/2}{#4}{#5+1}
                  \else
                  \draw ({#1}, {#3}) -- ({#2}, {#4});
                  \fi
              }
              \cantor{0}{3}{0}{3}{0}
        \end{tikzpicture}\]
        Построив по данной функции меру $\mu(e) = \int\limits_{e}C(x)\d x$, мы получим меру, сосредоточенную на канторовом множестве меры нуль.
        Она сингулярна относительно стандартной меры Лебега на $\R$.
    }
\end{document}


