\documentclass[a4paper]{report}

\usepackage{../mathstemplate}

\date{IV семестр, весна 2024 г.}
\title{Алгебра. Неофициальный конспект}
\author{Лектор: Алексей Владимирович Степанов\\ Конспектировал Леонид Данилевич}

\begin{document}
    \shorthandoff{"}
    \maketitle
    \tableofcontents
    \newpage
    \setcounter{lection}{0}


    \chapter{Гомологическая алгебра}
    \newlection{12 февраля 2024 г.}


    \section{Абелевы категории}
    Напомним некоторые определения из предыдущей лекции.
    \definition[Предаддитивная категория $\cat{A}$]{
        $\forall A, B \in \cat{A}: \Mor_{\cat{A}}(A, B)$ образует абелеву группу, и везде, где определена, выполнена дистрибутивность: \[\alpha(\beta+\gamma) = \alpha\beta + \alpha\gamma \qquad (\beta + \gamma)\alpha = \beta\alpha + \gamma\alpha\]
    }
    \definition[Бипроизведение]{ Такая диаграмма, что
    % https://q.uiver.app/#q=WzAsMyxbMSwwLCJDIl0sWzAsMCwiQSJdLFsyLDAsIkIiXSxbMSwwLCJcXGlvdGFfMSIsMix7Im9mZnNldCI6MX1dLFswLDEsIlxccGlfMSIsMix7Im9mZnNldCI6MX1dLFswLDIsIlxccGlfMiIsMCx7Im9mZnNldCI6LTF9XSxbMiwwLCJcXGlvdGFfMiIsMCx7Im9mZnNldCI6LTF9XV0=
        \[\begin{tikzcd}[ampersand replacement=\&]
              A \& C \& B
              \arrow["{\iota_1}"', shift right, from=1-1, to=1-2]
              \arrow["{\pi_1}"', shift right, from=1-2, to=1-1]
              \arrow["{\pi_2}", shift left, from=1-2, to=1-3]
              \arrow["{\iota_2}", shift left, from=1-3, to=1-2]
        \end{tikzcd}\]
        \numbers{
            \item $\pi_1\iota_1 = \id_A$.
            \item $\pi_2\iota_2 = \id_B$.
            \item $ \iota_2 \pi_2+ \iota_1\pi_1  = \id_C$.
            \item $\pi_2\iota_1 = 0$.
            \item $\pi_1\iota_2 = 0$.
        }
    }
    \definition[Аддитивная категория]{
        Предаддитивная категория с финальным объектом и произведениями (любых двух объектов).
    }
    Эквивалентно, существуют инициальный объект и копроизведения, эквивалентно существуют нулевой объект и бипроизведения.
    \definition[Предабелева категория]{
        Аддитивная категория, в которой у всех морфизмов есть ядро и коядро.
    }
    \definition[(Ко)нормальный мономорфизм (эпиморфизм)]{
        Он является (ко)эквалайзером (какой-то, неважно какой, пары стрелок).
    }
    \definition[Абелева категория]{
        Предабелева категория, в которой все мономорфизмы нормальны.
    }
    Пусть $\cat C$ --- категория.
    Вспомним про категорию стрелок $\cat{Arr}\cat{C}$, в которой объекты --- стрелки из $\Mor(\cat C)$, множество морфизмов между $\phi, \psi$ --- это\[\Mor_{\cat{Arr}_{\cat{C}}}(\phi, \psi) = \defset{(\alpha, \beta)}{\alpha: \source(\phi) \map \source(\psi), \beta: \target(\phi) \map \target(\psi), \beta \phi = \psi \alpha}\]
    % https://q.uiver.app/#q=WzAsNCxbMCwwLCJcXGJ1bGxldCJdLFsxLDAsIlxcYnVsbGV0Il0sWzAsMSwiXFxidWxsZXQiXSxbMSwxLCJcXGJ1bGxldCJdLFswLDEsIlxccGhpIl0sWzIsMywiXFxwc2kiXSxbMCwyLCJcXGFscGhhIl0sWzEsMywiXFxiZXRhIl1d
    \[\begin{tikzcd}[ampersand replacement=\&]
          \bullet \& \bullet \\
          \bullet \& \bullet
          \arrow["\phi", from=1-1, to=1-2]
          \arrow["\psi", from=2-1, to=2-2]
          \arrow["\alpha", from=1-1, to=2-1]
          \arrow["\beta", from=1-2, to=2-2]
    \end{tikzcd}\]
    Далее будем обозначать за $\ker f$ ядро стрелки, как уравнитель стрелки и нуля, а за $\Ker f \coloneqq \source(\ker f)$ --- объект (в конкретных категориях типа $\modR{R}$ это докатегорное понятие ядра --- подмодуль без стрелки-вложения).
    \lemma{
        $\ker, \coker$ --- функторы $\cat{Arr} \cat A \map \cat{Arr} \cat A$.
        \provehere{
            Достаточно доказать для ядер, для коядер двойственно.

        Определим действие $\ker$ на морфизмах:
        % https://q.uiver.app/#q=WzAsNixbMSwwLCJBIl0sWzEsMSwiQSciXSxbMiwwLCJCIl0sWzIsMSwiQiciXSxbMCwwLCJcXEtlciBmIl0sWzAsMSwiXFxLZXIgZiciXSxbMCwyLCJmIl0sWzEsMywiZiciXSxbMCwxLCJcXGFscGhhIl0sWzIsMywiXFxiZXRhIl0sWzQsMCwiXFxrZXIgZiJdLFs1LDEsIlxca2VyIGYnIl0sWzQsNSwiXFxleGlzdHMhXFxwaGkiLDAseyJsYWJlbF9wb3NpdGlvbiI6NDAsInN0eWxlIjp7ImJvZHkiOnsibmFtZSI6ImRhc2hlZCJ9fX1dXQ==
            \[\begin{tikzcd}[ampersand replacement=\&]
            {\Ker f} \& A \& B \\
            {\Ker f'} \& {A'} \& {B'}
            \arrow["f", from=1-2, to=1-3]
            \arrow["{f'}", from=2-2, to=2-3]
            \arrow["\alpha", from=1-2, to=2-2]
            \arrow["\beta", from=1-3, to=2-3]
            \arrow["{\ker f}", from=1-1, to=1-2]
            \arrow["{\ker f'}", from=2-1, to=2-2]
            \arrow["{\exists!\phi}"{pos=0.4}, dashed, from=1-1, to=2-1]
            \end{tikzcd}\]
            $f \cdot \ker f = 0 \then \beta \cdot f \cdot \ker f = 0 \then f' \cdot \alpha \cdot \ker f = 0$, откуда по универсальному свойству ядра $\exists !\phi: \ker f' \cdot \phi = \alpha \cdot \ker f$.

            Положим $\ker(\alpha, \beta) = (\phi, \alpha)$.
            Далее несложно проверить, что данное определение сохраняет композицию и $\id$.
        }
    }
    \definition[Точный функтор]{Функтор, сохраняющий ядра и коядра.}
    \intfact[Теорема Фрейда --- Митчелла (Freyd --- Mitchell)] {\label{mitchell}
    Для любой малой абелевой категории $\cat{A}$: $\exists R \in \cat{Ring}$ (необязательно коммутативное кольцо с единицей) и строгий, полный, точный функтор $\cat{A} \map \modR{R}$.
    }
    \proposal{
        Для всякого морфизма $f: A \map B$ найдётся пунктирная стрелка, делающая диаграмму коммутативной.
% https://q.uiver.app/#q=WzAsNixbMCwwLCJcXEtlciBmIl0sWzEsMCwiQSJdLFsyLDAsIkIiXSxbMywwLCJcXENvS2VyIGYiXSxbMSwxLCJcXENvS2VyIFxca2VyIGYiXSxbMiwxLCJcXEtlciBcXGNva2VyIGYiXSxbMSwyLCJmIl0sWzAsMSwiXFxrZXIgZiJdLFsyLDMsIlxcY29rZXIgZiJdLFsxLDQsIlxcY29rZXJcXGtlciBmIiwyXSxbNSwyLCJcXGtlciBcXGNva2VyIGYiLDJdLFs0LDUsIlxcZXhpc3RzISIsMCx7InN0eWxlIjp7ImJvZHkiOnsibmFtZSI6ImRhc2hlZCJ9fX1dXQ==
        \[\begin{tikzcd}[ampersand replacement=\&]
        {\Ker f} \& A \& B \& {\CoKer f} \\
        \& {\CoKer \ker f} \& {\Ker \coker f}
        \arrow["f", from=1-2, to=1-3]
        \arrow["{\ker f}", from=1-1, to=1-2]
        \arrow["{\coker f}", from=1-3, to=1-4]
        \arrow["{\coker\ker f}"', from=1-2, to=2-2]
        \arrow["{\ker \coker f}"', from=2-3, to=1-3]
        \arrow["{\exists!}", dashed, from=2-2, to=2-3]
        \end{tikzcd}\]
        Более того, в абелевой категории эта стрелка --- изоморфизм.
        \provehere{
            Следует из эпи-моно разложения, доказанного на прошлой лекции, или из теоремы Митчелла.

            Само построение пунктирной стрелки получается из универсальных свойств, а доказательство того, что это --- изо --- непростое.
        }
    }
    \lemma{ Пусть $\cat C$ --- полная подкатегория в абелевой категории $\cat A$. Следующие условия равносильны
    \bullets{
        \item $\cat C$ является абелевой.
        \item \bullets{
            \item $0_{\cat A} \in \cat C$, здесь, как обычно, $0_{\cat A}$ --- нулевой объект категории $\cat A$.
            \item $\cat C$ содержит бипроизведение любых двух своих объектов.
            \item Ядра и коядра (взятые в $\cat A$) любых морфизмов из $\cat C$ лежат в $\cat C$.
        }
    }
    \provewthen{
        Очевидно.
    }{
        Чуть сложнее, доказывать не будем (и использовать тоже).
    }
    }
    \section{Компл$\acute{\text{е}}$ксы}
    Если противное не оговорено, то всё происходит в абелевой категории $\cat A$, большими буквами обозначены объекты данной категории, маленькими --- морфизмы.
    \definition[Компл$\acute{\text{е}}$кс]{
        Такая диаграмма, что $\forall k \in \Z: d_k \cdot d_{k+1} = 0$.
        % https://q.uiver.app/#q=WzAsNSxbMSwwLCJDX3tuKzF9Il0sWzIsMCwiQ19uIl0sWzMsMCwiQ197bi0xfSJdLFswLDAsIlxcY2RvdHMiXSxbNCwwLCJcXGNkb3RzIl0sWzMsMCwiZF97bisxfSJdLFswLDEsImRfbiJdLFsxLDIsImRfe24tMX0iXSxbMiw0LCJkX3tuLTJ9Il1d
        \[\begin{tikzcd}[ampersand replacement=\&]
              \cdots \& {C_{n+1}} \& {C_n} \& {C_{n-1}} \& \cdots
              \arrow["{d_{n+1}}", from=1-1, to=1-2]
              \arrow["{d_n}", from=1-2, to=1-3]
              \arrow["{d_{n-1}}", from=1-3, to=1-4]
              \arrow["{d_{n-2}}", from=1-4, to=1-5]
        \end{tikzcd}\]
    }
    Альтернативно, комплекс можно рассматривать, как функтор из категории $(\Z, \ge)$ (полученной из частично упорядоченного множества) в $\cat A$ (при котором образ композиции любых двух нетождественных морфизмов нулевой).

    Ещё один, следующий, взгляд на комплексы работает только для конкретной категории, уже вложенной в $R$-модули.
    \definition[Градуированный объект]{
        $C_{\bullet} = \bigoplus\limits_{n \in \Z}C_n$ с морфизмом $d: C_\bullet \map C_\bullet$, таким, что $d(C_n) \subset C_{n+p}$ для некоторой фиксированной \emph{степени объекта} $p$ (чаще всего она равна $\pm1$).
    }
    \definition[Дифференциальный модуль]{
        Градуированный объект $(C_\bullet, d)$ со свойством $d^2 = 0$.
    }
    \definition[Комплекс]{ Дифференциальный модуль степени $-1$. }
    При развороте стрелок получается дифференциальный модуль степени $+1$, также известный, как \emph{кокомплекс}:
    % https://q.uiver.app/#q=WzAsNSxbMCwwLCJcXGNkb3RzIl0sWzEsMCwiQ157bisxfSJdLFsyLDAsIkNee259Il0sWzMsMCwiQ157bi0xfSJdLFs0LDAsIlxcY2RvdHMiXSxbMSwwLCJkXntuKzJ9IiwyXSxbMiwxLCJkXntuKzF9IiwyXSxbMywyLCJkXm4iLDJdLFs0LDMsImRee24tMX0iLDJdXQ==
    \[\begin{tikzcd}[ampersand replacement=\&]
          \cdots \& {C^{n+1}} \& {C^{n}} \& {C^{n-1}} \& \cdots
          \arrow["{d^{n+2}}"', from=1-2, to=1-1]
          \arrow["{d^{n+1}}"', from=1-3, to=1-2]
          \arrow["{d^n}"', from=1-4, to=1-3]
          \arrow["{d^{n-1}}"', from=1-5, to=1-4]
    \end{tikzcd}\]
    \precaution{
    У кокомплекса несколько другая нумерация стрелок, но кокомплексы мы практически не будем использовать.
    }
    \definition[Сдвиг комплекса $(C_\bullet, d)$ на $p \in \Z$]{
        Комплекс $(C[p]_\bullet, d[p])$, где $C[p]_n = C_{n+p}$ и $d[p]_n = (-1)^{p}d_{n+p}$.
    }
\end{document}