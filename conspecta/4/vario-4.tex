\documentclass[a4paper]{article}

\usepackage{../mathstemplate}

\date{IV семестр, весна 2024 г.}
\title{Вариационное исчисление. Неофициальный конспект}
\author{Лектор: Роман Владимирович Романов \\ Конспектировал Леонид Данилевич}

\begin{document}
    \shorthandoff{"}
    \maketitle
    \tableofcontents
    \newpage
    \setcounter{lection}{0}
    \newlection{15 февраля 2023 г.}
    \section{Что мы будем изучать}
    Вариационное исчисление занимается поиском экстремумов в задаче, где число переменных бесконечно.

    Рассмотрим конечномерную ситуацию.
    Пусть имеется $f: M \map \R$, где $M$ --- какое-то многообразие.

    При поиске экстремумов формируеются следующие направления:
    \numbers{
        \item Необходимое условие: $(\grad f)(x) = 0$.
        \item Достаточное: форма $(D^2 f)(x)$ знакоопределён ($> < 0$).
        \item Поиск экстремумов сужения $f\big|_{N}$ на подмногообразие (метод множителей Лагранжа).
    }
    В случае вариационного исчисления вместо $M$ стоит некоторое бесконечномерное пространство, например, пространство функций.
    В основном мы будем заниматься аналогами 1 и 3 пунктов.

    Функция, которая в свою очередь задана на пространстве функций часто называется \emph{функционал}.
    Чтобы визуально различать <<обычные>> функции, и функционалы, образ точки $f$ под действием функционала $J$ будем обозначать $J[f]$.

    Пускай $X$ --- (пока произвольное) метрическое пространство, $J: X \map \R$ --- функция.
    \definition[$x \in X$ --- cтрогий локальный минимум]{
        $\exists \delta > 0: \forall y \in U_{\delta}(x): J[y] > J[x]$. Квадратные скобочки --- косметическое.
    }
    Аналогично определяются нестрогий минимум и максимумы.
    Также стоит вспомнить про существование глобальных строгих и нестрогих минимумов и максимумов.

    \example[Чего такого особенного в бесконечномерии?]{
        Пусть $X = \defset{f \in C[0, 1]}{f(0) = f(1) = 1}$, норма на $C[0, 1]$ определена формулой $\|f\| = \max\limits_{x \in [0, 1]}|f(x)|$.

        Пусть $J[f] \coloneqq \int\limits_{0}^{1}f^2(x)\d x$. Очевидно, $J$ непрерывен.

        Ясно, что $\forall f \in X: J[f] > 0$.
        С другой стороны, $\inf\limits_{f \in X}J[f] = 0$ --- можно рассматривать функции вида
        \[\begin{tikzpicture}
%            \draw[scale=1,domain={0}:{0.1},smooth,variable=\x,blue,line width=0.8pt] plot ({\x},{sqrt(2*\c*\x)});
            \draw[->] (-1,0) -- (3,0) node[right] {$x$};
            \draw[->] (0,-1) -- (0,3) node[above] {$y$};
            \fill (2,0) circle (1.5pt) node[below] {$1$};
            \fill (0,2) circle (1.5pt) node[left] {$1$};
            \fill (0,0) circle (1.5pt) node[below left] {$0$};
            \draw[blue, line width=0.8pt] (0,2) -- (0.1,0);
            \draw[blue, line width=0.8pt] (0.1,0) -- (1.9,0);
            \draw[blue, line width=0.8pt] (1.9,0) -- (2,2);
            \draw[dashed] (0,2) -- (2,2);
            \draw[dashed] (2,0) -- (2,2);
        \end{tikzpicture}\]

        С третьей стороны, $X$ замкнуто: равномерный предел равномерных непрерывен, и условия на значения на концах уважают предел.
        Получается, в данном случае теорема Кантора не работает. В чём дело?

        Оказывается, проблема в том, что нет компактности: в бесконечномерном пространстве замкнутое ограниченное множество необязательно компактно.
    }
    \subsection{Интегральные функционалы}
    В дальнейшем мы будем рассматривать не произвольные функционалы, а ограничимся некоторым их подмножеством.

    Пусть задано непрерывное $L: [a, b] \times \R^n \times \R^n \map \R$, положим $J[u] \coloneqq \int\limits_{a}^{b}L(t, u(t), \dot{u}(t))\d t$.
    Мы будем заниматься множеством $X = C^1[a, b] = C^1([a, b] \map \R^n)$ (далее не будем указывать область значений, ясно из контекста) и его замкнутыми подмножествами.

    Такие $J$ называются \emph{интегральные функционалы}.
    Мы их изучаем, так как на них возможна богатая теория, и вместе с тем, интегральные функционалы часто встречаются в приложениях.
    \examples{
        \item $X = \defset{u \in C^1[a, b]}{u(a) = u_a, u(b) = u_b}, J[u] = \int\limits_{a}^{b}\sqrt{1 + (u')^2}\d x$ --- функционал длин графиков кривых.
        \item $J = \int\limits_{a}^{b}(\frac{\dot u^2}{2} - V(u))\d x$, где $V$ --- заданная функция. В механике называется \emph{действием}.
    }
    Сначала убедимся, что они непрерывны.
    \note[О норме]{
        Для $f \in C^1[a, b]$: $\|f\| = \max\limits_{x \in [a, b]}|f(x)| + \max\limits_{x \in [a, b]}|f'(x)|$ --- очевидно норма. В дальнейшем мы всегда будем использовать такую норму для $C^1$.
    }
    \proposal{
        Пусть $X = C^1[a, b], L \in C([a, b] \times \R^n \times \R^n)$. Тогда интегральный функционал $J$ непрерывен на $X$.
        \provehere{
            Пусть $u, \tilde{u} \in X, \|u - \tilde{u}\| < \delta < 1$. \[\left|J[u] - J[\tilde{u}]\right| = \left|\int\limits_{a}^{b}L(x, \tilde{u}(x), \dot{\tilde{u}}(x)) - L(x, u(x), \dot{u}(x))\d x\right| \circlesign{\le}\]
            Заметим, что    $\|(x, \tilde{u}(x), \dot{\tilde{u}}(x)) - (x, u(x), \dot{u}(x))\|_{\R^{2n + 1}} < \delta$

            Рассмотрим $K = [a, b] \times \overline{B_{\|u\|_X + 1}} \times \overline{B_{\|u\|_{X} + 1}}$ --- компакт в $\R^{2n + 1}$.

            \[\circlesign{\le} \int\limits_{a}^{b}\omega_{L\big|_K}(\delta)\d x = (b - a)\omega_{L\big|_K}(\delta) \underset{\delta \to 0}\Map 0\]
            где $\omega$ --- модуль непрерывности. Он определён, так как $L\big|_K$ непрерывна на компакте.
        }
    }
    Пусть $X$ --- нормированное пространство (необязательно замкнутое), $J: X \map \R$.

    \definition[Производная функционала $J$ в точке $x$ по направлению $h \in X$]{
        \[\delta J[x, h] = \frac{\d}{\d t}\Big|_{t = 0}J[x + th]\]
        Иначе эту штуку называют \emph{вариация} $J$ по направлению $h$.
    }
    \properties[Вариация]{
    \item Однородность: $\delta J[x, ch] = c \cdot \delta J[x, h]$.
    \item Не следует ожидать аддитивность.
        Так, $\exists \delta J[x, h_1], \delta J[x, h_2]$ не влечёт существование $\delta J[x, h_1 + h_2]$, а если последнее и существует, то не обязано быть суммой.

        Примеры этого были в анализе, здесь бесконечномерной специфики нет.
        \item Как и в конечномерном анализе, в критической (экстремальной) точке вариация (коли $\exists$) должна обращаться в нуль.

        А именно, $x \in X$ --- локальный экстремум $J$, тогда $\forall h: \exists \delta J[x, h] \then \delta J[x, h] = 0$.
        \provehere{
            Сужение $\alpha(t) = J[x + th]$ тоже имеет локальный экстремум, значит, если производная в $t = 0$ есть, то нуль.
        }
    }

    \section{Формула первой вариации. Уравнение Эйлера --- Лагранжа}
    \subsection{Лемма Дюбуа-Реймона}
    \lemma[Дюбуа-Реймона]{\label{du-Bois-Reymond}
        Пускай $f \in C[a, b]$, и для всех $\omega \in C^1[a, b]$, таких, что $\omega(a) = \omega(b) = 0$, известно, что $\int\limits_{a}^{b}f \omega' = 0$.

        Тогда $f \equiv \const$.
        \provehere{
            Если бы $f$ сама была гладкой, то можно было бы интегрировать по частям. $\int f'\omega = 0 \then f' \equiv 0$ --- можно взять $\omega$, сосредоточенную там, где $f'$ одного знака.

            Мы надеемся, что $f$ --- константа, то есть равна своему среднему $\overline{f} \bydef \frac{1}{b - a}\int\limits_{a}^{b}f$.

            Проинтегрируем $f - \overline{f}$: $\omega(x) \coloneqq \int\limits_{a}^{x}\left(f(x') - \overline{f}\right)\d x'$. Понятно, что $\omega \in C^1$. Более того, несложно видеть, что $\omega(a) = \omega(b) = 0$.

            Подставим данную $\omega$ в посылку теоремы. \[0 = \int\limits_{a}^{b}f \omega' = \int\limits_{a}^{b}(f - \overline{f})\omega' = \int\limits_{a}^{b}(f - \overline{f})^2\d x\] Так как интеграл нуль, то получаем $f \equiv \overline{f}$.
        }
    }
    \subsection{Формула первой вариации}
    Опять $X = C^1[a, b]$, и функционал того же самого вида $J[u] = \int\limits_{a}^{b}L(t, u(t),\dot{u}(t))\d t$.
    \lemma[Формула первой вариации]{
        Пусть $L \in C^1([a, b] \times \R^n \times \R^n)$.
        Градиент $L$ по второму и третьему аргументам будем обозначать $\nabla_u L$ и $\nabla_{\dot{u}} L$ соответственно, это векторы из $\R^n$.

        Тогда производная $J$ в точке $u$ по направлению $h$ существует, и равна \[\int\limits_a^b\left[\Big\langle(\nabla_u L)(t, u(t), \dot{u}(t)), h(t)\Big\rangle + \angles{(\nabla_{\dot{u}} L)(t, u(t), \dot{u}(t)), \dot{h}(t)}\right]\d t\]
        \provehere{
        $J[u + \tau h] - J[u] = \int\limits_{a}^{b}\left[ L(t, u(t) + \tau h(t) , \dot{u}(t) + \tau \dot{h}(t)) - L(t, u(t), \dot{u}(t))\right]\d t$.

        Применяя формулу Лагранжа, получаем для некой $\tau_* = \tau_*(t) \in [0, \tau]$:
            \multline{J[u + \tau h] - J[u] = \tau\int\limits_{a}^{b} \biggl[\angles{ (\nabla_u L)(t, u(t) + \tau_* h(t), \dot{u}(t) + \tau_* \dot{h}(t)), h(t)} +\\+ \angles{(\nabla_{\dot{u}} L)(t, u(t) + \tau_* h(t), \dot{u}(t) + \tau_* \dot{h}(t)), \dot{h}(t)}\biggr]\d t}

            Поделив на $\tau$, получаем $\frac{J[u + \tau h] - J[u]}{\tau} = \int\limits_{a}^{b} \dots$ --- вот тот, что выше.

            Сперва разберёмся с первым слагаемым. Покажем, что \[\underbrace{\int\limits_{a}^{b}\angles{(\nabla_u L)(t, u(t) + \tau_* h(t), \dot{u}(t) + \tau_* \dot{h}(t)), h(t)}\d t}_{I} \underset{\tau \to 0}\Map \underbrace{\int\limits_{a}^{b}\angles{(\nabla_{u} L)(t, u(t), \dot{u}(t)), h(t)}\d t}_{\II}\]

            Модуль разности аргументов не превосходит $\tau_* \|h\|_{X}$. Отсюда $\|\nabla_u L(\dots) - \nabla_u L(\dots)\|_{\R^n} \le \omega_{L\big|_K}(\tau_* \| h\|_X)$, здесь $K \coloneqq [a, b] \times \overline{B_{\|u\| + \|h\|}} \times \overline{B_{\|u\| + \|h\|}}$. %Здесь, видимо, $\tau_* \le 1$, чтобы задать именно такой компакт.

            Значит, $|(I) - (\II)| \le \int\limits_{a}^{b}\omega_{L\big|_K}(\tau_* \|h\|)\d t \le (b - a)\omega_{L\big|_K}(\tau \|h\|)\d t \underset{\tau \to 0}\Map 0$.

            Таким образом, у первого слагаемого под интегралом --- естественный предел.
            Аналогично со вторым слагаемым, получаем утверждение леммы.
        }
    }
    \subsection{Уравнение Эйлера --- Лагранжа}
    Пусть $u \in X$ --- экстремум.
    Тогда $\forall h \in X: \delta J[u, h] = 0$

    Условие обнуления градиента --- некое уравнение на точку.
    Мы хотим уравнение на $u(t)$, избавимся от $h$.
    Подгоним под лемму Дюбуа-Реймона~(\cref{du-Bois-Reymond}).

    Введём $R(x) \coloneqq \int\limits_{a}^{x}(\nabla_u L)(t, u(t), \dot{u}(t))\d t$.
    Тогда $\delta J[x, h] = \int\limits_a^b \angles{\dot{R}(t), h(t)} + \angles{(\nabla_{\dot{u}} L)(t, u(t), \dot{u}(t)), \dot{h}(t)}\d t$
    Интегируя по частям, получим (поскольку $R(a) = 0$) $\angles{R(b), h(b)} + \int\limits_{a}^{b}\Bigl\langle\underbrace{(\nabla_{\dot{u}}L)(t, u(t), \dot{u}(t)) - R(t)}_{\xi(t)}, \dot{h}(t)\Bigr\rangle\d t$

    И это равно нулю $\forall h \in C^1[a, b]$.
    Рассмотрим $h$, обращающийся на концах в ноль: $h(a)=h(b)=0$.
    Теперь $\int\limits_{a}^{b}\angles{\xi(t), \dot{h}(t)}\d t = 0$, и мы покомпонентно можем применить лемму Дюбуа-Реймона, получая $\xi(t) = C \equiv \const$.
    Но $R(t) \in C^1$, значит, $\nabla_{\dot{u}}L(t, u(t), \dot{u}(t)) \in C^1$ тоже.

    Дифференцируя $\xi$, получаем уравнение: $\frac{\d}{\d t}(\nabla_{\dot{u}} L)(t, u(t), \dot{u}(t)) - (\nabla_u L)(t, u(t), \dot{u}(t)) = 0$.
    Оно называется \emph{уравнение Эйлера --- Лагранжа}, это основное уравнение вариационного исчисления.

    \note{
        В случае общего положения уравнение Эйлера --- Лагранжа --- дифференциальное второго порядка, что соответствует $u \in C^2$: при вычислении $\frac{\d}{\d t}(\nabla_{\dot{u}} L)(t, u(t), \dot{u}(t))$ появится в общем случае вторая производная $u$.
        Это, на самом деле, довольно общая ситуация: экстремаль <<регулярнее>>, чем произвольный элемент своего пространства.
    }
    \subsection{Случай свободных концов}
    Теперь рассмотрим совсем произвольную $h \in C^1$, и получим уравнение на вариацию \[0 = \delta J[u, h] = \angles{R(b), h(b)} + \int\limits_{a}^{b}\angles{C, \dot{h}(t)}\d t = \angles{R(b), h(b)} + \angles{C, h(b)} - \angles{C, h(a)}\]

    \numbers{
        \item Рассмотрим такую $h$, что $h(b) = 0, h(a) = C$.
        Для неё $\delta J[u, h] = -\|C\|^2$, значит, $\xi = C = 0$.

    Подставляя в определение $\xi$, получаем $R(a) = 0$, то есть $(\nabla_{\dot{u}} L)(a, u(a), \dot{u}(a)) = 0$.

    \item Теперь рассмотрим такую $h$, что $h(b) = R(b)$. В этом случае $\delta J[u, h] = \|R(b)\|^2 \then R(b) = 0$.
        Получили $(\nabla_{\dot{u}} L)(b, u(b), \dot{u}(b)) = 0$.
    }
    Итак, помимо уравнения Эйлера --- Лагранжа, мы получили два условия (но в разных точках) на уравнение второго порядка, можно надеяться, что хватит, чтобы найти решения (но это совсем не факт --- так, может существовать одно решение, а может их вовсе не быть, или быть бесконечно много).

    Подытожим в теорему.
    \theorem[Задача со свободными концами]{
        Пусть $L \in C^1([a, b] \times \R^n \times \R^n)$, пусть $X = C^1[a, b]$, пусть $u$ --- локальный экстремум $J$.

        Тогда
        \numbers{
            \item $\left(\nabla_{\dot{u}} L\right)(t, u(t), \dot{u}(t)) \in C^1[a, b]$.
            \item $\frac{\d }{\d t} \nabla_{\dot{u}} L = \nabla_{u} L$ --- уравнение Эйлера --- Лагранжа.
            \item $(\nabla_{\dot{u}} L)(a, u(a), \dot{u}(a)) = 0$
            \item $(\nabla_{\dot{u}} L)(b, u(b), \dot{u}(b)) = 0$
        }
    }
    \subsection{Случай фиксированных концов}
    Теперь обсудим, что происходит, если концы несвободны.

    Рассмотрим $X = \defset{f \in C^1[a, b]}{f(a) = f_a, f(b) = f_b}$.
    Это не подпространство (не имеет линейной структуры), нельзя определить производную по направлению.

    Функционал $J: X \map \R$ задан той же формулой.

    Какая здесь характеризация локальных экстремумов?

    Рассмотрим $\tilde{J}: C^1[a, b] \map \R$ --- с той же формулой, что и $J$.
    Тогда $\forall u, h: \exists \delta\tilde{J}[u, h]$.

    С другой стороны, если $h \in C^1[a, b], h(a) = h(b) = 0$, то $\forall u \in X, t \in \R: u + th \in X$
    Имеем право рассмотреть $J[u + th]$. Если $u$ --- локальный экстремум, то $\frac{\d }{\d t}\big|_{t = 0}J[u + th] = 0$.
    Она существует, так как это $\frac{\d}{\d t}\tilde{J}[u + th]$.

    Тем самым, такие функции $h$ прибавлять можно, будем это тоже называть вариацией: $\delta J[u, h]$ задаётся той же формулой.
    Дальше работает то же самое рассуждение, все действия те же самые, только при интегрировании по частям внеинтегральный член занулится, никаких дополнительных соотношений не возникнет.
    \theorem[Задача с фиксированными концами]{
        Пусть $L \in C^1([a, b] \times \R^n \times \R^n)$, пусть $X = \defset{f \in C^1[a, b]}{f(a) = f_a, f(b) = f_b}$, пусть $u$ --- локальный экстремум $J$. Тогда
        \numbers{
            \item $(\nabla_{\dot{u}} L)(t, u(t), \dot{u}(t)) \in C^1[a, b]$.
            \item $\frac{\d }{\d t} \nabla_{\dot{u}} L = \nabla_{u} L$ --- уравнение Эйлера --- Лагранжа.
        }
    }
    Заметим, что у нас по-прежнему два условия (теперь уже данные в самой задаче) и уравнение второго порядка, значит, по-прежнему, данных для решения задачи как раз столько, что стоит надеяться на получение решения.
\end{document}
