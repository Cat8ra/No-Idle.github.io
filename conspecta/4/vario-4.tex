\documentclass[a4paper]{report}

\usepackage{../mathstemplate}

\date{IV семестр, весна 2024 г.}
\title{Вариационное исчисление. Неофициальный конспект}
\author{Лектор: Роман Владимирович Романов \\ Конспектировал Леонид Данилевич}

\begin{document}
    \shorthandoff{"}
    \maketitle
    \tableofcontents
    \newpage
    \setcounter{lection}{0}
    поиск экстремумов, где переменных бесконечно;

    $f: M \map \R$.
    \numbers{
        \item Необходимое: $(\grad f)(x) = 0$
        \item Достаточное $(D^2 f)(x)$ знакоопределена ($> < 0$). Будем рассматривать мало.
        \item Экстремум $f\big|_{N}$ --- ? (метод множителей Лагранжа)
    }
    $M$ --- б/м пространство, например, функций. $f$ --- функционал.

    Пускай $X$ --- метрическое пространство, $J: X \map \R$ --- функция.
    \definition[$x \in X$ --- cтрогий локальный минимум]{
        $\forall y \in U_{\delta}(x): \exists \delta > 0: J[y] > J[x]$. Квадратные скобочки --- косметическое.
    }
    Аналогично определяются нестрогий минимум и максимумы.

    \example{
        Пусть $X = \defset{f \in C[0, 1]}{f(0) = f(1) = 1}$, где $\|f\| = \max\limits_{x \in [0, 1]}|f(x)|$.

        Пусть $J[f] = \int\limits_{0}^{1}f^2(x)\d x$. $J$ непрерывен.

        Ясно, что $\forall f \in X: J[f] > 0$. С другой стороны, $\inf\limits_{f \in X}J[f] = 0$ --- можно рассматривать такие функции:...

        С другой стороны, $X$ замкнут. Получается, теорема Кантора не работает. В чём дело? Нет компактности, замкнутое ограниченное в бесконечномерном случае необязательно компактно.
    }

    Пусть $L: [a, b] \times \R^n \times \R^n \map \R$, $J[u] = \int\limits_{a}^{b}L(t, u(t), \dot{u}(t))\d t$.
    Здесь выберем $X = C^1[a, b] = C^1([a, b] \map \R^n)$ (далее не будем указывать область значений, ясно из контекста) и его замкнутые подмножества (не подпространства, нет линейной структуры).

    Пусть $L \in C([a, b] \times \R^n \times \R^n)$.

    Они называются интегральные функционалы --- богатая теория, но часто встречаются в приложениях.
    \examples{
        \item $u \in C^1, u(a) = u_a, u(b) = u_b, J[u] = \int\limits_{a}^{b}\sqrt{1 + (u')^2}\d x$ --- функционал длин графиков кривых.
        \item $J = \int\limits_{a}^{b}(\frac{u^2}{2} - V(u))\d x$, где $V$ --- заданная функция. В механике называется действием.
    }
    Сначала убедимся, что они непрерывны.
    \note[О норме]{
        Для $f \in C^1[a, b]$: $\|f\| = \max\limits_{x \in [a, b]}|f(x)| + \max\limits_{x \in [a, b]}|f'(x)|$ --- очевидно норма. Всегда будем использовать такую норму для $C^1$.
    }
    \proposal{
        Пусть $X = C^1[a, b], L \in C([a, b] \times \R^n \times \R^n)$. Тогда $J$ (определена где-то выше) --- непрерывна на $X$.
        \provehere{
            Пусть $u, \tilde{u} \in X, \|u - \tilde{u}\| < \delta < 1$. $J[u] - J[\tilde{u}] = \int\limits_{a}^{b}L(x, \tilde{u}(x), \dot{\tilde{u}}(x)) - L(x, u(x) \dot{u}(x))\d x \circlesign{\le}$
            Заметим, что    $\|(x, \tilde{u}(x), \dot{\tilde{u}}(x)) - (x, u(x) \dot{u}(x))\|_{\R^{2n + 1}} < \delta$

            Рассмотрим $K = [a, b] \times \overline{B_{\|u\|_x + 1}} \times \overline{B_{\|u\|_x + 1}}$ --- компакт в $\R^{2n + 1}$.

            \[\circlesign{\le} \int\limits_{a}^{b}\omega_{L\big|_K}(\delta)\d x = (b - a)\omega_{L\big|_K}(\delta) \underset{\delta \to 0}\Map 0\] где $\omega$ --- модуль непрерывности.

            Пользовались тем, что $L\big|_K$ непрерывна на компакте.
        }
    }
    Пусть $X$ --- нормированное пространство (необязательно замкнутое), $J: X \map \R$.

    \definition[Производная функционала $J$ в точке $x$ по направлению $h \in X$]{
        $\delta J[x, h] = \frac{1}{\d t}\big|_{t = 0}J[x + th]$.
        Иначе говоря, \emph{вариация} $J$ по направлению $h$.
    }
    Вариация однородна: $\delta J[x, ch] = c \cdot \delta J[x, h]$. Неаддитивна: $\exists \delta J[x, h_1], \delta J[x, h_2]$ --- не следует существование $J[x, h_1 + h_2]$, а если и есть, то необязана быть суммой.
    Примеры были в анализе, нет б/м специфики.

    \properties{
        \item Как и в к/м анализе, в критической точке вариация (коли $\exists$) должна обращаться в нуль.

        А именно, $x \in X$ --- локальный экстремум $J$, тогда $\forall h: \exists \delta J[x, h] \then \delta J[x, h] = 0$.
        \provehere{
            Сужение $\alpha(t) = J[x + th]$ тоже имеет локальный экстремум, значит, если производная в $t = 0$ есть, то нуль.
        }
    }

    \subsection{Необходимые условия}
    \lemma[Ди Буа-Руйона, что-то такое]{
        Пускай $f \in C[a, b], \omega \in C^1[a, b], \omega(a) = \omega(b) = 0$, известно, что $\int\limits_{a}^{b}f \omega' = 0$ для всех таких $\omega$.

        Тогда $f \equiv \const$.
        \provehere{
            Если бы $f$ сама была гладкой, то можно было бы интегрировать по частям. $\int f'\omega = 0 \then f' \equiv 0$ --- можно взять $\omega$, сосредоточенную там, где $f'$ одного знака.

            Надеемся, что $f = \overline{f} = \frac{1}{b - a}\int\limits_{a}^{b}f$

            Проинтегируем $f - \overline{f}$. $\omega(x) \coloneqq \int\limits_{a}^{x}f(x') - \overline{f}\d x'$ --- функция из $C^1$.

            Дальше $\omega(a) = \omega(b) = 0$.

            $0 = \int\limits_{a}^{b}f \omega' = \int\limits_{a}^{b}(f - \overline{f})\omega' = \int\limits_{a}^{b}(f - \overline{f})^2\d x$, упс, противоречие, интеграл нуль, значит, $f \equiv \overline{f}$.
        }
    }
    Опять $X = C^1[a, b]$, функционал того же самого вида $J[u] = \int\limits_{a}^{b}L(t, u(t),\dot{u}(t))\d t$.
    \lemma[Формула первой вариации]{
        Давайте дифференцировать по всевозможным направлениям. Потребуем для этого $L \in C^1([a, b] \times \R^n \times \R^n)$.

        Пусть $u, h \in X$. $J[u + th] - J[u] = \int L(t, u(t) + \tau h(t) + \dot{u}(t) + \tau \dot{h}(t)) - L(t, u(t), \dot{u}(t))\d t$.

        Формула Лагранжа.

        $\grad_u L$ --- вектор из $\R^n$, градиент

        $\tau\int\limits_{a}^{b} \angles{ (\grad_u L)(t, u(t) + \tau_* h(t), \dot{u}(t) + \tau_* h'(t)), h(t)} + \angles{(\grad_{\dot{u}} L)(\dots), \dot{h}(t)}\d t$ где $\tau_* = \tau_*(t) \in [0, \tau]$.

        Значит, $\frac{J[u + \tau h] - J[u]}{\tau} = \int\limits_{a}^{b} \dots$ --- вот тот, что выше

        $\int\limits_{a}^{b}\angles{(\grad_u L)(t, u(t) + \tau_* h(t), \dot{u}(t) + \tau_* h'(t)), h(t)} \Map \int\limits_{a}^{b}\angles{(\grad_{u} L)(t, u(t), \dot{u}(t)), h(t)}\d t$

        Модуль разности аргументов не превосходит $\tau_* \|h\|_{X}$. Значит, $\|\grad_u L(\dots) - \grad_u L(\dots)\| \le \omega_{L\big|_K}(\tau_* \| h\|_X)$. Здесь $K \coloneqq [a, b] \times \overline{B}_{\|u\| + \|h\|} \times \overline{B_{\|u\| + \|h\|}}$.

        Значит, модуль разности интегралов I и II (где один стремится к другому) не превосходит $|(I) - (II)| \le \int\limits_{a}^{b}\omega_{L\big|_K}(\tau_* \|h\|)\d t \le (b - a)\omega_{L\big|_K}(\tau \|h\|)\d t \underset{t \to 0}\Map 0$.

        Разобрались с первым слагаемом под интегралом --- естественный предел. Аналогично со вторым слагаемым, значит, $\frac{J[u + \tau h] - J[u]}\tau \underset{t \to 0}\Map \int\limits_a^b\angles{(\grad_u L)(t, u(t), \dot{u}(t)), h(t)} + \angles{(\grad_{\dot{u}} L)(t, u(t), \dot{u}(t)), \dot{h}(t)}\d t$. Оказалось, что производная по любому направлению $\exists $ и равна тому, что слева.
    }
    Пусть $u \in X$ --- экстремум.
    Тогда $\forall h \in X: \delta J[u, h] = 0$

    Градиент нуль --- уравнение на точку. Хотим уравнение на $u(t)$, избавимся от $h$. Подгоним под лемму Ди-кого?

    Введём $R(x) = \int\limits_{a}^{x}(\grad_u L)(t, u(t), \dot{u}(t))\d t$.
    Тогда $\delta J[x, h] = \int\limits_a^b \angles{\dot{R}(t), h(t)} + \angles{\grad_{\dot{u}} L(t, u(t), \dot{u}(t)), h'(t)}$ Интегируем по частям.

    $\angles{\grad_{\dot{u}}}$
    Дальше я записал в тетрадку кое-что

    $\xi(t) = \const$. Но теперь $R(t) \in C^1$, значит, $\grad_{\dot{u}}L(t, u(t), \dot{u}(t)) \in C^1$ тоже.

    Дифференцируя: $\frac{\d}{\d t}(\grad_{\dot{u}} L)(t, u(t), \dot{u}(t)) - (\grad_u L)(t, u(t), \dot{u}(t)) = 0$. Ого, уравнение на $L$. Уравнение Эйлера --- Лагранжа, основное уравнение вариационки.

    Примечание: в случае общего положения уравнение Э --- Л --- второго порядка ($u \in C^2$), потому что экстремаль \comment{как правильно сказать?}

    $C \coloneqq \xi$. Теперь $h$ опять произвольный $\delta J[u, h] = \angles{R(b), h(b)} + \angles{C, \dot{h}(t)}\d t = \angles{R(b), h(b)} + \angles{C, h(b)} - \angles{C, h(a)}$.

    Теперь в качестве $h$ возьмём $h(b) = 0, h(a) = C$. Для него $\delta J[u, h] = -\|C\|^2$, значит, $\xi = C = 0$.

    Что это означает? См. определение $\xi$. $R(a) = 0$, значит, $(\grad_{\dot{u}} L)(a, u(a), \dot{u}(a)) = 0$.

    Теперь наоборот, $h(b) = R(b)$. Тогда $\delta J[u, h] = \|R(b)\|^2 \then R(b) = 0$.
    Аналогично $(\grad_{\dot{u}} L)(b, u(b), \dot{u}(b)) = 0$.

    Два условия (но в разных точках) на уравнение второго порядка, можно надеяться, что хватит (но это совсем не факт).

    Подытожим в теорему.
    \theorem[Задача со свободными концами]{
        Пусть $J \in C^1([a, b] \times \R^n \times \R^n)$, пусть $X = C^1[a, b]$, пусть $u$ --- локальный экстремум $J$.

        Тогда
        \numbers{
            \item $(\grad_{\dot{u}} L)(t, u(t), \dot{u}(t)) \in C^1[a, b]$.
            \item $\frac{\d }{\d t} \grad_{\dot{u}} L = \grad_{u} L$ --- Э --- Л
            \item $(\grad_{\dot{u}} L)(a, u(a), \dot{u}(a)) = 0$
            \item $(\grad_{\dot{u}} L)(b, u(b), \dot{u}(b)) = 0$
        }
    }

    Теперь обсудим, что, если концы не свободны.

    Рассмотрим $X = \defset{f \in C^1[a, b]}{f(a) = f_a, f(b) = f_b}$. Это не подпространство (не имеет линейной структуры), нельзя определить производную по направлению.

    $J: X \map \R$ тот же.

    Какая здесь характеризация локальных экстремумов?

    Рассмотрим $\tilde{J}: C^1[a, b] \map \R$ --- с той же формулой, что и $J$. Тогда $\forall u, h: \exists \delta\tilde{J}[u, h]$.

    С другой стороны, если $h \in C^1[a, b], h(a) = h(b) = 0$, то $\forall u \in X, t \in \R: u + th \in X$
    Имеем право рассмотреть $J[u + th]$. Если $u$ --- локальный экстремум, то $\frac{\d }{\d t}\big|_{t = 0}J[u + th] = 0$.
    Она существует, так как это $\frac{\d}{\d t}\tilde{J}[u + th]$.

    Тем самым, такие функции $h$ прибавлять можно, будем это тоже называть вариацией: $\delta J[u, h]$ задаётся той же формулой.
    Дальше работает то же самое рассуждение, все действия те же самые, только при интегрировании по частям внеинтегральный член занулится, никаких дополнительных соотношений не возникнет.
    \theorem[ с фиксированными концами]{
        $L \in C^1(...), X = ...$, пусть $u$ --- локальный экстремум от $J$. Тогда
        \numbers{
            \item $(\grad_{\dot{u}} L)(t, u(t), \dot{u}(t)) \in C^1[a, b]$.
            \item $\frac{\d }{\d t} \grad_{\dot{u}} L = \grad_{u} L$ --- Э --- Л
        }
    }
    Заметим, что у нас по-прежнему два условия и уравнение второго порядка.
\end{document}
