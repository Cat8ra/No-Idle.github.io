\documentclass[a4paper]{report}

\usepackage{../mathstemplate}

\date{IV семестр, весна 2024 г.}
\title{Комплексный анализ. Неофициальный конспект}
\author{Лектор: Сергей Витальевич Кисляков \\ Конспектировал Леонид Данилевич}

\begin{document}
    \shorthandoff{"}
    \maketitle
    \tableofcontents
    \newpage
    \setcounter{lection}{0}
    \chapter{Комплексный анализ}
    \newlection{16 февраля 2024 г.}
    Пусть $f: G \map \C$, где открытое $G \subset \C$.
    \definition[$f$ голоморфна в $z_0 \in G$]{
    $\exists \lim\limits_{z \to z_0}\frac{f(z) - f(z_0)}{z - z_0} \bydef f'(z_0)$.
    }
    Во втором семестре мы проверяли, что $f = u + iv$ (где $u, v: G \map \R$) голоморфна в $z_0 \iff f$ дифференцируема в вещественном смысле, и выполняются уравнения Коши --- Римана:
    \[\der{u}{x} = \der{v}{y} \qquad \der{u}{y} = -\der{v}{x}\]
    \definition[$f$ аналитична в $G$] {
    $\forall z_0 \in G: \exists c_j \in \C$: \[f(z) = \sum\limits_{j = 0}^{\infty}c_j(z - z_0)^j\tag{$*$}\label{analytic}\] где ряд сходится не только при $z = z_0$.
    }
    \theorem{
    $f$ аналитична в $G \iff f$ голоморфна во всех точках $G$.
    \provetwhen{
        Доказали во втором семестре, несложно.
    }{
        Скоро займёмся, время пришло.
    }
    }
    Из представления~(\ref{analytic}) следует, что производная в точке $z$ считается почленно: $f'(z) = \sum\limits_{j = 1}^{\infty}j c_j (z - z_0)^{j - 1}$.
    В частности, отсюда получается, что $f'(z_0) = c_1$, и вообще $f^{(n)}(z_0) = j! \cdot c_j$.

    Вскоре мы увидим, что ситуация разительно отличается от вещественной: в вещественном случае были разные классы --- дифференцируемые функции, $C^1$, $C^\infty$, аналитичные, и множество промежуточных классов.

    В комплексном же случае, если функция хотя бы один раз дифференцируема, то окажется, что этого достаточно, чтобы она была не просто дифференцируема, а непрерывно дифференцируема, бесконечно дифференцируема, и даже аналитична.
    \section{Интеграл от дифференциальной формы вдоль пути}
    \subsection{Про дифференциальные формы}
    \definition[Линейная функция $l: \R^n \map \C$]{${\forall \alpha, \beta \in \R, x, y \in \R^n: l(\alpha x + \beta y) = \alpha l(x) + \beta l(y)}$.}
    \definition[Линейная форма на множестве $G \subset \R^n$]{
    Функция двух переменных ${\phi: G \times \R^n \map \C}$, линейная по второму аргументу.
    }
    В пространстве $\R^n$ имеется базис $(e_j)$: $h = e_1 h_1 + \dots + e_n h_n$.

    Тем самым, $\phi(x, h) = \sum\limits_{j = 1}^{n}\underbrace{\phi(x, e_j)}_{\eqqcolon g_j(x)}h_j = \sum\limits_{j = 1}^{n}g_j(x)h_j$.

    Введём \emph{базисные линейные формы} $\d x_j(u, h) = h_j$, игнорирующую первую координату, и возвращающая $j$-ю компоненту второго аргумента.
    Теперь $\phi(x, h)$ разложилась в сумму $\sum\limits_{j = 1}^{n}g_j \d x_j$.

    \example{
        Пусть $f: G \map \C$ --- дифференцируемая в $G$ функция.
        Заметим, что её дифференциал $\d_f(x, \_)$ --- в точности линейная форма на $G$.

        При разложении по базису получится $\d_f(x, \_) = \sum\limits_{j = 1}^{n}\der{f}{x_j}(x)\d x_j$.
    }
    Вскоре мы увидим, что далеко не всякая линейная форма является чьим-то дифференциалом.

    Если $\phi = \sum\limits_{j = 1}^{n} g_j\d x_j$ --- дифференциал функции $f$, то непременно $g_j = \der{f}{x_j}$.

    Тот факт, что $\phi$ является дифференциалом $f$, можно сказать наоборот: $f$ является первообразной $\phi$.
    \subsection{Про интегрирование}
    Рассмотрим монотонную функцию $\Phi: \angles{a, b} \map \R$.
    Как и при определении стилтьесовой длины, будем считать, что $\Phi$ определена на некотором открытом множестве, содержащем $\angles{a, b}$.
    Обозначим за $l_\Phi$ стилтьесову длину, отвечающую функции $\Phi$.

     Пускай $\lambda_\Phi$ --- продолжение стилтьесовой длины $l_\Phi$ по Лебегу --- Каратеодори.

    Она, как водится, определена на некоторой $\Sigma$-алгебре, в которой есть борелевские множества, но измеримы могут быть и какие-то другие множества, зависящие от конкретной функции $\Phi$.
    \examples{
        \item Так, функция $\phi(x) = \all{0,&x < 0 \\ 1,& x \ge 0}$ порождает дельта-меру $\delta_0$, относительно которой все множества измеримы.

        Кроме того, эта мера сингулярна относительно стандартной меры Лебега.

    \item Может показаться, что так происходит из-за разрывности $\phi$, но это не так.

        Рекурсивно определим канторову лестницу $C: [0, 1] \map [0, 1]$:
        \[\begin{tikzpicture}[scale=1]
              \draw[->] (-0.5,0) -- (3.5,0) node[above]{$x$};
              \draw[->] (0,-0.5) -- (0,3.5) node[right]{$y$};
              \fill (0, 0) circle (1.5pt) node[below left] {$0$};
              \fill (1, 0) circle (1.5pt) node[below] {$\nicefrac13$};
              \fill (2, 0) circle (1.5pt) node[below] {$\nicefrac23$};
              \fill (3, 0) circle (1.5pt) node[below] {$1$};
              \fill (0,1.5) circle (1.5pt) node[left] {$\nicefrac12$};
              \fill (0,3) circle (1.5pt) node[left] {$1$};
              \newcommand{\cantor}[5]{ %l, r, d, u, depth
                  \pgfmathsetmacro{\depth}{\numexpr#5}

                  \draw ({(#1+2*#2)/3},{(#3+#4)/2}) -- ({(2*#1+#2)/3},{(#3+#4)/2});

                  \ifnum\depth<5
                  \cantor{#1}{(2*#1+#2)/3}{#3}{(#3+#4)/2}{#5+1}
                  \cantor{(#1+2*#2)/3}{#2}{(#3+#4)/2}{#4}{#5+1}
                  \else
                  \draw ({#1}, {#3}) -- ({#2}, {#4});
                  \fi
              }
              \cantor{0}{3}{0}{3}{0}
        \end{tikzpicture}\]
        Построив по данной функции стилтьесову длину $\lambda_C$, мы получим меру, сосредоточенную на канторовом множестве меры нуль.

        Её носитель --- само канторово множество, так как на всех отрезках вне канторова множества $\lambda_C$ равна нулю.
        Она сингулярна относительно стандартной меры Лебега на $\R$, и её измеримые множества разительно отличаются от измеримых множеств меры Лебега.
    }
    По мере Стилтьеса можно интегрировать: если $v$ является $\lambda_\Phi$ измеримой (в частности, измерима по Борелю, или даже непрерывна), то определён интеграл $\int\limits_{\angles{a, b}}v \d\lambda_{\Phi}$
    Иногда пишут просто $\int\limits_{\angles{a, b}}v \d \Phi$. %Так складывается ощущение, что из измеримости следуют и борелевская измеримость, и непрерывность.
    \ok
    Теперь пусть $I = [a, b]$, и $\Psi: [a, b] \map \R$ --- функция ограниченной вариации.
    В таком случае $\Psi = \Phi_1 - \Phi_2$, где некие $\Phi_1, \Phi_2$ возрастают.
    Можно определить знакопеременную меру $\lambda_\Psi \bydef \lambda_{\Phi_1} - \lambda_{\Phi_2}$, понятно, что определение корректно.
    \subsection{Интеграл от дифференциальной формы вдоль пути}
    Пускай $\gamma: [a, b] \map G \subset \R^n$ --- спрямляемый путь (путь конечной длины).
    Пускай $U = \sum\limits_{j = 1}^{n}u_j \d x_j$ --- дифференциальная форма в области $G$.
    Если не сказано противное, будем считать, что $u_j$ --- непрерывные функции.
    \definition[Интеграл от $U$ вдоль пути $\gamma$]{
    $\int\limits_{\gamma}U \bydef \sum\limits_{j = 1}^{n}\int\limits_{[a, b]}u_j(\gamma(t))\d \gamma_j(t)$.
    }
    Здесь $\gamma = (\gamma_1, \dots, \gamma_n)$.
    Так как путь спрямляем, то все $\gamma_j$ --- ограниченной вариации, каждая порождает свою меру Стилтьеса, и определение интегрирует композицию $U \circ \gamma$ по данной мере.
    \subsection{Сумма путей}
    Пускай имеются два отрезка $[a, c]$ и $[c, d]$, и на них заданы пути $\gamma_1: [a, c] \map G$, $\gamma_2: [c, d] \map G$.
    Предположим, что $\gamma_1(c) = \gamma_2(c)$.

    Тогда можно устроить путь $\gamma = \gamma_1 \oplus \gamma_2: [a, d] \map G$, $\gamma(t) \bydef \all{\gamma_1(t),&t \in [a, c] \\ \gamma_2(t),& t \in [c, d]}$.

    \note{
    Интеграл аддитивен по множеству: $\int\limits_{\gamma_1 \oplus \gamma_2}U = \int\limits_{\gamma_1}U + \int\limits_{\gamma_2}U$. %Мб тут вставить ", поэтому", а тто так не выглядит, как по множеству.
    }
    \subsection{Альтернативное определение}
    Далее мы не интересуемся никакими чудесами вроде канторовых лестниц, и считаем, что $\Phi$ такова, что $\lambda_\Phi$ абсолютно непрерывна относительно стандартной меры Лебега.

    А раз так, то по теореме Радона --- Никодима $\exists$ суммируемая $w: [a, b] \map \R$, такая, что \[\lambda_\Phi(e) = \int\limits_{e}w(x)\d x\tag{$+$}\label{density}\]
    \fact{\label{density-is-derivative}
        Формула~(\ref{density}) заведомо верна, если $\Phi$ непрерывно дифференцируема на $[a, b]$, тогда $w = \Phi'$.
        \provehere{
            Введём меру $\nu(e) = \int\limits_{e}\Phi'(x)\d x$, заданную на измеримых по Лебегу множествах.
            $\Phi'$ непрерывна, и, следовательно, измерима.

            Если $\angles{c, d} \subset [a, b]$, то $\nu(\angles{c, d}) = \int\limits_{\angles{c, d}}\Phi'(x)\d x = \Phi(d) - \Phi(c) = l_\Phi(\angles{c, d})$.

            Таким образом, из теоремы единственности, продолжение $l_\Phi$ по Лебегу --- Каратеодори совпадает с $\int\limits_{e}\Phi'(x)\d x$.
        }
    }
    \note{
        Утверждение~(\cref{density-is-derivative}) сохраняет силу, если $\Phi$ непрерывна и кусочно-непрерывно дифференцируема.
    }
    \comment{Далее где-то используется $\Phi$, а где-то $\beta$, надо убедиться, что это везде одно и то же, и заменить. } % тогда и я трогать эти две буквы пока не буду
    Пускай $\beta: [a, b] \map \R$ --- функция ограниченной вариации, кусочно-непрерывно дифференцируемая: $\exists a = a_0 < a_1 < \dots < a_k = b$, такие, что $\beta$ непрерывно дифференцируема на $[a_s, a_{s+1}]$ при $0 \le s < k$.
    Введём $\rho(e) = \int\limits_{e}\beta'(x)\d x$ --- это знакопеременная вещественная мера.

    У данной меры возникают (см. разложение Хана) положительная и отрицательная части $\rho_+(e) \bydef \int\limits_{e}(\beta')_+(x)\d x$ и $\rho_-(e) \bydef \int\limits_{e}(\beta')_-(x)\d x$
    
    Если обозначить за $\Phi_+(t) = \int\limits_{0}^{t}(\beta')_+(x)\d x$ и $\Phi_-(t) = \int\limits_{0}^{t}(\beta')_-(x)\d x$, то окажется, что соответствующие меры Стилтьеса совпадают с $\rho_+$ и $\rho_-$.
    
    Более того, $\beta = \Phi_+ - \Phi_-$ --- получили разложение функции ограниченной вариации в положительную и отрицательную части.
    \note{
    Это разложение экономнее, чем то, которое было получено ранее --- ранее в качестве $\Phi_+$ выбиралась вариация $\Phi$.
    }
    
    Если всё, что написано выше, собрать вместе, то получится 
    \encircle{\int\limits_{[s, t]}v \d \Phi = \int\limits_{[s, t]}v(x)\beta'(x)\d x}
    Далее <<гладкий>> используется, как синоним к непрерывно-дифференцируемому.
    \corollary[Можно считать определением]{
        Если $U = \sum\limits_{j = 1}^{n}u_j \d x_j$ --- дифференциальная форма в $G$ с непрерывными коэффициентами, а $\gamma = (\gamma_1, \dots, \gamma_n): [a, b] \map G$ --- спрямляемый кусочно-гладкий путь, то
    \[\int\limits_{\gamma}U = \sum\limits_{j = 1}^{n}\int\limits_{a}^{b}u_j(\gamma(t))\gamma_j'(t)\d t\]
    }
    \subsection{(Не)зависимость от параметризации}
    Пускай $\gamma: [a, b] \map G$ --- кусочно-гладкий путь, $\psi: [c, d] \map [a, b]$ --- гладкий гомеоморфизм.

    Теперь $\tilde{\gamma} = \gamma \circ \psi$ --- перепараметризация $\gamma$
    \lemma{
        Для всякой формы $U$:
        \[\int\limits_{\tilde{\gamma}}U = \pm\int\limits_{\gamma}U\]
        Знак $+$ выбирается, если $\psi$ возрастает, и $-$ --- если убывает.
        \provehere{
            Предположим, что $\gamma$ --- гладкий путь, иначе применяем к кусочкам гладкости по отдельности.

            $\int\limits_{\tilde{\gamma}}U = \sum\limits_{j = 1}^{n}\int\limits_{c}^{d}u_j(\gamma(\psi(t)))\gamma_j'(\psi(t))\cdot \psi'(t)\d t = \left\|\arr{c}{\tau = \psi(t) \\ \d \tau = \psi'(t)\d t}\right\| = \sum\limits_{j = 1}^{n}\int\limits_{\psi(c)}^{\psi(d)}u_j(\gamma(\tau))\gamma_j'(\tau)\d \tau = \pm \int\limits_{\gamma}U$
        }
    }
    Про $\psi$ также можно считать, что это он не гладкий, а лишь кусочно-гладкий.

    Тем самым, можно определить сумму путей для несоприкасающихся отрезков: для двух путей $\gamma_1: [a, b] \map G, \gamma_2: [c, d] \map G$ (при условии $\gamma_{1}(b) = \gamma_2(c)$) можно один их отрезков-прообразов линейным возрастающим преобразованием перевести в отрезок, соприкасающийся со вторым (например, $t \mapsto t + (b - c)$).

    Также есть понятие обратного пути $\gamma_{-1}(t) = \gamma(a + b - t)$.
    Для любой формы $U$: \[\int\limits_{\gamma \oplus \gamma_{-1}}U = \int\limits_{\gamma}U + \int\limits_{\gamma_{-1}}U = \int\limits_{\gamma}U - \int\limits_{\gamma}U = 0\]
    \section{Условия существования первообразной у дифференциальной формы}
    \theorem{
    Если у дифференциальной формы $U$ в открытом множестве $G \subset \R^n$ имеется первообразная $F$, то для всякого кусочно-гладкого пути $\gamma: [a, b] \map G$
    \[\int\limits_{\gamma}U = F(\gamma(b)) - F(\gamma(a))\]
    \provehere{
        $U = \sum\limits_{j = 1}^{n}g_j \d x_j$, где $g_j(w) = \der{}{x_j}F(w)$.
        Считаем, что путь гладкий.
    \[\int\limits_{\gamma}U = \sum\limits_{j = 1}^{n}\int\limits_{a}^{b}\der{}{x_j}F(\gamma(t))\gamma_j'(t)\d t = \int\limits_{a}^{b}\frac{\d}{\d t}(F \circ \gamma)(t)\d t = F(\gamma(b)) - F(\gamma(a))\]
    Если же путь всего лишь кусочно-гладкий, то надо разбить отрезок на подотрезки гладкости, и сложить.
    }
    }
    \corollary{
    Если у дифференциальной формы $U$ есть первообразная, то её интегралы по всем путям с данными началом и концом, равны.
    }
    Оказывается, верно и обратное.
    \theorem{
        Пусть дифференциальная форма $U$ с непрерывными коэффициентами в области $G$ такова, что $\forall x, y \in G: \exists c \in \C: \forall$ кусочно-гладкого пути $\gamma$ с началом в $x$ и концом в $y$: $\int\limits_{\gamma}U = c$.
        Эквивалентно, для всех замкнутых кусочно-гладких путей $\gamma: \int\limits_{\gamma}U = 0$.

        Тогда $U$ обладает первообразной.
        \provehere{
            \indentlemma{
            Пусть $G$ --- область в $\R^n$, тогда любые две её точки можно соединить ломаной (кусочно-линейным путём).
            }{

            }
        }
    }
   %Надо пересмотреть написание "кусочно-непрерывно дифференцируемый", я нашёл вариант с двумя дефисами.
\end{document}
