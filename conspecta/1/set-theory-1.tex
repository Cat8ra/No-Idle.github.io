\documentclass[a4paper]{report}

\usepackage{../mathstemplate}

\date{I семестр, осень 2022 г.}
\title{Основы теории множеств. Неофициальный конспект}
\author{Лектор: Виктор Львович Селиванов \\Конспектировал Леонид Данилевич}

\begin{document}
    \maketitle
    \tableofcontents
    \newpage
    \setcounter{lection}{0}
    \newlection{6 сентября 2022 г.}
    \section*{Литература}
    \begin{enumerate}
        \item Н.\ К.\ Верещагин, А.\ Шень.
        <<Лекции по математической логике и теории алгоритмов.\ ч.\ 1.
        Начала теории множеств>>.\ --- М.: МЦНМО, 2012.
        \item К.\ Куратовский, А.\ Мостовский.
        <<Теория множеств>>.\ М.: Мир, 1970.
        \item Т.\ Йех, <<Теория множеств и метод форсинга>>.\ М.: Мир, 1973
        \item И.\ А.\ Лавров, Л.\ Л.\ Максимова, <<Задачи по теории множеств, математической логике и теории алгоритмов>>.\ М.: Наука, 2001.
    \end{enumerate}


    \chapter{Наивная теория множеств}


    \section{Множества. Отношения и операции}
    Множества бывают конечные (с $n \in \N_0$ элементами) и бесконечные.

    Конечные множества можно задать перечислением $\{1, 3, 8, 21\}$ или свойством $\{x | \phi(x) \}$, например, $\{x|x\text{ --- чётное натуральное}\}$.

    Равенство $A = B \iff \forall x : (x \in A \iff x \in B)$.

    Включение $A \subset B \iff \forall x : (x \in A \then x \in B)$.

    Пересечение $A \cap B = \{x | x \in A \land x \in B\}$.
    Ассоциативно и коммутативно.
    Дистрибутивно относительно $\triangle$.

    Объединение $A \cup B = \{x | x \in A \lor x \in B\}$.
    Ассоциативно и коммутативно.

    Разность $A \cup B = \{x | x \in A \land x \notin B\}$.

    Симметрическая разность $A \triangle B \bydef (A \bs B) \cup (B \bs A)$.
    Ассоциативно и коммутативно.

    Дополнение $A\inv \bydef U \bs A$, если все рассматриваемые множества содержатся в унивёрсуме $U$.

    Булеан --- множество всех подмножеств $A$.
    Обозначается $\mathcal{P}(A) = 2^A = \{X | X \subset A\}$.

    \subsection{Отношения}
    Декартово произведение $A \times B = \{ (a, b) | a \in A \land b \in B\}$.
    Подмножества $R \subset A\times B$ называются бинарными отношениями между $A$ и $B$.
    Запись $(a, b) \in R$ иногда упрощают до $aRb$.
    Так, типичным отношением является <<\text{<}>>.
    Тогда пишут $a<b$.

    Композиция отношений $R$ (между $A$ и $B$) и $S$ (между $B$ и $C$): \[S \circ R = \{ (a, c) \in A \times C | \exists b \in B: ((a, b) \in R \land (b, c) \in S)\}\]

    $R^{-1} = \{(b, a) | (a, b) \in \R\}; R^{-1} \subset B \times A$ --- обратное отношение. $\left(R^{-1}\right)^{-1} = R$.

    $dom(R) = \{a | \exists b :~(a R b) \}$ --- область определения $R$.

    $rng(R) = \{b | \exists a :~(a R b) \}$ --- область значений $R$.

    Образ $R(A') = \{b \in B | \exists a \in A':~a R b\}$ для $A' \subset A$.

    Прообраз $R^{-1}(B') =\{a \in A | \exists b \in B': ~ a R b\}$ для $B' \subset B$.

    $R$ является функциональным отношением $\iff \forall a, b, b':~((a R b) \land (a R b')) \then b = b')$.

    $R$ --- функция, если оно функционально, и $dom(R) = A$.
    В таком случае пишут $R(a) = b$ для того единственного $b \in B: a R b$.

    Можно подчеркнуть, что $R$ --- тотальная функция, а если $dom(R) \ne A$, и $R : \subset A \map B$, то это частичная функция.

    Функция называется инъекцией, если $\forall a, a' \in A: a \ne a' \then f(a) \ne f(a')$.

    Функция называется сюръекцией, если $rng(f) = B$.

    Функция называется биекцией, если она одновременно является и сюръекцией, и инъекцией.

    \subsubsection{Типы внутренних бинарных отношений $R \subset A \times A$}

    Рефлексивность $\forall a \in A:~a R a$.

    Антирефлексивность $\forall a \in A:~!(a R a)$.

    Симметричность $\forall a, b \in A:~a R b \iff b R a$.

    Антисимметричность $\forall a, b \in A:~((a R b) \land (b R a)) \then a = b$.

    Транзитивность $\forall a, b, c \in A:~(a R b) \land (b R c) \then a R c$.

    Предпорядок --- отношение с рефлексивностью и транзитивностью.
    Обозначается $\le, \preceq, \subseteq, \sqsubseteq$.

    Частичный порядок --- антисимметричный предпорядок.

    Строгий порядок --- антирефлексивность и транзитивность.
    Обозначается $<, \prec, \subset, \sqsubset$.

    Эквивалентность --- рефлексивность, симметричность, транзитивность. $=, \equiv, \approx, \cong, \simeq$.

    \subsubsection{Классы эквивалентности}

    Рассмотрим некое множество $A$ и отношение эквивалентности на нём $\equiv$.

    Пусть $[~]: A \map 2^A, a \mapsto [a]$, где $[a] = \{x \in A | x \equiv a\}$ --- класс эквивалентности, (порождённый элементом | элемента) $a$.

    Предложение: классы эквивалентности образуют разбиение $A$, т.\ е.\ $\forall a \in A: [a] \subset A \land [a] \ne \varnothing$, а кроме того $\forall x, y \in A: ([x] = [y]) \lor ([x] \cap [y] = \varnothing)$ и $\underset{a \in A}{\cup}[a] = A$.

    \begin{proof}
        $[a] \ne \varnothing$, так как $a \in [a]$.
        По этой же причине $\left(\underset{a \in A}{\cup}[a]\right) \supset \left(\underset{a \in A}{\cup}a\right) \supset A$, но так как $\forall a \in A: [a] \subset A$, то  $\underset{a \in A}{\cup}[a] = A$.

        Если $a \equiv b$, то $\forall x \in [a] : x \in [b]$, (так как раз $a \equiv x$, то по транзитивности $b \equiv x$).

        Если же $a \not\equiv b$, то $[a] \cap [b] = \varnothing$.
        От противного: пусть $\exists x \in A: x \in [a] \land x \in [b]$.
        Тогда по транзитивности $a \equiv b$, противоречие.
    \end{proof}

    Фактор множества $A$ по отношению эквивалентности $\equiv$ обозначается $A_{/\equiv}$.

    $A_{/\equiv}\bydef\{s \subset A | \exists a \in A : s = [a]\}$.

    \newlection{13 сентября 2022 г.}


    \section{Мощность множества. Сравнение мощностей}

    О мощности множества можно думать, как о количестве его элементов.
    Однако непонятно, как быть с бесконечными множествами.
    \definition[Равномщность]{$A$ и $B$ равномощны --- $A \simeq B$ --- если существует биекция $f : A \map B$.}

    \subsection{Свойства отношения равномощности}

    Отношение рефлексивно, симметрично, транзитивно.

    $A \simeq A$, так как $id$ --- искомая биекция.

    $A \simeq B \then B \simeq A$, так как существование биекции $f : A \map B$ влечёт существование обратной биекции $f^{-1} : B \map A$.

    $A \overset{f}{\simeq} B \land B \overset{g}{\simeq} C \then A \overset{f \circ g}{\simeq} C$.

    Таким образом, $\simeq$ является отношением эквивалентности, но ввести фактор множества всех множеств нельзя, так как множества всех множеств не существует.

    \definition{Множество $A$ не превосходит по мощности множество $B$ ($A \preceq B$), если существует инъекция $f : A \map B$. }

    \theorem[Теорема Кантора-Шрёдера-Бернштейна]{
        $A \preceq B \land B \preceq A \then A \simeq B$.
        \provehere{
            Пусть $A \overset{f}{\map} B \overset{g}{\map} A$ --- две инъекции.

            Пусть $h = f \circ g$. Как композиция инъекций, она является инъекцией.

            Пусть $\switch{A_0 = A\\A_1 = g(B)\\A_2 = h(A)}$ Заметим, что $A_2 \subseteq A_1 \subseteq A_0$.

            $A_1 \simeq B$, потому что $g : B \map A_1$ --- биекция.

            Аналогично $h : A_0 \map A_2$ --- биекция.

            Утверждается, что достаточно доказать, что $A_0 \simeq A_1$.

            Определим бесконечную последовательность $A_{n+2} = h(A_n)$. Из этого определения видно, что $A_{n+1} \subseteq A_n$ и множества, равномощные $A_0$ --- с чётными номерами, а равномощные $A_1$ --- с нечётными.

            Пусть $C_n = A_n \bs A_{n+1}$. Нетрудно видеть, что $h : C_0 \map C_2$ --- биекция. Вообще говоря, все $C_{2n}$ равномощны. После этого из картинки видно, что $C_{2n}$ уплотняются, а остальные могут тождественно перейти в себя. Формальнее, \[u : A_0 \map A_1 \quad u(a) = \switch{h(a),&\exists n \in \N: a \in C_{2n}\\a,&\text{otherwise}}\]
            Можно увидеть, что $u$ --- искомая биекция.

        }}

    \definition{Множество $A$  меньше по мощности  $B$ ($A \prec B$), если

        $\switch{A\preceq B\\B\not\preceq A}\overset{\text{здесь равносильно}}{\iff}\switch{A \preceq B\\A\not\simeq B}$}

    \theorem[Теорема Кантора]{
        Для любого множества $A$: $A \prec 2^A$.
        \begin{proof}
            \note{Если $A$ --- конечно и имеет $n$ элементов, то теорема верна, так как $n < 2^n$ для любого $n \in \N_0$.}

            $A \preceq 2^A$ --- рассмотрим инъекцию $a \in A \mapsto \{a\}$.
            Теперь докажем, что $A \not\simeq 2^A$.
            Предположим противное: $A \simeq 2^A$.
            Тогда есть биекция $g : A \map 2^A$.
            Теперь, сходно с диагональным аргументом для $\N \not\simeq \R$, определим
            \([B \subseteq A:\quad B =\{a \in A | a \notin g(a) \}\)
            Очевидно, $\nexists a \in A: B = g(a)$.
            Однако $B \subseteq A \then B \in g(A)$, противоречие.
        \end{proof}
    }

    \subsection{Некоторые виды множеств по мощностям}
    \numbers{
        \item Конечные множества. \emph{Комбинаторика}
        \item Счётные множества --- множества, равномощные $\N$. \emph{Информатика}
        \item Континуальные множества --- множества, равномощные $2^\N$. \emph{Матанализ}
    }

    \[\underset{\text{счётное}}{\N} \subset \underset{\text{счётное}}{\Z} \subset \underset{\text{счётное}}{\Q} \subset \underset{\text{континуальное}}{\R} \subset \underset{\text{континуальное}}{\C}\]

    \subsubsection{Шкала мощностей}
    \[0, 1, 2, 3, \dots, (\omega = |\N|), \dots, (\mathbf{C} = |2^\N|), \dots, \]
    \proposal{Для любого бесконечного множества $A: \N \preceq A$.
    \provehere{
        Пусть $A$ --- бесконечное множество.
        Тогда $\exists a_0 \in A$.
        Заметим, что $A \bs \{a_0\}$ тоже бесконечно.
        Дальше по индукции мы можем найти $a_n$ для любого $n \in \N$.
        Таким образом, мы нашли инъекцию $\N \map A$.
    } }

    \question{Существует ли множество $A$ : $\N \prec A \prec \R$?}

    Континуум гипотеза, CH, утверждает, что таких множеств не существует.

    \newlection{20 сентября 2022 г.}

    \question{Пусть даны множества $A$ и $B$: $\left[\begin{aligned}
                                                         A \simeq B \\ A \prec B \\ B \prec A
    \end{aligned}\right.$. Правда ли, что другого исхода не бывает?

    Наиболее популярная система аксиом утверждает, что всё исчерпывается этими тремя случаями.
    }


    \section{Числовые структуры в теории множеств}

    \subsection{Натуральные числа}

    Определим натуральное число, как мощность конечного множества.

    \[0 \coloneqq |\o|; \quad 1 \coloneqq |\{\o\}|\]
    Сложение: для непересекающихся множеств $|A| + |B| = |A| \sqcup |B|$, но так как множества могут пересекаться, то мы можем их сделать искусственно непересекающимися:
    \[|A| + |B| = |(\{0\} \times A) \cup (\{1\} \times B)|\]
    Умножение:
    \[|A| \cdot |B| = |A \times B|\]
    Степень не является основной операцией, но её можно определить красиво: \[|A|^{|B|} = |A^B|\]
    Упорядоченность: \[|A| \le |B| \overset{def}{\iff} = A \prec B\]

    После определения структуры надо доказать свойства (ассоциативность и коммутативность $+$ и $\cdot$, дистрибутивность $\cdot$ относительно $+$, нейтральность $0$ и $1$, $0 < 1 < 2 < \dots$, между соседними числами нет других чисел, аксиому индукции), но мы этого делать не будем.

    Любая структура, удовлетворяющая этим свойствам, изоморфна $\N$.

    \subsection{Целые числа}

    Построим целые числа из натуральных --- $(\N, +, \cdot, \le, 0, 1)$.

    Определим $\Z = \N \times \N / \sim$, где $(a, b) \sim (c, d) \overset{def}{\iff} a + d = b + c$.
    Паре $(a, b)$, неформально говоря, будет соответствовать $a - b$.
    \note{Здесь и далее тильда над плюсом: $\tilde{+}$  не имеет никакого отношения к отношению эквивалентности $\sim$, она лишь показывает, что данное сложение отличается от сложения в предыдущей структуре.}
    \begin{gather*}
    [a, b]
        ~\tilde{+} ~[c, d] \coloneqq [a + c, b + d]\\
        [a, b]~\tilde{\cdot} ~[c, d] \coloneqq [ac + bd, ad + bc]\\
        [a, b] ~\tilde{\le}~ [c, d] \coloneqq (a + d) \le (b + c)\\
        \tilde{0} \coloneqq [0, 0]; \quad \tilde{1} \coloneqq [1, 0]\\
    \end{gather*}

    После определения операций, и проверки, что эквивалентные пары после равных операций эквивалентны, надо проверить свойства целых чисел:


    Это упорядоченное кольцо, то есть:

    $\tilde{+}, ~\tilde{\cdot}$ ассоциативны и коммутативны; $\tilde{\cdot}$ дистрибутивна относительно $\tilde{+}$, $\tilde{0}$ нейтральны относительно $\tilde{+}, \tilde{\cdot}$,
    \begin{gather*}
        \forall x:~ \exists y : x + y = 0\\
        \forall x, y, z : x \le y \then x + z \le y + z\\
        \forall x, y, z : x \le y \land z > 0 \then xz \le yz\\
    \end{gather*}

    \subsection{Рациональные числа в теории множеств}

    Уже есть $\N \subset \Z$ --- внутри $Z$ есть подмножество, изоморфное $\N$.

    Рассмотрим $\Q \coloneqq (\Z \times (\N \bs \{0\}))/\sim$, где $(a, b) \sim (c, d) \overset{def}{\iff} ad = bc$.

    Теперь введём операции:

    \begin{gather*}
    [a, b]
        \tilde{+} [c, d] \bydef [ad + bc, bd]\\
        [a, b] \tilde{\cdot} [c, d] \bydef [ac, bd]\\
        [a, b] \le [c, d] \overset{def}{\iff} ad \le bc\\
        \tilde{0} \coloneqq [0, 1]; \quad \tilde{1} \coloneqq [1, 1]\\
    \end{gather*}

    После определения операций, и проверки, что эквивалентные пары после равных операций эквивалентны, надо проверить свойства рациональных чисел:

    Это упорядоченное поле, такое, что любой элемент получается делением целого числа на натуральное. % , то есть

% $\tilde{+}, ~\tilde{\cdot}$ ассоциативны и коммутативны; $\tilde{\cdot}$ дистрибутивна относительно $\tilde{+}$, $\tilde{0}$ нейтральны относительно $\tilde{+}, \tilde{\cdot}$, 
% \[ \forall x: ~ \exists y : x + y = 0  \]
% \[\forall x \ne 0:~ \exists y = x^{-1} : x \cdot y = 1\]
% \[\forall x:\exists a \in \Z, b \in \N: x = a \cdot b^{-1}\]
% \[\forall x, y, z : x \le y \then x + z \le y + z\]
% \[\forall x, y, z : x \le y \land z > 0 \then xz \le yz\] 

    \subsection{Вещественные числа в теории множеств}
    Уже определены $\N \subset \Z \subset \Q$.

    Определим $\R \coloneqq S/\sim$, где $S$ --- множество всех последовательностей Коши $\{q_i\}_{i \in \N}$ рациональных чисел: $\forall n \in \N: \exists m \in \N: \forall i, j \in \N : i, j > m: \quad (|q_i| - |q_j|) < 2^{-n}$.

    \begin{gather*}
        \{q_i\} \sim \{r_i\} \overset{def}{\iff} \lim\limits_{i \map \infty}(q_i - r_i) = 0\\
        [\{q_i\}]~\tilde{+}~[\{r_i\}] \bydef [\{q_i + r_i\}]\\
        [\{q_i\}]~\tilde{\cdot}~[\{r_i\}] \bydef [\{q_i \cdot r_i\}]\\
        [\{q_i\}]~\tilde{\le}~[\{r_i\}]~ \overset{def}{=} ~\exists n, m \in \N: \forall i, j \in \N:i, j > m:\quad q_i - r_j < -2^{-n}\\
        \tilde{0} \coloneqq [\{0, 0, \dots \}]; \quad \tilde{1} \coloneqq [\{1, 1, \dots \}]\\
    \end{gather*}

    После определения операций, и проверки, что эквивалентные последовательности Коши после равных операций эквивалентны, надо проверить, что получилось полное упорядоченное поле, (то есть любое непустое ограниченное сверху множество имеет супремум).

    \subsection{Комплексные числа}
    Уже определены $\N \subset \Z \subset \Q \subset \R$.

    Из аксиом упорядоченного кольца $R$ можно доказать $\nexists i \in R : i^2 = -1$.
    Поле комплексных чисел есть наименьшее расширение поля вещественных чисел, обладающее таким элементом.

    Определим $\C \coloneqq \R \times \R$.

    Теперь введём операции

    \begin{gather*}
    (a, b)
        \tilde{+} (c, d) \bydef [a + c, b + d]\\
        (a, b) \tilde{\cdot} (c, d) \bydef (ac - bd, ad + bc)\\
        \tilde{0} \coloneqq (0, 0); \quad \tilde{1} \coloneqq (1, 0); \quad i = (0, 1)\\
    \end{gather*}

    Можно проверить, что полученная структура --- поле, являющееся расширением $\R$ (содержит подмножество, изоморфное $\R$) и содержащее мнимую единицу.


    \chapter{Аксиоматика Цермело --- Френкеля с аксиомой выбора.}


    \section{Противоречивость наивной теории множеств}
    К сожалению, наивная теория множеств противоречива.
    Например, вот пример противоречия: $y \coloneqq \{x | x \notin x\}$.
    Тогда $y \in y \iff y \notin y$.
    \newlection{21 сентября 2022 г.}


    \section{Аксиомы Цермело --- Френкеля с аксиомой выбора, ZFC}
    Множества обозначаются латинскими буквами, переменными: $x, y, z, \dots$.
    Для формул $\phi, \psi$ определены также формулы $(\phi \lor \psi), (\phi \land \psi), (\phi \then \psi), ((\phi \iff \psi) \bydef (\phi \then \psi \land \psi \then \phi)), \neg \phi$.
    Также для получения новых формул пишут $\forall x \phi$ или $\exists x \phi$.

    Запись $A = \{ x | \phi(x) \}$ определяет не множество, но новый класс, который может не быть множеством.
    Класс --- неформальное понятие о формуле.

    Для классов определены булевские операции $A \cup B, A \cap B, \neg A$, что на самом деле просто модифицирует задающие класс формулы.
    Так, $A \cup B = \{ x | \phi_A(x) \lor \phi_B(x) \}$.
    \numbers{
        \item[0.] Существует хотя бы одно множество. $\exists x : ~ x = x$. Аксиома не всегда приводится, иногда опускается.
        \item Аксиома объёмности. $\forall X, Y: ~ (\forall u : (u \in X \iff u \in Y)) \iff X = Y$.
        \item Аксиома (неупорядоченной) пары. $\forall u, v:~(\exists \{u, v\} = X \text{ (это такое обозначение множества)} : \forall z : (z \in X \iff z = u \lor z = v))$.
        \definition[Упорядоченная пара]{Упорядоченной парой из элементов $x, y$ называется множество $(x, y) \bydef \{x, \{x, y\}\}$.}
        \definition[Одноэлементное множество]{$\{x\} \bydef \{x, x\}$.}
        \proposal{$(x, y) = (x', y') \iff x = x' \land y = y'$. }
        \item Аксиома выделения. $\forall X, \phi(u) \text{ ($\phi(u)$ --- формула от свободной переменной)} : ~(\exists \{x \in X | \phi(x) \} = Y : ~ u \in Y \iff (u \in X \land \phi(u)))$. Пересечение класса со множеством -- множество.
        \theorem{
            Существует пустое множество
            \provehere{
                Рассмотрим множество из Аксиомы $0$, назовём его $x$, рассмотрим

                $\o \bydef \{u \in x | \neg(u = u) \}$. Видно, что $\forall x: x \in \o \iff x \ne x$, откуда получаем $\forall x : \neg(x \in o)$.
            }
        }
        \theorem{Существует разность множеств $X \bs Y$. \begin{proof}
                                                             Определим её, как $\{u \in X | u \notin Y\}$.
        \end{proof}}
        \item Аксиома объединения. $\forall X: ~\exists Y : ~ (\forall z : ~u \in z \land z \in X \then u \in Y)$. Аксиома говорит, что существует множество, содержащее объединение элементов $X$.
        \theorem{Существует объединение элементов $X$, обозначаемое $\left(\bigcup\limits_{z \in X}z\right)$.
        \provehere{
            Рассмотрим для данного $X$ $Y$ из данной аксиомы. Используя аксиому выделения, получим $\left(\bigcup\limits_{z \in X}z\right) \bydef \{u \in Y | \exists z \in X: u \in z \}$.
        }}
        \theorem{Существует пересечение элементов $X$, обозначаемое $\left(\bigcap\limits_{z \in X}z\right)$
            \provehere{
                Рассмотрим для данного $X$ $Y$ из данной аксиомы. Используя аксиому выделения, получим $\left(\bigcap\limits_{z \in X}z\right) \bydef \{u \in Y | \forall z \in X: u \in z \}$.
            }}
        \item Аксиома степени. $\forall X :~ \exists \mathcal{P}(X) = 2^X : ~ (u \in 2^X \iff u \subseteq X)$.
        \definition[Подмножество]{$Y \subseteq X \overset{def}{\iff} \forall u : (u \in Y \then u \in X)$.}
        \theorem{Для множеств $A, B$ существует множество $A \times B = \defset{(a, b)}{ a \in A \land b \in B}$.
        \provebullets{
            \item $\{x\} \in \mathcal{P}(X) \subseteq \mathcal{P}({X \cup Y})$
            \item $\{x, y\} \in \mathcal{P}({X \cup Y})$
            \item $(x, y)  = \{x, \{x, y\}\} \in \mathcal{P}(\mathcal{P}(X \cup Y))$.
            \item $X \times Y \bydef \{z \in \mathcal{P}(\mathcal{P}(X \cup Y)) | \exists x \in X, y \in Y : z = (x, y)\}$
        }
        }
        \item Аксиома замены.

        $\forall \phi(u, v): (\forall x, y, y' : ~ \phi(x, y) \land \phi(x, y') \then y = y') \then \forall X : ~ (\exists Y: ~(\forall u, v : u \in X \land \phi(u, v) \then v \in Y)$. Неформальнее, если $\phi$ --- функциональное отношение (быть может не везде определённая функция), то существует множество, содержащее образ $\phi(X)$.

        Используя аксиому выделения, можно доказать существование множества, \emph{являющегося} образом $\phi(X)$.
        \item Аксиома бесконечности. $\exists Y : (\o \in Y \land (\forall y : y \in Y \then (y \cup \{y\}) \in Y))$.

        Несложно видеть, что $\o \in Y, \{\o\} \in Y, \{\o, \{\o\}\} \in Y, \dots$
        \item Аксиома фундирования (иногда называется аксиомой регулярности).

        $\forall X:~(X \ne \o \then \exists x : ~(x \in X \land \forall u \in x : u \notin X))$.

        Неформально говоря, для бинарного отношения $\in$ на непустом множестве $X$ существует минимальный элемент внутри $X$.

        \begin{samepage}
            \proposal {$\nexists X : X \in X$.
            \provehere{
                Предположим, что существует $X \in X$.
                Для противоречия рассмотрим $\{X\}$.
                Из определения $\{X\}$ единственный $Y \in \{X\}$ --- это $X$.
                Но тогда противоречие с аксиомой фундирования, ведь $X \in X$.
            }}
        \end{samepage}
        \item Аксиома выбора. $\forall X : ~\exists f : (2^X \bs \{\o\}) \map X : ~ \forall Y \subseteq X : ~(Y = \o \lor f(Y) \in Y)$.
        \note{Функция $f$ --- особое множество пар.}

        Часто использование аксиомы выбора подчёркивается отдельно, так как она неконструктивна и из неё подчас следуют странные, контринтуитивные вещи.
    }

    \fact{В ZF AC равносильна следующему: $\forall \text{ бесконечного } A : A \simeq A \times A$.}
    \theorem{$A \prec B \lor B \prec A \lor A \simeq B$ в ZFC
    \provehere{Будет подальше~(\cref{aleb_and_blea_then_aeqb})}}
    \newlection{22 сентября 2022 г.}


    \section{Вполне упорядоченные множества. Ординалы}
    Пусть $(P; \le)$ --- частичный порядок: антирефлексивность $(x < y \iff x \le y \land x \ne y)$, транзитивность.

    Можно писать и $(P; \le)$, и $(P; <)$, так как понятно, как из $<$ получить $\le$, и наоборот (равенство считаем уже заданным на множестве).

    \definition[Фундированный порядок]{$P$ --- фундированный порядок, если любое подмножество имеет минимальный элемент: $\forall X \subseteq P : \exists x \in P: \nexists y : y < x$.}
    \definition[Линейный порядок]{Любые два элемента сравнимы: $\forall x, y \in P: \left[\begin{aligned}
                                                                                              x < y \\ y < x \\ x = y
    \end{aligned}\right.$}

    \definition[Верхняя граница для множества $X \subseteq P$]{Такое число $y \in P: \forall x \in X: x \le y$.}
    \definition[Точная (наименьшая) верхняя граница, supremum]{Наименьшее число в множестве верхних границ. $y = \sup X$.}
    Рассмотрим множество $M = \defset{x \in \Q}{x^2 < 2}$ в каком-то порядке.
    Тогда $\sup_{(\R; \le)}M = \sqrt{2}$; $\nexists \sup_{(\Q; \le)} M$.
    \definition[Начальный сегмент, задаваемый элементом $p$]{$\hat{p} = \defset{x \in P}{x < p}$.}
    Пусть $(P, <)$ и $(Q, \prec)$ --- частичные порядки.
    \definition[Изоморфизм из $P$ на $Q$]{Биекция $f : P \map Q$, такая, что  $\forall x, y \in P: {x < y \iff f(x) \prec f(y)}$.}
    \definition[Изоморфность частичных порядков $P$ и $Q$]{Существование изоморфизма из $P$ в $Q$.}
    \fact{Изоморфизм --- отношение эквивалентности.}
    \definition[Вложение]{Инъекция $f : P \map Q$, сохраняющая порядок: $\forall x, y \in P: {x < y \iff f(x) \prec f(y)}$.}
    \definition[Полный порядок или вполне упорядоченное множество $(P; <)$]{Линейный фундированный порядок $(P; <)$: в любом подмножестве есть минимум, все элементы сравнимы.}

    \subsection{Свойства полных порядков}
    $(P; <)$ и $(Q; \prec)$ ниже --- полные порядки.
    \numbers{
        \item Для любого вложения в себя $f : P \map P$ верно: $\forall p : p \le f(p)$.
        \provehere{
            От противного: $\exists p \in P: p \not\le f(p)$. Тогда $\defset{p \in P}{f(p) < p} \ne \o$, а ещё в этом множестве есть минимальный элемент $p_0$.
            Минимальность означает следующее: \[{\forall x \in P : x < p \then x \le f(x)}\]
            Но вложение сохраняет порядок, из $f(x) < f(p_0)$ и транзитивности следует $\forall x \in \hat{p_0}: x< f(p_0)$
        Тогда $f(p_0)$ -- верхняя грань $\hat{p_0}$. В то же время $p_0 = \sup \hat{p_0}$, откуда $p_0 \le f(p_0)$
        }
        \item Никакой полный порядок не может быть изоморфен своему начальному сегменту $\forall p \in P: P \ncong \hat{p}$.
        \provehere{Допустим, для некоего $p$ существует вложение $f : P \map \hat{p}$. Тогда $f(p) < p$, противоречие.}
        \item Для любых $P, Q: \left[\begin{aligned}
                                         P \cong Q \\ \exists p \in P: \hat{p} \cong Q \\ \exists q \in Q: P \cong \hat{q}
        \end{aligned}\right.$, причём выполняется ровно одно.
        \provebullets{
            \item Если выполняются одновременно первое и ещё какое-то, то вполне упорядоченное множество изоморфно своему начальному сегменту.
            \item Если одновременно выполняются второе и третье, то тоже существует вложение из $P$ в некое несобственное подмножество --- композиция изоморфизмов.
            \item Докажем, что выполняется хотя бы одно.

            \bullets{\item Введём отношение $f \bydef \defset{(p, q)\in P \times Q}{\hat{p} \cong \hat{q}}$.
            Это отношение функционально: если $f(p, q)$ и $f(p, q')$, то $\hat{q} \cong \hat{q}'$, откуда если $q \ne q'$, то больший из $q$ и $q'$ порождает полный порядок, изоморфный своему начальному сегменту (порождённому меньшим из $q$ и $q'$). Аналогично это инъекция. Будем писать $f(p) = q; ~ f^{-1}(q) = p$.

            \item Утверждается, что либо $\dom f = P$, либо $\rng f = Q$.

            \item Заметим, что если $p \in \rng f$, то $\forall x < p : ~x \in \rng f$. Рассмотрим некий $x < p$ и покажем, что действительно $\exists y \in Q : \hat{x} \cong \hat{y}$.

            Известно, что $\hat{p} \overset{f_p}{\cong} \hat{q}$. Пусть данный изоморфизм переводит $x$ в $y = f_p(x)~ (y \in Q)$. Утверждается, что $\hat{x} \cong \hat{y}$. Ну, в самом деле, $\forall a < x: f_p(a) < f_p(x)$ --- изоморфизм сохраняет порядок; $\forall b \prec y : f_p^{-1}(b) < f_p^{-1}(y)$ --- обратный изоморфизм тоже сохраняет порядок.

            \item Аналогично если $q \in \dom f$, то $\forall y \prec q : ~ y \in \dom f$.

            \item Предположим противное: $\dom f \subsetneq P$. Пусть $p$ --- наименьший элемент $P\bs(\dom f)$ (существует из-за фундированности). Аналогично, $q$ --- наименьший элемент, такой, что $q \notin \rng(f)$. Заметим, что $\hat{p} = \dom f; \quad \hat{q}~ = \rng f$.

            Утверждается, что $f : \hat{p} \map \hat{q}$ --- изоморфизм, так как для любых $p_1, p_2: {p_1 < p_2 < p}$ изоморфизм, переводящий $\hat{p_2}$ в $\hat{q_2}$, переводит $p_1$ в некий $q_1:q_1 \prec q_2$. Значит, порядок сохраняется.

            Но тогда получается $\hat{p} \cong \hat{q}$, противоречие.

            \item Итак, $\dom f = P \land \rng f = Q$. В любом случае мы нашли изоморфизм между одним порядком, и подмножеством другого. А подмножество --- начальный сегмент: уже доказано, что $\dom f$ и $\rng f$ каждый если не совпадают с порядком, то являются начальными сегментами.
            }
        }
    }
    \newlection{4 октября 2022 г.}
    Комментарии к пункту 3 из предыдущей лекции:
    Для доказательства достаточно рассмотреть три случая: \emph{Хотя я пока не очень понимаю, почему недостаточно того, что написано выше}
    \numbers{
        \item $P, Q$ не имеют наибольшего элемента. Этот случай, как сказано на лекции, полностью покрывается приведённым выше доказательством
        \item Ровно один порядок, без потери общности, $P$ --- содержит наибольший элемент. Тогда у него есть несколько, из-за фундированности --- конечное число --- предшественников $p_0, p_1, \dots, p_n$, таких, что $\hat{p_n}$ не имеет наибольшего элемента.
        \item И $P$, и $Q$ содержат наибольший элемент. \dots
    }
    \note{Фундированное множество --- именно то множество, на котором можно использовать метод математической индукции. Для полного порядка $(P; <)$ определим множество $A \subseteq P$, такое, что \[\forall p \in P : (\hat{p} \subseteq A \then p \in A) \then A = P\]
        \provehere{От противного --- найти минимальный элемент в $P \bs A$.}}

    \subsection{Ординалы}
    \definition[Транзитивное множество $S$]{$\forall x, y: (x \in y \land y \in S) \then x \in S$.}\definition[Ординал или порядковое число]{Такое транзитивное множество $S$, что
        \begin{equation}
            \label{eq:ord}\forall x, y \in S: \any{x \in y\\y \in x \\ x = y}
        \end{equation}
        Несложно видеть, что из-за аксиомы фундированности (регулярности) возможно лишь одно из трёх.}
    Обозначим ординалы греческими буквами $\alpha, \beta, \dots$, и класс ординалов обозначим $\Ord$.

    Пусть $<$ --- сужение отношения $\in$ на $\Ord$.
    Иначе говоря, для $a, b \in \Ord$ вместо $a \in b$ будем (иногда) писать $a < b$.

    \subsubsection{Свойства ординалов}
    \numbers{
        \item $x \in \alpha \then x \in \Ord$
        \provehere{$\forall u \in v \in x:$ так как $\alpha$ --- ординал, то $u, v \in \alpha$, и для $u, v$ выполняется конъюнкция~\cref{eq:ord}. Кроме того, она выполняется для $u$ и $x$, откуда $u \in x$ (остальные альтернативы --- $x \in u \lor x = u$ --- вызывают противоречие с фундированностью).}
        \item $\alpha = \{\beta | \beta < \alpha\}$.
        \provehere{Оставлю, как упражнение. }
        \item Вполне упорядоченные множества изоморфны $(\alpha, <) \cong (\beta, <)$, если и только если они равны $\alpha = \beta$.
        \provewthen{Очевидно}{$(\alpha, <) \cong (\beta, <) \then \exists \text{ биекция } f : \alpha \map \beta$. Докажем, что $\forall x \in \alpha : x = f(x)$. \indent{Пусть, это не так. Возьмём наименьшее $x \ne f(x)$. Тогда $\forall z < x: f(z) = z$. $x = \{z \in \alpha|z < x\}$. С другой стороны, $f(x) = \defset{f(z)}{z \in \alpha \land z < x}$, откуда $x = f(x)$, противоречие.}Но тогда получается, что $\alpha \subseteq \beta$, а по симметрии --- $\alpha = \beta$.}
        \singlepage{
            \item $\alpha < \beta \lor \beta < \alpha \lor \alpha = \beta$.
            \provehere{Вытекает из теоремы о вполне упорядоченных множеств и предыдущего свойства.}
        }
        \item $\alpha \le \beta \iff \alpha \subseteq \beta$
        \provehere{Докажем, что $\alpha \in \beta \iff \alpha \subsetneq \beta$. В правую сторону очевидно, $\forall x \in \alpha : x \in \beta$ из транзитивности. Но $\alpha \ne \beta$, откуда $\alpha \subsetneq \beta$. В левую сторону --- $\alpha \in (\beta \bs \alpha)$, минимальный элемент разности.}
        \definition[Наименьший ординал, больший $\alpha$]{$\alpha + 1 \bydef \alpha \cup \{\alpha\}$.
        Несложно показать, что $\alpha \cup \{\alpha\}$ --- ординал, проверить транзитивность и конъюнкцию~\cref{eq:ord}.}
        \item $\nexists \beta \in \Ord : \alpha < \beta < \alpha + 1$. \provehere{От противного: $\beta \in \alpha \cup \{\alpha\}$. Либо $\alpha = \beta$, либо противоречие с аксиомой фундированности, так как $\alpha \in \beta$.}
        \item Любое множество ординалов $A$ вполне упорядоченно отношением $<$ (из п. 4), причём $\bigcup A = \sup A$.
        \provebullets{
            \item Любые два ординала сравнимы, причём если ординалы $x, y, z \in A$ и $x \in y \in z$, то $x \in z$.
            \item $\bigcup A$ --- ординал. Проверим транзитивность: $x \in y \in \bigcup A$. Но тогда $\exists \alpha \in A: y \in \alpha$. Тогда $x \in \alpha$ по транзитивности, откуда $x \in \bigcup A$.
            \item Покажем, что $\bigcup A$ --- верхняя граница $A$ по отношению $<$. $\forall \alpha \in A: \alpha \le \bigcup A$. Это всё равно, что $\alpha \subseteq \bigcup A$.
            \item Покажем, что $\bigcup A = \sup A$ --- наименьшая верхняя граница. Покажем, что для любой верхней границы $\beta: \bigcup A \le \beta$. Это верно, так как $\forall \alpha \in A : \alpha \subseteq \beta$.
        }
        \item Класс $\Ord$ не является множеством. \provehere{Пусть, является. Тогда $\alpha \coloneqq \bigcup \Ord = \sup \Ord$ --- наибольший ординал. Но тогда рассмотрим $\alpha + 1$.}
        \item Любое вполне упорядоченное множество изоморфно единственному ординалу.
        \provehere{
            Единственность очевидна, так как изоморфные ординалы равны.

            Рассмотрим вполне упорядоченное множество $(P; \sqsubset)$. Сначала заметим, что $\forall p \in P : \exists \alpha \in \Ord : \hat{p} \cong \alpha$. Это верно из принципа наименьшего элемента во вполне упорядоченных множествах --- для минимального $p$ такого, что $\nexists \alpha \cong \hat{p}$ подойдёт ординал $\bigcup\defset{\alpha \in \Ord}{\exists q \in \hat{p} : \alpha \cong \hat{q} }$.

            Теперь рассмотрим $M = \{\alpha | \exists p \in P : \alpha \cong \hat{p}\}$. Это множество по аксиоме замены. Но тогда $\bigcup M \cong P$.  }
    }
    \singlepage{
        \newlection{11 октября 2022 г.}
        \subsubsection{Типы ординалов}
    }
    \numbers{
        \item[0.] Нулевой ординал $\o$.
        \item Последовательные ординалы (последователи) --- ординалы вида $\{\alpha\} + 1$
        \item Предельные ординалы --- все остальные ординалы
    }
    \definition[Натуральное число]{Непредельный ординал, все элементы которого также не являются предельными. Множество натуральных чисел обозначается $\omega = \{0, 1, 2, \dots\}$.}
    \proposal{Множество $\omega$ существует.\provehere{По аксиоме бесконечности $\exists X : (\o \in X \land \forall x \in X: (x \cup \{x\}) \in X))$.

    Заметим, что $0 = \o \in X$.

    Теперь заметим, что $1 = \{0\} \cup \o \in X$.

    Можно доказать по индукции, что любое натуральное число содержится в $X$.

    Теперь воспользуемся аксиомой выделения, получим множество натуральных чисел \[\omega = \defset{x \in X}{x \text{ --- натуральное}}\]}}
    \definition[Конечное множество]{Множество, равномощное некоторому натуральному числу.}

    \subsubsection{Шкала ординалов}
    \[0, 1, \dots, \omega, \omega + 1, (\omega + 2 = (\omega + 1) + 1), \dots, (\omega \cdot 2 = \omega + \omega), \omega \cdot 2+ 1, \dots, \omega\cdot 3, \dots, \omega \cdot \omega, \dots\]

    \theorem[О рекурсивных определениях по ординалам]{
        Для любой функции-класса $G: V \map V$, где $V$ --- класс всех множеств, $\exist!$ функция-класс $F: \Ord \map V: F(\alpha) = G\left(F\big|_\alpha\right)$, где $F\big|_\alpha$ --- функция, ограниченная на $\alpha$, а именно, $F\big|_\alpha \bydef \defset{(\beta, y) \in F}{\beta < \alpha}$.
        Напоминание: $F(x) = y \overset{def}{\iff} (x, y) \in F$
    }
    \provebullets{
        \item Единственность: пусть существуют две такие функции $F, F'$. Утверждается, что $\forall \alpha \in \Ord: F(\alpha) = F'(\alpha)$. Предположим, что это не так, возьмём наименьшее $\alpha$ такое, что это не так. Тогда $F\big|_\alpha = F'\big|_\alpha$, откуда $F(\alpha) = F'(\alpha) = G\left(F\big|_\alpha\right)$, противоречие.
        \item Существование: рассмотрим некоторый класс функций \[C = \defset{f:\alpha \map V}{\alpha \in \Ord \land (\forall \beta < \alpha : f(\beta) = G\left(f\big|_\beta\right)}\]
        Заметим, что если $f, f' \in C$, то $f \subseteq f' \lor f' \subseteq f$. Утверждается, что искомая функция-класс \[F = \bigcup C\]
        В самом деле, можно заметить, что если некое $\alpha \notin \dom F$, то найдётся функция $H $, такая, что $\dom H = \alpha + 1$, определённая так: $H(\beta) = \all{F(\beta),& \beta < \alpha\\ G\left(F\big|_\beta\right),&\beta = \alpha}$.
    }


    \section{Эквивалентные формулировки аксиомы выбора}

    \subsection{О наибольшем и максимальном элементах в $(X, \sqsubset)$}

    \definition[Наибольший элемент]{Элемент $x \in X$ такой, что $\forall  y \in X : y \sqsubseteq x$. }
    \definition[Максимальный элемент в $(X, \sqsubset)$]{Элемент $x \in X$ такой, что $\nexists y : x \sqsubset y$.} В слове \emph{наибольший} есть подстрока <<больший>>, этот элемент, в отличие от максимального, действительно больше остальных.

    \subsection{Формулировки}
    \theorem[Лемма Цорна, принцип максимального элемента]{Если в частичном порядке ${(X, \sqsubset)}$ любое линейно-упорядоченное множество (любая цепь) имеет верхнюю границу, то в $X$ имеется максимальный элемент.}
    \theorem[Теорема Цермело, принцип полного упорядочивания]{
        Любое множество $A$ можно вполне упорядочить:
        \begin{gather*}
            \exists \text{ бинарное отношение }R \subseteq A \times A : (A, R) \text{ --- вполне упорядоченное множество}
        \end{gather*}
    }
    \theorem{
        Из аксиом $ZF$ следует эквивалентность следующих утверждений:
        \numbers{
            \item Аксиома выбора, $AC$.
            \item Лемма Цорна, $ZL$.
            \item Теорема Цермело, $ZT$.
        }\provebullets{
            \item $AC \then ZL$.

            Рассмотрим некоторое частично-упорядоченное множество $(X, \sqsubset)$, в котором любая цепь ограничена сверху. Докажем, что есть максимальный элемент от противного.

            $\forall x \in X : \exists y \in X : x \sqsubset y$. Рассмотрим $\mathscr{L} = \defset{L \subseteq X}{(L, \sqsubset) \text{ --- лум}}$. Определим $B(L) = \defset{y}{\forall x \in L: (x \sqsubset y)}$. Из посылки теоремы: $\defset{y}{\forall x \in L: (x \sqsubseteq y)} \ne \o$; пусть его элемент $y$. Тогда $B(L) \ne \o$ тоже, так как для $y$ существует $y' : y \sqsubset y'$, такой $y'$ уже строго больше всех элементов из $L$.

            Заметим, что $B: \mathscr{L} \map \left(2^X\bs \{\o\}\right)$. Также, по аксиоме выбора, есть функция ${f : \left(2^X \bs \{\o\}\right) \map X}$. Рассмотрим композицию этих функций $g = f \circ B : g(L) = f(B(L))$. Тогда заметим, что $\forall L \in \mathscr{L}, \forall x \in L : x \sqsubset g(L)$.

            Пусть $x_0 = g(\o); x_0 \in X$. Фактически, $x_0$ --- любой элемент из $X$. Построим некоторую функцию $F$ по рекурсии. Для этого сначала скажем, что \[G : V \map V, G(z) = \all{g(\rng(z)),&z \text{ --- бинарное отношение (множество пар), и } \rng(z) \in \mathscr{L}\\x_0,&\text{иначе}}\]
            Теперь определим $F : \Ord \map X;\quad F(\alpha) = g\left(\rng\left(F\big|_\alpha\right)\right) = g(\{F(\beta)|\beta < \alpha\})$. Здесь я пишу первую строчку из определения $G$, так как доказуемо для всех $\alpha \in \Ord : F(\alpha) \in \mathscr{L}$. Заметим, что $F$ --- инъекция, так как разные ординалы переходят в разные элементы. Отсюда $F^{-1} : X \map \Ord$ --- сюръекция. Тогда по аксиоме замены класс $\Ord$ является множеством, противоречие.

            \item $ZL \then ZT$.

            Пусть $A$ --- любое множество. Рассмотрим множество \[X = \defset{f: \alpha \map A}{\left(\alpha \in \Ord\right) \land \left(f \text{ --- инъекция}\right)}\]
            Удостоверимся, что $X$ --- множество: рассмотрим другое множество \[Y = \defset{(P; \sqsubset)}{P \subseteq A \text{ и $(P; \sqsubset)$ --- полный порядок}}\] $Y$ является множеством, так как $Y \subseteq 2^A \times (A \times A)$. Тогда утверждается, что всякому элементу $Y$ соответствует ровно один ординал $\alpha_p$. Несложно видеть, что тогда только множество этих $\{\alpha_p\}$ может быть областью определений функций из $X$.

            Утверждается, что для $(X, \subseteq)$ применима лемма Цорна: любое линейно-упорядоченное подмножество в $X$ ограничено сверху. В самом деле, для $L \in X: \bigcup L \in X$ и $\bigcup L$ --- верхняя граница. Значит, существует максимальный элемент в $X$. Обозначим $(u : \alpha \map A)$ --- максимальный элемент в $(X, \subseteq)$.

            Докажем, что $u$ --- ещё и сюръекция: пусть существует $y \in X$, такой, что $u^{-1}(y) = \o$. Но тогда рассмотрим $u' : (\alpha + 1) \map A; \quad u'(\beta) = \all{u(\beta),&\beta < \alpha\\y,&\beta = \alpha}$, противоречие с максимальностью $u$. Отсюда $u: \alpha \map A$ --- биекция. Тогда определим полный порядок на множестве $A$ следующим образом: $a < b \iff u^{-1}(a) < u^{-1}(b)$.
            \item $ZT \then AC$

            Докажем, что для любого $X : \exists f : \left(2^X\bs \{\o\}\right) \map X$, такая, что $f(S) \in S$. Для этого всего лишь найдём полный порядок по теореме Цермело, после чего возьмём минимальный элемент, пользуясь нашей операцией сравнения.
        }}
    \newlection{18 октября 2022 г.}


    \section{Сравнимость мощностей, шкала кардиналов, кумулятивная иерархия}
    \theorem{Для любых множеств $A$ и $B$ выполняется ровно одно из условий: $\any{A \cong B \\ A \prec B \\ B \prec A}$}
    \provehere{\label{aleb_and_blea_then_aeqb}Мы уже удостоверялись, что любые два условия не могут выполняться одновременно.

    По теореме Цермело, любое множество можно вполне упорядочить. Тогда рассмотрим полные порядки $(A; \sqsubset_A)$ И $(B; \sqsubset_B)$.

    Но тогда выполняется ровно одно из следующих условий: $\any{A \simeq B\\\exists! q\in B:A \simeq \hat{q} \\\exists! p \in A: B \simeq \hat{p}}$

        Отсюда очевидно, что есть либо инъекция из $A$ в $B$, либо --- наоборот --- инъекция из $B$ в $A$, либо вдруг даже биекция.
    }
    \definition[Мощность]{Мощность $|A|$ множества $A$ --- наименьший ординал, изоморфный $A$.}
    \definition[Кардинал]{Ординал, не равномощный никакому меньшему ординалу. Класс всех кардиналов обозначается $\Card$.}
    \note{$\omega + 1 \simeq \omega \cup \{\omega\}$ --- тоже счётное множество;\ $\omega + 1$ --- не является кардиналом.
    Более того, $\omega + \omega$ и даже $\omega \cdot \omega$ не являются кардиналами, они все счётны.
    }
    $\omega_1$ --- наименьший несчётный ординал.
    Из аксиом $ZFC$ не ясно, континуален ли $\omega_1$.
    \definition[Следующий кардинал]{Для кардинала $\varkappa$ существует $\varkappa^+$ --- наименьший ординал, больший $\varkappa$.}
    Определим $F$, используя рекурсию по ординалам: $F(\alpha) = \all{0,&\alpha = 0\\ F(\beta)^+,&\alpha = \beta + 1\\\sup\limits_{\gamma < \alpha} F(\gamma),&\alpha \text{ --- предельный ординал}}$
    Несложно видеть, и несложно доказать по индукции, что $F(\alpha) = \alpha$ для любого конечного $\alpha \in \omega$. $F(\omega) = \omega$, $F(\omega + 1) = \omega^+  = \omega_1$, $F(\omega + \omega) = \sup\{\omega, \omega^+, (\omega^+)^+, \dots\}$\ldots

    \proposal{
        Функция $F$ --- функция-класс, устанавливающая изоморфизм между классом ординалов $(\Ord, <)$ и классом кардиналов $(\Card, <)$.
        \provehere{
            Заметим, что $F$ возрастает, а именно, $\forall \alpha < \beta: F(\alpha) < F(\beta)$.
            Это несложно проверить.
            \bullets{
                \item Докажем по индукции, что $F(\alpha)$ --- кардинал для всякого $\alpha \in \Ord$.
                Достаточно убедиться про $F(\alpha)$, где $\alpha$ --- предельный, остальное очевидно.
                От противного: пусть $F(\alpha) \cong \delta$, где $\delta$ --- кардинал, меньший $F(\alpha)$.
                Есть два случая:
                \bullets{
                    \item $\forall \psi < \alpha: F(\psi) < \delta$.
                    В таком случае $F(\alpha) = \sup\limits_{\psi < \alpha}F(\psi) \le \delta$ и никак не может быть больше $\delta$.
                    \item $\exists \psi < \alpha: F(\psi) \ge \delta$.
                    Так как $\alpha$ --- предельный, то $\psi + 1 < \alpha$ тоже.
                    Но $F(\psi) \prec F(\psi + 1)$, откуда $F(\alpha) \prec F(\psi + 1)$.
                    Тогда получаем противоречие, ведь очевидно, что мощность $|F(\dots)|$ возрастает по мере возрастания аргумента.
                }

                \item Теперь проверим, что всякий ординал лежит в образе $F$.
                Опять же пойдём от противного: пусть наименьший ординал, не достигающийся функцией, равен $\delta$.

                Все меньшие достигались, обозначим $\mathcal{M}$ за прообраз всех меньших ординалов.

                Покажем, что $F(\sigma) = \delta$, где $\sigma$ --- наименьший элемент, не лежащий в $\mathcal{M}$.
                \bullets{
                    \item Если $\sigma$ --- предельный, то $F(\sigma) = \sup\limits_{\xi < \sigma}F(\xi)$, что не больше $\delta$.
                    \item Иначе $\sigma = \xi + 1$ для некоего ординала $\xi$. $F(\xi) < \delta$, значит по определению $F(\sigma) \le \delta$.
                }
                Но $\sigma \notin \mathcal{M}$, значит, $F(\sigma) = \delta$.

            }
        }
    }

    \subsection{Шкала бесконечных кардиналов}
    $\{\aleph_\alpha\}_{\alpha \in \Ord} \bydef F(\omega + \alpha)$.
    \definition[Сумма ординалов]{Сумма ординалов $\alpha + \beta$ --- ординал, изоморфный полному порядку $(P; <)$, где $P = \left(\alpha \times \{0\}\right) \cup \left(\beta \times \{1\} \right)$ и \[(x, i) < (y, j) \iff \all{x < y, & i = j\\i < j,& i \ne j}\] По сути, мы пририсовали к $\alpha$ справа $\beta$ и рассмотрели это как новый полный порядок.}

    \subsection{Кумулятивная иерархия}
    Определим рекурсией по ординалам $\{V_\alpha\}_{\alpha \in \Ord}$:
    \[V_\alpha = \all{0, &\alpha = 0\\ 2^{F(\beta)},&\alpha = \beta + 1\\\bigcup_{\gamma < \alpha}V_\gamma,&\alpha\text{ --- предельный ординал}}\]
    Эта последовательность $V_\alpha$ --- кумулятивная иерархия
    \theorem[Фон Нейман]{Всякое множество встретится в $V: \bigcup_{\alpha \in \Ord} = V$, где $V$ --- множество всех множеств.}

    \subsection{Арифметика кардиналов}
    $\varkappa^+$ --- наименьший кардинал, больший $\varkappa$

    $\varkappa + \lambda \bydef |\{\{0\}\times \varkappa\}\cup \{\{1\}\times \lambda\}|$

    $\varkappa \cdot \lambda \bydef |\varkappa \times \lambda|$

    $\varkappa^\lambda \bydef |\{f:\varkappa \map \lambda\}|$

    \subsubsection{Свойства}
    \numbers{
        \item Сложение и умножение коммутативны и ассоциативны
        \item Умножение дистрибутивно относительно сложения
        \item 0, 1 --- нейтральны относительно понятно чего
        \item $(\varkappa \cdot \lambda)^\mu = \varkappa^\mu \cdot \lambda^\mu$
        \item $(\varkappa^\lambda)^\mu = \varkappa^{(\lambda\cdot \mu)}$
        \item Нетривиальное свойство, с доказательством от Хаусдорфа: $(\varkappa^+)^\lambda = \varkappa^+ \cdot \varkappa^\lambda$
        \item $\aleph_\alpha + \aleph_\beta = \aleph_\alpha \cdot \aleph_\beta = \max\{\aleph_\alpha, \aleph_\beta\}$.
        Часть про произведение равносильно аксиоме выбора.
        \item $\alpha \le \beta \iff \aleph_\alpha^{\aleph_\beta} = 2^{\aleph_\beta}$
    }
    Видимо, первые 5 считаются очевидными, остальные --- нетривиальными.
    Как бы то ни было, на лекции не было ни одного доказательства\ldots

    \subsection{Арифметика ординалов}
    Сумма ординалов уже определена выше.

    Произведение ординалов: $\alpha \cdot \beta = (P, \sqsubset)$, где $P = \alpha \times \beta$ и \[(a, b) \sqsubset (a', b') \iff (b < b')\lor(b=b'\land a < a')\]

    Также операции можно определить рекурсивно:

    $\alpha + \beta = \all{\alpha, &\beta = 0\\ (\alpha + \gamma) + 1,&\beta = \gamma + 1\\\sup\limits_{\gamma < \beta}\alpha + \gamma,&\beta\text{ --- предельный}}$


    $\alpha \cdot \beta = \all{0, &\beta = 0\\ (\alpha \cdot \gamma) + \alpha,&\beta = \gamma + 1\\\sup\limits_{\gamma < \beta}\alpha \cdot \gamma,&\beta\text{ --- предельный}}$

    $\alpha^\beta = \all{1, &\beta = 0\\ (\alpha^\gamma) \cdot \alpha,&\beta = \gamma + 1\\\sup\limits_{\gamma < \beta}\alpha^\gamma,&\beta\text{ --- предельный}}$

    \subsubsection{Свойства}
    \numbers{
        \item $+, \cdot$ не коммутативны, но ассоциативны.
        \item $\cdot$ дистрибутивно слева относительно $+$ (но не справа).
        \item 0, 1 нейтральны относительно $\cdot, +$.
        \item $\alpha^\beta \cdot \alpha^\gamma = \alpha^{\beta + \gamma}$.
        \item $(\alpha^\beta)^\gamma = \alpha^{(\beta\cdot\gamma)}$.
    }
\end{document}
