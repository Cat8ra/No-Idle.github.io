\documentclass[a4paper]{report}
\usepackage{../mathstemplate}

\date{I семестр, осень 2022 г.}
\title{Дискретка. Неофициальный конспект}
\author{Лектор: Светалана Александровна Пузынина \and \\ Конспектировал Леонид Данилевич}

\begin{document}
    \maketitle
    \tableofcontents
    \newpage
    \setcounter{lection}{0}


    \chapter{Булевы функции}
    \newlection{1 сентября 2022 г.}


    \section{Введение в булевы функции}

    \emph{Нотация:}

    Ниже я пишу $f(g(x_i))$, где $f$ --- некая булева функция, $g$ --- булева функция одной переменной, имея в виду $f(g(x_1), g(x_2), \dots, g(x_n))$, где $n$ --- количество переменных, принимаемых $f$.

    $!$ --- логическое отрицание. $(x = y)$ --- функция равенства --- равна 1 $\iff x = y$.

    $f_?$ по умолчанию имеет $n$ переменных.
    \ok
    \definition[Булева функция ]{Функция $f: \{0; 1\}^n \map \{0; 1\}$}

    \subsection{Примеры}
    Бывает удобно интерпретировать 0 как ложь, 1 --- как истину.
    А именно, по этой причине основные функции имеют следующие названия:

    \bullets{
        \item \emph{Конъюнкция (логическое <<И>>)} --- результат истинен, если и один, \textbf{и} другой аргументы истинны.
        Обозначается $x \land y$, или $x @ y$, или $x \& y$ или $xy$ (без знака, как умножение (на деле $\land$ --- действительно умножение, умножение в $\Ff_2$)).
        \[x \land y = 1 \overset{def}{\iff} x = 1 \text{ и } y = 1\]

        \item \emph{Дизъюнкция (логическое <<ИЛИ>>)} --- результат истинен, если один, \textbf{или} другой аргумент истинен.
        Обозначается $x \lor y$, или $x | y$.
        \[x \lor y = 1 \overset{def}{\iff} x = 1 \text{ или } y = 1 \text{ (или и то, и то)}\]

        \item \emph{Импликация (логическое <<следует>>)}. Предположим, нам известно, что из $x$ следует $y$. Тогда результат импликации истинен, если возможна ситуация, когда $x$ имеет истинность $a$, в то время как $y$ имеет истинность $b$.
        Обозначается $x \then y$, или $x \to y$.
        \[x \to y = 1 \overset{def}{\iff} x = 0 \text{ или } y = 1 \text{ (или и то, и то)}\]

        \item \emph{Симметрическая разность (логическое <<либо-либо>>)} --- результат истинен, если либо один, \textbf{либо} другой аргумент истинен.
        Обозначается $x \xor y$, или $x \ne y$.
        \[x \xor y = 1 \overset{def}{\iff} x \ne y\]

        \item \emph{Отрицание (логическое <<не>>)} --- результат истинен, если аргумент ложен.
        Обозначается $!x$, или $\neg x$.
        \[\neg x = 1 \overset{def}{\iff} x = 0\]
    }

    \subsection{Основные эквивалентности}
    Истинность любой не слишком большой булевой формулы можно проверить перебором --- две функции равны, если их результат совпадает на любом наборе значений.
    \bullets{
        \item $\neg\neg x = x$
        \item $x \then y = \neg x \lor y$
        \item $x \then y = \neg y \then \neg x$
        \item $x \lor y = y \lor x$ --- коммутативность дизъюнкции
        \item $x \land y = y \land x$ --- коммутативность конъюнкции
        \item $(x \lor y) \lor z = x \lor (y \lor z)$ --- ассоциативность дизъюнкции
        \item $(x \land y) \land z = x \land (y \land z)$ --- ассоциативность конъюнкции
        \item $x \lor (y \land z) = (x \lor y) \land (x \lor z)$ --- дистрибутивность дизъюнкции относительно конъюнкции
        \item $x \land (y \lor z) = (x \land y) \lor (x \land z)$ --- дистрибутивность конъюнкции относительно дизъюнкции
        \item $\neg(x \lor y) = \neg x \land \neg y$ --- закон де Моргана
        \item $\neg(x \land y) = \neg x \lor \neg y$ --- закон де Моргана
    }

    \subsection{Базис}
    \definition[Базис]{Некоторое подмножество булевых функций $\Fc$.}
    \definition[Формула над базисом $\Fc$]{
        Определение по индукции:
        Во-первых (база), всякая функция $f \in \Fc$ является формулой над $\Fc$.
        Во-вторых (переход), для $\text{Ф}_1, \dots, \text{Ф}_n$, каждая из которых --- либо формула над базисом $\Fc$, либо переменная, формула $f(\text{Ф}_1, \dots, \text{Ф}_n)$, где $f \in \Fc$ --- функция, принимающая $n$ аргументов, является формулой над $\Fc$.
    }
    Так, $(x \lor y) \land y$ --- формула над базисом $\{\land, \lor\}$.


    \section{Способы задания булевых функций}

    \subsection{Таблица истинности}
    Булеву функцию можно задать таблицей истинности:
    \begin{table}[!ht]
        \centering
        \begin{tabular}{c | c}
            x & $\neg x$ \\
            \hline
            0 & 1        \\
            1 & 0        \\
        \end{tabular}
        \begin{tabular}{c c c}
            & &
        \end{tabular}
        \begin{tabular}{c c | c c c c}
            x & y & $x \land y$ & $x \lor y$ & $x \then y$ & $x \xor y$ \\
            \hline
            0 & 0 & 0           & 0          & 1           & 1          \\
            0 & 1 & 0           & 1          & 1           & 1          \\
            1 & 0 & 0           & 1          & 0           & 1          \\
            1 & 1 & 1           & 1          & 1           & 0          \\
        \end{tabular}
    \end{table}
    Если хочется, то строчки в таблице можно упорядочить в лексикографическом порядке (смотрим на первое различие, сравниваем там), как в таблицах выше.
    Тогда для описания функции достаточно только результатов, для $2^n$ возможных значений $n$ переменных.
    Это называется задание \emph{булевым вектором}, например, булев вектор функции логического <<И>> --- это \mono{0001}.

    \fact{
        Если все наборы $(\sigma_1, \dots, \sigma_n)$ пронумеровать в лексикографическом порядке, то номер очередного набора --- его дешифровка, как числа в двоичной системе счисления.
        Так, номер набора $(0, 1, 0, 1, 1)$ --- это $01011_2 = 11_{10}$.
        Это же можно записать в виде $\sum\limits_{i = 1}^{n}\sigma_i 2^{n-i}$.
    }

    \subsection{Дизъюнктивная нормальная формула}
    Будем обозначать $x^\sigma = \all{x, & \sigma = 1 \\ \neg \sigma, & \sigma = 0}$
    \definition[Простая конъюнкция]{
        Конъюнкция нескольких переменных, возможно, с отрицаниями.
        Каждая переменная встречается не более одного раза.
    }
    \definition[Дизъюнктивная нормальная форма, ДНФ]{
        Представление булевой формулы в виде дизъюнкции простых конъюнкций.
    }
    Так, $(x \land \neg y) \lor z$ --- дизъюнктивная нормальная форма.
    \definition[Совершенная дизъюнктивная нормальная форма, СДНФ]{
        Дизъюнктивная нормальная форма, в каждой конъюнкции которой --- все переменные данной формулы.
        Ещё можно потребовать, чтобы все конъюнкции были различны.
    }
    \note{
        У любой формулы есть совершенная дизъюнктивная нормальная форма;\ её можно построить по таблице истинности.

        А именно: для всякого набора переменных $(\sigma_1, \dots, \sigma_n)$ такого, что при данных значениях переменных результат функции истинен, добавить в текущую СДНФ (изначально пустую) конъюнкцию $(x_1^{\sigma_1} \land \dots \land x_n^{\sigma_n})$.

        Таким образом, СДНФ равна \[\bigvee\limits_{f(\sigma_1, \dots, \sigma_n) = 1}\left(x_1^{\sigma_1} \land \dots \land x_n^{\sigma_n}\right)\]

        Несложно убедиться, что данная СДНФ описывает именно данную булеву формулу.
    }

    \subsection{Конъюнктивная нормальная формула}
    \definition[Простая дизъюнкция]{
        Дизъюнкция нескольких переменных, возможно, с отрицаниями.
        Каждая переменная встречается не более одного раза.
    }
    \definition[Конъюнктивная нормальная форма, КНФ]{
        Представление булевой формулы в виде конъюнкции простых дизъюнкций.
    }
    Так, $(x \land \neg y) \lor z$ --- конъюнктивная нормальная форма.
    \definition[Совершенная конъюнктивная нормальная форма, СКНФ]{
        Конъюнктивная нормальная форма, в каждой дизъюнкции которой --- все переменные данной формулы.
        Ещё можно потребовать, чтобы все дизъюнкции были различны.
    }
    \note{
        У любой формулы есть совершенная конъюнктивная нормальная форма;\ её можно построить по таблице истинности.

        А именно: для всякого набора переменных $(\sigma_1, \dots, \sigma_n)$ такого, что при данных значениях переменных результат функции ложен, добавить в текущую СКНФ (изначально пустую) дизъюнкцию $(x_1^{\neg\sigma_1} \lor \dots \lor x_n^{\neg\sigma_n})$.

        Таким образом, СКНФ равна \[\bigwedge\limits_{f(\sigma_1, \dots, \sigma_n) = 0}\left(x_1^{\neg\sigma_1} \lor \dots \lor x_n^{\ne\sigma_n}\right)\]

        Несложно убедиться, что данная СДНФ описывает именно данную булеву формулу.
    }

    \subsection{Многочлен Жегалкина}
    Взаимоисключающее <<или>> конъюнкций (допускается слагаемое 1) без повторений слагаемых.

    Ещё его можно описать, как обычный многочлен над $\Ff_2$.

    Так, $f(x, y, z) = 1 \xor x \xor (x \land y \land z)$ --- многочлен Жегалкина.
    \emph{Иногда константу 0 не считают многочленом Жегалкина, но это что-то странное.}
    \theorem{Всякая функция имеет единственное представление многочленом Жегалкина
    \provebullets{
        \item \underline{Существование:} заметим, что $x \lor y = x \xor y \xor (x \land y)$. Ещё заметим, что $\neg x = (x \xor 1)$.
        Наконец заметим дистрибутивность $\xor$ относительно $\land$ --- $(x \xor y) \land z = (x \land z) \xor (y \land z)$, <<можно раскрывать скобки>>.

        Таким образом можно преобразовать ДНФ данной формулы, получив многочлен Жегалкина --- одинаковые слагаемые сокращаются, так как $x \xor x = 0$.

        \item \underline{Единственность:} применим количественный аргумент. Всего многочленов Жегалкина $2^{2^n}$ --- всякая конъюнкция может либо встретиться, либо нет.
        Но булевых функций столько же.
    }}


    \section{Замкнутые классы}
    Рассмотрим множество булевых формул $\Fc$.
    \definition[Замыкание $\Fc$]{
        Множество всех булевых функций, представимых формулами над $\Fc$.
        Обозначают $[\Fc]$.
    }
    Так, $[\o] = \o$; $[\neg] = \{\id, \neg\}$, $[\lor] = \defset{x_1 \lor \dots \lor x_n}{n >1}$.
    Пояснение к последнему примеру: $\lor$ --- функция двух аргументов, логическое <<ИЛИ>>.
    Замыкание класса, состоящего из этой функции, равно функции логического <<ИЛИ>> многих (хотя бы двух) переменных.
    \definition[Замкнутый класс]{ Класс, равный своему замыканию.}

    \subsection{Примеры замкнутых классов}
    \bullets{
        \item $T_0$ --- класс функций, сохраняющий $0$.
        \[f \in T_0 \iff f(0, \dots, 0) = 0\]
        Так, $\land, \lor, \xor, 0$ сохраняют $0$ (0 --- функция, возвращающая ноль при любых значениях аргумента; тождественный ноль).

        \item $T_1$ --- класс функций, сохраняющий $1$.
        \[f \in T_1 \iff f(1, \dots, 1) = 1\]
        Так, $\land, \lor, 1$ сохраняют $1$ (1 --- функция, возвращающая единицу при любых значениях аргумента; тождественная единица).

        \item \up \definition[Двойственная функция к $f$]{$f^*(x_1, \dots, x_n) = \neg f(\neg x_1, \dots, \neg x_n)$}
        Так, $\land$ и $\lor$ двойственны друг другу.
        \definition[Самодвойственная функция $f$]{Функция, двойственная сама себе: $f^* = f$.}
        Так, $\neg$ самодвойственно.

        Класс самодвойственных функций $S$ замкнут.
        \provehere{
            Для доказательства достаточно убедиться, что все одноэтажные формулы ($f(\text{Ф}_1, \dots, \text{Ф}_n)$, где $\text{Ф}_i$ --- либо функция из $S$, либо переменная) над $S$ самодвойственны.

            Тогда, строя формулу над $S$ по индукции, можно убедиться, что на каждом шаге будет самодвойственная функция.

            Утверждение проверяется так: самодвойственная функция --- функция, при замене истинности всех аргументов которой меняется результат.
            Но тогда при замене истинности всех аргументов $\text{Ф}_i$ на противоположную, истинность $\text{Ф}_i$ тоже изменится (если это переменная --- как истинность переменной, иначе --- как самодвойственная функция).
            Отсюда и истинность $f(\text{Ф}_1, \dots, \text{Ф}_n)$, где $\text{Ф}_i$ сменится на противоположную.
        }

        \item Введём частичный порядок на множестве двоичных наборов: $\left((x_1, \dots, x_n) \le (y_1, \dots, y_n)\right) \iff \forall i: (x_i \le y_i)$.
        \definition[Монотонная функция]{
            $f: \forall \alpha, \beta \in \{0, 1\}^n: \left(\alpha \le \beta\then f(\alpha) \le f(\beta)\right)$.
        }
        Иными словами, при изменении какого-то аргумента с лжи на истину, значение функции не может поменяться в обратную сторону.
        Так, $\lor$ и $\land$ --- монотонны, а $\neg, \xor, \then$ --- нет.

        Класс монотонных функций $M$ замкнут.
        \provehere{
            Аналогично, убедимся, что все одноэтажные формулы ($f(\text{Ф}_1, \dots, \text{Ф}_n)$, где $\text{Ф}_i$ --- либо функция из $M$, либо переменная) над $M$ монотонны.

            Тогда, строя формулу над $M$ по индукции, можно убедиться, что на каждом шаге будет монотонная функция.

            Несложно видеть, что при замене аргумента со лжи на истину, истинность $\text{Ф}_i$ не поменяется в обратную сторону, откуда и истинность $f(\text{Ф}_1, \dots, \text{Ф}_n)$ не поменяется в обратную сторону.
        }
        \item \up
        \definition[Линейная функция]{
            Функция, многочлен Жегалкина которой не использует нетривиальные (с хотя бы двумя переменными) конъюнкции.
        }
        Иными словами, $\xor$ нескольких переменных, и, возможно, 1.

        Класс линейных функций $L$ замкнут.
        \provehere{
            Аналогично, убедимся, что все одноэтажные формулы ($f(\text{Ф}_1, \dots, \text{Ф}_n)$, где $\text{Ф}_i$ --- либо функция из $L$, либо переменная) над $L$ монотонны.

            Тогда, строя формулу над $L$ по индукции, можно убедиться, что на каждом шаге будет линейная функция.

            Для доказательства просто заметим, что $\xor$ ассоциативен, коммутативен, и прочая, и прочая, откуда можно просто раскрыть скобки и получить опять же линейую функцию.
        }
    }
    \newlection{5 сентября 2022 г.}


    \section{Теорема Поста}
    \theorem[Пост, 1921]{
        Множество булевых функций $\Fc$ является полной системой $\iff$ $\Fc$ не содержится ни в одном из пяти классов $T_0, T_1, M, S, L$.

        \provetwhen{ Если $\Fc \subset A$ для некоего $A \in \{T_0, T_1, S, M, L\}$, то $[\Fc] = A$ и $A \ne U$}{

            \numbers{
                \item Рассмотрим $f_0 \notin T_0$.
                Тогда $f(0) = 1$.
                Если $f(1) = 1$, то $f(x) = 1$ и нами получена константа $1$.
                Иначе $f(1) = 0$, то $f(x) = ! x$, и нами получено отрицание.

                Аналогично для $f_1 \notin T_1$ мы получаем либо константу $0$, либо отрицание.

                Покамест нами получены либо обе константы, либо отрицание, либо и то, и то (из константы и отрицания получается и другая константа тоже).

                \item Пусть получено отрицание;\ получим константы.
                Рассмотрим $f_S \notin S$.
                Для некоего набора $\{\sigma_i\}$ верно $f_S(\sigma_i) = f_S(\neg\sigma_i)$.
                Тогда рассмотрим $f_S(x = \sigma_i)$ для некой переменной $x$.

                $\forall x \in \{0, 1\}$ $(x = \sigma_i) \in \{ \sigma_i, !\sigma_i\}$, откуда $f_S(x = \sigma_i)$ не зависит от $x$; это константа.
                С помощью отрицания получаем другую константу.

                \item Пусть нами получены константы;\ получим отрицание.
                Рассмотрим $f_M \notin M$.
                Существуют два набора переменных $\{\alpha_i\}, \{\beta_i\}: \alpha < \beta$, однако $f_M(\alpha_i) = 1 \land f_M(\beta_i) = 0$.

                Пусть $I = \{ i \in \N | \alpha_i \ne \beta_i\}, M_1 = \{ i \in \N | \alpha_i = 1 = \beta_i\}, M_0 = \{ i \in \N | \alpha_i = 0 = \beta_i\}$.
                Очевидно $I \sqcup M_1 \sqcup M_0 = \{i \in N | 1 \le i \le n\}$.

                Тогда $f_M\left(\begin{cases}
                                    1, &i \in M_1,\\ 0, & i \in M_0,\\ x, &i \in I
                \end{cases}\right) = !x$.
                Мы получили отрицание.

                \item Выразим конъюнкцию $(\land)$ из $f_L \notin L$.
                Рассмотрим представление $f_L$ через многочлен Жегалкина.
                В нём существует нетривиальная конъюнкция, имеющая хотя бы две переменные.
                Пусть они $x_j, x_k$.
                Несложно видеть, что всегда многочлен Жегалкина разбивается следующим образом: \[f_L(x_i) = \left(x_j \land x_k \land P(\dots)\right) \xor \left(x_j \land Q(\dots)\right) \xor \left(x_k \xor R(\dots)\right) \xor S(\dots)\] Здесь $P, Q, R, S$ --- булевы функции $n - 2$ переменных. $P$ не является константой $0$.

                Для некоего $\{\alpha_i\}_{i = 1}^{n-2}$: $P(\alpha_i) = 1$.
                Подставив $x = x_j, y = x_k$, остальные $x = \alpha$ мы получим $f_L(x_i) = (x \land y) \xor (x \land a) \xor (y \land b) \xor c$ для неких $a, b, c \in \{0, 1\}$.
                Тогда подставив вместо $x, y$ пару $x \xor b, y \xor a$, мы получим \[\left((x \xor b) \land (y \xor a)\right) \xor \left((x \xor b) \land a\right) \xor \left((y \xor a) \land b\right) \xor c = (x \land y) \xor (a \land b) \xor c\] Отсюда возможным отрицанием получаем чистую конъюнкцию.

            }
            Мы выразили константы, отрицание, конъюнкцию, значит, мы выразили базис.
        }
    }


    \chapter{Комбинаторика, выборки, числа Каталана}


    \section{Выборки}

    Пусть дано некое множество элементов $A$.

    \emph{Выборки} --- некие наборы $M$, такие, что $set(M) \subset A$.
    Выборки бывают \emph{упорядоченными} ($M$ --- упорядоченный массив) и \emph{неупорядоченными} ($M$ --- (мульти)множество), \emph{с повторениями} (в $M$ возможны одинаковые элементы) и \emph{без повторений} (все элементы в $M$ уникальны).

    Правила суммы и произведения я пропущу.

    Формулы:

    \begin{tabular}{| c | c | c |}
        \hline
        & Упорядоченные                           & Неупорядоченные                                         \\
        \hline
        С повторениями & $n^k$                                   & $\hat{C}_n^{k} = C_{n + k - 1}^{k} = \binom{n+k-1}{k}$  \\
        \hline
        Без повторений & Размещения $A_n^k = \dfrac{n!}{(n-k)!}$ & Сочетания $C_n^k = \binom{n}{k} = \dfrac{n!}{(n-k)!k!}$ \\
        \hline
    \end{tabular}

    Предлагается выводить формулу $\hat{C}_n^{k} = \binom{n + k - 1}{k}$ из бинарного вектора, а именно:
    \provebullets{
        \item $\hat{C}_n^{k}$ равно числу решений уравнения в неотрицательных числах $\sum\limits_{i = 1}^{n}x_i = k$.
        \item Закодируем некоторое решение $\{x_i\}_{1\le i \le n}$ двоичным вектором, где число $x_i$ кодируется $x_i$ единицами, а между числами стоит $0$. Заметим, что существует биекция всех двоичных строк длины $n + k - 1$ с $k$ единицами на решения вышеприведённого уравнения.
        \item Но таких строк $\binom{n+k-1}{k}$
    }

    Теорема. $\binom{n}{k} = \binom{n -1}{k - 1} + \binom{n - 1}{k}$.
    \begin{proof}
        $ \binom{n -1}{k - 1} + \binom{n - 1}{k} = \dfrac{(n-1)!}{(k-1)!(n-k)!} + \dfrac{(n-1)!}{k!(n-1-k)!} = \dfrac{(n-1)!k}{k!(n-k)!} + \dfrac{(n-1)!(n-k)}{k!(n-k)!} = \dfrac{(n-1)! \cdot n}{k!(n-k)!} = \dfrac{n!}{k!(n-k)!} = \binom{n}{k}$
    \end{proof}

    \subsection{Треугольник Паскаля}

    \emph{Извините, он у меня не треугольник}

    \begin{tabular}{|c|c|c|c|c|c|}
        \hline
        1 & 1 & 1  & 1     & 1     & 1     \\
        \hline
        1 & 2 & 3  & 4     & 5     & 6     \\
        \hline
        1 & 3 & 6  & 10    & 15    & 21    \\
        \hline
        1 & 4 & 10 & 20    & 35    & \dots \\
        \hline
        1 & 5 & 15 & 35    & \dots & \dots \\
        \hline
        1 & 6 & 21 & \dots & \dots & \dots \\
        \hline
    \end{tabular}

    \subsection{Бином Ньютона}

    $(a + b)^n = \sum\limits_{i = 0}^{n}\binom{n}{i}a^i b^{n-i}$.

    \begin{proof}
        Раскроем скобки.
        В итоговой сумме все слагаемые $a^k b^l$ удовлетворяют тождеству $k+l=n$; количество способов выбрать $k$ раз $a$, а остальные разы --- $b$ --- равно $\binom{n}{k}$
    \end{proof}

    \subsection{Оценки на факториал}

    Для $n > 1$

    \[\left(\dfrac{n}{e}\right)^n < n! < n^n\]

    Правое очевидно, докажем левое неравенство по индукции.

    \begin{proof}
        База: $n = 1$, $\left(\dfrac{1}{e}\right)^1 < 1$, верно.
        Переход: Пусть $\left(\dfrac{n}{e}\right)^n < n!$.
        Докажем, что $\left(\dfrac{n + 1}{e}\right)^{n+1} < (n+1)!$.

        Домножим обе части индукционного неравенства на $(n + 1)$. $(n + 1)\left(\dfrac{n}{e}\right)^n < (n+1)!$.
        Покажем, что $(n + 1)\left(\dfrac{n}{e}\right)^n > \left(\dfrac{n+1}{e}\right)^{n+1}$, или же --- $e\cdot n^n > (n+1)^n$.
        Это верно, так как $e > \left(1 + \frac{1}{n}\right)^n$, что мы узнаем в курсе матанализа.
        Отсюда переход верен и индукция верна.

    \end{proof}

    \subsection{Асимптотические оценки}

    $f(n) = o(g(n))$ по определению означает $\lim\limits_{n \map \infty}\dfrac{f(n)}{g(n)} = 0$. $f(n) = o(1) \iff \lim_{n \map \infty}f(n) = 0$.

    \subsubsection{Формула Стирлинга}

    \[n! = (1+ o(1))\sqrt{2\pi n}\left(\frac{n}{e}\right)^n\]

    Доказательство будет в курсе матанализа.
    \newlection{12 сентября 2022 г.}


    \section{Числа Каталана}
    \definition[Язык Дика]{
        Определён над алфавитом $\{\mono{(.)}\}$ (в такой нотации \mono{.} --- разделитель между символами). Правильной скобочной последовательностью ПСП называется

        пустая строка $\eps$,

        $(u)$ для строки над языком Дика $u$,

        $uv$, для строк $u$ и $v$ над этим языком.
    }
    \definition[Числа Каталана]{
        Числом Каталана $D_n$ называется количество строк над языком Дика длины $2n$.

        \url{http://oeis.org/A000108}
    }
    \theorem[Рекурсивная формула для чисел Каталана]{
        $D_n = \switch { 1,&n = 0\\ \sum\limits_{k = 0}^{n - 1}D_k D_{n - 1 - k},&n > 0}$
        \begin{proof}
            База очевидна.

            Рассмотрим произвольную непустую ПСП $w$.
            Она начинается с символа \mono{(}.
            Найдём парную ей скобку и назовём строку между скобками $u$, а строку после закрывающейся скобки $v$.
            Тогда $w = \mono{(}u\mono{)}v$.
            Заметим, что любая пара $(u, v)$ подходящих длин подходит.
            Формула отсюда очевидна.
        \end{proof}
    }

    \subsection{Числа Каталана через монотонные пути}
    Пусть дан квадрат $[0; n] \times [0; n]$.
    Назовём монотонным путём длины $n$ конечную последовательность длины $2n + 1$, состоящую точек $(x_i, y_i)$, таких, что $(x_1, y_1) = (0, 0)$; $(x_{2n+1}, y_{2n+1}) = (n, n)$; $\forall i : 0 \le i \le 2n : (x_{i+1}, y_{i+1})=\left[\begin{aligned}(x_i + 1, y_i)
                                                                                                                                                                                                                                               &\text{--- горизонтальный переход}\\(x_i, y_i+1)&\text{--- вертикальный переход}
    \end{aligned}\right.$

    Дополнительным условием на <<правильные>> монотонные пути поставим $\forall i:x_i \ge y_i$.
    Это обозначает, что все точки пути лежат ниже или на прямой $y = x$.

    \fact{Для $\forall$ префикса ПСП $s$: количество \mono{(} не меньше количества \mono{)} на этом префиксе. Доказывается по индукции.}

    Этот факт показывает возможность биекции монотонных путей длины $2n + 1$ и ПСП длины $2n$\ldots

    \theorem{Аналитическая формула для чисел Каталана} $D_n = \frac{1}{n+1}\binom{2n}{n}$.
    \begin{proof}
        Рассмотрим аналогичный путь, в котором все $y_i$ уменьшены на $1$.
        Тогда в новом пути $y_i < x_i \forall i$.
        Посчитаем количество путей из $(0; -1)$ в $(n; n - 1)$ как общее число монотонных путей за вычетом тех путей, у которых присутствует точка $(x_i, y_i) : x_i \ge y_i$.

        Общее число путей очевидно $\binom{2n}{n}$ --- среди $2n$ переходов выбрать $n$ горизонтальных.

        Очевидно, что в каждом неправильном пути с $\exists (x_i, y_i) : x_i \ge y_i$ присутствует точка $(a; a)$ для некоего $a$.
        Пусть $a$ --- первая точка пересечения пути с прямой $y = x$.
        Тогда отразим первую половину пути до $a$ от прямой $y = x$ и получим путь из $(-1; 0)$ в $(n; n - 1)$.
        Заметим, что существует биекция между неправильными путями (пересекающими $y = x$) из $(0, 0)$ в $(n; n)$ и путями из $(0; -1)$ в $(n-1;n)$, а последних $\binom{2n}{n-1}$ --- в каждом таком пути $n + 1$ горизонтальный переход и $n - 1$ вертикальный. \mono{// todo: нарисовать картинку}

        $D_n = \binom{2n}{n} - \binom{2n}{n-1} = \dfrac{(2n)!}{n!n!}-\dfrac{(2n)!}{(n-1)!(n+1)!} = \dfrac{(2n)!}{n!(n+1)!} = \frac{1}{n+1}\binom{2n}{n}$.
    \end{proof}

    \theorem[Асимптотика чисел Каталана]{
        $D_n = (1 + o(1))\dfrac{4^n}{n^{\frac{3}{2}}\sqrt{\pi}}$.
        \begin{proof}
            Использование формулы Стирлинга.
        \end{proof}
    }

    \subsection{Другие комбинаторные объекты}

    Есть очень много разумных комбинаторных объектов, количество которых для данного размера --- число Каталана.


    \chapter{Графы}


    \section{Вступление}
    Графическое представление --- вершины --- точки, рёбра --- линии.

    Морально --- объекты и связи между ними.

    История про Кенигсбергские мосты --- можно ли обойти все мосты, пройдя ровно по разу через каждый из мостов? \mono{\\todo: прикрепить картинку.} Данный граф --- мультиграф, так как присутствуют кратные рёбра, то есть некоторые вершины связаны дважды.
    Эйлеров обход графа --- путь, проходящий по каждому ребру графа ровно один раз.

    История про 3 дома, 3 колодца и тропинки между каждой парой разнотипных объектов --- является ли граф $K_{3,3}$ планарным?
    Можно ли граф $K_{3,3}$ расположить на плоскости, чтобы кривые, соответствующие рёбрам, не пересекались?

    История про знакомых юношей и девушек в деревне, требуется сыграть максимальное количество свадьб между знакомыми парами --- нахождение максимального паросочетания в двудольном графе.

    История о раскраске карты, чтобы граничащие по области ненулевой длины государства --- односвязные множества плоскости --- имели разный цвет.
    Минимальное количество цветов для раскраски? (Максимальное хроматическое число планарных графов).
    Вообще задача о раскраске графа --- нахождении хроматического числа.

    \subsection{Формально}
    Формально $(V; E)$, где $V$ --- \emph{как правило конечное} множество вершин, любое множество; $E \subset V \times V$.

    Граф можно задать квадратной матрицей смежности $a_{i,j}$ порядка $|V|$:

    $a_{i,j}=\switch{1,&(i,j)\in E\\0, &\text{otherwise}}$.

    \definition[(Не)ориентированный граф] {$(u,v) \in E \iff (v, u) \in E$. Иначе граф называется ориентированным.}
    \definition[Мультиграф]{Мультиграф задаётся матрицей смежности, здесь $a_{i,j}\in \N$.}
    \definition{Смежные вершины --- для неориентированного графа} Вершины $u, v$ --- смежные, если $(u, v) \in E$.
    \definition[Инцидентные вершина и ребро]{Ребро $(u,v)$ и вершина $u$ для некоего $v$.}
    \definition[Петля]{Ребро $(u,u)$ для некоего $u$.}
    \definition[Степень вершины $v: \deg(v)$] {$\deg(v) = \sum\{a_{v,u}|u \in V \} +a_{v,v}$.
    Иначе говоря, количество рёбер, инцидентных $v$, где петли считаются дважды.
    В ориентированных графах выделяют \emph{исходящую} и \emph{входящую} степени.}

    \subsection{Факты}
    \fact{В неориентированном графе сумма степеней всех вершин равна удвоенному числу рёбер.}
    \fact{В ориентированном графе сумма входящих степеней равна сумме исходящих степеней.}
    \fact{Всякий конечный граф содержит чётное число вершин нечётной степени.}

    \subsection{Пути}
    \definition[Путь]{
        Конечная последовательность $a_i$ нечётной длины $2k + 1$. $a_i \in \switch{V,& 2 \not|~ i\\E,&2~|~i}$
        Необходимое требование на путь: $\forall i \in \{2, 4, \dots, 2k\}$: $a_{i}$ --- ребро между $a_{i-1}$ и $a_{i+1}$.

        Различные вершины в пути $\then$ вершинно-простой путь.

        Различные рёбра в пути $\then$ рёберно-простой путь.

        Цикл --- путь с $a_1 = a_{2k+1}$. Цикл простой, если все его вершины, кроме $a_{2k+1}$ различны.

        Цикл --- рёберно-простой, если соответствующий путь рёберно-простой.

        \fact{Если существует путь, соединяющий $u$ и $v$, то существует и простой путь, соединяющий $u$ и $v$.}

        \definition[Связанные вершины]{
            Две вершины $u$ и $v$ называются связанными, если существуют пути из $u$ в $v$ и из $v$ в $u$. (В неориентированных графах очевидно достаточно существования одного, любого, пути). (В ориентированных графах наличие одного пути влечёт не сильную, обычную для ориентированных графов, связность, а так называемую слабую связность).
        }
        \fact{Связанность --- отношение эквивалентности}
        \definition[Компоненты связности]{Классы эквивалентности для отношение связанности. В ориентированном случае называются компонентами сильной связности.}
        Граф \emph{связный}, если в нём ровно одна (хотя бы одна) компонента связности. В ориентированном случае граф сильно связный.
        \lemma{Компонента связности --- связный граф}
        \provehere{
            Докажем для любых вершин компоненты $u$ и $v$, что существует путь между ними, использующий только вершины из данной компоненты.
            Так как $u$ и $v$  в одной компоненте связности, то существует путь в исходном графе между ними.
            Но все промежуточные вершины по определению связаны с $u$ и $v$, поэтому этот путь --- в компоненте связности.
        }
    }
    \newlection{15 сентября 2022 г.}


    \section{Эйлеровы и Гамильтоновы пути и циклы}

    \subsection{Эйлеров путь}
    \definition[Эйлеров путь]{Рёберно-простой путь, содержащий все рёбра графа.}
    \definition[Эйлеров цикл]{Эйлеров путь, являющийся циклом.}
    \theorem{Связный граф содержит эйлеров цикл $\iff$ все вершины в нём имеют чётную степень. Связный граф содержит эйлеров путь $\iff$ все вершины, кроме быть может ровно двух, в нём имеют чётную степень.}
    \provetwhen{
        Очевидно}{Индукция по числу рёбер. Индукционное предположение --- верность сразу обоих предположений, и про пути, и про циклы.

    Пусть верно для всех графов с $\le n$ рёбрами. Рассмотрим граф с $n + 1$ ребром.
    \bullets{
        \item
        Рассмотрим две вершины нечётной степени. Найдём между ними простой путь и удалим его. Граф, возможно, распадётся на компоненты связности, в каждой все степени вершин чётны. Значит, по предположению, во всех компонентах есть эйлеровы циклы (во всех компонентах рёбер не больше $n$).

        Для воссоздания эйлерового пути будем двигаться по найденному пути. Встречая вершину из необойдённой компоненты, обходим по эйлеровому циклу её компоненту, после чего продолжаем движение по пути.
        \item
        Если все вершины чётной степени, то будем удалять не путь, а цикл, всё аналогично. Формально --- упражнение читателю. \qedhere
    }
    }
    \theorem{Сильно связный ориентированный граф содержит эйлеров цикл $\iff$ каждая его вершина имеет равные степени захода и исхода. Для эйлерового пути --- все вершины таковы, кроме быть может двух, для которых одна имеет степень захода на 1 больше степени исхода, а другая --- на 1 меньше. }
    \definition[Граф де Брейна (de Bruijn) порядка $n$ для $k$-символьного алфавита $\Sigma$]{
        Множество вершин $V = \Sigma^n$

        Множество рёбер: у каждой вершины есть $k$ исходящих дуг.
        Слово из букв $w_1 w_2\dots w_n$ имеет $i$-ю ($1 \le i \le k$) из инцидентных дуг с вторым концом $w_2 w_3\dots w_n\Sigma_i$, здесь $\Sigma_i$ --- $i$-й символ алфавита.

        Иначе говоря, ориентированная дуга $(u, v)$ есть $\iff$ \mono{u[1; n) = v[0; n - 1)} для 0-индексированных строк.
        Здесь \mono{s[i; j)} означает подстроку строки \mono{s} с $i$-го символа включительно по $j$-й невключительно.}
    \theorem{В графе де Брейна есть эйлеров цикл.}\begin{proof}
                                                      В каждой вершине и входящая, и исходящая степень равны $k$.
    \end{proof}\corollary{Существует строка над алфавитом $\Sigma$ длины $k^{n+1}+n$, у которой все подстроки длины $n + 1$ различны; их множество совпадает с множеством всех строк длины $n + 1$ над алфавитом $\Sigma$.}

    Так, для $n = 2, k = 2$ в графе де Брейна $k^n = 4$ вершин и $k^{n+1} = 8$ рёбер.
    Существует строка длины $k^{n+1} + n = 10$, содержащая все строки длины $n + 1 = 3$ над двухсимвольным алфавитом по разу.

    Алгоритм построения строки де Брейна: \numbers{\item записывается произвольная начальная строка $w_0$. \item находится эйлеров цикл. \item Начиная с $w_0$ производится движение по циклу, символы, соответствующие посещённым рёбрам, дописываются в конец строки.}\mono{//todo: нарисовать картинку.}

    \subsection{Гамильтонов цикл}
    \definition[Гамильтонов цикл]{Простой цикл в графе называется гамильтоновым, если он проходит через каждую вершину ровно один раз.}
    \note{Вероятно, не существует полиномиального алгоритма о поиске гамильтонова цикла. $NP$-полная задача, никто, вероятно, не знает, существует ли алгоритм.}

    Достаточное условие существования гамильтонова цикла:
    \theorem[Дирак, 1952]{В графе $G = (V, E): |V| \ge 3$ существует гамильтонов путь (цикл), если сумма степеней любых двух вершин хотя бы $n - 1$ ($n$)}\prove{
        \lemma{В графе с $|V| = k \ge 3$ с гамильтоновым путём сумма степеней концов вершин хотя бы $k$. Тогда в графе существует гамильтонов цикл.
            \begin{proof}[Доказательство леммы]
                Пусть гамильтонов путь в данном графе содержит вершины в порядке следования $\{A_i\}_{1 \le i \le k}$.
                Пусть $adj_i$ --- множество вершин, смежных с $i$-й.
                Если существуют вершины $u \in adj_1, v \in adj_k : u + 1 = v$, то существует понятный гамильтонов цикл

                $1, 2, \dots, u-1, u, k, k-1, \dots, v + 1, v, 1$. \mono{//todo: нарисовать картинку}.
                Иначе $|adj_1| + |adj_k| < k$, противоречие.
            \end{proof}}
        Из леммы понятно, что достаточно доказать для пути.
        Рассмотрим самый длинный в данном графе простой путь $P$.
        Предположим, что он не гамильтонов: $|P| = k < n$.
        Рассмотрим подграф $H = \left(set(P), \{(u, v) \in E | u, v \in set(P)\}\right)$, где $set(P)$ --- множество вершин в пути.
        По предположению не существует рёбер из концов пути во внешний граф --- в любую из вершин $V \bs set(P)$.
        Тогда в графе $H$ по лемме существует гамильтонов цикл, так как $n - 1 \ge k$.

        Но цикл практически всегда можно удлиннить: если цикл проходит по вершинам $\{A_i\}_{1 \le i \le k}$, то любое ребро вида $(a_i, b)$ при условии $b \notin A$ можно добавить, как первое ребро нового пути.
        Получаем противоречие, так как мы нашли более длинный путь.

        Если же такого ребра не существует, то степени всех вершин пути не больше $k-1$, а степени остальных вершин не больше $n - 1 - k$.
        Противоречие с суммой степеней.
    }


    \section{Деревья}
    \definition[Дерево]{Связный граф без циклов}
    \definition[Лес]{Граф без циклов}
    \definition[Ориентированное дерево]{орграф (ориентированный граф) без циклов, в котором ровно одна вершина имеет степень захода 0, а остальные --- степень захода 1. Здесь вершина с нулевой степенью захода называется \emph{корень дерева}, а вершины с нулевой степенью исхода --- \emph{листьев}.}
    \definition[Мост]{Ребро, удаление которого увеличивает число компонент связности графа.}
    \theorem[Теорема о мостах]{Ребро является мостом $\iff$ оно не принадлежит ни одному (простому) циклу.}
    \begin{proof}
        $\then$ (Мост $\then$ не принадлежит циклам).
        От противного --- пусть ребро содержится в цикле.
        Покажем, что связность вершин $u$ и $v$ не зависит от наличия данного ребра.
        В самом деле, если путь не $ u - v$ не содержал это ребро, то он остался.
        Если пути между $u$ и $v$ не существовало, то он не появился.
        Иначе путь между $u$ и $v$ проходил по данному ребру, но тогда ребро можно обойти по остальным рёбрам цикла, содержащего ребро, противоречие.

        $\when$ (Не принадлежит циклам $\then$ мост).
        Изначально концы ребра $x$ и $y$ были в одной компоненте связности.
        От противного --- пусть после удаления ребра они остались в одной компоненте связности.
        Тогда существует путь между $x$ и $y$, не содержащий данное ребро.
        Значит, после добавления ребра получится цикл, противоречие.

        \mono{//todo: нарисовать картинку}
    \end{proof}

    \theorem{Следующие утверждения для (простого) графа равносильны:
    \numbers{
        \item $G$ --- дерево
        \item $\forall u, v \in V: $ существует ровно один простой путь из $u$ в $v$.
        \item $G$ не содержит циклов, но при добавлении любого несуществующего ребра теряет это свойство.
        \item $G$ --- связный граф и $|E| = |V| - 1$.
        \item $G$ не содержит циклов и $|E| = |V| - 1$.
        \item $G$ --- связный граф и каждое ребро является мостом.
    }}
    \definition[\text{$H$ --- остовный подграф $G$}]{$V_H = V_G; E_H \subset E_G$.}
    \definition[Остовное дерево]{ Остовный подграф, являющийся деревом.}
    \fact{Всякий связный граф содержит остовное дерево.
    \provehere{
        Пока в $G$ есть ребро, содержащееся в цикле, удалим его.
        Это не мост, граф останется связным.

        Теперь в $G$ нет циклов.
        Значит, это дерево.
    }}
    \corollary{Связный граф с $n$ вершинами содержит хотя бы $n - 1$ ребро.}
    \definition[\text{$H$ --- индуцированный подграф $G$}]{$V_H \subset V_G; E_H = \{(u, v) \in E_G | u, v \in V_H\}$.}


    \section{Изоморфность}
    \definition[Графы $G_1$ и $G_2$ изоморфны]{ Существует биекция $f : V_1 \map V_2: \forall u, v \in V_1 : \left((u, v) \in E_1 \iff (f(u), f(v)) \in E_2\right)$.}
    \fact{Проверка двух графов на изоморфность --- $NP$-трудная задача.}

    \newlection{19 сентября 2022 г.}


    \section{Планарные и плоские графы}
    \definition[Плоский граф]{Граф, существует укладка которого на плоскость, то есть существует отображение из вершин графа в точки плоскости, а из рёбер --- в кривые, с концами --- образами вершин, инцидентных ребру, причём кривые не должны пересекаться. Формальнее, рёбра можно отображать в ломаные с конечным числом звеньев.

    В сигнатуру плоского графа входит ещё множество областей, на которые образы рёбер разбивают плоскость --- граней. Неограниченная область тоже является гранью (называется внешняя грань). $G = (V, E, F)$. }
    \definition[Планарный граф]{Граф, изоморфный некоему плоскому графу.}
    \note{Плоский граф знает своё расположение на плоскости, а планарный граф --- нет.}

    \fact{Граф плоский /планарный $\then$ существует укладка графа на сфере.

    // Доказательство получается из стереографической проекции.}

    \subsection{Двойственные графы}
    Пусть дан плоский связный мультиграф $G = (V, E, F)$.
    \definition[Граф, двойственный $G$]{Новый плоский граф $H$.
    Грани графа $G \map \text{вершины графа $H$}$.
    В графе $H$ существует ребро между новыми вершинами, если в графе $G$ прообразы вершин --- грани --- имели общую границу ненулевой длины.}
    \fact{Для плоского графа $G$ граф $G^*$ --- тоже плоский; $(G^*)^* = G$. Я пропустил, было ли доказательство...}
    \note{Изоморфные плоские графы могут иметь неизоморфные графы, двойственные им. Так, например, графы <<бамбук длины 3>> и <<солнышко с тремя листьями>> имеют изоморфные двойственные им графы. \mono{//todo: нарисовать картинку.}}
    \theorem[Эйлер, 1758]{Во всяком плоском связном графе $|V| - |E| + |F| = 2$.
    \note{Для укладки не на плоскость, а на сферу с $k$ ручками выполняется $|V| - |E| + |F| = 2 - 2k$. Доказательства и даже точной формулировки, увы и ах, не будет...}
    }
    \begin{proof}[Доказательство случая для плоскости]
        Индукция по числу граней $|F|$.

        База: $|F| = 1 \then $ граф без циклов.
        Из связности следует, что это дерево, значит, действительно $|V| - |E| = 1$.

        Переход: Пусть $|F| \ge 2$.
        Удалим какое-нибудь ребро, входящее в цикл.
        Ребро, входящее в цикл, разделяло две грани, образы которых в двойственном графе были смежны.

        После удаления ребра количество и рёбер, и граней уменьшилось ровно на $1$.
        Итак, $(V, E, F) \mapsto (V', E', F') : \quad |V| = |V'|; ~|E'| = |E| - 1; ~|F'| = |F| - 1$, и мы можем воспользоваться индукционным предположением.
    \end{proof}

    \corollary{В планарном графе без петель и кратных рёбер $G = (V, E) : |E| \le \dfrac{l}{l - 2}(|V| - 2)$. Здесь за $l$ обозначается длина наименьшего цикла в графе.\begin{proof}

                                                                                                                                                                            Заметим, что $|F| \le \dfrac{2}{l}|E|$, так как каждой грани соответствует не менее $l$ рёбер, а каждому ребру --- ровно $2$ грани.
                                                                                                                                                                            Подставив в формулу Эйлера, получим $|E| = |V| + |F| - 2 \then \left(1 - \dfrac{2}{l}\right)|E| \ge |V| - 2$, откуда выражается требуемое.

    \end{proof}
    }
    \corollary{Во всяком планарном графе без петель и кратных рёбер есть вершина степени не более, чем $5$. \begin{proof}
                                                                                                                От противного: пусть степень любой вершины хотя бы $6$.
                                                                                                                Тогда несложно получить $|E| \ge 3|V|$, это противоречит предыдущему следствию.
    \end{proof}}
    \corollary{Графы $K_{3,3}$ (полный двудольный граф, размер каждой доли 3) и $K_5$ (полный граф на 5 вершинах) не являются планарными.}
    \theorem{Граф $G$ планарен $\iff$ любой его подграф не гомеоморфен ни графу $K_5$, ни графу $K_{3,3}$.
    \indent{
        \definition[Операция разбиения ребра $e$ в неориентированном графе $G$]{Удаление ребра $e$, создание новой вершины $w$, добавление в граф двух рёбер $(u, w)$ и $(v, w)$, где $e = (u, v)$.}
        \definition[Гомеоморфные графы]{Графы $G_1$ и $G_2$ гомеоморфны, если, применяя к \emph{каждому} из них произвольное количество раз операцию разбиения ребра, можно получить изоморфные графы.}
    }
    \note{В одну сторону доказательство очевидное --- если существует и укладка, и подграф, гомеоморфный запретному $K_5$ или $K_{3,3}$, то из укладки для данного графа можно извлечь укладку для $K_5$ или $K_{3,3}$, что приводит к противоречию.}}

    \subsection{Теорема о художественной галерее}
    Пусть дан произвольный $n$-угольник.
    Требуется расставить минимальное количество <<сторожей>> --- точек внутри многоугольника, чтобы всякая точка многоугольника была обозреваема каким-либо сторожем --- существовал отрезок из <<сторожа>> в данную точку, не выходящий за пределы многоугольника.

    \note{Для выпуклого многоугольника достаточно одного <<сторожа>> --- в любой точке данного многоугольника.}

    \theorem[Хватал, 1975]{Для любого $n \ge 3$ достаточно $\left\lfloor \frac{n}{3}\right\rfloor$ сторожей в некоторых его вершинах. }
    \begin{proof}
        Нижняя оценка --- гребёнка Хватала.
        Для $n = 3k$ необходимо минимум $k$ <<сторожей>> в некоем специфическом многоугольнике, называемом <<гребёнка Хватала>>. \mono{//todo: добавить картинку.}
        \indent{\lemma{Всякий многоугольник разбиваем диагоналями на треугольники. Граф с вершинами, совпадающими с вершинами многоугольника, и рёбрами, соответствующими диагоналям / сторонам, раскрашиваем в 3 цвета.}\provehere{
            По индукции.

            База: $n = 3$.
            Треугольник разбит и раскрашиваем.

            Переход: $n \ge 4$.
            Пусть $\angle A_{i - 1}A_i A_{i+1} < 180^o$. Такой угол есть, например, в качестве точки $A_i$ можно взять крайнюю в каком-то направлении.
            \bullets{
                \item
                Пусть отрезок $A_{i - 1}A_{i + 1}$ лежит внутри многоугольника. Тогда применим индукционное предположение для многоугольника $A_1\dots A_{i-1}A_{i+1}A_{|A|}$ --- многоугольника $A$ с выкинутой вершиной $A_i$, после чего покрасим $A_i$ в подходящий цвет (запрещены всего 2 цвета, цвета смежных с ней вершин).
                \item
                Пусть отрезок $A_{i - 1}A_{i + 1}$ не лежит внутри многоугольника. Тогда треугольник $A_{i - 1}A_i A_{i + 1}$ содержит некие вершины многоугольника. Сопоставим каждой такой вершине прямую, проходящую через неё, и параллельную $A_{i - 1}A_{i + 1}$. Возьмём среди всех вершин ту, у которой соответствующая прямая ближе всего к точке $A_i$, пусть это точка $A_j$. Утверждение: отрезок $A_i A_j$ лежит внутри многоугольника. \mono{// нарисовать картинку.} Тогда применим индукционное предположение для половинок и склеим их. \mono{написать подробнее}
            }
        }}
        По лемме строим разбиение многоугольника на треугольники, красим вершины в 3 цвета.
        Выбираем среди трёх цветов вершин тот, который используется нестрого реже (не чаще) остальных цветов.
        Ему соответствует $\left\lfloor\frac{n}{3}\right\rfloor$ вершин.
        Расставим сторожей именно в этих вершинах.
        Тогда каждый треугольник из триангуляции обозреваем.
    \end{proof}
    \theorem[Фари, 1948]{Для любого планарного графа существует отображение на плоскость, такое, что рёбра переходят в отрезки
    \newlection{26 сентября 2022 г.}
        \begin{proof}
% Предположим, что граф связен. Иначе применим теорему для всех его компонент связности.

            Докажем, что существует преобразование произвольной укладки в укладку, где рёбрам соответствуют отрезки, сохраняющее множество граней.

            Добавим в граф рёбра, пока все грани не станут треугольниками.
            Для жизнеспособности доказательства удалим добавленные рёбра после построения укладки.

            \indentlemma[О триангуляции]{
                Пусть $G$ --- планарный граф без петель и кратных рёбер. Тогда существует триангуляция $T$ --- плоский граф, каждая грань в котором --- треугольник (с возможно кривыми рёбрами), в котором $G$ содержится, как остовный подграф.
            }{
                Так как $G$ --- без петель и кратных рёбер, то существует грань $f$, на границе которой больше 3 вершин.
                \bullets{
                    \item Грань $f$ имеет связную границу.

                    Рассмотрим граф $H$, индуцированный на множестве вершин грани $f$. Он не является кликой${}^*$, можно добавить ребро (в $H$ могут быть рёбра между несоседними вершинами $f$, проведённые вне $f$).

                    ${}^*$ -- при $|V_H| \ge 5$ $H$ содержит индуцированный $K_5$, не являющийся планарным. Иначе $|V_H| = 4$ и надо разобрать случаи:
                    \numbers{ \item Граница --- треугольник и ребро вовнутрь. Тогда есть висячая вершина, можем провести ребро \item Граница --- четырёхугольник. Не могут без пересечений вне него быть проведены всевозможные рёбра.}

                    \item Граница грани $f$ несвязна. Тогда она содержит вершины из хотя бы двух компонент связности, можем соединить любые две из разных компонент.
                }
                В таком процессе на каждом шагу увеличивается количество рёбер, но мы поддерживаем граф простым; значит, если шаг возможен, то граф ещё не полный --- можно добавить ребро.
                Иначе доказательство завершено за конечное число шагов.
            }
            Индукция по числу вершин.

            База: $|V| = 3$ --- можем нарисовать подходящий треугольник.

            Шаг индукции: В планарном графе есть вершина $v : \deg v \le 5$.
            Докажем, что есть такая, не лежащая на внешней грани.
            \indent{От противного: пусть у всех $|V| - 3$ вершин не на внешней грани степень хотя бы $6$. Тогда сумма степеней всех вершин (внешняя грань связана с внутренностью, значит, на ней есть вершина степени хотя бы 3) $\sum\limits_v\deg(v) \ge 6(|V| - 3) + 3 + 2 + 2 = 6|V| - 11 > 6|V| - 12$, но мы знаем, что $2|E| \le 6|V| - 12$. Противоречие.}

            Рассмотрим такую вершину $v$.
            Из неё исходит $\deg v$ рёбер, $\deg v$ граней, примыкающих к (содержащих) $v$ являются треугольниками.
            Значит, $v$ лежит внутри некоего (не-более-чем-пяти)угольника. $\deg v \ge 3$, так как все грани являются треугольниками.

            Применим индуктивное предположение теоремы Фари к графу $G$ без вершины $v$ и инцидентных её рёбер.

            Дальше по теореме Хватала во всяком пятиугольнике есть вершина, которая <<обозревает>> весь пятиугольник.
            Расположим удалённую вершину $v$ в данной точке.
        \end{proof}}


    \section{Раскраски графов}

    \definition[Раскраска графа]{Раскраска графа цветами из множества $C$ --- функция $f : V \map C$. }
    \definition[Правильная раскраска графа]{Такая функция, что $(u, v) \in E \then f(u) \ne f(v)$. Я пишу граф красится в $k$ цветов $\equiv$ существует правильная раскраска при $|C| \le k$. }

    \numbers{
        \item Двудольный граф красится в два цвета.
        \item Полный граф на $n$ вершинах не красится в менее чем $n$ цветов.
    }
    \theorem[Хивуд]{
        Всякий планарный граф красится в $5$ цветов.
        \prove{
            Индукция по числу вершин.

            База: $|V| \le 5$, $f = \id$.

            Шаг индукции: рассмотрим вершину минимальной степени $v$. Удалим её и покрасим остальной граф. Дальше её надо добавить, сохранив правильность раскраски.

            Если $\deg v \le 4$, то её можно покрасить в тот цвет, которым не покрашен ни один из соседей $v$ --- смежных с ней вершин.

            Если $\deg v = 5$, то рассмотрим соседей вершины и пронумеруем их в порядке укладки на плоскости: $\{v_i\}_{i \in \{1, 2, 3, 4, 5\}}$. Добавим каждое ребро в случае его отсутствия --- между $v_i$ и $v_j$ если $i + 1 \underset{5}{\equiv} j$. Какой-то из диагонали $(v_i, v_j)$ нет для $i + 2 \underset{5}{\equiv} j$, так как $K_5$ непланарен.

            \definition[Склейка вершин в графе $G/uv$]{Отождествление вершин $u$ и $v$.
            В новом графе есть ребро, если между любыми прообразами концов данного ребра есть хотя бы одно ребро.}

            Склеим такие вершины $v_i$ и $v_j$, а ещё вершину саму $v$. Раскрасим склеенный граф, дальше $v_i$ и $v_j$ оставим такого цвета, а $v$ покрасим в какой-то пятый цвет, не совпадающий с цветом ни одного из её соседей.
        }
    }

    \intfact[Аппель, Хакен, 1977]{Всякий планарный граф красится в 4 цвета. Доказательство использует компьютерный перебор --- первое в истории такое доказательство. }
    \ok
    \definition[Хроматическое число графа $G$]{Минимальное число цветов, в которое красится граф $G$.
    Обозначается $\chi(G)$.}
    В частности, хроматическое число любого планарного графа не больше $4$.
    \definition[$k$-дольный граф]{Граф с хроматическим числом $k$}
    \fact{Граф двудольный $\iff$ он не содержит циклов нечётной длины.}

    \lemma{Если граф $H$ не красится в $k$ цветов, то он содержит индуцированный подграф, в котором степени всех вершин хотя бы $k$.}{\begin{proof}
                                                                                                                                          Удалим вершину $v : \deg v < k$.
                                                                                                                                          Покрасим граф и вернём её обратно.
                                                                                                                                          Нельзя покрасить, если и только если остался граф, где степени всех вершин $\ge k$.
    \end{proof}}
    \corollary{Пусть вершины $v_1, \dots, v_n$ графа $G$ пронумерованы так, что любая вершина $v_k$ имеет не более $d$ соседей среди вершин $v_1, \dots, v_{k - 1}$. Тогда граф красится в $d + 1$ цвет.}{\provehere{От противного: применим лемму, в индуцированном подграфе вершина с максимальным номером имеет степень не более $d$, противоречие.}}

    \theorem[Брукс, 1941]{
        В графе $G$ степени всех вершины $\le d$. Для $d \ge 3$: если ни одна из компонент связности $G$ не является полным графом, то $\chi(G) \le d$.
        \prove{
            Решаем для каждой компоненты связности независимо, предполагаем $G$ связным.

            Пусть $u, v$ --- несмежные вершины графа $G$. Рассмотрим следующие графы:
            \bullets{
                \item $G/uv$ --- склейка $G$ по вершинам $u$ и $v$.
                \item $G + (u, v)$ --- граф с добавленным ребром $(u, v)$.
            }

            Заметим, что если $G$ красится в $k$ цветов, то и хотя бы один из данных двух графов красится в $k$ цветов. (Рассмотреть случаи $u$ и $v$ одного цвета, либо разных)

            От противного: возьмём граф при наименьшем $|V|$, такой, что он не красится в $d$ цветов. По лемме в нём есть индуцированный подграф, где все вершины степени хотя бы $d$. Взяли минимальный граф, значит, в индуцированный подграф совпадает с ним самим, все вершины степени хотя бы $d$.

            Рассмотрим любую вершину $p$ степени $d$, у неё есть два соседа, несмежных между собой, так как $G \ne K_{d + 1}$. Тогда так как $G$ не красился, то и $G/uv$ равно как и $G + (u, v)$ не красятся в $d$ цветов.

            $G/uv$:
            \bullets{
                \item Не красится в $d$ цветов
                \item Граф связен
                \item $\deg p < d; \forall w \in V : w \ne uv: \deg w \le d$.
                \item В $G/uv$ по лемме есть индуцированный подграф, содержащий вершину $uv$ (иначе есть меньший контрпример), который не красится в $d$ цветов. Он не содержит вершину $p$, так как её степень меньше $d$.
                \item Все вершины степени $d$ в данным подграфе не имеют рёбер вовне, так как их степень в графе $G$ равна $d$. Единственной вершиной, могущий иметь рёбра вовне, является $uv$.
            }

            $G + (u, v)$:
            \bullets{
                \item Определим $H'$ как тот индуцированный подграф из $G/uv$, который мы нашли, только в $H'$ вершины $u$ и $v$ расклеены. Проведём в графе $H'$ ребро $(u, v)$.
                \item Определим $\tilde{H}$ --- индуцированный на $\left(V_G \bs V_{H'}\right)\cup \{u, v\}$ вершинах граф. В нём также проведём ребро $(u, v)$.
                \item $H' \cap \tilde{H} = (u, v)$.
                \item Цель --- покрасить $\tilde{H}$ и $H'$ в $d$ цветов каждый. Тогда склеивая эти раскраски по ребру мы получим раскраску графа $G$.
                \item Докажем, что все вершины в $\tilde{H}$ имеют степень не больше $d$. Для этого достаточно проверить, что $\exists w_u \in V_{H'}: (u, w_u) \in E$. И для $v$ надо найти аналогичную $w_v$. Но если бы для $u$ или $v$ таких рёбер не было, то в $H'$ все степени были бы не больше $d$, противоречие с минимальностью графа.
            }

            Тогда для $H'$ и $\tilde{H}$ выполняется утверждение теоремы, но раз граф $G$ не красится, то $H' = K_{d - 1} \lor \tilde{H} = K_{d - 1}$. Но в $H'$ вершины $u$ и $v$ не имеют общих соседей; в $\tilde{H}$ тоже что-то не так (пойму позже)
        }
    }
    \newlection{3 октября 2022 г.}


    \section{Паросочетания}
    \definition[Паросочетание]{Подмножество рёбер $M \subseteq E$, такое, что все рёбра не имеют общих концов. \emph{Совершенное паросочетание} --- паросочетание, в котором участвуют все вершины.}
    \theorem[Холл, 1935]{В двудольном графе $(G, V_1, V_2)$ существует паросочетание, покрывающее $V_1 \iff$ $\forall U \subset V_1: \left(\bigcup_{u \in U}\adj_u\right) \ge |U|$, где $\adj_u$ --- множество вершин, смежных с $u$. Очевидно, что для $u \in V_1: \adj_u \subset V_2$. \prove{\bullets{\item[$\then$] очевидно: если есть паросочетание, насыщающее $V_1$, то $\left(\bigcup_{u \in U}\adj_u\right) \subset \left(\bigcup_{u \in U}\{\pair_u\}\right) = |U|$, где $\pair_u$ --- соответствующая $u$ вершина в паросочетании.

    \item[$\when$] Докажем по индукции, по числу вершин в левой доле $|V_1|$.

    Предположим, что $\exists U_1 \subsetneq V_1: \left(\bigcup_{u \in U_1}\{\adj_u\}\right) = |U_1|$. \indent{Для $U_1$ выполняется предположение индукции, есть паросочетание в подграфе, индуцированном на $U_1 \cup \left(\bigcup_{u \in U_1}\{\adj_u\}\right)$.
    По условию теоремы $W = V_1 \bs U_1: \left(\bigcup_{u \in (U_1 \cup W)}\{\pair_u\}\right) \ge |V_1| - |U_1|$. Так как $\left(\bigcup_{u \in W}\{\adj_u\}\right) \cap\left(\bigcup_{u \in U_1}\{\adj_u\}\right) = \o$, то $\left(\bigcup_{u \in W}\{\adj_u\}\right) \ge |W|$ и по предположению индукции есть паросочетание, насыщающее и $W_1$ тоже.
    }
    Иначе $\forall U \subset V_1: \left(\bigcup_{u \in U}\adj_u\right) > |U|$. Будем удалять из графа рёбра, пока это верно. В какой-то момент наступит равенство, и можно будет применить предыдущее решение.
    }
    }}

    \subsection{Паросочетания в графах общего вида}
    Назовём компоненту связности \emph{нечётной}, если в ней нечётное количество вершин.
    Для $U \subset V_G$ обозначим $G \bs U$ --- индуцированный на $V \bs U$ подграф.

    \theorem[Татт, 1947]{В графе $G(V, E)$ есть совершенное паросочетание $\iff$ $\forall U \subset V: \odd(G\bs U) \le |U|$, где $\odd(H)$ --- количество нечётных компонент связности в $H$.

    В частности, для $U = \o: \odd(G) = 0$, то есть $G$ состоит из нескольких компонент связности из чётного количества вершин. \prove{\bullets{\item[$\then$] В графе существует паросочетание $\then$ все компоненты связности чётного размера. После удаления $|U|$ вершин станет не более $|U|$ нечётных компонент связности, так как нечётными могли стать только компоненты, содержащие $\pair_u$ для $u \in U$.

    \item[$\when$] Предположим, что совершенного паросочетания нет. Рассмотрим $\hat{G} = (V, \hat{E})$ --- такой граф, где $E \subset \hat{E}$, и $\hat{E}$ --- максимально по размеру, и таково, что совершенного паросочетания всё ещё нет. Иначе говоря, докинем рёбер, пока можно.

% В целях получения противоречия построим $U \subset V$ так, что $\odd(\hat{G}\bs U) > |U|$. Так как $\odd(G \bs U) \ge \odd(\hat{G} \bs U)$ (добавление ребра не увеличивает количество нечётных компоненты связности в графе), то это действительно приведёт к противоречию.

    Покажем, что на деле в $\hat{G}$ есть совершенное паросочетание.

    Положим $U = \defset{v \in V}{deg_{\hat{G}}v = |V| - 1}$ --- множество вершин, связанных со всеми в новом графе $\hat{G}$. Не исключено, что $U = \o$.
    \indent{
        \proposal{В графе $\hat{G} \bs U$ всякая компонента связности --- полный граф.
        \provehere{
            От противного: пусть есть $C$ --- компонента связности в $\hat{G} \bs U$, не являющаяся полным графом. Тогда $\exists u, v, w \in C: \left((u, w) \in \hat{E}\right) \land \left((v, w) \in \hat{E}\right) \land \left((u, v) \notin \hat{E}\right)$. (Это упражнение читателю, верно для любой компоненты связности, не являющейся полным графом).

            Так как $w \notin U$, то $\exists p \in V_1: (p, w) \notin \hat{E}$. По построению $\hat{G}$ при добавлении ребра $(p, w)$ появится совершенное паросочетание $M_1$, а при добавлении ребра $(u, v)$ появится совершенное паросочетание $M_2$. Посмотрим на рёбра из $M_1 \cup M_2$. Так как оба паросочетания совершенны, то каждой вершине инцидентны по два ребра из $M_1 \cup M_2$, то есть рёбра формируют циклы чётной длины из чередующихся $M_1$ и $M_2$ рёбер и изолированные рёбра, принадлежащие $M_1 \cap M_2$.

            Так как в $\hat{G}$ нет совершенного паросочетания, то $(u, v)$ и $(w, p)$ лежат в одном и том же цикле, образовавшемся на рёбрах $M_1 \cup M_2$ --- иначе можно перекомбинировать паросочетания $M_1$ и $M_2$ так, что появится совершенное паросочетание на $\hat{G}$. Вспомним, что $(w, u), (w, v) \in \hat{E}$. Можно заметить, что в таком случае паросочетания всё ещё можно перекомбинировать так, чтобы по-прежнему быть полным, но не использовать ни одно из рёбер $(u, v)$ и $(w, p)$. Противоречие с определением $\hat{G}$
        }
        }
    }
    Зная устройство графа $\hat{G}$, несложно построить в нём совершенное паросочетание --- чётные компоненты <<замкнём>> сами на себя, а к нечётным компонентам связности присоединим одну вершину из $U$. Это возможно, так как по условию $|U| \ge \odd(\hat{G} \bs U)$.
    }}}
    \theorem[Берж, 1958]{Наименьшее число вершин, непокрытых паросочетанием, равно дефекту графа $d(G) \bydef \max\limits_{U \subset V}\left(\odd(G \bs U) - |U|\right)$.}
    \provetwhen {
        Рассмотрим максимальное паросочетание $M$. Оно оставляет непокрытыми $m$ вершин. Очевидно, $\forall U \subset V: \odd(G \bs U) \le |U| + m$, так как в каждой нечётной компоненте связности должна быть либо непокрытая $M$ вершина, либо непарная --- с соседом в $U$. Неравенство перепишу как $d(G) \le m$ для любого паросочетания $M$.
    }{
        Добавим в граф $d(G)$ ($d(G)$ --- дефект графа) новых вершин $\{v_1, \dots, v_{d(G)}\}$, соединим каждую из них со всеми вершинами из $V_G$. Покажем, что полученный граф $G'$ удовлетворяет условию теоремы Татта, то есть содержит совершенное паросочетание. Для всякого $U' \subset V \cup \{v_1, \dots, v_{d(G)}\}$ рассмотрим два случая:
        \bullets{
            \item $\{v_1, \dots, v_{d(G)}\} \bs U' \ne \o$. В таком случае $G' \bs U'$ имеет ровно одну компоненту связности, для $U' = \o$ она ещё и чётная.
            \item Иначе $\{v_1, \dots, v_{d(G)}\} \subset U'$. Тогда $G' \bs U' = G \bs U$ для $U = U' \bs \{v_1, \dots, v_{d(G)}\}$. Заметим, что $|U| = |U'| - d(G)$. Применив определение $d(G) = \max\limits_{U \subset V}\left(\odd(G \bs U) - |U|\right)$, получаем, что $d(G) \ge \odd(G \bs U) - |U| = \odd(G' \bs U') - (|U'| - d(G))$, то есть $|U'| \ge \odd(G \bs U')$, что и требовалось.
        }
        Итак, в графе $G'$ есть совершенное паросочетание. Выкинув вершины $\{v_1, \dots, v_{d(G)}\}$ получаем в графе $G$ паросочетание, оставляющее непокрытыми не более $d(G)$ вершин. Но так как всякое паросочетание оставляет непокрытыми хотя бы $d(G)$ вершин (см. $\then.$), то $d(G)$ в точности равно количеству вершин, оставленных непокрытыми максимальным паросочетанием.
    }
    \definition[$k$-регулярный граф]{Граф, степень каждой вершины которого равна $k$.}
    \theorem[Петерсон, 1981]{Всякий $3$-регулярный граф без мостов содержит совершенное паросочетание.\provehere{Применим теорему Татта. Для этого рассмотрим произвольное $U \subset V$. Утверждается, что нечётных компонент связности в $G \bs U$ не более $|U|$. Пусть этих компонент $k$. Посчитаем количество рёбер, соединяющих вершины из $U$ и нечётные компоненты извне.

    С одной стороны, их не больше, чем $3|U|$ --- сумма степеней вершин из $U$.

    С другой стороны, их не меньше, чем $3k$ --- каждая нечётная компонента связности должна иметь хотя бы 3 ребра, соединяющих её и остальной граф. Этих рёбер должно быть нечётное число (следует из нечётности каждой степени вершины), а ещё больше одного, так как граф без мостов.

    Отсюда $k \le |U|$, и теорема Татта применима.}}

    \newlection{10 октября 2022 г.}

    \subsection{Устойчивое паросочетание}
    Рассмотрим полный двудольный граф $G(V_1, V_2, E)$.

    Зададим порядок $<$ на множестве рёбер для каждой конкретной вершины $v \in V_1 \cup V_2$.

    \emph{Неформально говоря, считаем, что пара $(u, v_1)$ предпочтительнее для $u$, чем пара $(u, v_2)$, если $(u, v_1) <_u (u, v_2)$.}

    \definition[Устойчивое паросочетание $M$]{ Паросочетание $M$, такое, что не существует ребра $(v_1, v_2) \in E\bs M$, для которого выполнены каждое из двух условий:
    \bullets{
        \item $(v_1, v_2) <_{v_1} (v_1, \pair_{v_1})$
        \item $(v_2, v_1) <_{v_2} (v_2, \pair_{v_2})$
    }
    }
    \theorem[Гейл, Шепли, 1962]{
        Во всяком полном двудольном графе $G = (V_1, V_2, E)$, для всяких предпочтений $\{<_v\}_{v \in V_1 \cup V_2}, |V_1| = |V_2|$ существует устойчивое паросочетание.
    }
    \prove{
        Построим какое-то нас устойчивое паросочетание, запустив алгоритм:

        Изначально паросочетание $M$ пусто.

        Заведём множество $\{A_v\}_{v \in V_1}: A_v \subset V_2$. Изначально $\forall v \in V_1: A_v = V_2$, потом множества будут уменьшаться.

        \emph{Можно думать об $A_v$, как о множестве вершин из $V_2 = \{v_{2,1}, v_{2,2}, \dots\}$ таких, что ранее данной стадии алгоритма ещё не было попытка провести ребро $(v_1, v_{2,i})$.}

        Пока паросочетание $M$ не максимально по размеру, рассмотрим любую вершину $v_1 \in V_1$, не насыщенную паросочетанием. Рассмотрим ребро $(v_1, u)$, где $u$ --- наиболее предпочтительная для $v_1$ вершина среди $A_{v_1}$ и удалим вершину $u$ из множества $A_{v_1}$; проведём это ребро если и только если вершине $u$ не инцидентно никакого ребра из текущего паросочетания, или же $(u, v_1) <_u (u, \pair_u)$.

        Алгоритм очевидно конечен (на каждом шаге $\sum\limits_v|A_v|$ уменьшается на 1), докажем его корректность:

        Рассмотрим пару $(v_1, v_2) \in E\bs M$. Если $(v_1, \pair_{v_1}) < (v_1, v_2)$, то пара не является причиной неустойчивости. Иначе $(v_1, \pair_{v_1}) > (v_1, v_2)$ и в какой-то момент была попытка провести ребро $(v_1, v_2)$, но оказалась (раньше или позже) такая вершина $v_1'$, что $(v_1', v_2) < (v_1, v_2)$. Тогда пара $(v_1, v_2)$ тоже не является причиной неустойчивости.
    }
    Свойства устойчивого паросочетания $M_{GS}$ (ниже $\pair_x$ относится к паре $x$ в данном паросочетании $M_{GS}$), найденного данным алгоритмом:

    \bullets{
        \item В $K_{n,n}$ образуется максимальное по размеру паросочетание $|M| = n$, которое является совершенным.
        \item Паросочетание <<самое хорошее>> для каждой $v_1 \in V_1$: пара $(v_1, \pair_{v_1})$ среди всех устойчивых паросочетаний для $v_1$ наиболее предпочтительная.
        \prove{
            Назовём пару $(x, y)$ возможной, если $\exists$ стабильное паросочетание, в котором данная пара входит в паросочетание.

            Докажем от противного: пусть есть определённая возможная пара $(x, y) \in V_1 \times V_2$ такая, что $\nexists y' \in V_2: (x, y') <_x (x, y)$, для возможной пары $(x, y')$, но $(x, y) \notin M_{GS}$. Существует паросочетание $M' \ni (x, y)$.

            Рассмотрим пары $(x, \pair_x) \in M$ и $(x, y) \in M'$. Пусть $y'$ --- пара $\pair_x$ в $M'$, то есть $\exists y' : (\pair_x, y') \in M'$. Из работы алгоритма $GS$ следует, что $(\pair_x, y') <_{y'} (x, \pair_x)$ и получаем противоречие со стабильностью $M_{GS}$.
        }
        \item Паросочетание <<самое плохое>> для каждой $v_2 \in V_2$: пара $(v_2, \pair_{v_2})$ среди всех устойчивых паросочетаний для $v_2$ наименее предпочтительная.
    }


    \section{Связность и разделяющие множества}
    Рассмотрим граф $G = (V, E)$.
    Множество $X \subset V$ назовём $(V_1, V_2)$-разделяющим, если в графе $G\bs X:\forall v_1 \in V_1 \bs X; \forall v_2 \in V_2 \bs X$ нет пути из $v_1$ в $v_2$, то есть $v_1$ и $v_2$ лежат в разных компонентах связности.
    \theorem[Геринг, 2000]{
        Пусть $V_1, V_2 \subset V; k \in \N$. Тогда ровно одно из двух условий верно:
        \numbers{
            \item $\exists U \subset V : |U| < k$ и $U$ --- $(V_1, V_2)$-разделяющее.
            \item Существует по крайней мере $k$ вершинно простых путей из $V_1$ в $V_2$, попарно не имеющих общих вершин.
        }
        \provebullets{
            \item Если верно $(2)$, то $\neg(1)$ очевидно --- надо удалить по крайней мере по одной вершине из каждого пути.
            \item Докажем импликацию $\neg (1) \then (2)$.

            По индукции:

            \underline{База:} $k = |U| = 1$, очевидна --- из $\neg(1)$ следует, что $V_1$ и $V_2$ не разделены пустым множеством, то есть $\exists (v_1, v_2) \in V_1 \times V_2: v_1 \text{ и } v_2$ соединены путём.

            \underline{Переход:} Пусть $\exists V_1, V_2 \subset V$ такие, что $\forall U \subset V: (U$ --- $(V_1, V_2)$-разделяющее$) \then (|U| \ge k)$. Будем считать, что $|V_1 \cap V_2| < k$, иначе $k$ разделяющих путей уже нашлись --- одновершинные.

            Будем удалять рёбра из графа до тех пор, пока в новом графе противное предположение всё ещё верно.

            Теперь при удалении любого ребра $(x, y) \in E'$: появляется $(V_1, V_2)$-разделяющее множество $U: |U| < k$. Отсюда видно, что до удаления $(x, y)$ множество $U \cup \{x\}$ --- разделяющее, равно как и $U \cup \{y\}$.
            \bullets{
                \item В случае $\{U \cup \{x\}, U \cup \{y\}\} = \{V_1, V_2\}$ в качестве $k$ путей можно взять $k - 1$ одновершинных путей $V_1 \cap V_2$, а ещё ребро $(x, y)$.

                \item В случае $U \cup \{x\} \notin \{V_1, V_2\}$ возьмём $W = U \cup \{x\}$. Иначе $U \cup \{y\} \notin \{V_1, V_2\}$, возьмём $W = U \cup \{y\}$. Теперь $|W| = k$, а ещё $W \notin \{V_1, V_2\}$, но $W$ --- $(V_1, V_2)$ разделяющее.

                $W$ --- разделяющее $\then$ все пути от $v \in V_1$ до $w \in W$ не заходят в $V_2$. Рассмотрим граф $G_1 = G \bs (V_2 \bs W)$. Граф уменьшился, так как $V_2 \bs W \ne \o$. Любое $(V_1, W)$-разделяющее множество в новом графе является $(V_1, W)$-разделяющим и в старом, поскольку вершины из $V_2 \bs W$ не лежат ни на каком пути из $W$ в $V_1$. Следовательно, в любом $(V_1, W)$-разделяющем множестве хотя бы $k$ вершин.

                По предположению индукции для графа $G_1$ имеется $k$ непересекающихся путей из $V_1$ в $W$.

                Аналогичными рассуждениями есть $k$ путей из $V_2$ В $W$. Но $|W| = k$, значит, данные пути можно <<склеить>> и получить $k$ путей из $V_1$ в $V_2$. \qedhere
            }}
    }
    \newlection{13 октября 2022 г.}
    \theorem[Менгер, 1927]{Пусть дан граф $G = (V, E)$. Пусть $a, b \in V$. Тогда наименьшее число вершин $(a, b)$-разделяющего множества (не включающего ни $a$, ни $b$) равно наибольшему числу непересекающихся по вершинам (за исключением $a$ и $b$) путей, соединяющих $a$ и $b$.\provehere{Применить теорему Геринга к графу, индуцированному на $V \bs \{a, b\}$, и применить к множествам соседей, $\adj_a$ и $\adj_b$.}}

    \definition[Вершинное покрытие]{Такое множество вершин, что каждое ребро содержит хотя бы одну их них (ребро рассматривается здесь, как множество инцидентных ей вершин).}
    \theorem[Кёниг, 1931]{Наибольшее число рёбер в паросочетании двудольного графа $G$ равно наименьшему числу вершин в вершинном покрытии.\provehere{
        Применим теорему Геринга к графу $G$ и множествам, являющимися левой и правой долями.
        Наибольшее количество путей --- наибольшее паросочетание. Наименьшее вершинное покрытие --- наименьшее разделяющее множество.
    }}


    \section{Рёберные раскраски}
    Для $C$ --- множества цветов
    \definition[Рёберная раскраска]{Отображение $c: C \map E$. Раскраска правильная, если $c(e) \ne c(e')$ для всех смежных рёбер $e$ и $e'$.}
    \note{Для любого цвета $c \in C$ множество рёбер цвета $c$ образует паросочетание.}
    \theorem[Кёниг, о раскраске рёбер]{В двудольном графе $G = (V_1, V_2, E)$ существует правильная раскраска рёбер в $D$ цветов, где $D = \max\limits_{v \in V_1 \cup V_2}\deg v$.}
    \provehere{
        Необходимость очевидна --- у вершины степени $D$ все инцидентные рёбра должны быть разноцветными.

        Достаточность: зафиксируем $D$.

        Будем доказывать по индукции, перебирая графы по параметру $d = \min\limits_{v \in V_1 \cup V_2}\deg v$.
        \bullets{
            \item База: $d = D$, $D$-регулярный граф.

            Для данного графа выполняется условие леммы Холла: $\forall U \subset V_1 \bigcup\limits_{u \in U} \adj_u \ge |U| \cdot \frac{|D|}{|D|}$. Значит, есть совершенное паросочетание.

            Удалим его (предварительно покрасив в цвет $D$), останется $D - 1$-регулярный граф. В нём покрасим паросочетание в цвет $D - 1$. И так $D$ раз.

            \item Переход: $d < D$. Пусть $G' = (V_1', V_2', E')$ --- копия $G$. Объединим $G$ и $G'$ в один граф $G'' = (V_1 \cup V_2', V_2 \cup V_1', E \cup E' \cup E_0)$, где $E_0 = \defset{(v, v')}{v \in V_1 \cup V_2 \land v' \text{ --- копия } v \land \deg v = d}$.

            В $G''$ наибольшая степень вершины всё ещё $D$, а наименьшая --- $d + 1$. По предположению индукции $G''$ красится $D$ цветов, значит, индуцированный на $V_1 \cup V_2$ подграф $G$ тоже красится в $D$ цветов.
        }
    }
    \theorem[Визинг, 1964]{В произвольном графе существует раскраска рёбер в $D + 1$ цвет, где $D$ --- наибольшая степень вершины.
    \prove{
        \indentlemma{
            Рассмотрим граф $G = (V, E)$. Пусть $v \in V$. Тогда при условиях \[\deg v \le k \land \forall u \in \adj_v : \left(\deg u \le k ~\land ~\#\defset{u \in \adj_v}{\deg u = k} \le 1\right)\] выполняется следующее: если рёбра графа $G \bs \{v\}$ можно покрасить в $k$ цветов, то и рёбра графа $G$ можно покрасить в $k$ цветов.
        }{Индукция по $k$.

        \underline{База:} $k = 1$. $v$ --- либо изолированная вершина, либо инцидентна изолированному ребру.

        \underline{Переход:} $m \coloneq \deg v; \{u_1, \dots, u_m\} = \adj_v$, причём $\deg u_1 \le k$, $\forall i = 2, \dots, m: \deg u_i < k$.

        Рассмотрим $c$ --- раскраску $G' = G \bs \{v\}$ в цвета $\{1, \dots, k\}$.

        Добавим в граф $G'$ новые рёбра от $u_i$ до новых вершин так, чтобы выполнялись равенства $\deg u_i = \all{k,&i=1\\k-1,&i\ge2}$.

        Обозначим $X_i = \defset{u \in \adj_v~}{~\text{В графе $G'$ вершине $u$ не инцидентно рёбер цвета $i$}}$.

        Зная степени вершин $u_1, \dots, u_m$ понимаем, что для $u_1$ $\exists! j : u_1 \in X_j$; для $u_i~(i \ge 2): \exists! \{j_1, j_2\}, j_1 \ne j_2: u_i \in X_{j_1} \land u_i \in X_{j_2}$.

        Тогда $\sum\limits_{i = 1}^k|X_i| = 2\deg v - 1 = 2m -1  < 2k$.

        Пусть $\exists i, j : |X_i| > |X_j| + 2$. Рассмотрим подграф $G_{i,j}'$ графа $G'$, образованный рёбрами цветов $i$ и $j$. В $G_{i,j}'$ всякая компонента связности --- путь или чётный цикл с чередующимися рёбрами цвета $i$ и $j$. Все вершины, кроме $X_i \cap X_j$, не являются изолированными в $G_{i,j}'$. Есть компонента связности в $G_{i,j}'$, в которой больше вершин из $X_i$, чем из $X_j$. Тогда это простой путь, начинающийся с ребра цвета $j$ в вершине из $X_i$, и не заканчивающийся ребром цвета $i$ в вершине из $X_j$. Поменяем в этом пути все вершины цвета $i$ на вершины цвета $j$ и наоборот. Одно такое перекрашивание увеличивает $|X_j|$ на $k$ и уменьшает $|X_i|$ на $k$, где $k$ --- количество вершин из $X_i$ на концах данного пути (1 или 2). Повторив это сколько надо раз, получим: теперь для всех $i, j: ||X_i| - |X_j|| \le 2$.

        Так как $\sum\limits_{i = 1}^k|X_i|$ нечётна, то $\exists i: |X_i| = 1$, так как есть нечётный, меньший 2 (если все хотя бы 2, то сумма хотя бы $2k$).

        Отсюда получается, что есть цвет $i: |X_i| = 1$. Пусть $X_i = \{\tilde{u}\}$. Всякой другой вершине $u_j \ne \tilde{u}$ инцидентно ребро цвета $i$ в графе $G'$. Построим граф $\tilde{G} = \left(V; \tilde{E}\right)$, где $\tilde{E} = E \bs \left(\{\left(v, \tilde{u}\right)\} \cup \{\text{Рёбра цвета $i$ в графе $G'$}\}\right)$. В графе $\tilde{G}$ степени $v$ и всех её соседей уменьшились на единицу $\then$ так как $\tilde{G}\bs \{v\}$ красится в $k-1$ цвет (той же раскраской, что $G'$ красится в $k$ цветов), то применимо индукционное предположение.

        А именно, $\tilde{G}$ красится в $k - 1$ цвет. Тогда вернём рёбра из $E_G \bs E_{\tilde{G}}$, покрасив их в цвет $k$.
        }
        Для графа $G = (V, E)$ обозначим $D = \max\limits_{v \in V}\deg v$. Пусть $U \subset V$. Докажем индукцией по $|U|$, что рёбра графа, индуцированного на $U$, красятся в $D + 1$ цветов.

        \underline{База:} $|U| = 1$, рёбер нет, любая раскраска годится.

        \underline{Переход:} Пусть $U = U' \sqcup \{v\}$. По предположению индукции рёбра между вершинами из $U'$ красятся в $D$ цветов. Так как все степени вершин не превышают (на деле строго меньше) $D + 1$, то по лемме граф, индуцированный на $U'$ тоже красится в $D + 1$ цвет.
    }}

    \subsection{Классы графа по отношению к теореме Визинга}
    \numbers{
        \item Графы, рёбра которых красятся в $D \bydef \max\limits_{v \in V}\deg v$ цветов.

        Некоторые из них перечислены ниже:
        \bullets{
            \item Двудольные графы.
            \item Планарные графы при $D \ge 7$.
            \item Почти все случайные графы
        }
        \item Графы, рёбра которых не красятся в $D \bydef \max\limits_{v \in V}\deg v$ цветов. Однако, по теореме Визинга, они красятся в $D + 1$ цвет.
        Некоторые из них перечислены ниже:
        \bullets{
            \item Некоторые планарные графы для $D \le 5$.
        }
    }
    \note{Вопрос, существует ли планарный граф с максимальной степенью вершин $D = 6$, рёбра которого красятся только в $D + 1$ цвет, является открытым.}
    \note{Алгоритм проверки, принадлежит ли данный граф первому классу, является $NP$-полной задачей. }
    \newlection{17 октября 2022 г.}


    \section{Теория Рамсея}

    \subsection{Числа Рамсея}
    Поиск регулярной структуры в, казалось бы, хаотическом устройстве.

    Так, в полном графе на шести вершинах есть либо треугольник, либо антитреугольник.
    \ok
    Рассмотрим гиперграф, в котором гиперрёбра соответствуют каждому множеству из $k$ человек, всякое гиперребро может быть двух цветов (\emph{знакомы-незнакомы}).

    Рассмотрим множество $M$ мощности $N$.
    Пусть есть произвольная покраска всех $k$-элементных подмножеств $M$ в $d$ цветов.

    \definition[Свойство Рамсея]{$N$ обладает свойством Рамсея $\mathcal{R}(k | m_1, \dots, m_d)$, если для всякой раскраски $M$ найдётся какое-то подмножество $A \subset M$ такое, что все его $k$-элементные подмножества покрашены в один и тот же из $k$ цветов (пусть это цвет $c$), и ещё $|A| = m_c$.}
    \definition[Число Рамсея]{$R(k; m_1, \dots, m_d)$ --- наименьшее натуральное число, удовлетворяющее свойству Рамсея $\mathcal{R}(k | m_1, \dots, m_d)$.}
    Пример: $R(2; 3, 3) = 6$.
    Можно проверить, например, полным перебором (проверить, что $6$ обладает свойством $\mathcal{R}(2 | 3, 3)$, а все меньшие натуральные числа --- нет).
    \theorem[Рамсей, 1930]{$R(k; m_1, \dots, m_d)$ существует (и конечно) для данных параметров $k, m_1, \dots, m_d$.}
    \provebullets{
        \item Найдём явно $R(1; m_1, \dots, m_d)$. Здесь красятся одноэлементные подмножества. По принципу Дирихле $\left(\sum\limits_{i = 1}^d m_i\right) - d + 1$ обладает свойством Рамсея $\mathcal{R}(1 | m_1, \dots, m_d)$. С другой стороны, все меньшие числа не обладают свойством --- контрпримером является $m_i - 1$ вершин цвета $i$ для $\left(\sum\limits_{i = 1}^d m_i\right) - d$, или сужение раскраски на меньшее множество.
        \item Если $\min\limits_{1 \le i \le d}m_i < k$, то $R(k; m_1, \dots, m_d) = \min\limits_{1 \le i \le d}m_i$. Достаточно выбрать подмножество мощности $\min\limits_{1 \le i \le d}m_i$: в нём все $k$-элементные подмножества имеют необходимый цвет (так как их нет).
        \item Будем доказывать существование чисел Рамсея индукцией, в порядке лексикографического возрастания $\left(k; \sum\limits_{i = 1}^d m_i\right)$.

        Считаем, что $\min\limits_{1 \le i \le d}m_i \ge k$. Обозначим $Q_i = R(k | m_1, \dots, m_i - 1, \dots, m_d)$. Эти числа существуют по индукционному предположению.

        Тогда утверждается, что $R(1; m_1, \dots, m_d) \le 1 + R(k - 1 | Q_1, \dots, Q_d)$. Обозначим данную оценку сверху $N \coloneqq R(k - 1 | Q_1, \dots, Q_d)$.

        Покажем, что $N$ обладает свойством Рамсея $\mathcal{R}(k | m_1, \dots, m_d)$: рассмотрим множество $M_N = \{1, \dots, N\}$ и произвольную покраску его $k$-элементных подмножеств в $d$ цветов. Построим другую покраску другого множества $M_{N-1}\{1, \dots, N - 1\}$: красить теперь будем $(k-1)$- элементные подмножества. Подмножество $A \subset M_{N-1}$ покрасим в тот же цвет, что и $A \cup \{N\}$ покрашено в множестве $M_N$.

        Но так как $|M_{N-1}| = R(k | Q_1, \dots, Q_d)$, то существует $B \subset M_{N-1}: |B| = Q_i$, в котором все $(k-1)$ элементные подмножества имеют цвет $i$. Но по определению $Q_i = R(k | m_1, \dots, m_i - 1, \dots, m_d)$. Значит, в $B$ содержится, как подмножество
        \singlepage{
            \bullets{
                \item либо для некоего $j \in \{1, \dots, d\} \bs \{i\}$ подмножество мощности $m_j$, все $k$-элементные подмножества которого имеют цвет $j$.
                \item либо подмножество мощности $m_i - 1$, все $k$-элементные подмножества которого имеют цвет $i$. Пусть оно $C$, тогда $C \cup \{N\}$ таково, что все его $k$-элементные подмножества имеют цвет $i$.  \qedhere
            }
        }
    }

    \subsubsection{Некоторые оценки на числа Рамсея}
    Назовём $R(n, m) = R(2 | n, m)$.
    \theorem[Верхняя оценка числа Рамсея]{$R(n, m) \le \binom{n+m-2}{n-1}$.\prove{
        Из доказательства теоремы Рамсея, имеем $R(2; n, m) \le 1 + R(1 | R(2, n - 1, m), R(2, n, m - 1)) = R(n - 1, m) + R(n, m - 1)$.

        Тогда из индукции действительно $\binom{n+m-2}{n-1} \le \binom{n+m-3}{n-2}+\binom{n+m-3}{n-1}$ (вообще говоря, в формуле равенство).
    }}
    \corollary{Асимптотическими оценками на биномиальные коэффициенты, получаем \[R(n, n) \le (1 + o(1))\dfrac{4^{n-1}}{\sqrt{2 \pi n}}\]}
    \theorem[Нижняя оценка на числа Рамсея]{
        \begin{gather*}
            R(n, n) \ge 2^{\nicefrac{n}{2}}
        \end{gather*}
        \provehere{
            Рассмотрим $n \ge 3$. Пусть $N < 2^{\nicefrac{n}{2}}$. Различных графов на $N$ вершинах $2^\frac{N(N-1)}2$. Ребро проведено --- первый цвет, ребра нет --- второй цвет.

            Покажем, что графов на $N$ вершинах, содержащих клику на $n$ вершинах, строго меньше, чем $\dfrac{2^{\frac{N(N-1)}{2}}}{2}$. Тогда очевидно, что графов, содержащих антиклику такое же количество, откуда есть граф на $N$ вершинах, не содержащий ни того, ни другого.

            Оценка на количество графов на $N$ вершинах, содержащих клику на $n$ вершинах:

            Для данных $n$ вершин есть $2^{\frac{N(N-1)}{2} - \frac{n(n-1)}{2}}$ графов, в которых эти $n$ вершин --- клика. Но тогда всего графов размера $N$, содержащих клику\[\binom{N}{n} \cdot 2^{\frac{N(N-1)}{2} - \frac{n(n-1)}{2}} < \frac{N^n}{n!}\cdot 2^{\frac{N(N-1)}{2} - \frac{n(n-1)}{2}}\]

            При $N < (n!)^\frac{1}{n} \cdot 2^{\frac{n-1}2-\frac1n}$ это действительно меньше, чем $2^{\nicefrac{N(N-1)}2}$. Так как $n! > 2^{\nicefrac{n}{2}+1}$ для $n \ge 3$, то $N = \lfloor2^{\nicefrac{n}{2}}\rfloor$ подойдёт.
        }}
    Открытая задача: для каких $\lambda$ верна асимптотическая оценка $R(n, n) > \lambda^{n+o(n)}$?

    \subsubsection{Применение чисел Рамсея}
    \theorem[Шур, 1917]{Если натуральный ряд покрашен в конечное число цветов, то уравнение $x + y = z$ имеет одноцветное решение.\prove{Рассмотрим полный граф, построенный на множестве вершин $\N$. Ребро $(i, j)$ покрасим в цвет $|i - j|$. По теореме Рамсея, в этом графе найдётся одноцветный треугольник, то есть $a < b < c: b - a, c - b, c - a$ одного цвета, откуда нашлось решение в виде $(c - a) = (c - b) + (b - a)$. }}

    \theorem[Фолькман --- Радо --- Сандерс]{Для натурального ряда, покрашенного в конечное количество цветов, найдётся сколь угодно большое конечное подмножество, такое, что суммы всех его подмножеств имеют один и тот же цвет.}
    \theorem[Hindman, 1974]{Для натурального ряда, покрашенного в конечное количество цветов, найдётся бесконечное подмножество, такое, что суммы всех его конечных подмножеств имеют один и тот же цвет.}
    \newlection{24 октября 2022 г.}
    \theorem{
        Для любого $k \in \N: \exists n \in \N: $ среди любых $n$ точек общего положения найдутся $k$, образующих вершины выпуклого многоугольника.
        \prove{
            \indent{
                \fact{
                    Среди любых 5 точек общего положения найдётся 4, формирующих вершины выпуклого четырёхугольника.
                    \provehere{
                        Пусть выпуклая оболочка данного набора точек содержит всего 3 точки, формируют треугольник ABC.
                        Тогда остальные две лежат внутри, посмотрим на прямую, проходящую через эти две точки.
                    }
                }
                \fact{
                    Если в наборе из $n \ge 4$ точек любые 4 формируют вершины выпуклого четырёхугольника, то все точки являются вершинами некоего выпуклого $n$-угольника.
                    \provehere{
                        От противного: пусть есть точка $P$, лежащая внутри выпуклой оболочки данного набора точек.
                        Тогда есть треугольник, содержащий данную точку, противоречие.
                    }
                }
            }
            В качестве подходящего $n$ рассмотрим $R(4 | 5, k)$. Здесь в первый цвет красятся четвёрки точек в невыпуклом положении, а во второй цвет --- четвёрки точек в выпуклом положении.

            Используя доказанные факты видим, что не может найтись пяти точек, любые 4 из которых в невыпуклом положении.
            Отсюда найдутся $k$ точек, любые 4 из которых в выпуклом положении.
            То есть по сути все точки, среди данных $k$, формируют вершины некоего выпуклого $k$-угольника.
        }
    }

    \subsection{Числа ван дер Вардена}
    \theorem[Ван Дер Варден, 1927] {
        Пусть натуральный ряд раскрашен в конечное количество цветов, $c$ цветов.
        Утверждается, что для данного $k \in \N$ определено число ван дер Вардена $W(k, c)$, такое, что среди первых $W(k, c)$ натуральных чисел найдётся одноцветная прогрессия длины $k$.
        \provebullets{
            \item \emph{Спойлер: все пункты данного списка, кроме последнего, бесполезны.}
            \item $W(1, c) = 1$ --- ищем арифметическую прогрессию длины 1.
            \item $W(k, 1) = k$ --- ищем арифметическую прогрессию длины $k$, один цвет.
            \item $W(2, c) = c + 1$ --- ищем арифметическую прогрессию длины $2$, по сути, два одноцветных элемента.
            \item $W(3, 2) - ?$. Перебор показывает, что это 9, докажем без перебора худшую оценку.
            \lemma{\label{a_lemma_1}
            Рассмотрим блок из пяти подряд идущих чисел $B = \{n, n + 1, \dots, n + 4\}$.
            Утверждается, что существуют $a, d: \all{d > 0 \\ \{a, a + d, a + 2d\} \subset B \\ \chi(a) = \chi(a + d)}$.
            \provehere{Среди первых трёх $\{n, n + 1, n + 2\}$ есть два одноцветных, возьмём их в качестве $a$ и $a + d$.}
            }
            Разобьём натуральный ряд на блоки размера $5: (1, 2, 3, 4, 5), (6, 7, 8, 9, 10), \ldots$
            \indent{
                \lemma{\label{dirichlet_lemma_1}
                Среди $165 = 5 \cdot (2^5+1)$ подряд идущих чисел найдутся два одинаково раскрашенных блока $B_i, B_j$, раскрашенных одинаково:
                    \[B_i = \{5i + 1, \dots, 5i + 5\}; \quad B_j = \{5j + 1, \dots, 5j + 5\}; \quad \all{\chi(5i + 1) = \chi(5j + 1)\\ \vdots \\ \chi(5i + 5) = \chi(5j + 5)}\]
                    \provehere{По принципу Дирихле.}
                }
            }
            Используя (\cref{dirichlet_lemma_1}), найдём два одинаково раскрашенных блока $B_i, B_j ~(i < j)$ среди первых $165$ натуральных чисел.

            Согласно (\cref{a_lemma_1}), найдутся $a, d$: $\chi(5i + a) = \chi(5i + a + d)$.
            Если ещё и $\chi(5i + a) = \chi(5i + a + 2d)$, то одноцветная прогрессия нашлась.
            Значит, элемент $\chi(5i + a + 2d)$ другого цвета. Тогда либо $\chi(5i + a), \chi(5j + a + d), \chi(5(2j - i) + a + 2d)$, либо
            $\chi(5i + a + 2d), \chi(5j + a + 2d), \chi(5(2j - i) + a + 2d)$ формируют одноцветную прогрессию.

            Так как $5(2j - a) + (a + 2d) \le 5(2 \cdot 32 - 0) + 5 = 325$, то мы получили оценку $W(3, 2) \le 325$.

            \item $W(3, 3) - ?$. Перебор показывает, что это 27;\ докажем без перебора худшую оценку.

            \lemma{\label{a_lemma_2}
            Рассмотрим блок из семи подряд идущих чисел $B = \{n, n + 1, \dots, n + 6\}$.
            Утверждается, что существуют $a, d: \all{d > 0 \\ \{a, a + d, a + 2d\} \subset B \\ \chi(a) = \chi(a + d)}$.
            \provehere{Среди первых четырёх $\{n, n + 1, n + 2, n + 3\}$ есть два одноцветных, возьмём их в качестве $a$ и $a + d$.}
            }
            Разобьём натуральный ряд на блоки размера $7: (1, 2, 3, 4, 5, 6, 7), (8, 9, 10, 11, 12, 13, 14), \ldots$
            \indent{
                \lemma{\label{dirichlet_lemma_2}
                Среди $889 = 7 \cdot (2^7+1)$ чисел найдутся два одинаково раскрашенных блока $B_i, B_j$, раскрашенных одинаково:
                    \[B_i = \{7i + 1, \dots, 7i + 7\}; \quad B_j = \{7j + 1, \dots, 7j + 7\};\quad \all{\chi(7i + 1) = \chi(7j + 1)\\ \vdots \\ \chi(7i + 7) = \chi(7j + 7)}\]
                    \provehere{По принципу Дирихле.}
                }
            }
            Согласно (\cref{a_lemma_2}), найдутся $a, d$: $\chi(7i + a) = \chi(7i + a + d)$.
            Если ещё и $\chi(7i + a) = \chi(7i + a + 2d)$, то одноцветная прогрессия нашлась.
            Значит, считаем, что элемент $\chi(7i + a + 2d)$ другого цвета.

            Среди $7 \cdot (2 \cdot 2^7 + 1) = 1771$ натуральных чисел мы знаем, что $\chi(7i + a) = \chi(7j + a + d)$;
            кроме того, $\chi(7i + a + 2d) = \chi(7j + a + 2d)$, а ещё эти цвета разные.
            Таким образом, мы нашли две арифметические прогрессии, имеющие общую точку цвета $3$, цветовых типов $(1,1,3)$ и $(2,2,3)$.

            Пусть $U = 1771$. Разобьём первые $W$ натуральных чисел на блоки из $U$ подряд идущих ($W$ определим позднее).
            Каждый такой блок может быть раскрашен не более, чем $3^U$ способами.
            Рассмотрим раскраску $\left\lfloor \frac{W}{U} \right\rfloor$ блоков, каждый блок длины $U$.

            Хотим, чтобы нашлись два одинаковых блока, для этого возьмём $W = U \cdot (3^U + 1)$.
            Эти два блока, $B_i$ и $B_j$ $(i < j)$ каждый содержат арифметические прогрессии цветовых типов $(1,1,3)$ и $(2,2,3)$ с общей точкой цвета $3$.
            Пусть данные прогрессии с общей точкой имеют индексы внутри блоков $a, d, g$ и $e, f, g$ соответственно.

            Теперь надо увидеть, что среди $2W$ чисел точно найдётся одноцветная прогрессия длины $3$.
            Ну, посмотрим, какого цвета точка $(2j - i)U + g$.
            \bullets{
                \item  Если цвета $1$, то нашли прогрессию $iU + a, jU + d, (2j - i)U + g$.
                \item  Если цвета $2$, то нашли прогрессию $iU + e, jU + f, (2j - i)U + g$.
                \item  Если цвета $3$, то нашли прогрессию $iU + g, jU + g, (2j - i)U + g$.
            }
            Получили оценку $W(3, 3) \le 2W = 2U \cdot (3^U + 1) = 2 \cdot 1771 \cdot (3^{1771} + 1)$.
            \item $W(4, 2) - ?$. Перебор показывает, что это $35$;\ докажем без перебора худшую оценку (её я не смогу привести в численном виде, так как она основана на оценке на число $W(3, 2^{2W(3, 2)})$).

            Разобьём натуральный ряд на блоки длины $U \coloneqq 2W(3, 2)$. По определению $W(3, 2)$ в каждом таком блоке есть одноцветная прогрессия длины $3$.
            В блоке длины $U$ есть прогрессия длины $4$ цветового типа $(1,1,1,2)$.

            По определению $W(3, 2^{U})$ среди $U \cdot W(3, 2^{U})$  натуральных чисел найдутся три блока длины $U$, имеющих один тип, да ещё и формирующих прогрессию.

            Пусть это блоки $i < j < k$; пусть индексы прогрессии типа $(1,1,1,2)$ --- это $a, b, c, d$.

            Тогда одноцветную арифметическую прогрессию формируют либо \[iU + a, jU + b, kU + c, (2k - j)U + d\text{, либо }iU + d, jU + d, kU + d, (2k - j)U + d\]

            Получили оценку $W(4, 2) \le 2(U \cdot W(3, 2^U)) \le 4W(3, 2) \cdot W(3, 2^{2W(3, 2)})$ (дополнительное домножение на двойку нужно для того, чтобы $(2k - j)$ попало внутрь).

            \item \textbf{Общий случай\ldots}

            Зафиксируем $k, c \in \N$, оценим сверху $W(k, c)$. Будем действовать по индукции, по $k$.

            \underline{База:} $k = 1$, убедились, что $W(1, c) = 1$.

            \underline{Переход:} Зафиксируем $c$.
            Считаем, что $W(k - 1, c)$ уже ограничено сверху для любого $c$.

            \indentlemma{
                Для всякого $n \in \N$ найдётся такое $U(n):$ во всякой раскраске в $c$ цветов $\{1, \dots, U(n)\}$:

                если нет $n$ прогрессий длины $k$, имеющих общую точку цвета $n+1$, таких, что $i$-я из них имеет цветовой тип $(\underbrace{i,\dots,i}_{k-1},n + 1)$, то найдётся одноцветная прогрессия длины $k$.
                Иными словами --- если нет структуры, где в каждой из $n$ прогрессий сначала идут $k - 1$ число одного цвета, а потом число другого цвета (общее для всех прогрессий), причём цвета чисел в начале разные для всех $n$ прогрессий --- то найдётся одноцветная прогрессия длины $k$.
            }{
                По индукции, по $n$.

                \underline{База:} $n = 1$.
                Здесь $U(n) = 2W(k - 1, c)$: среди такого количества чисел найдётся одноцветная прогрессия длины $k - 1$ в первой половине.
                Если её $k$-й элемент окажется такого же цвета, то следствие импликации верно.
                Иначе не выполнится посылка --- на самом деле нашлась одна ($n = 1$) прогрессия, имеющая нужный цветовой тип и общую точку в конце (прогрессия одна, все точки общие).

                \underline{Переход:}
                Здесь $U(n) = 2U(n - 1) \cdot W(k - 1, c^{U(n - 1)})$.
                Разобьём эти $U(n)$ чисел на блоки размера $U(n - 1)$.

                Возьмём первую половину, первые $W(k - 1, c^{U(n - 1)})$ из них.
                Там найдётся арифметическая прогрессия из $k - 1$ такого блока, причём блоки, входящие в эту прогрессию, будут полностью совпадать по цвету.

                Если в этих блоках в каждом есть прогрессия длины $k$, то мы нашли прогрессию длины $k$ среди $U(n)$ чисел.
                Иначе в каждом таком блоке есть одинаковые структуры из $n - 1$ прогрессии с общей конечной точкой.
                Пусть это блоки $i_1, \dots, i_{k - 1}$.

                Пусть $j$-я прогрессия внутри каждого блока индексируется $\underbrace{\text{idx}_{j,1}, \dots, \text{idx}_{j,k-1}}_{\text{цвета }j}, \text{idx}_{j,k}$.
                $\text{idx}_{j,k}$ одинаковы для всех $j$. Обозначим их общее значение $\text{idx}_k$

                Тогда заметим структуру для $n$ арифметических прогрессий с общей точкой в конце:
                $i$-я прогрессия имеет вид
                \[\all{U(n - 1) \cdot i_1 + \text{idx}_{j,1}, \dots, U(n - 1) \cdot i_{k-1} + \text{idx}_{j,k-1}, U(n - 1) \cdot i_k + \text{idx}_{j,k},&j < n \\
                U(n - 1) \cdot i_1 + \text{idx}_{k}, \dots, U(n - 1) \cdot i_{k-1} + \text{idx}_{k}, U(n - 1) \cdot i_k + \text{idx}_{k},&j = n}\]

                Несложно убедиться, что все эти прогрессии имеют разные цвета в начале и общую точку в конце.
            }
            Применим лемму для $n = c$.
            Получим условие, что если среди $\{1, \dots, U(c)\}$ нет структуры, состоящей из $n$ прогрессий с разными цветами в начале и общей точкой другого цвета в конце --- а их действительно нет, для такой структуры нужен как минимум $n + 1$ цвет --- то найдётся одноцветная прогрессия длины $k$.
        }
    }
    \theorem[Семереди, 1975] {
        Для любой плотности $\delta \in (0; 1)$ и любого $k \in \N$ имеется число $N(k, \delta)$:
        любое подмножество $\{1, \dots, N(k, \delta)\}$ мощности $\lfloor \delta \cdot N(k, \delta) \rfloor$ содержит арифметическую прогрессию длины $k$.
    }
\end{document}

